% no breaks in math mode
\relpenalty=10000
\binoppenalty=10000

\newcommand{\emphindex}[2][]{%
\emph{#2}%
\ifthenelse{\equal{#1}{}}{\index{#2}}{\index{#1}}%
}

\newcommand{\abs}[1]{{\lvert#1\rvert}}
\newcommand{\norm}[1]{\left\Vert#1\right\Vert}
\newcommand{\RR}{\mathbb{R}}
\newcommand{\QQ}{\mathbb{Q}}
\newcommand{\ZZ}{\mathbb{Z}}
\newcommand{\NN}{\mathbb{N}}
\newcommand{\Ccal}{\mathcal{C}}
\newcommand{\Dcal}{\mathcal{D}}
\newcommand{\Ical}{\mathcal{I}}
\newcommand{\inv}{^{-1}}
\newcommand{\openinterval}[2]{\mathopen{}\left]#1,#2\right[\mathclose{}}
\newcommand{\AMod}{{\ensuremath{\mathbf{A}{\operatorname{\mathbf{-Mod}}}}}}
\newcommand{\Set}{{\ensuremath{{\operatorname{\mathbf{Sets}}}}}}
\newcommand{\FinSets}{{\texttt{FinSets}}}
\newcommand{\SkeletalFinSets}{{\texttt{SkeletalFinSets}}}
\newcommand{\Gset}{{\texorpdfstring{\ensuremath{\mathbf{G}{\operatorname{\mathbf{-Sets}}}}}{G-Sets}}}
\newcommand{\SkeletalGSets}{{\texorpdfstring{\ensuremath{{\operatorname{\mathbf{Skeletal-}}}\mathbf{G}{\operatorname{\mathbf{-Sets}}}}}{Skeletal-G-Sets}}}
\newcommand{\CapPkg}{\textsc{Cap}}
\newcommand{\ReprSet}{\mathcal{U}}

% greek letters
\renewcommand{\epsilon}{\varepsilon}
\renewcommand{\theta}{\vartheta}
\renewcommand{\phi}{\varphi}

% algorithms
\let\oldnl\nl% Store \nl in \oldnl
\newcommand{\nonl}{\renewcommand{\nl}{\let\nl\oldnl}}% Remove line number for one line
\newcommand{\rememberlines}{\nonl \hfill \ensuremath{\hookrightarrow\space} \xdef\rememberedlines{\number\value{AlgoLine}}}
\newcommand{\resumenumbering}{\setcounter{AlgoLine}{\rememberedlines}}
\SetKw{KwBreak}{break}
\SetKwRepeat{Do}{do}{while}
\SetKwInput{KwNote}{Note}
\SetKwInput{KwImpl}{Implementation}

% theorems
\theoremstyle{definition}
\newtheorem{defn}{Definition}[section]
\newtheorem{exmp}[defn]{Example}

\theoremstyle{plain}
\newtheorem{thm}[defn]{Theorem}
\newtheorem{prop}[defn]{Proposition}
\newtheorem{lem}[defn]{Lemma}
\newtheorem{cor}[defn]{Corollary}

\theoremstyle{definition}
\newtheorem{rem}[defn]{Remark}
\newtheorem{constr}[defn]{Construction}

\newtheoremstyle{break}
{}{}%
{}{}%
{\bfseries}{}%  % Note that final punctuation is omitted.
{\newline}{}
\theoremstyle{break}
\newtheorem{algo}[defn]{Algorithm}

\newcommand{\Defnref}[1]{%
	\hyperref[defn:#1]{Definition~\ref*{defn:#1}}%
}
\newcommand{\defnref}[1]{%
	\hyperref[defn:#1]{Definition~\ref*{defn:#1}}%
}
\newcommand{\Exmpref}[1]{%
	\hyperref[exmp:#1]{Example~\ref*{exmp:#1}}%
}
\newcommand{\exmpref}[1]{%
	\hyperref[exmp:#1]{Example~\ref*{exmp:#1}}%
}
\newcommand{\Thmref}[1]{%
	\hyperref[thm:#1]{Theorem~\ref*{thm:#1}}%
}
\newcommand{\thmref}[1]{%
	\hyperref[thm:#1]{Theorem~\ref*{thm:#1}}%
}
\newcommand{\Propref}[1]{%
	\hyperref[prop:#1]{Proposition~\ref*{prop:#1}}%
}
\newcommand{\propref}[1]{%
	\hyperref[prop:#1]{Proposition~\ref*{prop:#1}}%
}
\newcommand{\Lemref}[1]{%
	\hyperref[lem:#1]{Lemma~\ref*{lem:#1}}%
}
\newcommand{\lemref}[1]{%
	\hyperref[lem:#1]{Lemma~\ref*{lem:#1}}%
}
\newcommand{\Remref}[1]{%
	\hyperref[rem:#1]{Remark~\ref*{rem:#1}}%
}
\newcommand{\remref}[1]{%
	\hyperref[rem:#1]{Remark~\ref*{rem:#1}}%
}
\newcommand{\Figref}[1]{%
	\hyperref[fig:#1]{Figure~\ref*{fig:#1}}%
}
\newcommand{\figref}[1]{%
	\hyperref[fig:#1]{Figure~\ref*{fig:#1}}%
}
\newcommand{\Appref}[1]{%
	\hyperref[app:#1]{Appendix~\ref*{app:#1}}%
}
\newcommand{\appref}[1]{%
	\hyperref[app:#1]{Appendix~\ref*{app:#1}}%
}
\newcommand{\Algoref}[1]{%
	\hyperref[algo:#1]{Algorithm~\ref*{algo:#1}}%
}
\newcommand{\algoref}[1]{%
	\hyperref[algo:#1]{Algorithm~\ref*{algo:#1}}%
}
\newcommand{\Funcref}[1]{%
	\hyperref[func:#1]{Function~\ref*{func:#1}}%
}
\newcommand{\funcref}[1]{%
	\hyperref[func:#1]{Function~\ref*{func:#1}}%
}
