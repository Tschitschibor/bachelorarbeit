% mainfile: ../main.tex

\section{The Tannaka Reconstruction Theorem}
In this section, we will state a general version of the Tannaka Reconstruction Theorem and afterwards give a proof in our setting. We will not define all the notions in the general version. For details see \cite{Tannaka_duality_in_nLab}. Afterwards, we show that the results carry over if we replace $\Gset$ by a category $\Ccal$ which is equivalent to $\Gset$ in a suitable way. We start by defining some notions and proving some results which we will need later.

\begin{rem}[Opposite monoids and anti-isomorphisms]\label{rem:opposite_monoid}
To any monoid $M = (M,\cdot)$ we can associate the \emphindex{opposite monoid} $M^{\op} \coloneqq (M,*)$ with opposite multiplication $m * n \coloneqq n \cdot m$ for all $m,n \in M$. In this setting it makes sense to define \emphindex[anti-isomorphism]{anti-isomorphisms}, that is, bijective maps $\phi\colon M \to N$ between monoids such that $\phi(m \cdot n) = \phi(n) \cdot \phi(m)$ for all $m,n \in M$. Then the identity on sets $\id\colon M \to M^{\op}$ is an anti-isomorphism. Note that if we can invert elements, that is, if $M = G$ is a group, then inverting elements is an anti-isomorphism $G \to G$ respectively an isomorphism $G \to G^{\op}$. Since the composition of two anti-isomorphisms is an isomorphism, by composing with the inversion we can make any anti-isomorphism $M \to N$ of monoids into an isomorphism if $M$ or $N$ is a group. In particular, if we want to show that a group and a monoid are isomorphic as monoids, it suffices to find an anti-isomorphism between them. Additionally, if a group $G$ and a monoid $M$ are isomorphic as monoids, then $M$ is already a group and $G$ and $M$ are isomorphic as groups. To check this, let $\phi\colon G \to M$ be an isomorphism of monoids. For $m \in M$ define $m' \coloneqq \phi((\phi\inv(m))\inv)$. Then we have:
\begin{align*}
m'm &= \phi((\phi\inv(m))\inv) m\\
    &= \phi((\phi\inv(m))\inv) \phi(\phi\inv(m))\\
    &= \phi((\phi\inv(m))\inv \phi\inv(m))\\
    &= \phi(1)\\
    &= 1
\end{align*}
and analogously $m m' = 1$. Thus, $m'$ is an inverse of $m$, $M$ is a group, and $G$ and $M$ are isomorphic as groups via $\phi$.
\end{rem}

\begin{rem}[Natural transformations as elements of a product]
Let $\Ccal$ and $\Dcal$ be categories and let $F,F'\colon \Ccal \to \Dcal$ be functors. A natural transformation $\alpha\colon F \to F'$ is given by components $\alpha_C$ for all objects $C$ in $\Ccal$ where $\alpha_C\colon F(C) \to F'(C)$ is a morphism in $\Dcal$. Thus, we can think of $\alpha$ as the tuple $(\alpha_C)_{C \in \Obj(\Ccal)} \in \prod_{C \in \Obj(\Ccal)} \Hom_{\Dcal}(F(C),F'(C))$. Conversely, let $(\alpha_C)_{C \in \Obj(\Ccal)} \in \prod_{C \in \Obj(\Ccal)} \Hom_\Dcal(F(C),F'(C))$. Then we can check if for all $C_1,C_2 \in \Obj(\Ccal)$ and $\phi \in \Hom_\Ccal(C_1,C_2)$ the following diagram commutes:
\[
\begin{tikzcd}
F(C_1) \arrow[r, "F(\phi)"] \arrow[d, "\alpha_{C_1}"] & F(C_2) \arrow[d, "\alpha_{C_2}"] \\
F'(C_1) \arrow[r, "F'(\phi)"]                         & F'(C_2)                         
\end{tikzcd}
\]
If this is the case, then $(\alpha_C)_{C \in \Obj(\Ccal)}$ defines a natural transformation $\alpha\colon F \to F'$ with components $\alpha_C$. In this sense, the class of natural transformations $F \to F'$ is a subclass of $\prod_{C \in \Obj(\Ccal)} \Hom_{\Dcal}(F(C),F'(C))$.
\end{rem}

\begin{lem}\label{lem:isomorphic_homs}
Let $\Ccal$ be a category and $A \cong B$ isomorphic objects in $\Ccal$. Then $\Hom_\Ccal(A,A)$ and $\Hom_\Ccal(B,B)$ are isomorphic as monoids with regard to the composition of morphisms.
\end{lem}
\begin{proof}
Let $\phi\colon A \to B$ be an isomorphism. This isomorphism induces the map
\begin{align*}
f\colon \Hom_\Ccal(A,A) &\to \Hom_\Ccal(B,B)\\
\tau &\mapsto \phi \circ \tau \circ \phi\inv
\end{align*}
which makes the following diagram commute:
\[
\begin{tikzcd}
A \arrow[r, "\tau"] \arrow[d, "\phi"'] & A \arrow[d, "\phi"] \\
B \arrow[r, "f(\tau)"']                & B                  
\end{tikzcd}
\]
Then for all $\tau_1,\tau_2 \in \Hom_\Ccal(A,A)$ we have \[f(\tau_1 \circ \tau_2) = \phi \circ \tau_1 \circ \tau_2 \circ \phi\inv = \phi \circ \tau_1 \circ \phi\inv \circ \phi \circ \tau_2 \circ \phi\inv = f(\tau_1) \circ f(\tau_2),\] that is, $f$ is an isomorphism of monoids.
\end{proof}

For the statement of the theorem we need the notion of an \emph{end} of a functor. An end can be defined in many equivalent ways. We will give the definition as an equalizer in enriched category theory, which fits best into our setting.

\begin{defn}[End \cite{end_in_nLab}]\label{defn:general_end}
Let $V$ be a symmetric monoidal category, $\Ccal$ a $V$-enriched category and $F'\colon \Ccal^{\op} \times \Ccal \to V$ a $V$-enriched functor. The \emphindex[end!general definition]{end} of $F'$ is the equalizer $\End(F')$ of the morphisms $\rho$ and $\lambda$ defined as follows: \[\rho,\lambda\colon \prod_{C \in \Obj(\Ccal)} F'(C,C) \longrightarrow \prod_{C_1,C_2 \in \Obj(\Ccal)} [\Hom_\Ccal(C_1,C_2),F'(C_1,C_2)]\] where $[-,-]$ is the internal hom in $V$ and the components of $\rho$ and $\lambda$ are given by: \[\rho_{C_1,C_2}\colon \prod_{C \in \Obj(\Ccal)} F'(C,C) \xrightarrow{\pi_{C_1}} F'(C_1,C_1) \longrightarrow [\Hom_\Ccal(C_1,C_2),F'(C_1,C_2)]\] where the second arrow is the adjunct of $F'(C_1,-)\colon \Hom_\Ccal(C_1,C_2) \to [F'(C_1,C_1),F'(C_1,C_2)]$ and \[\lambda_{C_1,C_2}\colon \prod_{C \in \Obj(\Ccal)} F'(C,C) \xrightarrow{\pi_{C_2}} F'(C_2,C_2) \longrightarrow [\Hom_\Ccal(C_1,C_2),F'(C_1,C_2)]\] where the second arrow is the adjunct of $F'(-,C_2)\colon \Hom_\Ccal(C_1,C_2) \to [F'(C_2,C_2),F'(C_1,C_2)]$. For a functor $F\colon \Ccal \to V$ we define the \emph{end} $\End(F) \coloneqq \End(\Hom_V(F(-),F(-)))$.
\end{defn}

\begin{thm}[Tannaka Reconstruction Theorem \cite{Tannaka_duality_in_nLab}]\label{thm:general_tannaka}
Let $V$ be a locally small closed symmetric monoidal category with all limits and $A$ a monoid in $V$. Let \AMod{} be the $V$-enriched category of all $A$-modules in $V$ and let $F\colon \AMod \to V$ be the forgetful fiber functor. Then $A \cong \End(F)$, that is, we can reconstruct $A$ from the category \AMod{} and the forgetful fiber functor $F$.
\end{thm}

In our setting we choose $V = \Set$ and $A = G$ to be a finite group. The category \AMod{} is just \Gset{} and $F = U\colon \Gset \to \Set$ is the forgetful functor. We first show that in this setting $\End(U)$ coincides with the class of all natural transformations $U \to U$ and afterwards phrase and prove the theorem in our setting.

\begin{prop}\label{prop:end_as_equalizer_concrete}
Let $\Ccal$ be a category and $F\colon \Ccal \to \Set$ a functor. For $C \in \Obj(\Ccal)$ set \[S_C \coloneqq \Hom_\Set(F(C),F(C))\] and for $C_1,C_2 \in \Obj(\Ccal)$ set \[T_{C_1,C_2} \coloneqq \Hom_\Set(\Hom_\Ccal(C_1,C_2),\Hom_\Set(F(C_1),F(C_2))).\] We set \[S \coloneqq \prod_{C \in \Obj(\Ccal)} S_C\] and \[T \coloneqq \prod_{C_1,C_2 \in \Obj(\Ccal)} T_{C_1,C_2}.\] Let $\rho \colon S \to T$ be the unique morphism in \Set{} with components given by
\[
\rho_{C_1,C_2}\colon S \stackrel{\pi_{C_1}}{\longrightarrow} S_{C_1} \to T_{C_1,C_2}
\]
where the second arrow maps $f \in \Hom_\Set(F(C_1),F(C_1))$ to the map
\begin{align*}
\Hom_\Ccal(C_1,C_2) &\to \Hom_\Set(F(C_1),F(C_2)))\\
\phi &\mapsto F(\phi) \circ f
\end{align*}
Analogously, let $\lambda \colon S \to T$ be the unique morphism in \Set{} with components given by
\[
\lambda_{C_1,C_2}\colon S \stackrel{\pi_{C_2}}{\longrightarrow} S_{C_2} \to T_{C_1,C_2}
\]
where the second arrow maps $f \in \Hom_\Set(F(C_2),F(C_2))$ to the map
\begin{align*}
\Hom_\Ccal(C_1,C_2) &\to \Hom_\Set(F(C_1),F(C_2)))\\
\phi &\mapsto f \circ F(\phi)
\end{align*}
Then the equalizer $E$ of $\rho$ and $\lambda$ is the class of all natural transformations $F \to F$, that is, $E = \Hom(F,F)$.
\end{prop}
\begin{proof}
Let $\alpha = (\alpha_C)_{C \in \Obj(\Ccal)} \in S$. Then $\alpha$ is an element of $E$ if and only if $\rho(\alpha) = \lambda(\alpha)$. By the universal property of the product this is equivalent to $\rho_{C_1,C_2}(\alpha) = \lambda_{C_1,C_2}(\alpha)$ for all $C_1,C_2 \in \Obj(\Ccal)$. Since $\rho_{C_1,C_2}(\alpha)$ and $\lambda_{C_1,C_2}(\alpha)$ are elements of $T_{C_1,C_2}$, that is, maps themselves, they are equal if and only if $\rho_{C_1,C_2}(\alpha)(\phi) = \lambda_{C_1,C_2}(\alpha)(\phi)$ for all $\phi \in \Hom_\Ccal(C_1,C_2)$. By the definition of $\rho_{C_1,C_2}$ and $\lambda_{C_1,C_2}$, this is equivalent to $F(\phi) \circ \alpha_{C_1} = \alpha_{C_2} \circ F(\phi)$ for all $\phi \in \Hom_\Ccal(C_1,C_2)$.

Summing up, we get that $\rho(\alpha) = \lambda(\alpha)$ if and only if $F(\phi) \circ \alpha_{C_1} = \alpha_{C_2} \circ F(\phi)$ for all $C_1,C_2 \in \Obj(\Ccal)$ and $\phi \in \Hom_\Ccal(C_1,C_2)$. This is exactly the definition of a natural transformation $F \to F$.
\end{proof}

\begin{rem}[Redefinition of the end]\label{rem:redefinition_of_the_end}
We have not shown that the assumptions in \propref{end_as_equalizer_concrete} match the definition of the end applied to the case $V = \Set$. Thus, for completeness we redefine the end in the case $V = \Set$: Let $\Ccal$ be a category and $F\colon \Ccal \to \Set$ a functor. Then we redefine the \emphindex[end!redefinition for $V = \Set$]{end} of $F$ as $\End(F) \coloneqq E = \Hom(F,F)$ where $E$ is the equalizer in \propref{end_as_equalizer_concrete}.
\end{rem}

\begin{thm}[Tannaka Reconstruction Theorem for \Gset \cite{Tannaka_duality_in_nLab}]\label{thm:tannaka_duality_for_Gset}
Let $G$ be a finite group. Let \Gset{} be the category of finite $G$-sets and let $U\colon \Gset \to \Set$ be the forgetful functor. Then $G$ can be reconstructed as the natural transformations $U \to U$ together with the composition, that is, $G \cong \End(U)$ as groups. In particular, the class of natural transformations $U \to U$ is a finite set.
\end{thm}
\begin{proof}
By \lemref{representability_of_U}, we know that $U \cong \Hom_{\Gset}(G,-)$ as functors $\Gset \to \Set$. Thus, we have the following canonical isomorphisms as sets, that is, bijective maps:
\begin{align}
\End(U) &= \Hom(U,U)\nonumber\\
&\cong \Hom(\Hom_{\Gset}(G,-),\Hom_{\Gset}(G,-)) \label{eq:isomorphic_homs}\\
&\cong \Hom_{\Gset}(G,G) \label{eq:Yoneda}\\
&\cong U(G) \label{eq:representability_of_U}
\end{align}
where the bijection in \eqref{eq:isomorphic_homs} is the claim of \lemref{isomorphic_homs}, the bijection in \eqref{eq:Yoneda} is the claim of the Yoneda lemma and the bijection in \eqref{eq:representability_of_U} is the map $\alpha_G'$ in \lemref{representability_of_U}. Now, we regard hom-sets as monoids with respect to the composition of morphisms, and $U(G)$ as a group with its former group structure $G$. We show that each of the bijections is an isomorphism or anti-isomorphism of monoids. Since the chain of bijections ends with a group, by \remref{opposite_monoid} this is sufficient to prove the claim.

By \lemref{isomorphic_homs}, we know that the bijection in \eqref{eq:isomorphic_homs} is an isomorphism of monoids. The bijection in \eqref{eq:Yoneda} is the map $f \mapsto f^*$ in the Yoneda lemma, which is an anti-isomorphism of monoids because $(f_1 \circ f_2)^* = f_2^* \circ f_1^*$. We show that $f = \alpha_G'$ is an isomorphism of monoids: Let $\phi_1,\phi_2 \in \Hom_{\Gset}(G,G)$. Then we have \[f(\phi_1 \circ \phi_2) = U(\phi_1 \circ \phi_2)(1) = U(\phi_1)(U(\phi_2)(1)) = U(\phi_1)(1) \cdot U(\phi_2)(1) = f(\phi_1) \cdot f(\phi_2)\] by the same arguments as in \remref{composition_of_G_morphisms}.

Summing up, we get $\End(U) \cong G$ as groups.
\end{proof}

In our setting we do not know the group $G$ and thus do not have the category \Gset{}. We only have a category $\Ccal$ and a functor $F\colon \Ccal \to \Set$ such that there exists an equivalence of categories $Z\colon \Ccal \to \Gset$ for some group $G$ with $F \cong U \circ Z$. Yet, we still want to reconstruct $G$. To this end, we show that we can also apply the theorem above for $\Ccal$ and $F$ instead of $\Gset$ and $U$. For this, we first need a lemma.

\begin{lem}\label{lem:natural_transformation_on_full_subcategory}
Let $\Ccal, \Dcal$ be categories and $F,F' \colon \Ccal \to \Dcal$ be functors. Then any natural transformation $\alpha\colon F \to F'$ is already determined by its values on representatives of isomorphism classes of objects in $\Ccal$. Additionally, let $\Ccal'$ be a full subcategory of $\Ccal$ such that $\Obj(\Ccal')$ contains at least one representative of each isomorphism class of objects in $\Ccal$. Then any natural transformation $\beta\colon F\vert_{\Ccal'} \to F'\vert_{\Ccal'}$ extends uniquely to a natural transformation $F \to F'$.
\end{lem}
\begin{proof}
Let $\alpha\colon F \to F'$ be a natural transformation, let $C_1$ and $C_2$ be objects in $\Ccal$ and let $\phi \colon C_1 \to C_2$ be an isomorphism. Then we get the following commutative diagram:
\[
\begin{tikzcd}
F(C_1) \arrow[r, "F(\phi)"] \arrow[d, "\alpha_{C_1}"]  & F(C_2) \arrow[d, "\alpha_{C_2}"] \\
F'(C_1) \arrow[r, "F'(\phi)"]                         & F'(C_2)                        
\end{tikzcd}
\]
In particular, the morphisms $F(\phi)$ and $F'(\phi)$ are also isomorphisms and thus we have \[\alpha_{C_1} = (F'(\varphi))\inv \circ \alpha_{C_2} \circ F(\phi),\] so $\alpha_{C_1}$ is uniquely determined by $\alpha_{C_2}$.

Now let $\beta\colon F\vert_{\Ccal'} \to F'\vert_{\Ccal'}$ be a natural transformation. Let $C$ be an object in $\Ccal$. By assumption there exists an object $C'$ in $\Ccal'$ such that $C \cong C'$. Fix an isomorphism $\phi_C \colon C \to C'$. Then we define $\alpha_C \coloneqq (F'(\phi_C))\inv \circ \beta_{C'} \circ F(\phi_C)$ and claim that this yields a natural transformation $\alpha\colon F \to F'$. For this, let $C_1$ and $C_2$ be objects in $\Ccal$ and $\phi \colon C_1 \to C_2$ a morphism. We have to show that the following diagram commutes:
\[
\begin{tikzcd}
F(C_1) \arrow[r, "F(\phi)"] \arrow[d, "\alpha_{C_1}"]  & F(C_2) \arrow[d, "\alpha_{C_2}"] \\
F'(C_1) \arrow[r, "F'(\phi)"]                         & F'(C_2)                        
\end{tikzcd}
\]
Every square in the following diagram commutes:
\[
\begin{tikzcd}
F(C_1) \arrow[d, "\alpha_{C_1}"] \arrow[rr, "F(\phi_{C_1})"] &  & F(C_1') \arrow[rrr, "F(\phi_{C_2} \circ \phi \circ \phi_{C_1}\inv)"] \arrow[d, "\beta_{C_1'}"] &  &  & F(C_2') \arrow[rr, "F(\phi_{C_2}\inv)"] \arrow[d, "\beta_{C_2'}"] &  & F(C_2) \arrow[d, "\alpha_{C_2}"] \\
F'(C_1) \arrow[rr, "F'(\phi_{C_1})"]                         &  & F'(C_1') \arrow[rrr, "F'(\phi_{C_2} \circ \phi \circ \phi_{C_1}\inv)"]                         &  &  & F'(C_2') \arrow[rr, "F'(\phi_{C_2}\inv)"]                         &  & F'(C_2)                         
\end{tikzcd}
\]
The left and the right square commute by definition of $\alpha$ and because functors commute with taking inverses. The central square commutes by assumption. Note that we need that $\Ccal'$ is a full subcategory since otherwise $\phi_{C_2} \circ \phi \circ \phi_{C_1}\inv$ might not be a morphism in $\Ccal'$. Thus, the outer rectangle commutes, which is just the diagram above.

The uniqueness of $\alpha$ follows from the first part.

Additionally, the uniqueness shows that the construction of $\alpha$ is independent of the choice of the isomorphisms $\phi_C$. In particular, for $C' \in \Obj(\Ccal')$ we have $\alpha_{C'} = \beta_{C'}$ because we can choose $\phi_{C'} = \id_{C'}$.
\end{proof}

\begin{prop}\label{prop:generalization_via_equivalence}
Let $\Ccal'$, $\Ccal$, and $\Dcal$ be categories. Let $Z\colon \Ccal' \to \Dcal$ be an equivalence of categories and let $F'\colon \Ccal' \to \Dcal$ and $F\colon \Ccal \to \Dcal$ be functors such that $F' \cong F \circ Z$. That is, the following diagram commutes up to isomorphism of functors:
\[
\begin{tikzcd}
\Ccal' \arrow[rr, "Z"] \arrow[rd, "F'"'] &       & \Ccal \arrow[ld, "F"] \\
                                         & \Dcal &                      
\end{tikzcd}
\]
Then $\Hom(F',F')$ and $\Hom(F,F)$ are isomorphic as monoids.
\end{prop}
\begin{proof}
By \lemref{isomorphic_homs}, we have $\Hom(F',F') \cong \Hom(F \circ Z,F \circ Z)$ as monoids. Thus, we only have to show $\Hom(F \circ Z,F \circ Z) \cong \Hom(F,F)$ as monoids. For this, let $\tilde \Ccal$ be the subcategory of $\Ccal$ consisting of all objects $Z(C')$ with $C' \in \Obj(\Ccal')$ and all morphisms $Z(\phi)$ where $\phi$ is a morphism in $\Ccal'$. Let $\beta \in \Hom(F \circ Z,F \circ Z)$. Then, for all objects $C'_1$ and $C'_2$ in $\Ccal'$ and all morphisms $\phi\colon C'_1 \to C'_2$ the following diagram commutes:
\[
\begin{tikzcd}
F(Z(C'_1)) \arrow[r, "F(Z(\phi))"] \arrow[d, "\beta_{C'_1}"] & F(Z(C'_2)) \arrow[d, "\beta_{C'_2}"] \\
F(Z(C'_1)) \arrow[r, "F(Z(\phi))"]                           & F(Z(C'_2))                         
\end{tikzcd}
\]
This implies that $\beta$ is a natural transformation $F\vert_{\tilde \Ccal} \to F\vert_{\tilde \Ccal}$. Since $Z$ is an equivalence of categories, it is essentially surjective and fully faithful, so by \lemref{natural_transformation_on_full_subcategory} we can extend $\beta$ uniquely to a natural transformation $\alpha\colon F \to F$. This extension is injective since restricting $\alpha$ to $\tilde \Ccal$ gives back $\beta$. 
Again by \lemref{natural_transformation_on_full_subcategory}, every natural transformation $F \to F$ arises in this way. Thus, we have a bijection. It remains to show that this is an isomorphism of monoids. For this, let $\beta_1, \beta_2 \in \Hom(F\vert_{\tilde \Ccal},F\vert_{\tilde \Ccal})$ and $\beta = \beta_1 \circ \beta_2$. Let $\alpha_1$, $\alpha_2$ and $\alpha$ be the extensions of $\beta_1$, $\beta_2$ and $\beta$ to $\Hom(F,F)$ respectively. We have to show that $\alpha = \alpha_1 \circ \alpha_2$, that is, that for all objects $C$ in $\Ccal$ we have $\alpha_C = (\alpha_1)_C \circ (\alpha_2)_C$. Let $C'$ be an object in $\tilde \Ccal$ with $C \cong C'$ and let $\phi\colon C \to C'$ be an isomorphism. Then by definition we have
\begin{align*}
\alpha_C &= (F(\phi))\inv \circ \beta_{C'} \circ F(\phi)\\
&= (F(\phi))\inv \circ (\beta_1)_{C'} \circ (\beta_2)_{C'} \circ F(\phi)\\
&= (F(\phi))\inv \circ (\beta_1)_{C'} \circ F(\phi) \circ (F(\phi))\inv \circ (\beta_2)_{C'} \circ F(\phi)\\
&= (\alpha_1)_C \circ (\alpha_2)_C.
\end{align*}
This finishes the proof.
\end{proof}

\begin{cor}
Let $\Ccal$ be a category equivalent to \Gset{} via an equivalence $Z \colon \Ccal \to \Gset$ and let $F\colon \Ccal \to \Set$ be a functor such that $F \cong U \circ Z$. Then $\End(F) \cong G$ as groups.
\end{cor}
\begin{proof}
By \remref{redefinition_of_the_end}, \propref{generalization_via_equivalence}, and \thmref{tannaka_duality_for_Gset}, we have \[\End(F) = \Hom(F,F) \cong \Hom(U,U) \cong G\] as monoids and, since $G$ is a group, also as groups.
\end{proof}
