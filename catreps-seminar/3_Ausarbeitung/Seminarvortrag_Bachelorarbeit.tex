\documentclass[12pt,compress]{beamer}

\setbeamertemplate{navigation symbols}{}

\usepackage[utf8]{inputenc}
\usepackage{tikz}
\usetikzlibrary{mindmap,trees,shadows}
\usepackage{tikz-cd}
\usepackage{mathtools}
\usepackage{bbm}
\usepackage{makecell}
\usepackage{ifthen}
\usetikzlibrary{calc}


%%% For algorithm styles
\usepackage{xspace}
\usepackage[linesnumbered,ruled]{algorithm2e}
\usepackage{algpseudocode}
\SetKw{Continue}{continue}
\SetKw{Break}{break}
\SetKw{Not}{not\xspace}
\SetKw{AndAlg}{and\xspace}

\usepackage{algcompatible}

\usepackage{adjustbox}
\usepackage{tabularx,booktabs}
\usepackage{xcolor,colortbl,etoolbox}
 
\makeatletter
\long\def\long@firstofone#1{#1}
\long\def\@getnextbraced#1#2#3{#2\long@firstofone{#3{#1}}}
\long\def\@braced@unexpanded#1\long@firstofone#2#3{%
   #1\long@firstofone{#2{#3}}}
\protected\long\def\@braced@expandedfully#1\long@firstofone#2#3{%
   \edef\@expandargs@internal{{#3}}%
   \expandafter\@getnextbraced\@expandargs@internal{#1}{#2}%
}
\protected\long\def\expandtabularargs{%
   \@braced@unexpanded
   \@braced@expandedfully
   \long@firstofone
}
\makeatother
 
\makeatletter
\newenvironment{breakalgo}[2][alg:\thealgorithm]{%
  \def\@fs@cfont{\bfseries}%
  \let\@fs@capt\relax%
  \par\noindent%
  \medskip%
  \rule{\linewidth}{.8pt}%
  \vspace{-3pt}%
  \captionof{algorithm}{#2}\label{#1}%
  \vspace{-1.7\baselineskip}%
  \noindent\rule{\linewidth}{.4pt}%
  \vspace{-1.3\baselineskip}%
}{%
  \vspace{-.75\baselineskip}%
  \rule{\linewidth}{.4pt}%
  \medskip%
}
\makeatother

%%% For proper underline
\usepackage{soul}
%\setuldepth{gjpqy}
%\setuldepth\strut
\setuldepth{-1}

%%% Macros for our recurring categories
\newcommand{\kmat}{\Bbbk\textnormal{-}\mathbf{mat}}
\newcommand{\kAlgebroid}{\Bbbk\textnormal{-}\mathrm{algebroid}}
\newcommand{\Rmat}{R\textnormal{-}\mathbf{mat}}
\newcommand{\HomAkmat}{\mathrm{Hom}_{\Bbbk}(\mathcal{A},\Bbbk\textnormal{-}\mathrm{Mat})}
\newcommand{\HomARmat}{\mathrm{Hom_{R}}(\mathcal{A},\Rmat)}
\newcommand{\HomA}{\mathrm{Hom}_{\mathcal{A}}}
\newcommand{\FinSets}{\mathrm{FinSets}}
\newcommand{\Cat}{\mathrm{\textbf{Cat}}}
\newcommand{\Set}{\mathrm{\textbf{Set}}}
\newcommand{\Quiv}{\mathrm{\textbf{Quiv}}}
\newcommand{\kChat}{\widehat{\Bbbk\mathcal{C}}}
\newcommand{\fpC}{\mathrm{fp}\mathcal{C}}

\newcommand\overviewslide{1}

\newcounter{saveenumi}
\newcommand{\seti}{\setcounter{saveenumi}{\value{enumi}}}
\newcommand{\conti}{\setcounter{enumi}{\value{saveenumi}}}

\resetcounteronoverlays{saveenumi}


\newcommand{\visiblesub}[2]{
  \ifthenelse{\ifnum#1>#2}
  { 0; }
  { 1; }
}





% If you wish to uncover everything in a step-wise fashion, uncomment
% the following command: 

\beamerdefaultoverlayspecification{<+->}

\title[The category of representations]
{The category of representations of a concrete category as a functor category}

\author{Tibor Gr{\"u}n}

%\date{October 30, 2020} % date of printed version
%\date{November 3, 2020} % date of submission
%\date{November 16, 2020} % date of seminar presentation
\date{March 22, 2021} % date of seminar presentation
\begin{document}

  %% to uncover arrows and nodes one after another
  %% use: 
  %% |[visible on=<1->]|\mathrm{FinSets}
  % in front of node
  %% \arrow[rd, visible on=<0>]
  % inside arrow
  % 1- means from first slide on
  % 2 only on second slide
  % -3 on slides 1,2,3
  % 0 always invisible
    \tikzset{
      invisible/.style={opacity=0.3},
      visible on/.style={alt={#1{}{invisible}}},
      alt/.code args={<#1>#2#3}{%
        \alt<#1>{ \pgfkeysalso{#2} }{ \pgfkeysalso{#3} }%
      }
  }

  % Keys to support piece-wise uncovering of elements in TikZ pictures:
  % \node[visible on=<2->](foo){Foo}
  % \node[visible on=<{2,4}>](bar){Bar}   % put braces around comma expressions
  %
  % Internally works by setting opacity=0 when invisible, which has the 
  % adavantage (compared to \node<2->(foo){Foo} that the node is always there, hence
  % always consumes space plus that coordinate (foo) is always available.
  %
  % The actual command that implements the invisibility can be overriden
  % by altering the style invisible. For instance \tikzset{invisible/.style={opacity=0.2}}
  % would dim the "invisible" parts. Alternatively, the color might be set to white, if the
  % output driver does not support transparencies (e.g., PS) 
  %

\begin{frame}
  \titlepage
\end{frame}

\section{Short introduction to Category theory}
\tikzset{invisible/.style={opacity=0.0}}
\begin{frame}[fragile]
\uncover<1->{
A \ul{quiver} (also called \ul{directed graph}) $q$ consists of a class of \ul{objects} (or \ul{vertices}) $q_{0} = \mathrm{Obj}\,q$
}
\uncover<3->{
and a class of
\ul{morphisms} (or \ul{arrows}) $q_{1} = \mathrm{Mor}\,q$
}
\[
\begin{tikzcd}[ampersand replacement=\&]
|[visible on=<2->]|1 \arrow[rddd, "b", visible on=<4->] \arrow[rrd, "a", visible on=<4->] \&         \&                    \&         \&                    \\
                                 \&         \& |[visible on=<2->]|3 \arrow[rd, "e", shift left, visible on=<4->] \&         \& |[visible on=<2->]|4 \arrow[ll, "d", visible on=<4->] \arrow[ll, "c"', bend right, shift right, visible on=<4->]  \\
                                 \&         \&                    \& |[visible on=<2->]|5 \arrow[lu, "f", shift left, visible on=<4->] \&                    \\
                                 \& |[visible on=<2->]|2 \&                    \&         \&                   
\end{tikzcd}
\]
\end{frame}

\begin{frame}
\uncover<1->{
together with two defining maps $s : q_{1} \longrightarrow q_{0}$ called \ul{source}, and
$t : q_{1} \longrightarrow q_{0}$ called \ul{target}.\\
}
\[
\begin{tikzcd}[ampersand replacement=\&]
|[visible on=<2->]|1 \arrow[rddd, "b", visible on=<2->] \arrow[rrd, "a", visible on=<2->] \&         \&                    \&         \&                    \\
                                 \&         \& |[visible on=<2->]|3 \arrow[rd, "e", shift left, visible on=<2->] \&         \& |[visible on=<2->]|4 \arrow[ll, "d", visible on=<2->] \arrow[ll, "c"', bend right, shift right, visible on=<2->]  \\
                                 \&         \&                    \& |[visible on=<2->]|5 \arrow[lu, "f", shift left, visible on=<2->] \&                    \\
                                 \& |[visible on=<2->]|2 \&                    \&         \&                   
\end{tikzcd}
\]
\uncover<3->{
For example
\begin{align*}
s(a) = s(b) = 1 \\
t(b) = 2 \\
t(a) = t(c) = t(d) = t(f) = 3
\end{align*}
etc. \\
}
\end{frame}

\begin{frame}
\uncover<1->{
In the finite case - which we will be dealing with most of the time here - we call $q_{0}$ \ul{set} of objects and and $q_{1}$ \ul{set} of morphisms.
}
\[
\begin{tikzcd}[ampersand replacement=\&]
|[visible on=<2->]|1 \arrow[rddd, "b", visible on=<2->] \arrow[rrd, "a", visible on=<2->] \&         \&                    \&         \&                    \\
                                 \&         \& |[visible on=<2->]|3 \arrow[rd, "e", shift left, visible on=<2->] \&         \& |[visible on=<2->]|4 \arrow[ll, "d", visible on=<2->] \arrow[ll, "c"', bend right, shift right, visible on=<2->]  \\
                                 \&         \&                    \& |[visible on=<2->]|5 \arrow[lu, "f", shift left, visible on=<2->] \&                    \\
                                 \& |[visible on=<2->]|2 \&                    \&         \&                   
\end{tikzcd}
\]
\uncover<3->{
So in the quiver $q$ we have
}
\uncover<4->{
\[
q_{0} = \{1,2,3,4,5\}
\]
and
\[
q_{1} = \{a,b,c,d,e,f\}
\]
}
\end{frame}

\begin{frame}
\uncover<1->
{
Another map relates the arrows with the objects. That is the Hom-set, i.e. set of morphisms, between two objects (order matters):
\[
\mathrm{Hom} : q_{0} \times q_{0} \longrightarrow \mathcal{P}(q_{1})
\]
}
\[
\begin{tikzcd}[ampersand replacement=\&]
|[visible on=<2->]|1 \arrow[rddd, "b", visible on=<2->] \arrow[rrd, "a", visible on=<2->] \&         \&                    \&         \&                    \\
                                 \&         \& |[visible on=<2->]|3 \arrow[rd, "e", shift left, visible on=<2->] \&         \& |[visible on=<2->]|4 \arrow[ll, "d", visible on=<2->] \arrow[ll, "c"', bend right, shift right, visible on=<2->]  \\
                                 \&         \&                    \& |[visible on=<2->]|5 \arrow[lu, "f", shift left, visible on=<2->] \&                    \\
                                 \& |[visible on=<2->]|2 \&                    \&         \&                   
\end{tikzcd}
\]
\uncover<3->
{
\begin{align*}
\mathrm{Hom}(1,3) &= \{a\} \\
\mathrm{Hom}(4,3) &= \{c,d\} \\
\mathrm{Hom}(3,4) &= \{\}
\end{align*}
}
\end{frame}

\begin{frame}[fragile]
\uncover<1->{
For the same object $X \in q_{0}$ we call $\mathrm{Hom}(X,X) = \mathrm{End}(X)$ the endomorphism set.
}
\[
\begin{tikzcd}
|[visible on=<2->]|1 \arrow["a"', loop, distance=2em, in=305, out=235, visible on=<2->] \arrow[rr, "b", visible on=<2->] &  & |[visible on=<2->]|2 \arrow["c"', loop, distance=2em, in=305, out=235, visible on=<2->]
\end{tikzcd}
\]
\uncover<3->{
\begin{align*}
\mathrm{End}(1) = \{a\} \\
\mathrm{End}(2) = \{c\} \\
\mathrm{Hom}(1,2) = \{b\}
\end{align*}
}
\end{frame}

\begin{frame}
\begin{centering}
You now know what a quiver is.
\end{centering}
\end{frame}
% you now know what a quiver is

\begin{frame}
A \ul{category} $\mathcal{C}$ is a quiver with two further maps:
\begin{description}
\item[($\mathbbm{1}$)] For every object $X \in \mathcal{C}_{0}$ there is the \ul{identity map} 
\begin{align*}
\mathbbm{1} : \mathcal{C}_{0} \longrightarrow \mathcal{C}_{1} \\
X \longmapsto \mathbbm{1}_{X} : X \longrightarrow X
\end{align*}
\item[($\mu$)] For two \ul{composable} morphisms $\varphi$ and $\psi \in \mathcal{C}_{1}$, i.e. with $t(\varphi) = s(\psi)$ there is
the \ul{composition map}
\begin{align*}
\mu &: \mathcal{C}_{1} \times \mathcal{C}_{1} \longrightarrow \mathcal{C}_{1} \\
\varphi &: A \longrightarrow B \\
\psi &: B \longrightarrow C \\
(\varphi, \psi) &\longmapsto \mu(\varphi, \psi) := \varphi\psi : A \longrightarrow C
\end{align*}
\end{description}
\end{frame}

\begin{frame}
The defining poperties for $\mathbbm{1}$ and $\mu$ are:
\begin{enumerate}
\item $s(\mathbbm{1}_{M}) = M = t(\mathbbm{1}_{M})$, i.e. $\mathbbm{1}_{M} \in \mathrm{End}(M)$.
\item $s(\varphi\psi) = s(\varphi)$ and $t(\varphi\psi) = t(\psi)$, i.e. for objects $M, L, N \in \mathcal{C}_{0}$ we have
\[
\mu : \mathrm{Hom}(M,L) \times \mathrm{Hom}(L,N) \longrightarrow \mathrm{Hom}(M,N)
\]
\item $(\varphi\psi)\rho = \varphi(\psi\rho)$, i.e. composition is \ul{associative}
\item $\mathbbm{1}_{s(\varphi)}\varphi = \varphi = \varphi\mathbbm{1}_{t(\varphi)}$, i.e. the identity is a left and right \ul{unit} of the composition.
\end{enumerate}
\uncover<5->{These properties make each endomorphism set $\mathrm{End}(M)$ for $M \in \mathcal{C}$ together with the composition into a monoid, called the \ul{endomorphism monoid} $(\mathrm{End}(M), \mu)$.}
\end{frame}

\begin{frame}
So when you define a category, you always answer the four questions
\begin{itemize}[<.->]
\item What are the objects?
\item What are the morphisms? Especially what are the identity morphisms?
\item How do you compose morphisms?
\item Why is the composition associative? Why is the identity a unit for the composition?
\end{itemize}
\end{frame}
\begin{frame}
\begin{centering}
You now know what a category is
\end{centering}
\end{frame}
% you now know what a category is
\begin{frame}
A small example for a category: The symmetric group on two objects $S_{2}$.
\[
\begin{tikzcd}
{\{1,2\}} \arrow["{\mathbbm{1}_{\{1,2\}}}"', loop, distance=2em, in=215, out=145] \arrow["{(1,2)}"', loop, distance=2em, in=35, out=325]
\end{tikzcd}
\]
The rule that the composition of $(1,2)$ with itself results in the identity $\mathbbm{1}_{\{1,2\}}$ makes sure there are only 2 morphisms in total.\\

\uncover<2->{
This is an example of a category which is a sub-category of the category \textsc{Sets} with sets as objects and functions between sets as morphisms.
Between those categories lies the category \textsc{FinSets} in which the objects are finite sets and morphisms are functions between finite sets.
}
\end{frame}

\begin{frame}[fragile]
\uncover<1->{
As another example of a finite subcategory of \textsc{FinSets}, take the category with two objects that are sets with $3$ elements.
}
\uncover<4->{
The endomorphisms at each objects are the cyclic permutations on the $3$ elements $\{1,2,3\}$
}
\uncover<6->{
and on the $3$ elements $\{4,5,6\}$.
}
\uncover<8->{
There is also a morphism from the first to the second object, mapping $1 \mapsto 4, 2 \mapsto 5$ and $3 \mapsto 6$.
}
\[
\begin{tikzcd}[ampersand replacement=\&]
|[visible on=<2->]|\{1,2,3\} \arrow["{(2,3,1)}"',pos=.52, loop, distance=2em, in=120, out=40, visible on=<5->]
\arrow["{(3,2,1)}"', loop, distance=2em, in=220, out=140, visible on=<5->]
\arrow["{\mathbbm{1}_{\{1,2,3\}}}"', pos=.48, loop, distance=2em, in=320, out=240, visible on=<5->]
\arrow[rr, "\begin{pmatrix} 1\mapsto 4 \\ 2\mapsto 5\\ 3\mapsto 6\end{pmatrix}", visible on=<9->] \&  \& |[visible on=<3->]|\{4,5,6\}
\arrow["{(5,6,4)}"', pos=.48, loop, distance=2em, in=140, out=60, visible on=<7->]
\arrow["{(6,5,4)}"', loop, distance=2em, in=40, out=320, visible on=<7->]
\arrow["{\mathbbm{1}_{\{4,5,6\}}}"', pos=.52, loop, distance=2em, in=300, out=220, visible on=<7->]
\end{tikzcd}
\]
\uncover<10->{
The endomorphism monoid on each object is the goup $C_{3}$, i.e. the cyclic group on three elements.
This also means that each endomorphism is invertible. We thus call this category $C_{3}C_{3}$.
}
\end{frame}

\begin{frame}
\uncover<1->{
You may have noticed that this picture is not complete when we look back to the axioms for a category: We have three endomorphisms at the first object and three endomorphisms at the second object. Since we also have a morphism from the first to the second object, we also need all the possible compositions of those morphisms.
}
\[
\begin{tikzcd}[ampersand replacement=\&]
|[visible on=<1->]|\{1,2,3\} \arrow["{(2,3,1)}"',pos=.52, loop, distance=2em, in=120, out=40, visible on=<1->]
\arrow["{(3,2,1)}"', loop, distance=2em, in=220, out=140, visible on=<1->]
\arrow["{\mathbbm{1}_{\{1,2,3\}}}"', pos=.48, loop, distance=2em, in=320, out=240, visible on=<1->]
\arrow[rr, "\begin{pmatrix} 1\mapsto 4 \\ 2\mapsto 5\\ 3\mapsto 6\end{pmatrix}", visible on=<1->] \&  \& |[visible on=<1->]|\{4,5,6\}
\arrow["{(5,6,4)}"', pos=.48, loop, distance=2em, in=140, out=60, visible on=<1->]
\arrow["{(6,5,4)}"', loop, distance=2em, in=40, out=320, visible on=<1->]
\arrow["{\mathbbm{1}_{\{4,5,6\}}}"', pos=.52, loop, distance=2em, in=300, out=220, visible on=<1->]
\end{tikzcd}
\]
\uncover<2->{
Of the missing two morphisms from $\{1,2,3\}$ to $\{4,5,6\}$, the first one is mapping $1 \mapsto 5, 2 \mapsto 6$ and $3 \mapsto 4$, where the last one is mapping $1 \mapsto 6, 2 \mapsto 4$ and $3 \mapsto 5$.
}
\uncover<3->{
So our category $C_{3}C_{3}$ has two objects $\{1,2,3\}, \{4,5,6\}$ and nine morphisms in total.
}
\end{frame}

\begin{frame}[fragile]
The categories $S_{2}$, $C_{3}C_{3}$, \textsc{Set}, \textsc{FinSets} can themselves all be considered objects in a greater category, \textsc{Cat}, i.e. \ul{the category of categories}.
\uncover<2->{
\[
\begin{tikzcd}[ampersand replacement=\&]
S_{2} \arrow[rr] \&                                        \& S_{3} \arrow[r] \& C_{3} \arrow[d] \arrow[r] \& C_{3}C_{3}  \arrow[ld]  \\
                 \& \textsc{Sets} \arrow[rr, shift left=2] \&                             \& \textsc{FinSets} \arrow[ll, shift left] \arrow[ll, shift left=4] \arrow[lu]
\end{tikzcd}
\]
}
\uncover<3->{
We know what the objects are in \textsc{Cat}. But what are the morphisms? What is meant with an arrow from \textsc{Sets} to \textsc{FinSets}?
}
\end{frame}
\begin{frame}
\begin{centering}
You now know the objects in the category \textsc{Cat} of all categories.
\end{centering}
\end{frame}
% you now know the objects in the category Cat of all categories

\begin{frame}
\noindent A \ul{functor} $F : \mathcal{C} \rightarrow \mathcal{D}$, between categories $\mathcal{C}$ and $\mathcal{D}$, consists of the
following data:

\begin{itemize}[<.->]
\item An object $Fc\in\mathcal{D}_{0}$, for each object $c \in \mathcal{C}_{0}$.
\item A function $Ff : Fc \rightarrow Fc' \in \mathcal{D}_{1}$, for each morphism $f : c \rightarrow c' \in \mathcal{C}_{1}$, so that the
source and target of $Ff$ are, respectively, equal to $F$ applied to the source or target of $f$, in other words,
$s(Ff) = Fs(f)$ and $t(Ff) = Ft(f)$.
\end{itemize}

\noindent The assignments are required to satisfy the following two \ul{functoriality axioms}:
\begin{itemize}[<.->]
\item For any composable pair $f : M \rightarrow N, g : N \rightarrow L \in \mathcal{C}_{1}, F\,f \cdot F\,g = F(f \cdot g)$.
\item For each object $c \in \mathcal{C}_{0}, F(1_{c}) = 1_{Fc}$.
\end{itemize}

\end{frame}
\begin{frame}
\noindent So with functors you always answer the four questions
\begin{itemize}[<.->]
\item How does it work on objects?
\item How does it work on morphisms?
\item Why does it respect composition?
\item Why does it respect identity morphisms?
\end{itemize}
\end{frame}
\begin{frame}
\begin{centering}
You now know what a functor is.
\end{centering}
\end{frame}
% you now know what a functor is
\begin{frame}
\begin{centering}
You now know the objects and morphisms in the category \textsc{Cat} of all categories.
\end{centering}
\end{frame}
% you now know the objects and morphisms in the category Cat of all categories
\begin{frame}
Let us now take a look at just two categories, $\mathcal{C}$ and $\mathcal{D}$ as objects in \textsc{Cat}.\\
The Hom-set $\mathrm{Hom}(\mathcal{C},\mathcal{D})$ of all functors $F : \mathcal{C} \longrightarrow \mathcal{D}$ is itself a category, called the \ul{functor category}.

This makes the functors $F, G, H : \mathcal{C} \longrightarrow \mathcal{D}$ objects in $\mathrm{Hom}(\mathcal{C},\mathcal{D})$ when before they were considered morphisms.

\[
\begin{tikzcd}[ampersand replacement=\&]
|[visible on=<2->]|F \arrow[r, "\alpha", Rightarrow, visible on=<4->] \& |[visible on=<2->]|G \arrow[r, "\beta", Rightarrow, visible on=<4->] \& |[visible on=<2->]|H
\end{tikzcd}
\]

\uncover<3->{
As you can imagine, we are again looking for the morphisms in this category, i.e. what are morphisms between functors?
}
\end{frame}
\begin{frame}
\begin{centering}
You now know the objects in the category $\mathrm{Hom}(\mathcal{C},\mathcal{D})$ of all functors between categories
$\mathcal{C}$ and $\mathcal{D}$.
\end{centering}
\end{frame}
% you now know the objects in the category Hom(C,D) of all functors between categories C and D
\begin{frame}[fragile]

\noindent Given categories $\mathcal{C}$ and $\mathcal{D}$ and functors $F : \mathcal{C} \rightarrow \mathcal{D}$ and
$G : \mathcal{C} \rightarrow \mathcal{D}$, a \ul{natural transformation} $\alpha : F \Rightarrow G$ consists of:
\begin{itemize}
\item a morphism $\alpha_{c} : Fc \rightarrow Gc \in \mathcal{D}_{1}$ for each object $c \in \mathcal{C}_{0}$, the collection of which
define the \ul{components} of the natural transformation, so that, for any morphism $f : c \rightarrow c' \in \mathcal{C}_{1}$, the following
square of morphisms in $\mathcal{D}$
\[
\begin{tikzcd}[ampersand replacement=\&]
Fc \arrow[r, "\alpha_{c}"] \arrow[d, "Ff"'] \& Gc \arrow[d, "Gf"] \\
Fc' \arrow[r, "\alpha_{c'}"]                \& Gc'                
\end{tikzcd}
\]

\ul{commutes}, i.e., has a common composite $Fc \rightarrow Gc' \in \mathcal{D}_{1}$. This means explicitly that
\begin{align*}
Ff \alpha_{c'} = \alpha_{c} Gf,\,\forall f : c \rightarrow c' \in \mathcal{C}_{1}. \label{eq:naturality_condition}
\end{align*}
\end{itemize}
\end{frame}

\begin{frame}
\begin{centering}
You now know what a natural transformation is.
\end{centering}
\end{frame}
% you now know what a natural transformation is

\begin{frame}
\begin{centering}
You now know the objects and morphisms in the category $\mathrm{Hom}(\mathcal{C},\mathcal{D})$ of all functors between categories $\mathcal{C}$ and $\mathcal{D}$.
\end{centering}
\end{frame}
% you now know the objects and morphisms in the category Hom(C,D) of all functors between categories C and D

%--------
% ^
%  |
% Grundbegriffe Kategorientheorie

\begin{frame}
Now that we know what a functor category is, we want to work towards a special kind of functor category:
Given a finite concrete category $\mathcal{C}$ whose endomorphism monoids are explicitly cyclic, we can
calculate its $\Bbbk$-linear closure, i.e. the $\Bbbk$-Algebroid $\mathcal{A}$.

Then we can calculate the category of $\Bbbk$-linear functors from the $\Bbbk$-Algebroid $\mathcal{A}$
into the matrix category $\Bbbk\text{-}\mathrm{Mat}$ over the same field $\Bbbk$.
\[
\mathrm{Hom}_{\Bbbk}(\mathcal{A},\Bbbk\text{-}\mathrm{Mat})
\]
\end{frame}

\tikzset{invisible/.style={opacity=0.0}}
\begin{frame}[fragile]
\[
\begin{tikzcd}
|[visible on=<2->]|\mathrm{FinSets} \arrow[d, visible on=<2->]                                                        &                               &                                              \\
|[visible on=<2->]|{\text{finite concrete category}\,\mathcal{C}} \arrow[rd, visible on=<5->] \arrow[rrd,  visible on=<5->] \arrow[dd,  visible on=<4->]  &                               &                                              \\
                                                                                  & |[ visible on=<5->]|{\text{quiver}\,q} \arrow[end anchor={[xshift=1.25em, yshift=.75em]},  visible on=<5->]{ld} & |[ visible on=<5->]|{\text{relations}\,\mathtt{rel}} \arrow[end anchor={[xshift=3.00em, yshift=.75em]},  visible on=<5->]{lld} \\
|[ visible on=<4->]|{\mathclap{\text{finitely presented category}\,\mathrm{fp}\mathcal{C}}} \arrow[d,  visible on=<4->] &                               & |[ visible on=<3->]|{\text{field}\,\Bbbk} \arrow[lld,  visible on=<3->] \arrow[d,  visible on=<6->]  \\
|[ visible on=<3->]|{\Bbbk\text{-}\mathrm{Algebroid}\,\mathcal{A}} \arrow[end anchor={[xshift=-1.25em, yshift=.75em]},  visible on=<3->]{rd}                                    &                               & |[ visible on=<6->]|\Bbbk\text{-}\mathrm{Mat} \arrow[end anchor={[xshift=1.25em, yshift=.75em]},  visible on=<6->]{ld} \\
                                                                                  & |[ visible on=<1->]|{\mathclap{\text{category of $\Bbbk$-linear functors}\,\mathrm{Hom}_{\Bbbk}(\mathcal{A},\Bbbk\text{-}\mathrm{Mat})}}
\end{tikzcd}
\]
\end{frame}

\begin{frame}

\end{frame}
% Abelsche Kategorie
% |
% |
% v

% Funktorkategorie nach Abelsche Kategorie ist Abelsche Kategorie
% |
% |
% v

%\begin{frame}
%Up to definition of Functor category $\HomAkmat$.
%$\mathrm{Hom}_{\Bbbk}(\mathcal{A},\Bbbk\textnormal{-}\mathrm{Mat})$.
%\end{frame}

%% juxtaposition kmat <-> HomAkmat
\begin{frame}[fragile]
\begin{tabular}{p{.28\textwidth}p{.33\textwidth}p{.33\textwidth}}
$\mathbf{Category}\,\mathcal{C}$ & $\HomAkmat$ & $\kmat$ \\
\hline \\
\textbf{Objects} $\mathcal{C}_{0}$ & \makecell[l]{$\Bbbk$-linear functors $F :$\\ $\mathcal{A} \rightarrow \kmat$ \\ $c \in \mathcal{A}_{0} \mapsto Fc \in \kmat_{0}$ \\ $\varphi \in \mathcal{A}_{1} \mapsto F\varphi \in \kmat_{1}$} 
& \makecell[r]{natural numbers $\mathbb{N}_{0}$} \\
\textbf{Morphisms} $\mathcal{C}_{1}$ & natural transformations, components are matrices & $m\times n$-matrices \\
\textbf{Composition}: & & \\
\makecell[l]{$\varphi : A \rightarrow B$, \\$\psi : B \rightarrow C$ \\ $\varphi\psi : A \rightarrow C$}
& \makecell[l]{$\eta : F \Rightarrow G$, $\eta_{c} : Fc \rightarrow Gc$ \\
  $\varepsilon : G \Rightarrow H$, $\varepsilon_{c} : Gc \rightarrow Hc$ \\
  $\eta\varepsilon : F \Rightarrow H$, $\eta_{c}\varepsilon_{c} : Fc \rightarrow Hc$}
& \\
& (component-wise) & matrix multiplication
\end{tabular}
\end{frame}

\begin{frame}[fragile]
\begin{tabular}{p{.28\textwidth}p{.33\textwidth}p{.33\textwidth}}
$\mathbf{Category}\,\mathcal{C}$ & $\HomAkmat$ & $\kmat$ \\
\hline \\
\textbf{Direct Sum}: & & \\
\makecell[l]{Let $I = \{1,\dots,N\}$ \\ be a finite set} & \makecell[l]{ $\{F_{i}\}_{i\in I}$ a family \\ of objects in\\ $\HomAkmat_{0}$} & \makecell[l]{$\{n_{i}\}_{i\in I}$ a family\\ of objects in $\kmat_{0}$}\\
& & \\
 & \makecell[l]{$F : c \mapsto \sum_{i=1}^{N} F_{i}(c)$}
 & $n = \sum_{i=1}^{N} n_{i}$ \\
& & \\
& \makecell[l]{at each object the \\sum of natural \\numbers at that object} & \makecell[r]{sum of natural\\ numbers}
\end{tabular}
\end{frame}

\begin{frame}[fragile]
\begin{tabular}{p{.28\textwidth}p{.66\textwidth}}
Projections: & $\pi_{i} : F \rightarrow F_{i}$ with components \\
& $(\pi_{i})_{c} := \begin{pmatrix}
0_{F_{<i}(c), F_{i}(c)} \\[2pt]
1_{F_{i}(c)} \\[2pt]
0_{F_{>i}(c), F_{i}(c)}
\end{pmatrix}$ \\
& \\
Coprojections: & $\iota_{i} : F_{i} \rightarrow F$ with components \\
& $(\iota_{i})_{c} :=$ \\
& $\begin{pmatrix}
0_{F_{i}(c),\,F_{<i}(c)} & 1_{F_{i}(c)} & 0_{F_{i}(c),\,F_{>i}(c)}
\end{pmatrix}$
\end{tabular}
\end{frame}

\begin{frame}[fragile]
For a morphism $a : c \rightarrow c' \in \mathcal{A}_{1}$ we have a family of morphisms 
$\{F_{i} a : F_{i} c \rightarrow F_{i} c'\}_{i \in I}$. Then the direct sum F is defined on morphisms as

\begin{align*}
F a := \sum_{i \in I} (\pi_{i})_{c} F_{i} a (\iota_{i})_{c'} : Fc \rightarrow Fc'
\end{align*}
which satisfies
\begin{align*}
(\iota_{i})_{c} Fa (\pi_{i})_{c'} &= (\iota_{i})_{c} \sum_{j \in I} (\pi_{j})_{c} F_{j} a (\iota_{j})_{c'} (\pi_{i})_{c'} \\
&= \sum_{j \in I} (\iota_{i})_{c} (\pi_{j})_{c} F_{j} a (\iota_{j})_{c'}(\pi_{i})_{c'} \\
&= \sum_{j \in I} (\delta_{i,j})_{c} F_{j} a (\delta_{j,i})_{c'} \\
&= 1_{F_{i} c} F_{i} a 1_{F_{i} c'} \\
&= F_{i} a
\end{align*}
\end{frame}

\begin{frame}[fragile]
One step in the decomposition algorithm looks like this:
\[
\begin{tikzcd}
                                   & F                 &                                      \\
F \arrow[ru, "\eta", bend left=49] &                   & F \arrow[lu, "\eta"', bend right=49] \\
I \arrow[u, "\iota"] \arrow[r, dash]     & \oplus \arrow[uu, "I \oplus K = F" description] & K \arrow[u, "\kappa"'] \arrow[l, dash]    
\end{tikzcd}
\]

Then we have two morphisms $\iota, \kappa$ with target $F$ and sources $I, K$ such that $I \oplus K = F$.
\end{frame}

\begin{frame}
\begin{algorithm}[H]
    \caption{\texttt{DecomposeOnceByRandomEndomorphism}}\label{algo:DecomposeOnceByRandomEndomorphism}
        \LinesNumbered
	\SetKwInput{Input}{Input~}
	\SetKwInput{Output}{Output~}
	\Input{~a functor $F$ in a functor category}
	\Output{~a pair $[\iota : I \rightarrow F, \kappa : K \rightarrow F]$ of morphisms such that $I \oplus K = F$ with $I \neq 0$ and $K \neq 0$ or
	$\mathtt{fail}$ if it was unable to further decompose $F$; }
	\BlankLine
	$d := \max \{ \mathrm{dim}_{\Bbbk}Fc \}_{c \in \mathcal{A}_{0}}$\;
	\If{$d = 0$}{
	    \Return $\mathtt{fail}$\tcp*{the zero representation is indecomposable}
	}
	$\mathcal{B} = [\beta_{1},\dots,\beta_{h}]$ is a $\Bbbk$-basis of $\mathrm{Hom}_{\HomAkmat}(F,F)$\;
	add $0_{F,F}$ to $\mathcal{B}$\;
\end{algorithm}
\end{frame}

\begin{frame}
\begin{algorithm}[H]
        \LinesNumbered
        \setcounter{AlgoLine}{6}
	$n := \lfloor\log_{2}(d)\rfloor + 1$\;
	\BlankLine
	\For{$b \in [h+1, h, \dots,2]$}{
	    $\alpha := \beta_{b} + \mathrm{random}(\Bbbk) \cdot \beta_{b-1}$\tcp*{a heuristic ansatz for a random endomorphism}
	    \For{$i \in [ 1, \dots, n ]$}{
	        $\alpha_{2} := \alpha^{2}$\;
	        \nl\tcc{We do not expect the exponentiation to produce an idempotent, still this is a very cheap test:}
	        \If{$\alpha = \alpha_{2}$}{
	            \Break\;
	        }
	        $\alpha := \alpha_{2}$\;
	    }
	    \BlankLine
	    
	    \If{$\alpha = 0$}{
	        \Continue\tcp*{try another endomorphism}
	    }
	}
\end{algorithm}
\end{frame}

\begin{frame}
\begin{algorithm}[H]
        \LinesNumbered
        \setcounter{AlgoLine}{7}
	\For{$b \in [h+1, h, \dots,2]$}{
	    \dots
	    \BlankLine
            \setcounter{AlgoLine}{21}
	    $\kappa := \mathrm{KernelEmbedding}(\alpha)$\;
	    
	    \If{$\kappa = 0$}{
	        \Continue\tcp*{try another endomorphism}
	    }
	    \BlankLine
	    $\iota := \mathrm{ImageEmbedding}(\alpha)$\;
	    \Return $[ \iota, \kappa ]$\;
	}
	\BlankLine
	\Return $\mathtt{fail}$\tcp*{The input functor $F$ is indecomposable with a high probability.}
\end{algorithm}
\end{frame}

\begin{frame}[fragile]
Direct sum decomposition
\[
\adjustbox{scale=0.7,center}{%
\begin{tikzcd}
                                                   &                                                           &                                                         &                                                             & F                                                          &                                                           &                                                          &                                                          \\
                                                   &                                                           & F \arrow[rru, "\eta_{1}", bend left]                    &                                                             &                                                            &                                                           & F \arrow[llu, "\eta_{1}"', bend right]                   &                                                          \\
                                                   &                                                           & I_{1} \arrow[u, "\iota_{1}"] \arrow[rr, dash]         &                                                             & \bigoplus \arrow[uu, "I_{1} \oplus K_{1}" description]     &                                                           & K_{1} \arrow[u, "\kappa_{1}"'] \arrow[ll, dash]        &                                                          \\
                                                   & I_{1} \arrow[ru, "\eta_{11}"', bend left=49, shift right] &                                                         &                                                             & I_{1} \arrow[llu, "\eta_{11}", pos=0.35, bend right]                 & K_{1} \arrow[ru, "\eta_{12}"', pos=0.40, bend left=49, shift left] &                                                          & K_{1} \arrow[lu, "\eta_{12}", bend right=49, shift left] \\
                                                   & I_{11} \arrow[u, "\iota_{11}"] \arrow[r, dash]          & \bigoplus \arrow[uu, "I_{11}\oplus K_{11}" description] &                                                             & K_{11} \arrow[ll, dash] \arrow[u, "\kappa_{11}"]         & I_{12} \arrow[u, "\iota_{12}"] \arrow[r, dash]          & \bigoplus \arrow[uu, "I_{12} \oplus K_{12}" description] & K_{12} \arrow[u, "\kappa_{12}"'] \arrow[l, dash]       \\
I_{11} \arrow[ru, "\eta_{111}"', bend left=49]     &                                                           & I_{11} \arrow[lu, "\eta_{111}", bend right=49]          & K_{11} \arrow[ru, "\eta_{112}"', bend left=49]              &                                                            & K_{11} \arrow[lu, "\eta_{112}", bend right=49]            &                                                          &                                                          \\
I_{111} \arrow[r, dash] \arrow[u, "\iota_{111}"] & \bigoplus \arrow[uu, "I_{111}\oplus K_{111}" description] & K_{111} \arrow[l, dash] \arrow[u, "\kappa_{111}"]     & I_{112} \arrow[u, "\iota_{112}"] \arrow[r, dash]          & \bigoplus \arrow[uu, "I_{112} \oplus K_{112}" description] & K_{112} \arrow[u, "\kappa_{112}"] \arrow[l, dash]       &                                                          &                                                          \\
                                                   &                                                           & I_{112} \arrow[ru, "\eta_{1121}"', bend left=49]        &                                                             & I_{112} \arrow[lu, "\eta_{1121}", bend right=49]           &                                                           &                                                          &                                                          \\
                                                   &                                                           & I_{1121} \arrow[u, "\iota_{1121}"] \arrow[r, dash]    & \bigoplus \arrow[uu, "I_{1121}\oplus K_{1121}" description] & K_{1121} \arrow[u, "\kappa_{1121}"] \arrow[l, dash]      &                                                           &                                                          &                                                         
\end{tikzcd}
}
\]
\end{frame}

\begin{frame}
\begin{algorithm}[H]
    \setcounter{AlgoLine}{0}
    \caption{\texttt{WeakDirectSumDecomposition}}\label{algo:WeakDirectSumDecomposition}
	\SetKwInput{Input}{Input~}
	\SetKwInput{Output}{Output~}
	\Input{~a functor $F$ in a functor category}
	\Output{~a list $[\eta_i : F_{i} \rightarrow F]$ of embeddings.}
	\BlankLine
	$\mathtt{queue} := [ 1_{F} ]$\;
	$\mathtt{summands} := \emptyset$\;
	
	\While{ $\mathtt{queue} \neq \emptyset$ }{
	    let $\eta = \mathtt{remove(queue)}$\;
	    $\mathtt{result} := \mathtt{DecomposeOnceByRandomEndomorphism}(s(\eta))$\;
	    \eIf(\tcp*[f]{$s(\eta)$ indecomposable}){$\mathtt{result} = \mathtt{fail}$}{
	        add $\eta$ to $\mathtt{summands}$\;
	    }{
	        $[\iota,\kappa] = \mathtt{result}$\;
	        append $[\iota\eta, \kappa\eta]$ to $\mathtt{queue}$\;
	    }
	}
	\BlankLine
	\Return $\mathtt{summands}$\;
\end{algorithm}
\end{frame}

\begin{frame}[fragile]
Hom-based invariants: \\
\texttt{EmbeddingOfSumOfImagesOfAllMorphisms}:
\[
\begin{tikzcd}
\displaystyle{\oplus_{i\in \mathtt{hom}} M} \arrow[rr, "u_{\text{out}}(\mathtt{hom})"] \arrow[rd, "\pi"', two heads] &                              & N \\
                                                                                                            & I \arrow[ru, "\iota"', hook] &  
\end{tikzcd}
\]
\end{frame}

\begin{frame}[fragile]
\[
\begin{tikzcd}
1 \arrow["a"', loop, distance=2em, in=305, out=235] \arrow[rr, "b"] &  & 2 \arrow["c"', loop, distance=2em, in=305, out=235]
\end{tikzcd}
\]
\end{frame}

\end{document}