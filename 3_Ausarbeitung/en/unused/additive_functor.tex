
\begin{lemma}[Additive functor]\endnote{(Ref. Lemma 2.10 in \cite{[Posur]})}
Any additive functor $F$ commutes with direct sums.
Let $( \mathcal{A},\oplus^{\mathcal{A}} )$ and $( \mathcal{B}, \oplus^{\mathcal{B}} )$ be additive categories whose projections and injections
we denote by $\pi_{i}, \iota_{i}$ and $p_{i}, q_{i}$, respectively. An Ab-functor $F : \mathcal{A} \rightarrow \mathcal{B}$ between their underlying
Ab-categories preserves the additive structure. More precisely, $F$ can be equipped with the following data in a unique way:
\begin{enumerate}
\item A dependent function mapping a finite set $I$ and a collection $(A_{i})_{i\in I}$ of objects in $\mathcal{A}$ to an isomorphism
\[
\sigma_{F}( (A_{i})_{i\in I}) : F( \oplus_{i\in I}^{\mathcal{A}} A_{i} ) \xrightarrow{\sim} \oplus_{i\in I}^{\mathcal{B}} F( A_{i} ),
\]
which we simply denote by $\sigma_{F}$ or $\sigma$ if no confusion may occur.
\item The equalities $\sigma p_{i} = F(\pi_{i})$ and $F(\iota_{i}) \sigma = q_{i}$ hold.
\end{enumerate}
In particular, we get an isomorphism $z : F( 0^{\mathcal{A}} ) \xrightarrow{\sim} 0^{\mathcal{B}}$.
\end{lemma}
\begin{proof}

\end{proof}
