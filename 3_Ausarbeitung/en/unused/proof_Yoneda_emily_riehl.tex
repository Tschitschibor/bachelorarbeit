Proof of Yoneda's lemma (from [context] by Emily Riehl, pages 57-59)

\begin{proof}[Proof of the bijection]\begin{subproof}[\nopunct]\phantom{}\\
There is clearly a function $\Phi : \mathrm{Hom}(\mathrm{Hom}_{\mathcal{C}}(c,-), F) \rightarrow Fc$ that maps
a natural transformation $\alpha : \mathrm{Hom}_{\mathcal{C}}(c,-) \Rightarrow F$ to the image of $1_{c}$ under the component
function $\alpha_{c} : \mathrm{Hom}_{\mathcal{C}}(c,c) \rightarrow Fc$, i.e.,
\begin{align}
&\Phi : \mathrm{Hom}(\mathrm{Hom}_{\mathcal{C}}(c,-), F) \rightarrow Fc; \\
&\Phi(\alpha) := \alpha_{c}(1_{c}).
\end{align}
Our first aim is to define an inverse function $\Psi : Fc \rightarrow \mathrm{Hom}(\mathrm{Hom}_{\mathcal{C}}(c,-), F)$ that constructs a
natural transformation $\Psi(x) : \mathrm{Hom}_{\mathcal{C}}(c,-) \Rightarrow F$ from any $x \in Fc$ (note that $Fc$ is a set). To this end,
we must define components $\Psi(x)_{d} : \mathrm{Hom}_{\mathcal{C}}(c,d) \rightarrow Fd$ so that naturality squares, such as displayed for
$f : c \rightarrow d$ in $\mathcal{C}$, commute:
\[
\begin{tikzcd}
{\mathrm{Hom}_{\mathcal{C}}(c,c)} \arrow[r, "\Psi(x)_{c}"] \arrow[d, "f_{*}"'] & Fc \arrow[d, "Ff"] \\
{\mathrm{Hom}_{\mathcal{C}}(c,d)} \arrow[r, "\Psi(x)_{d}"']                    & Fd                
\end{tikzcd}
\]
where $f_{*} := \mathrm{Hom}_{\mathcal{C}}(c,f) : \mathrm{Hom}_{\mathcal{C}}(c,c) \rightarrow \mathrm{Hom}_{\mathcal{C}}(c,d)$
is the image of $f$ under the partial Hom-functor $\mathrm{Hom}_{\mathcal{C}}(c,-)$ as defined in \ref{ex:hom_functor}.
The image of the identity element $1_{c} \in \mathrm{Hom}_{\mathcal{C}}(c,c)$ under the left-bottom composite is
$\Psi(x)_{d}(f) \in Fd$, the value of the component $\Psi(x)_{d}$ at the element $f \in \mathrm{Hom}_{\mathcal{C}}(c,d)$. The image under
the top-right composite is $Ff(\Psi(x)_{c}(1_{c}))$. For $\Psi$ to define an inverse for $\Phi$, we must define
$\Psi(x)_{c}(1_{c}) = x$. Now, naturality forces us to define
\begin{align}\label{eq:Yoneda-Inverse_Psi}
&\Psi : Fc \rightarrow \mathrm{Hom}(\mathrm{Hom}_{\mathcal{C}}(c,-),F); \\
&\Psi(x)_{d}(f) := F\,f(x).
\end{align}
This condition completely determines the components $\Psi(x)_{d}$ of $\Psi(x)$.

It remains to verify that $\Psi(x)$ is natural. To that end, we must show for a generic morphism $g : d \rightarrow e$ in $\mathcal{C}_{1}$, (one
whose domain is not necessarily the distinguished object c), that the square
\[
\begin{tikzcd}
{\mathrm{Hom}_{\mathcal{C}}(c,d)} \arrow[r, "\Psi(x)_{d}"] \arrow[d, "g_{*}"'] & Fd \arrow[d, "Fg"] \\
{\mathrm{Hom}_{\mathcal{C}}(c,e)} \arrow[r, "\Psi(x)_{e}"']                    & Fe                
\end{tikzcd}
\]
commutes. The image of $f \in \mathrm{Hom}_{\mathcal{C}}(c,d)$ along the left-bottom composite is
$\Psi(x)_{e}(gf) := F(gf)(x)$. The image along the top-right composite is $Fg(\Psi(x)_{d}(f)) := Fg(Ff(x))$. By functoriality of $F$,
$F(gf) = Fg \cdot Ff$, so these elements agree. Thus, the formula \eqref{eq:Yoneda-Inverse_Psi} defines the function
$\Psi : Fc \rightarrow \mathrm{Hom}(\mathrm{Hom}_{\mathcal{C}}(c,-),F)$.

By construction, $\Phi\Psi(x) = \Psi(x)_{c}(1_{c}) = x$, so $\Psi$ is a right inverse to $\Phi$. We wish to show that
$\Psi\Phi(\alpha) = \alpha$, i.e., that the natural transformation $\Psi(\alpha_{c}(1_{c}))$ is $\alpha$. It suffices to show that
these natural transformations have the same components. By definition,
\begin{align*}
\Psi(\alpha_{c}(1_{c}))_{d}(f) = F\,f(\alpha_{c}(1_{c})).
\end{align*}
By naturality of $\alpha$, the square
\begin{align}
\begin{tikzcd}[ampersand replacement = \&]
{\mathrm{Hom}_{\mathcal{C}}(c,c)} \arrow[r, "\alpha_{c}"] \arrow[d, "f_{*}"'] \& Fc \arrow[d, "Ff"] \\
{\mathrm{Hom}_{\mathcal{C}}(c,d)} \arrow[r, "\alpha_{d}"']                    \& Fd                
\end{tikzcd}
\end{align}
commutes, from which we see that $F\,f(\alpha_{c}(1_{c})) = \alpha_{d}(f)$. Thus, $\Psi(\alpha_{c}(1_{c}))_{d} = \alpha_{d}$,
proving that $\Psi$ is also a left inverse to $\Phi$. The explicit bijections $\Phi$ and $\Psi$ prove that evaluation of a natural
transformation at the identity of the representing object defines an isomorphism
\[
\mathrm{Hom}(\mathrm{Hom}_{\mathcal{C}}(c,-), F)  \xrightarrow{\cong} Fc,
\]
as claimed.
\end{subproof}
\begin{subproof}[Proof of the naturality]\phantom{}\\
The naturality in the statement of the Yoneda lemma amounts to the following pair of assertions. Naturality in the functor
asserts that, given a natural transformation $\beta : F \Rightarrow G$, the element $Gc$ representing the composite natural transformation
$\beta\alpha : \mathrm{Hom}_{\mathcal{C}}(c,-) \Rightarrow F \Rightarrow G$ is the image under $\beta_{c} : Fc \rightarrow Gc$ of the element
of $Fc$ representing $\alpha : \mathrm{Hom}_{\mathcal{C}}(c,-) \Rightarrow F$, i.e., the diagram
\[

\]
commutes in $\Set$. By definition, $\Phy_{G}(\beta\cdot \alpha) = (\beta \cdot \alpha)_{c}(1_{c})$, which is
$\beta_{c}(\alpha_{c}(1_{c}))$ by the definition of vertical composition of natural transformations given in \ref{la:vert_composition_nat},
and this is $\beta_{c}(\Phi_{F}(\alpha))$.
...
\end{subproof}
\end{proof}
