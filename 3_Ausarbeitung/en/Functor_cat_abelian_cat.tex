\label{sect:abelian_cat}
From now on instead of dealing with a finite concrete category $\mathcal{C}$ directly, we can instead take its finite-dimensional $\Bbbk$-linear
closure, i.e. the algebroid $\mathcal{A}$. This is the source of our functors in the functor category $\HomAkmat$. The target of
our functors is the abelian category $\kmat$.

In the following sections we prove that the functor category with values in $\kmat$ is an Abelian category with enough
projectives (constructively) and direct sum decomposition (constructively).

\subsection{$\HomAkmat$ is a $\Bbbk$-linear abelian category}

\begin{definition}[The functor category $\HomAkmat$]
Let $\Bbbk$ be a commutative ring, and let $\mathcal{A}$ be a finite-dimensional algebroid over $\Bbbk$.\\
The \ul{functor category} $\HomAkmat$ has
\begin{itemize}
\item as objects $\Bbbk$-linear functors $F : \mathcal{A} \rightarrow \kmat \in \HomAkmat_{0}$, where each
object $i \in \mathcal{A}$ gets mapped to a natural number $F(i) \in \kmat_{0} = \mathbb{N}_{0}$, and each
morphism $b : i \rightarrow j \in \mathcal{A}_{1}$ gets mapped to an $F(i) \times F(j)$ matrix
$F(b) : F(i) \rightarrow F(j) \in \kmat_{1}$,
\item as morphisms $\alpha : F \rightarrow G \in \HomAkmat_{1}$ natural transformations $\alpha : F \Rightarrow G$ with each component
$\alpha_{i} : F(i) \rightarrow G(i) \in \kmat_{1}$ for an object $i \in \mathcal{A}$ being an $F(i) \times G(i)$ matrix satisfying the
naturality conditions in \eqref{eq:naturality_condition}, i.e.
\[
\begin{tikzcd}
F(i) \arrow[rr, "\alpha_{i}"] \arrow[dd, "Fb"] &  & G(i) \arrow[dd, "Gb"] \\
                                             &  &                     \\
F(j) \arrow[rr, "\alpha_{j}"]                &  & G(j)                
\end{tikzcd}
\]
commutes for all $b : i \rightarrow j \in \mathcal{A}_{1}$, i.e.
\begin{align}
\alpha_{i}\,Gb &= Fb\,\alpha_{j}
\end{align}
\end{itemize}
\end{definition}

\begin{theorem}\label{thm:functor_category_abelian}
Let $\mathcal{A}$ be a finite-dimensional algebroid over some field $\Bbbk$. The functor category $\HomAkmat$
is a $\Bbbk$-linear abelian category.
\end{theorem}
\begin{proof}
We prove that $\HomAkmat$ is a $\Bbbk$-linear category, then that it is also a $\Bbbk$-linear additive category and a $\Bbbk$-linear pre-abelian
category and finally that it is a $\Bbbk$-linear abelian category.
\begin{enumerate}
\renewcommand{\labelenumi}{(\theenumi)}
\item In order to show that $\HomAkmat$ is a $\Bbbk$-linear category, we must show that 
\begin{enumerate}[align=left, leftmargin=0pt]
\renewcommand{\labelenumii}{(\roman{enumii})}
\item for any two objects $F,G \in \HomAkmat_{0}$ the hom-set
\[
\mathrm{Hom}_{\HomAkmat}(F,G)
\]
between them is a $\Bbbk$-module, and
\item that the composition of two morphisms
\[
\mu : \mathrm{Hom}_{\HomAkmat}(F,G) \times \mathrm{Hom}_{\HomAkmat}(G,H) \rightarrow \mathrm{Hom}_{\HomAkmat}(F,H)
\]
is a $\Bbbk$-bilinear map, i.e. for
\begin{align*}
\eta, \varepsilon &\in \mathrm{Hom}_{\HomAkmat}(F,G), \\
\varphi, \psi &\in \mathrm{Hom}_{\HomAkmat}(G,H)\,\text{and for} \\
x &\in \Bbbk
\end{align*}
\begin{align*}
(\eta + \varepsilon)\varphi &= \eta\varphi + \varepsilon\varphi \\
\eta(\varphi + \psi) &= \eta\varphi + \eta\psi \\
(x\eta)\varphi &= \eta(x\varphi) = x(\eta\varphi).
\end{align*}
\end{enumerate}

\begin{subproof}[Proof of (i)]
For any object $c \in \mathcal{A}$, the set of components $\mathrm{Hom}_{\kmat}(Fc,Gc)$ is a $\Bbbk$-module.

We define the addition and scalar multiplication
\begin{align*}
+ :&& \mathrm{Hom}_{\HomAkmat}(F,G) &\times \mathrm{Hom}_{\HomAkmat}(F,G) \rightarrow \mathrm{Hom}_{\HomAkmat}(F,G)\\
\cdot :&& \Bbbk &\times \mathrm{Hom}_{\HomAkmat}(F,G) \rightarrow \mathrm{Hom}_{\HomAkmat}(F,G)
\end{align*}
component-wise: $\forall \eta, \varepsilon \in \mathrm{Hom}_{\HomAkmat}(F,G), \forall x \in \Bbbk, \forall c \in \mathcal{A}$
\begin{align}
(\eta+\varepsilon)_{c} &:= \eta_{c} + \varepsilon_{c}\\
(x \eta)_{c} &:= x\,\eta_{c}
\end{align}
where the right-hand side is the usual addition and scalar multiplication of matrices. This includes the additive inverse $-\eta$ for
$x = -1$ and the zero morphism $0_{F,G}$ for $x = 0$:

We identify as the neutral element $0_{F,G} \in \mathrm{Hom}_{\HomAkmat}(F,G)$
the natural transformation $0_{F,G}$ with each component $(0_{F,G})_{c} = 0_{Fc,Gc}$ being the $Fc\times Gc$ zero matrix.
For each natural transformation $\eta$ the additive inverse $-\eta$ is defined component-wise as
\begin{align}
(-\eta)_{c} &:= -\eta_{c}.
\end{align}
We also confirm that the addition is commutative:
\begin{align}
(\eta+\varepsilon)_{c} &= \eta_{c} + \varepsilon_{c}\\
    &= \varepsilon_{c} + \eta_{c}\\
    &= (\varepsilon + \eta)_{c}
\end{align}
This concludes that for each $F, G \in \HomAkmat,\, \mathrm{Hom}_{\HomAkmat}(F,G)$ is a $\Bbbk$-module.
\end{subproof}
\begin{subproof}[Proof of (ii)]
Let $F, G, H \in \HomAkmat$ and let $\eta, \varepsilon \in \mathrm{Hom}_{\HomAkmat}(F,G)$,
$\varphi, \psi \in \mathrm{Hom}_{\HomAkmat}(G,H)$ and $x \in \Bbbk$.\\
The composition $\eta\varphi \in \mathrm{Hom}_{\HomAkmat}(F,H)$ is defined by component-wise matrix-multiplication,
i.e. $\forall c \in \mathcal{A}$
\begin{align*}
(\eta\varphi)_{c} := \eta_{c}\varphi_{c}
\end{align*}
and from this follows the $\Bbbk$-bilinearity of the composition, since the matrix multiplication is bilinear.\\
This concludes the first part of the proof, i.e. $\HomAkmat$ is a $\Bbbk$-linear category.
\end{subproof}

\item Next we show that $\HomAkmat$ is a $\Bbbk$-linear additive category, i.e. it is
\begin{enumerate}[align=left, leftmargin=0pt]
\renewcommand{\labelenumii}{(\roman{enumii})}
\item A $\Bbbk$-linear category with
\item A dependent function $\oplus$ mapping a finite set $I$ and a collection $(F_{i})_{i\in I}$ of objects in $\HomAkmat$
to a corresponding direct sum
\begin{align*}
( \oplus_{i\in I} F_{i}, (\pi_{i})_{i\in I}, (\iota_{i})_{i\in I}, u_{\mathrm{in}}, u_{\mathrm{out}} ).
\end{align*}
\end{enumerate}
\begin{subproof}[Proof of (ii)]
We will make extensive use of the direct sum in $\kmat$ as described in \ref{ex:kmat_additive} and \ref{ex:block_diagonal_matrix} to
define the direct sum in $\HomAkmat$.
Let $I = \{1,\dots,N\}$ be a finite set, and $\{F_{i} \}_{i \in I}$ a family of objects in $\HomAkmat_{0}$.
\begin{itemize}
\item The object $F := \oplus_{i \in I} F_{i}$ is a functor defined by its image on objects $c \in \mathcal{A}_{0}$ and its
image on morphisms $a : c \rightarrow c' \in \mathcal{A}_{1}$ in the following way:
\begin{itemize}
\item For an object $c \in \mathcal{A}_{0}$ we have a family $\{F_{i}\,c \}_{i \in I}$ of objects in $\kmat_{0}$ for which we can define
their direct sum
\begin{align}
F\,c := \left(\bigoplus_{i = 1}^{N} F_{i}\right)c := \bigoplus_{i = 1}^{N} (F_{i}\,c) := \sum_{i = 1}^{N} (F_{i}\,c).
\end{align}
\item The projections $\pi_{i} : F \rightarrow F_{i}$ and coprojections $\iota_{i} : F_{i} \rightarrow F$ are defined component-wise
exactly as in \eqref{eq:projection_direct_sum_matrix} and \eqref{eq:coprojection_direct_sum_matrix}, i.e.
\begin{align}
(\pi_{i})_{c} :=
\begin{pmatrix}
0_{F_{<i}(c), F_{i}(c)} \\[2pt]
1_{F_{i}(c)} \\[2pt]
0_{F_{>i}(c), F_{i}(c)}
\end{pmatrix} \\[1ex]
(\iota_{i})_{c} := 
\begin{pmatrix}
0_{F_{i}(c),\,F_{<i}(c)} & 1_{F_{i}(c)} & 0_{F_{i}(c),\,F_{>i}(c)}
\end{pmatrix}
\end{align}

We verify the property of a direct sum, that

\begin{align*}
\sum_{i \in I} (\pi_{i}) (\iota_{i}) &= 1_{F}\,\text{ and } \\
(\iota_{i}) (\pi_{j}) &= (\delta_{i,j}) = \begin{cases}
1_{F_{i}} & \text{ if } i = j \\
0_{F_{i}, F_{j}} & \text{ if } i \neq j
\end{cases}
\end{align*}
which again can be checked component-wise
\begin{align*}
\sum_{i \in I} (\pi_{i})_{c} (\iota_{i})_{c} &= 1_{Fc}\,\text{ and } \\
(\iota_{i})_{c}(\pi_{j})_{c} &= (\delta_{i,j})_{c} = \begin{cases}
1_{F_{i}c} & \text{ if } i = j \\
0_{F_{i}c, F_{j}c} & \text{ if } i \neq j
\end{cases}
\end{align*}

\item For a morphism $a : c \rightarrow c' \in \mathcal{A}_{1}$ we have a family of morphisms 
$\{F_{i} a : F_{i} c \rightarrow F_{i} c'\}_{i \in I}$. Analogous to \ref{ex:block_diagonal_matrix} we define

\begin{align*}
F a := \sum_{i \in I} (\pi_{i})_{c} F_{i} a (\iota_{i})_{c'} : Fc \rightarrow Fc'
\end{align*}
which satisfies
\begin{align*}
(\iota_{i})_{c} Fa (\pi_{i})_{c'} &= (\iota_{i})_{c} \sum_{j \in I} (\pi_{j})_{c} F_{j} a (\iota_{j})_{c'} (\pi_{i})_{c'} \\
&= \sum_{j \in I} (\iota_{i})_{c} (\pi_{j})_{c} F_{j} a (\iota_{j})_{c'}(\pi_{i})_{c'} \\
&= \sum_{j \in I} (\delta_{i,j})_{c} F_{j} a (\delta_{j,i})_{c'} \\
&= 1_{F_{i} c} F_{i} a 1_{F_{i} c'} \\
&= F_{i} a
\end{align*}
\end{itemize}
This defines how $F$ works on objects and on morphisms.

\item For a family $\tau = \{ \tau_{i} : G \rightarrow F_{i} \}_{i \in I}$ of natural transformations with the same source
$G \in \HomAkmat_{0}$ we have for each object $c \in \mathcal{A}_{0}$ a family
$\tau_{c} = \{ (\tau_{i})_{c} : Gc \rightarrow F_{i}c \}_{i \in I}$ of morphisms in $\kmat_{1}$ with same source $Gc \in \kmat_{0}$.
For these we have by \eqref{eq:u_in_direct_sum_matrix} the morphism $u_{\text{in}}(\tau_{c}) : Gc \rightarrow Fc$ such that
\begin{align*}
u_{\text{in}}(\tau_{c}) (\pi_{i})_{c} =
(\tau_{i})_{c}
\end{align*}
We can thus define the natural transformation
\begin{align*}
u_{\text{in}}(\tau) : G \rightarrow F \in \HomAkmat_{1}
\end{align*}
by the components
\begin{align*}
(u_{\text{in}}(\tau))_{c} := u_{\text{in}}(\tau_{c})
\end{align*}
and we can verify that $u_{\text{in}}(\tau) (\pi_{i})$ is the natural transformation with components
\begin{align*}
(u_{\text{in}}(\tau) (\pi_{i}))_{c} &= (u_{\text{in}}(\tau))_{c} (\pi_{i})_{c} \\
&= u_{\text{in}}(\tau_{c}) (\pi_{i})_{c} \\
&= (\tau_{i})_{c}
\end{align*}
per definition, i.e. $u_{\text{in}}(\tau)$ is the natural transformation fulfilling $u_{\text{in}}(\tau) (\pi_{i}) = \tau_{i}$.
\item Analogous for a family $\rho = \{ \rho_{i} : F_{i} \rightarrow H \}$ of natural transformations with the same target
$H \in \HomAkmat_{0}$ we have for each object $c \in \mathcal{A}_{0}$ a family
$\rho_{c} = \{ (\rho_{i})_{c} : F_{i} c \rightarrow Hc \}_{i \in I}$ and get the
natural transformation
\begin{align*}
u_{\text{out}}(\rho) : F \rightarrow H \in \HomAkmat_{1}
\end{align*}
with components
\begin{align*}
(u_{\text{out}}(\rho))_{c} := u_{\text{out}}(\rho_{c})
\end{align*}
fulfilling $\iota_{i} u_{\text{out}}(\rho) = \rho_{i}$.
\end{itemize}
All in all we have 
\begin{align*}
\forall c \in \mathcal{A}_{0},&& &&  (\oplus_{i \in I} F_{i}) c &= \oplus_{i \in I} (F_{i} c) \\
\forall c \in \mathcal{A}_{0},&& \forall i \in I,&& (\pi_{i} : F \rightarrow F_{i})_{c} &= (\pi_{i})_{c} : Fc \rightarrow F_{i} c \\
\forall c \in \mathcal{A}_{0},&& \forall i \in I,&& (\iota_{i} : F_{i} \rightarrow F)_{c} &= (\iota_{i})_{c} : F_{i} c \rightarrow Fc \\
\forall c \in \mathcal{A}_{0},&& \forall \tau = \{ \tau_{i} : G \rightarrow F_{i} \}_{i = 1,\dots,n},&&
(u_{\mathrm{in}}(\tau))_{c} &= u_{\mathrm{in}}(\tau_{c}) \\
\forall c \in \mathcal{A}_{0},&& \forall \rho = \{ \rho_{i} : F_{i} \rightarrow H \}_{i = 1,\dots,n},&&
(u_{\mathrm{out}}(\rho))_{c} &= u_{\mathrm{out}}(\rho_{c})
\end{align*}

where $\tau_{c} = \{ (\tau_{i})_{c} : Gc \rightarrow F_{i}c \}_{i \in I}$ and $\rho_{c} = \{ (\rho_{i})_{c} : F_{i} c \rightarrow Hc \}_{i \in I}$.

And thus we proved that $\HomAkmat$ is a $\Bbbk$-linear additive category.
\end{subproof}

\item Next we show that $\HomAkmat$ is a $\Bbbk$-linear pre-abelian category, i.e.
\begin{enumerate}[align=left, leftmargin=0pt]
\renewcommand{\labelenumii}{(\roman{enumii})}
\item a $\Bbbk$-linear additive category with
\item a dependent function mapping a morphism $\eta : F \rightarrow G \in \HomAkmat_{1}$ to a kernel of $\eta$ and
\item a dependent function mapping a morphism $\eta : F \rightarrow G \in \HomAkmat_{1}$ to a cokernel of $\eta$.
\end{enumerate}
\begin{subproof}[Proof of (ii)]
For a morphism $\eta : F \rightarrow G \in \HomAkmat_{1}$ we want to define its kernel, i.e. the triple
\begin{itemize}
\item $K := \mathrm{KernelObject}(\eta)$
\item $\kappa := \mathrm{KernelEmbedding}(\eta) : K \rightarrow F$ such that $\kappa \eta = 0_{K,G}$.
\item For any other $\tau : T \rightarrow F$ such that $\tau\eta = 0_{T,G}$ we have a unique morphism
$(\tau / \eta) := \mathrm{KernelLift}(\eta,\tau)$ such that $\tau = (\tau / \eta) \kappa$
\end{itemize}

For the components $\eta_{c} : Fc \rightarrow Gc \in \kmat_{1}$ we have in $\kmat$ the kernel object
\[
Kc := \mathrm{KernelObject}(\eta_{c})
\]
and the kernel embedding
\[
\kappa_{c} := \mathrm{KernelEmbedding}(\eta_{c}) : Kc \rightarrow Fc
\]
such that
$\kappa_{c} \eta_{c} = 0_{Kc,Fc}$.
We define the kernel embedding of $\eta$ as
\[
\kappa : K \hookrightarrow F
\]
by its components
\[
\kappa_{c} = \mathrm{KernelEmbedding}(\eta)_{c} := \mathrm{KernelEmbedding}(\eta_{c}) : Kc \rightarrow Fc
\]
with source
$K = \mathrm{KernelObject}(\eta)$ defined on objects as $Kc := \mathrm{KernelObject}(\eta_{c})$ which is the source of the
kernel embedding of $\eta_{c}$.\\
Rather than proving that the so defined $\kappa : K \rightarrow F$ is a natural transformation, we assume it is and fill in the blanks, i.e. 
how $K$ acts on morphisms.

We have for each morphism $a : c \rightarrow c' \in \mathcal{A}_{1}$ the morphisms $Fa : Fc \rightarrow Fc'$ and
$Ga : Gc \rightarrow Gc'$. Then the kernel embedding $\kappa$ of a natural transformation $\eta : F \rightarrow G$ has components
$\kappa_{c} : Kc \rightarrow Fc$ with $\kappa_{c} = \mathrm{KernelEmbedding}(\eta_{c})$ and the kernel object $K$ acts on objects
as $\mathrm{KernelObject}(\eta)c = \mathrm{KernelObject}(\eta_{c}) = Kc$.
In the following we show that K acts on morphisms via the kernel lift
$Ka := (\kappa_{c}Fa/\kappa_{c'}) := \mathrm{KernelLift}(\eta_{c},\kappa_{c}Fa)$:

\[
\begin{tikzcd}
Kc \arrow[dd, "Ka"] \arrow[rr, "\kappa_{c}", hook] \arrow[rrdd, "\kappa_{c}Fa"] \arrow[dd, "(\kappa_{c}Fa/\kappa_{c'})"', dashed, shift right=2] &  & Fc \arrow[rr, "\eta_{c}"] \arrow[dd, "Fa"] &  & Gc \arrow[dd, "Ga"] \arrow[rr, "\varepsilon_{c}", two heads] \arrow[rrdd, "Ga\varepsilon_{c'}"] &  & Cc \arrow[dd, "Ca"'] \arrow[dd, "(\varepsilon_{c}\backslash Ga\varepsilon_{c'})", dashed, shift left=2] \\
                                                                                                                                                                                                       &  &                                            &  &                                                                                                                                                             &  &                                                                                                  \\
Kc' \arrow[rr, "\kappa_{c'}", hook]                                                                                                                                                                    &  & Fc' \arrow[rr, "\eta_{c'}"]                &  & Gc' \arrow[rr, "\varepsilon_{c'}", two heads]                                                                                                               &  & Cc'                                                                                             
\end{tikzcd}
\]

The composition morphism $\kappa_{c}\,Fa : Kc \rightarrow Fc'$ is a competing morphism to $\kappa_{c'} : Kc' \hookrightarrow Fc'$ in that
for both morphisms we have $\kappa_{c}\,Fa\,\eta_{c'} = 0_{Kc,Gc'}$ and $\kappa_{c'}\,\eta_{c'} = 0_{Kc',Gc'}$. The second equation
is just the definition of kernel embedding $\kappa_{c'}$, while the first equation comes from the commutative center
square $Fa\,\eta_{c'} = \eta_{c}\,Ga$ and thus $\kappa_{c}\,Fa\,\eta_{c'} = \kappa_{c}\,\eta_{c}\,Ga$ which again with the definition
of the kernel embedding $\kappa_{c}$ gives $0_{Kc,Gc}\,Ga = 0_{Kc,Gc'}$.

As we have seen in \ref{def:kernel}, the kernel embedding $\kappa_{c'}$ dominates $\kappa_{c}\,Fa$, i.e. there exists a
unique lift $(\kappa_{c}Fa/\kappa_{c'}) : Kc \rightarrow Kc'$ with $(\kappa_{c}Fa/\kappa_{c'})\,\kappa_{c'} = \kappa_{c}\,Fa$.
Naturality of the left square, i.e. $\kappa_{c}\,Fa = Ka\,\kappa_{c'}$ leads to
$Ka\,\kappa_{c'}\,\eta_{c'} = \kappa_{c}\,Fa\,\eta_{c'} = 0_{Kc,Gc'}$ and thus by the above argument, $Ka = (\kappa_{c}Fa/\kappa_{c'})$
what we wanted to show.

Now for the kernel lift $(\tau / \kappa) := \mathrm{KernelLift}(\eta, \tau)$:

\[
\begin{tikzcd}
K \arrow[r, "\kappa", hook]                                   & F \arrow[r, "\eta", shift left] \arrow[r, "{0_{F,G}}"', shift right] & G \\
T \arrow[ru, "\tau"] \arrow[u, "(\tau / \kappa)", dashed] &                     &  
\end{tikzcd}
\]

Let $\kappa : K \hookrightarrow F$ be the kernel embedding of the morphism $\eta : F \rightarrow G$, and let $\tau : T \rightarrow F$
with $\tau \eta = 0_{T,G}$. For all $c \in \mathcal{A}_{0}$ we have
$\tau_{c} : Tc \rightarrow Fc$ with $\tau_{c} \eta_{c} = 0_{Tc,Gc}$. Therefore in $\kmat$ we have
$(\tau_{c} / \kappa_{c} ) := \mathrm{KernelLift}( \eta_{c}, \tau_{c} )$ such that $\tau_{c} = (\tau_{c} / \kappa_{c} ) \kappa_{c}$.

The natural transformation $\lambda : T \rightarrow K$ defined by its components\\
$\lambda_{c} := (\tau_{c} / \kappa_{c}) := \mathrm{KernelLift}( \eta_{c}, \tau_{c} )$ satisfies
$\tau_{c} = \lambda_{c} \kappa_{c} = (\tau_{c} / \kappa_{c} ) \kappa_{c}$ and therefore
$\tau = \lambda \kappa$, i.e. $\lambda = (\tau / \kappa)$ is the kernel lift $\mathrm{KernelLift}(\eta, \tau)$.

This fully describes the kernel of $\eta$.
\end{subproof}
\begin{subproof}[Proof of (iii)]
For the same morphism $\eta : F \rightarrow G \in \HomAkmat_{1}$  we want to define its cokernel, i.e. the triple
\begin{itemize}
\item $C := \mathrm{CokernelObject}(\eta)$
\item $\varepsilon := \mathrm{CokernelProjection}(\eta) : G \twoheadrightarrow C$ such that $\eta\,\varepsilon = 0_{F,C}$.
\item For any other $\tau : G \rightarrow T$ such that $\eta\,\tau = 0_{G,T}$ we have a unique morphism
$(\varepsilon \backslash \tau) := \mathrm{CokernelColift}(\eta,\tau)$ such that $\tau = \varepsilon (\varepsilon \backslash \tau)$
\end{itemize}
We have for each morphism $a : c \rightarrow c' \in \mathcal{A}_{1}$ the morphisms $Fa : Fc \rightarrow Fc'$ and
$Ga : Gc \rightarrow Gc'$. Then the cokernel projection $\varepsilon : G \twoheadrightarrow C$ of a natural transformation
$\eta : F \rightarrow G$ has components
$\varepsilon_{c} : Gc \rightarrow Cc$ with $\varepsilon_{c} = \mathrm{CokernelProjection}(\eta_{c})$ and the Cokernel object $C$ acts on
objects as $\mathrm{CokernelObject}(\eta)c = \mathrm{CokernelObject}(\eta_{c}) = Cc$.
And on morphisms $Ca := \mathrm{CokernelColift}(\eta_{c},Ga\,\varepsilon_{c'})$.

\[
\begin{tikzcd}
Kc \arrow[dd, "Ka"] \arrow[rr, "\kappa_{c}", hook] \arrow[rrdd, "\kappa_{c}Fa"] \arrow[dd, "(\kappa_{c}Fa/\kappa_{c'})"', dashed, shift right=2] &  & Fc \arrow[rr, "\eta_{c}"] \arrow[dd, "Fa"] &  & Gc \arrow[dd, "Ga"] \arrow[rr, "\varepsilon_{c}", two heads] \arrow[rrdd, "Ga\varepsilon_{c'}"] &  & Cc \arrow[dd, "Ca"'] \arrow[dd, "(\varepsilon_{c}\backslash Ga\varepsilon_{c'})", dashed, shift left=2] \\
                                                                                                                                                                                                       &  &                                            &  &                                                                                                                                                             &  &                                                                                                  \\
Kc' \arrow[rr, "\kappa_{c'}", hook]                                                                                                                                                                    &  & Fc' \arrow[rr, "\eta_{c'}"]                &  & Gc' \arrow[rr, "\varepsilon_{c'}", two heads]                                                                                                               &  & Cc'                                                                                             
\end{tikzcd}
\]

Dually to the above, and by focusing on the naturality square on the right, we have the cokernel object $C$ with
$Cc = \mathrm{CokernelObject}(\eta_{c})$ and $Ca = \mathrm{CokernelColift}(\eta_{c}, Ga\varepsilon_{c'})$.

Again we get the cokernel colift $(\varepsilon \backslash \tau) = \mathrm{CokernelColift}(\eta, \tau)$
\[
\begin{tikzcd}
F \arrow[r, "\eta"] & G \arrow[r, "\varepsilon", two heads] \arrow[rd, "\tau"'] & C \arrow[d, "(\tau\backslash\eta)", dashed]                                           \\
                    &                                                & T 
\end{tikzcd}
\]
component-wise, i.e. for each $c \in \mathcal{A}_{0}$ we get
\[
(\varepsilon \backslash \tau)_{c} := (\varepsilon_{c} \backslash \tau_{c}).
\]

This fully describes the cokernel of $\eta$.
\end{subproof}

\item Finally to show that $\HomAkmat$ is an abelian category, we need to show that it is
\begin{enumerate}
\renewcommand{\labelenumii}{(\roman{enumii})}
\item a pre-abelian category with
\item a dependent function $(-/-)$ mapping a pair $\iota : K \rightarrow A, \tau : T \rightarrow A$ to a lift $\tau / \iota$ of
$\tau$ along $\iota$, where $A, K, T \in \HomAkmat_{0}$, $\iota$ is a monomorphism and $\tau \mathrm{CokernelProjection}(\iota) = 0$.
\item a dependent function $(-\backslash -)$ mapping a pair $\varepsilon : A \rightarrow C, \tau : A \rightarrow T$ to a colift
$\epsilon \backslash \tau$ of $\tau$ along $\varepsilon$, where $A, K, T \in \mathcal{A}_{0}$, $\varepsilon$ is an epimorphism and
$\mathrm{KernelEmbedding}(\varepsilon)\,\tau = 0$.
\end{enumerate}
\begin{subproof}
Let $A, K, T \in \HomAkmat_{0}$ and $\iota, \tau \in \HomAkmat_{1}$ such that 
\begin{itemize}
\item $\iota : K \rightarrow A$ is a monomorphism,
\item $\tau : T \rightarrow A$ is a morphism with $\tau \, \mathrm{CokernelProjection}(\iota) = 0$.
\end{itemize}
Then for each $c \in \mathcal{A}_{0}$ we have $\iota_{c}, \tau_{c} \in \kmat_{1}$ such that
\begin{itemize}
\item $\iota_{c} : Kc \rightarrow Ac$ is a monomorphism, and
\item $\tau_{c} \, \mathrm{CokernelProjection}(\iota_{c}) = 0$.
\end{itemize}
Since $\kmat$ is an abelian category, we have a lift $\tau_{c} / \iota_{c}$ of $\tau_{c}$ along $\iota_{c}$, i.e.
$(\tau_{c} / \iota_{c}) \iota_{c} = \tau_{c}$.
This lift $(\tau_{c} / \iota_{c}) \in \kmat_{1}$ defines the components for $(\tau / \iota) \in \HomAkmat_{1}$. And since
$\iota$ is a monomorphism, the lift along $\iota$ is necessarily unique.\\
Thus we have a dependent function $( - / - )$ mapping a
a pair $\iota : K \rightarrow A, \tau : T \rightarrow A$ to a lift $(\tau / \iota)$ of
$\tau$ along $\iota$, where $A, K, T \in \HomAkmat_{0}$, $\iota$ is a monomorphism and $\tau \, \mathrm{CokernelProjection}(\iota) = 0$
with $(\tau / \iota)$ defined as
\begin{align}
(\tau / \iota)_{c} := (\tau_{c} / \iota_{c}).
\end{align}
Analogous for (iii) we get the colift along epimorphisms $(\varepsilon \backslash \tau)$ defined by its components
\begin{align}
(\varepsilon \backslash \tau)_{c} := (\varepsilon_{c} \backslash \tau_{c})
\end{align}


\end{subproof}
\end{enumerate}
\end{proof}


