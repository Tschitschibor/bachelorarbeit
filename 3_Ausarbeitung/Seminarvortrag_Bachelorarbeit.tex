\documentclass[12pt,compress]{beamer}

\setbeamertemplate{navigation symbols}{}

\usepackage[utf8]{inputenc}
\usepackage{tikz}
\usetikzlibrary{mindmap,trees,shadows}
\usepackage{tikz-cd}
\usepackage{mathtools}
\usepackage{bbm}
\usepackage{makecell}
\usepackage{ifthen}
\usetikzlibrary{calc}

\usepackage{tabularx,booktabs}
\usepackage{xcolor,colortbl,etoolbox}
 
\makeatletter
\long\def\long@firstofone#1{#1}
\long\def\@getnextbraced#1#2#3{#2\long@firstofone{#3{#1}}}
\long\def\@braced@unexpanded#1\long@firstofone#2#3{%
   #1\long@firstofone{#2{#3}}}
\protected\long\def\@braced@expandedfully#1\long@firstofone#2#3{%
   \edef\@expandargs@internal{{#3}}%
   \expandafter\@getnextbraced\@expandargs@internal{#1}{#2}%
}
\protected\long\def\expandtabularargs{%
   \@braced@unexpanded
   \@braced@expandedfully
   \long@firstofone
}
\makeatother
 


%%% For proper underline
\usepackage{soul}
%\setuldepth{gjpqy}
%\setuldepth\strut
\setuldepth{-1}

%%% Macros for our recurring categories
\newcommand{\kmat}{\Bbbk\textnormal{-}\mathbf{mat}}
\newcommand{\kAlgebroid}{\Bbbk\textnormal{-}\mathrm{algebroid}}
\newcommand{\Rmat}{R\textnormal{-}\mathbf{mat}}
\newcommand{\HomAkmat}{\mathrm{Hom}_{\Bbbk}(\mathcal{A},\Bbbk\textnormal{-}\mathrm{Mat})}
\newcommand{\HomARmat}{\mathrm{Hom_{R}}(\mathcal{A},\Rmat)}
\newcommand{\HomA}{\mathrm{Hom}_{\mathcal{A}}}
\newcommand{\FinSets}{\mathrm{FinSets}}
\newcommand{\Cat}{\mathrm{\textbf{Cat}}}
\newcommand{\Set}{\mathrm{\textbf{Set}}}
\newcommand{\Quiv}{\mathrm{\textbf{Quiv}}}
\newcommand{\kChat}{\widehat{\Bbbk\mathcal{C}}}
\newcommand{\fpC}{\mathrm{fp}\mathcal{C}}

\newcommand\overviewslide{1}

\newcounter{saveenumi}
\newcommand{\seti}{\setcounter{saveenumi}{\value{enumi}}}
\newcommand{\conti}{\setcounter{enumi}{\value{saveenumi}}}

\resetcounteronoverlays{saveenumi}


\newcommand{\visiblesub}[2]{
  \ifthenelse{\ifnum#1>#2}
  { 0; }
  { 1; }
}





% If you wish to uncover everything in a step-wise fashion, uncomment
% the following command: 

\beamerdefaultoverlayspecification{<+->}


\title[The category of representations]
{The category of representations of a concrete category as a functor category}

\author{Tibor Gr{\"u}n}

%\date{October 30, 2020} % date of printed version
%\date{November 3, 2020} % date of submission
%\date{November 16, 2020} % date of seminar presentation
\date{March 22, 2021} % date of seminar presentation
\begin{document}

  %% to uncover arrows and nodes one after another
  %% use: 
  %% |[visible on=<1->]|\mathrm{FinSets}
  % in front of node
  %% \arrow[rd, visible on=<0>]
  % inside arrow
  % 1- means from first slide on
  % 2 only on second slide
  % -3 on slides 1,2,3
  % 0 always invisible
    \tikzset{
      invisible/.style={opacity=0.3},
      visible on/.style={alt={#1{}{invisible}}},
      alt/.code args={<#1>#2#3}{%
        \alt<#1>{ \pgfkeysalso{#2} }{ \pgfkeysalso{#3} }%
      }
  }

  % Keys to support piece-wise uncovering of elements in TikZ pictures:
  % \node[visible on=<2->](foo){Foo}
  % \node[visible on=<{2,4}>](bar){Bar}   % put braces around comma expressions
  %
  % Internally works by setting opacity=0 when invisible, which has the 
  % adavantage (compared to \node<2->(foo){Foo} that the node is always there, hence
  % always consumes space plus that coordinate (foo) is always available.
  %
  % The actual command that implements the invisibility can be overriden
  % by altering the style invisible. For instance \tikzset{invisible/.style={opacity=0.2}}
  % would dim the "invisible" parts. Alternatively, the color might be set to white, if the
  % output driver does not support transparencies (e.g., PS) 
  %

\begin{frame}
  \titlepage
\end{frame}

\section{Short introduction to Category theory}
\tikzset{invisible/.style={opacity=0.0}}
\begin{frame}[fragile]
\uncover<1->{
A \ul{quiver} (also called \ul{directed graph}) $q$ consists of a class of \ul{objects} (or \ul{vertices}) $q_{0} = \mathrm{Obj}\,q$
}
\uncover<3->{
and a class of
\ul{morphisms} (or \ul{arrows}) $q_{1} = \mathrm{Mor}\,q$
}
\[
\begin{tikzcd}[ampersand replacement=\&]
|[visible on=<2->]|1 \arrow[rddd, "b", visible on=<4->] \arrow[rrd, "a", visible on=<4->] \&         \&                    \&         \&                    \\
                                 \&         \& |[visible on=<2->]|3 \arrow[rd, "e", shift left, visible on=<4->] \&         \& |[visible on=<2->]|4 \arrow[ll, "d", visible on=<4->] \arrow[ll, "c"', bend right, shift right, visible on=<4->]  \\
                                 \&         \&                    \& |[visible on=<2->]|5 \arrow[lu, "f", shift left, visible on=<4->] \&                    \\
                                 \& |[visible on=<2->]|2 \&                    \&         \&                   
\end{tikzcd}
\]
\end{frame}

\begin{frame}
\uncover<1->{
together with two defining maps
\[
s : q_{1} \longrightarrow q_{0}
\]
called \ul{source}, and
\[
t : q_{1} \longrightarrow q_{0}
\]
called \ul{target}.\\
}
\uncover<2->{
In the finite case - which we will be dealing with most of the time here - we call $q_{0}$ \ul{set} of objects and and $q_{1}$ \ul{set} of morphisms.
So in the above quiver $q$ we have
}
\uncover<3->{
\[
q_{0} = \{1,2,3,4,5\}
\]
and
\[
q_{1} = \{a,b,c,d,e,f\}
\]
}
\end{frame}

\begin{frame}
\uncover<1->{
\begin{align*}
s(a) = s(b) = 1 \\
t(b) = 2 \\
t(a) = t(c) = t(d) = t(f) = 3
\end{align*}
etc. \\
}
\uncover<2->
{
Another map relates the arrows with the objects. That is the Hom-set, i.e. set of morphisms, between two objects (order matters):
\[
\mathrm{Hom} : q_{0} \times q_{0} \longrightarrow \mathcal{P}(q_{1})
\]
}
\uncover<3->
{
For example
\begin{align*}
\mathrm{Hom}(1,3) &= \{a\} \\
\mathrm{Hom}(4,3) &= \{c,d\} \\
\mathrm{Hom}(3,4) &= \{\}
\end{align*}
}
\uncover<4->{
For the same object $X \in q_{0}$ we call $\mathrm{Hom}(X,X) = \mathrm{End}(X)$ the endomorphism set.
}
\end{frame}
\begin{frame}
\begin{centering}
You now know what a quiver is.
\end{centering}
\end{frame}
% you now know what a quiver is

\begin{frame}
A \ul{category} $\mathcal{C}$ is a quiver with two further maps:
\begin{description}
\item[($\mathbbm{1}$)] For every object $X \in \mathcal{C}_{0}$ there is the \ul{identity map} 
\begin{align*}
\mathbbm{1} : \mathcal{C}_{0} \longrightarrow \mathcal{C}_{1} \\
X \longmapsto \mathbbm{1}_{X} : X \longrightarrow X
\end{align*}
\item[($\mu$)] For two \ul{composable} morphisms $\varphi$ and $\psi \in \mathcal{C}_{1}$, i.e. with $t(\varphi) = s(\psi)$ there is
the \ul{composition map}
\begin{align*}
\mu &: \mathcal{C}_{1} \times \mathcal{C}_{1} \longrightarrow \mathcal{C}_{1} \\
\varphi &: A \longrightarrow B \\
\psi &: B \longrightarrow C \\
(\varphi, \psi) &\longmapsto \mu(\varphi, \psi) := \varphi\psi : A \longrightarrow C
\end{align*}
\end{description}
\end{frame}

\begin{frame}
The defining poperties for $\mathbbm{1}$ and $\mu$ are:
\begin{enumerate}
\item $s(\mathbbm{1}_{M}) = M = t(\mathbbm{1}_{M})$, i.e. $\mathbbm{1}_{M} \in \mathrm{End}(M)$.
\item $s(\varphi\psi) = s(\varphi)$ and $t(\varphi\psi) = t(\psi)$, i.e. for objects $M, L, N \in \mathcal{C}_{0}$ we have
\[
\mu : \mathrm{Hom}(M,L) \times \mathrm{Hom}(L,N) \longrightarrow \mathrm{Hom}(M,N)
\]
\item $(\varphi\psi)\rho = \varphi(\psi\rho)$, i.e. composition is \ul{associative}
\item $\mathbbm{1}_{s(\varphi)}\varphi = \varphi = \varphi\mathbbm{1}_{t(\varphi)}$, i.e. the identity is a left and right \ul{unit} of the composition.
\end{enumerate}
\uncover<5->{These properties make each endomorphism set $\mathrm{End}(M)$ for $M \in \mathcal{C}$ together with the composition into a monoid, called the \ul{endomorphism monoid} $(\mathrm{End}(M), \mu)$.}
\end{frame}

\begin{frame}
So when you define a category, you always answer the four questions
\begin{itemize}[<.->]
\item What are the objects?
\item What are the morphisms? Especially what are the identity morphisms?
\item How do you compose morphisms?
\item Why is the composition associative? Why is the identity a unit for the composition?
\end{itemize}
\end{frame}
\begin{frame}
\begin{centering}
You now know what a category is
\end{centering}
\end{frame}
% you now know what a category is
\begin{frame}
A small example for a category: The symmetric group on two objects $S_{2}$.
\[
\begin{tikzcd}
{\{1,2\}} \arrow["{\mathbbm{1}_{\{1,2\}}}"', loop, distance=2em, in=215, out=145] \arrow["{(1,2)}"', loop, distance=2em, in=35, out=325]
\end{tikzcd}
\]
The rule that the composition of $(1,2)$ with itself results in the identity $\mathbbm{1}_{\{1,2\}}$ makes sure there are only 2 morphisms in total.\\

\uncover<2->{
This is an example of a category which is a sub-category of the category \textsc{Sets} with sets as objects and functions between sets as morphisms.
Between those categories lies the category \textsc{FinSets} in which the objects are finite sets and morphisms are functions between finite sets.
}
\end{frame}

\begin{frame}[fragile]
The categories $S_{2}$, \textsc{Set}, \textsc{FinSets} can themselves all be considered objects in a greater category, \textsc{Cat}, i.e. \ul{the category of categories}.
\uncover<2->{
\[
\begin{tikzcd}[ampersand replacement=\&]
S_{2} \arrow[rr] \&                                        \& S_{3} \arrow[r, shift left] \& C_{3} \arrow[d]                                                             \\
                 \& \textsc{Sets} \arrow[rr, shift left=2] \&                             \& \textsc{FinSets} \arrow[ll, shift left] \arrow[ll, shift left=4] \arrow[lu]
\end{tikzcd}
\]
}
\uncover<3->{
We know what the objects are in \textsc{Cat}. But what are the morphisms? What is meant with an arrow from \textsc{Sets} to \textsc{FinSets}?
}
\end{frame}
\begin{frame}
\begin{centering}
You now know the objects in the category \textsc{Cat} of all categories.
\end{centering}
\end{frame}
% you now know the objects in the category Cat of all categories

\begin{frame}
\noindent A \ul{functor} $F : \mathcal{C} \rightarrow \mathcal{D}$, between categories $\mathcal{C}$ and $\mathcal{D}$, consists of the
following data:

\begin{itemize}[<.->]
\item An object $Fc\in\mathcal{D}_{0}$, for each object $c \in \mathcal{C}_{0}$.
\item A function $Ff : Fc \rightarrow Fc' \in \mathcal{D}_{1}$, for each morphism $f : c \rightarrow c' \in \mathcal{C}_{1}$, so that the
source and target of $Ff$ are, respectively, equal to $F$ applied to the source or target of $f$, in other words,
$s(Ff) = Fs(f)$ and $t(Ff) = Ft(f)$.
\end{itemize}

\noindent The assignments are required to satisfy the following two \ul{functoriality axioms}:
\begin{itemize}[<.->]
\item For any composable pair $f : M \rightarrow N, g : N \rightarrow L \in \mathcal{C}_{1}, F\,f \cdot F\,g = F(f \cdot g)$.
\item For each object $c \in \mathcal{C}_{0}, F(1_{c}) = 1_{Fc}$.
\end{itemize}

\end{frame}
\begin{frame}
\noindent So with functors you always answer the four questions
\begin{itemize}[<.->]
\item How does it work on objects?
\item How does it work on morphisms?
\item Why does it respect composition?
\item Why does it respect identity morphisms?
\end{itemize}
\end{frame}
\begin{frame}
\begin{centering}
You now know what a functor is.
\end{centering}
\end{frame}
% you now know what a functor is
\begin{frame}
\begin{centering}
You now know the objects and morphisms in the category \textsc{Cat} of all categories.
\end{centering}
\end{frame}
% you now know the objects and morphisms in the category Cat of all categories
\begin{frame}
Let us now take a look at just two categories, $\mathcal{C}$ and $\mathcal{D}$ as objects in \textsc{Cat}.\\
The Hom-set $\mathrm{Hom}(\mathcal{C},\mathcal{D})$ of all functors $F : \mathcal{C} \longrightarrow \mathcal{D}$ is itself a category, called the \ul{functor category}.

This makes the functors $F, G, H : \mathcal{C} \longrightarrow \mathcal{D}$ objects in $\mathrm{Hom}(\mathcal{C},\mathcal{D})$ when before they were considered morphisms.

\[
\begin{tikzcd}[ampersand replacement=\&]
|[visible on=<2->]|F \arrow[r, "\alpha", Rightarrow, visible on=<4->] \& |[visible on=<2->]|G \arrow[r, "\beta", Rightarrow, visible on=<4->] \& |[visible on=<2->]|H
\end{tikzcd}
\]

\uncover<3->{
As you can imagine, we are again looking for the morphisms in this category, i.e. what are morphisms between functors?
}
\end{frame}
\begin{frame}
\begin{centering}
You now know the objects in the category $\mathrm{Hom}(\mathcal{C},\mathcal{D})$ of all functors between categories
$\mathcal{C}$ and $\mathcal{D}$.
\end{centering}
\end{frame}
% you now know the objects in the category Hom(C,D) of all functors between categories C and D
\begin{frame}[fragile]

\noindent Given categories $\mathcal{C}$ and $\mathcal{D}$ and functors $F : \mathcal{C} \rightarrow \mathcal{D}$ and
$G : \mathcal{C} \rightarrow \mathcal{D}$, a \ul{natural transformation} $\alpha : F \Rightarrow G$ consists of:
\begin{itemize}
\item a morphism $\alpha_{c} : Fc \rightarrow Gc \in \mathcal{D}_{1}$ for each object $c \in \mathcal{C}_{0}$, the collection of which
define the \ul{components} of the natural transformation, so that, for any morphism $f : c \rightarrow c' \in \mathcal{C}_{1}$, the following
square of morphisms in $\mathcal{D}$
\[
\begin{tikzcd}[ampersand replacement=\&]
Fc \arrow[r, "\alpha_{c}"] \arrow[d, "Ff"'] \& Gc \arrow[d, "Gf"] \\
Fc' \arrow[r, "\alpha_{c'}"]                \& Gc'                
\end{tikzcd}
\]

\ul{commutes}, i.e., has a a common composite $Fc \rightarrow Gc' \in \mathcal{D}_{1}$. This means explicitly that
\begin{align*}
Ff \alpha_{c'} = \alpha_{c} Gf,\,\forall f : c \rightarrow c' \in \mathcal{C}_{1}. \label{eq:naturality_condition}
\end{align*}
\end{itemize}
\end{frame}

\begin{frame}
\begin{centering}
You now know what a natural transformation is.
\end{centering}
\end{frame}
% you now know what a natural transformation is

\begin{frame}
\begin{centering}
You now know the objects and morphisms in the category $\mathrm{Hom}(\mathcal{C},\mathcal{D})$ of all functors between categories $\mathcal{C}$ and $\mathcal{D}$.
\end{centering}
\end{frame}
% you now know the objects and morphisms in the category Hom(C,D) of all functors between categories C and D

%--------
% ^
%  |
% Grundbegriffe Kategorientheorie

\begin{frame}

\end{frame}

% Abelsche Kategorie
% |
% |
% v

% Funktorkategorie nach Abelsche Kategorie ist Abelsche Kategorie
% |
% |
% v

\begin{frame}
Up to definition of Functor category $\HomAkmat$.
%$\mathrm{Hom}_{\Bbbk}(\mathcal{A},\Bbbk\textnormal{-}\mathrm{Mat})$.
\end{frame}

\section{How we get to $\mathrm{Hom}_{\Bbbk}(\mathcal{A},\Bbbk\textnormal{-}\mathrm{Mat})$ homakamat} %$\HomAkmat$}
\tikzset{invisible/.style={opacity=0.3}}
\subsection{$\FinSets$ and finite concrete categories}
%% start of subsection FinSets
\begin{frame}[fragile]
\[
\begin{tikzcd}
|[visible on=<1->]|\mathrm{FinSets} \arrow[d, visible on=<2>]                                                        &                               &                                              \\
|[visible on=<2>]|{\text{finite concrete category}\,\mathcal{C}} \arrow[rd, visible on=<0>] \arrow[rrd,  visible on=<0>] \arrow[dd,  visible on=<0>]  &                               &                                              \\
                                                                                  & |[ visible on=<0>]|{\text{quiver}\,q} \arrow[end anchor={[xshift=1.25em, yshift=.75em]},  visible on=<0>]{ld} & |[ visible on=<0>]|{\text{relations}\,\mathtt{rel}} \arrow[end anchor={[xshift=3.00em, yshift=.75em]},  visible on=<0>]{lld} \\
|[ visible on=<0>]|{\mathclap{\text{finitely presented category}\,\mathrm{fp}\mathcal{C}}} \arrow[d,  visible on=<0>] &                               & |[ visible on=<0>]|{\text{field}\,\Bbbk} \arrow[d,  visible on=<0>] \arrow[lld,  visible on=<0>]  \\
|[ visible on=<0>]|{\Bbbk\text{-}\mathrm{Algebroid}\,\mathcal{A}} \arrow[end anchor={[xshift=-1.25em, yshift=.75em]},  visible on=<0>]{rd}                                    &                               & |[ visible on=<0>]|\Bbbk\text{-}\mathrm{Mat} \arrow[end anchor={[xshift=1.25em, yshift=.75em]},  visible on=<0>]{ld} \\
                                                                                  & |[ visible on=<0>]|{\mathclap{\text{category of $\Bbbk$-linear functors}\,\mathrm{Hom}_{\Bbbk}(\mathcal{A},\Bbbk\text{-}\mathrm{Mat})}}
\end{tikzcd}
\]
\end{frame}

%% rest of subsection FinSets
\begin{frame}
The category of finite sets. We call a finite subcategory of $\FinSets$, i.e. a category with only finitely many objects (and the objects still being finite sets)
a \ul{finite concrete category}.
\end{frame}

\begin{frame}[fragile]
An example of a finite concrete category.
\[
\begin{tikzcd}
{\{1,2,3\}} \arrow["\begin{pmatrix} 1\mapsto 2 \\ 2\mapsto 3\\ 3\mapsto 1\end{pmatrix}"', loop, distance=2em, in=125, out=55] \arrow[rr, "\begin{pmatrix} 1\mapsto 4 \\ 2\mapsto 5\\ 3\mapsto 6\end{pmatrix}"] &  & {\{4,5,6\}} \arrow["\begin{pmatrix} 4\mapsto 5 \\ 5\mapsto 6\\ 6\mapsto 4\end{pmatrix}"', loop, distance=2em, in=125, out=55]
\end{tikzcd}
\]
\end{frame}

\subsection{Quivers, relations, multiplication with association and identity all together define an abstract category}
%% start of subsection Quivers
\begin{frame}[fragile]
\[
\begin{tikzcd}
|[visible on=<1->]|\mathrm{FinSets} \arrow[d, visible on=<1->]                                                        &                               &                                              \\
|[visible on=<1->]|{\text{finite concrete category}\,\mathcal{C}} \arrow[rd, visible on=<2->] \arrow[rrd,  visible on=<3>] \arrow[dd,  visible on=<0>]  &                               &                                              \\
                                                                                  & |[ visible on=<2->]|{\text{quiver}\,q} \arrow[end anchor={[xshift=1.25em, yshift=.75em]},  visible on=<0>]{ld} & |[ visible on=<3>]|{\text{relations}\,\mathtt{rel}} \arrow[end anchor={[xshift=3.00em, yshift=.75em]},  visible on=<0>]{lld} \\
|[ visible on=<0>]|{\mathclap{\text{finitely presented category}\,\mathrm{fp}\mathcal{C}}} \arrow[d,  visible on=<0>] &                               & |[ visible on=<0>]|{\text{field}\,\Bbbk} \arrow[d,  visible on=<0>] \arrow[lld,  visible on=<0>]  \\
|[ visible on=<0>]|{\Bbbk\text{-}\mathrm{Algebroid}\,\mathcal{A}} \arrow[end anchor={[xshift=-1.25em, yshift=.75em]},  visible on=<0>]{rd}                                    &                               & |[ visible on=<0>]|\Bbbk\text{-}\mathrm{Mat} \arrow[end anchor={[xshift=1.25em, yshift=.75em]},  visible on=<0>]{ld} \\
                                                                                  & |[ visible on=<0>]|{\mathclap{\text{category of $\Bbbk$-linear functors}\,\mathrm{Hom}_{\Bbbk}(\mathcal{A},\Bbbk\text{-}\mathrm{Mat})}}
\end{tikzcd}
\]
\end{frame}

%% rest of subsection Quivers
\begin{frame}[fragile]
The underlying quiver of the category. The objects are now denoted by numbers $1,2$ 
\[
\begin{tikzcd}
1 \arrow["a"', loop, distance=2em, in=305, out=235] \arrow[rr, "b"] &  & 2 \arrow["c"', loop, distance=2em, in=305, out=235]
\end{tikzcd}
\]
\end{frame}

\begin{frame}
We are now ``abstracting'' from the concrete category of sets and functions between those sets while retaining the
relations between those functions. $f, g, h \in \mathcal{C}$ with $fg = h$ then we want corresponding abstract morphisms
$a, b, c \in q$ with $ab = c$. This is accomplished by adding the relation $(ab, c)$ in our set of relations.
If at this stage, we introduce a field $\Bbbk$ we can write $(ab, c)$ as $ab - c = 0$. But at this stage we don't need
a field yet.
\end{frame}

\subsection{finitely presented category $\fpC$}
%% start of subsection fpC
\begin{frame}[fragile]
\[
\begin{tikzcd}
|[visible on=<1->]|\mathrm{FinSets} \arrow[d, visible on=<1->]                                                        &                               &                                              \\
|[visible on=<1->]|{\text{finite concrete category}\,\mathcal{C}} \arrow[rd, visible on=<1->] \arrow[rrd,  visible on=<1->] \arrow[dd,  visible on=<2->]  &                               &                                              \\
                                                                                  & |[ visible on=<1->]|{\text{quiver}\,q} \arrow[end anchor={[xshift=1.25em, yshift=.75em]},  visible on=<2->]{ld} & |[ visible on=<1->]|{\text{relations}\,\mathtt{rel}} \arrow[end anchor={[xshift=3.00em, yshift=.75em]},  visible on=<2->]{lld} \\
|[ visible on=<3>]|{\mathclap{\text{finitely presented category}\,\mathrm{fp}\mathcal{C}}} \arrow[d,  visible on=<0>] &                               & |[ visible on=<0>]|{\text{field}\,\Bbbk} \arrow[d,  visible on=<0>] \arrow[lld,  visible on=<0>]  \\
|[ visible on=<0>]|{\Bbbk\text{-}\mathrm{Algebroid}\,\mathcal{A}} \arrow[end anchor={[xshift=-1.25em, yshift=.75em]},  visible on=<0>]{rd}                                    &                               & |[ visible on=<0>]|\Bbbk\text{-}\mathrm{Mat} \arrow[end anchor={[xshift=1.25em, yshift=.75em]},  visible on=<0>]{ld} \\
                                                                                  & |[ visible on=<0>]|{\mathclap{\text{category of $\Bbbk$-linear functors}\,\mathrm{Hom}_{\Bbbk}(\mathcal{A},\Bbbk\text{-}\mathrm{Mat})}}
\end{tikzcd}
\]
\end{frame}

%% rest of subsection fpC
\begin{frame}
This is a finite presentation of our concrete category.
\end{frame}

\subsection{field $\Bbbk$ and the matrix category}
%% start of subsection field k
\begin{frame}[fragile]
\[
\begin{tikzcd}
|[visible on=<1->]|\mathrm{FinSets} \arrow[d, visible on=<1->]                                                        &                               &                                              \\
|[visible on=<1->]|{\text{finite concrete category}\,\mathcal{C}} \arrow[rd, visible on=<1->] \arrow[rrd,  visible on=<1->] \arrow[dd,  visible on=<1->]  &                               &                                              \\
                                                                                  & |[ visible on=<1->]|{\text{quiver}\,q} \arrow[end anchor={[xshift=1.25em, yshift=.75em]},  visible on=<1->]{ld} & |[ visible on=<1->]|{\text{relations}\,\mathtt{rel}} \arrow[end anchor={[xshift=3.00em, yshift=.75em]},  visible on=<1->]{lld} \\
|[ visible on=<1->]|{\mathclap{\text{finitely presented category}\,\mathrm{fp}\mathcal{C}}} \arrow[d,  visible on=<0>] &                               & |[ visible on=<2->]|{\text{field}\,\Bbbk} \arrow[d,  visible on=<3>] \arrow[lld,  visible on=<0>]  \\
|[ visible on=<0>]|{\Bbbk\text{-}\mathrm{Algebroid}\,\mathcal{A}} \arrow[end anchor={[xshift=-1.25em, yshift=.75em]},  visible on=<0>]{rd}                                    &                               & |[ visible on=<3>]|\Bbbk\text{-}\mathrm{Mat} \arrow[end anchor={[xshift=1.25em, yshift=.75em]},  visible on=<0>]{ld} \\
                                                                                  & |[ visible on=<0>]|{\mathclap{\text{category of $\Bbbk$-linear functors}\,\mathrm{Hom}_{\Bbbk}(\mathcal{A},\Bbbk\text{-}\mathrm{Mat})}}
\end{tikzcd}
\]
\end{frame}

%% rest of subsection field k
\begin{frame}
The matrix category $\kmat$ is an Abelian category.
\end{frame}

\subsection{$\Bbbk\text{-}\mathrm{Algebroid}$}
%% start of subsection k-Algebroid
\begin{frame}[fragile]
\[
\begin{tikzcd}
|[visible on=<1->]|\mathrm{FinSets} \arrow[d, visible on=<1->]                                                        &                               &                                              \\
|[visible on=<1->]|{\text{finite concrete category}\,\mathcal{C}} \arrow[rd, visible on=<1->] \arrow[rrd,  visible on=<1->] \arrow[dd,  visible on=<1->]  &                               &                                              \\
                                                                                  & |[ visible on=<1->]|{\text{quiver}\,q} \arrow[end anchor={[xshift=1.25em, yshift=.75em]},  visible on=<1->]{ld} & |[ visible on=<1->]|{\text{relations}\,\mathtt{rel}} \arrow[end anchor={[xshift=3.00em, yshift=.75em]},  visible on=<1->]{lld} \\
|[ visible on=<1->]|{\mathclap{\text{finitely presented category}\,\mathrm{fp}\mathcal{C}}} \arrow[d,  visible on=<2->] &                               & |[ visible on=<1->]|{\text{field}\,\Bbbk} \arrow[d,  visible on=<1->] \arrow[lld,  visible on=<2->]  \\
|[ visible on=<3>]|{\Bbbk\text{-}\mathrm{Algebroid}\,\mathcal{A}} \arrow[end anchor={[xshift=-1.25em, yshift=.75em]},  visible on=<0>]{rd}                                    &                               & |[ visible on=<1->]|\Bbbk\text{-}\mathrm{Mat} \arrow[end anchor={[xshift=1.25em, yshift=.75em]},  visible on=<0>]{ld} \\
                                                                                  & |[ visible on=<0>]|{\mathclap{\text{category of $\Bbbk$-linear functors}\,\mathrm{Hom}_{\Bbbk}(\mathcal{A},\Bbbk\text{-}\mathrm{Mat})}}
\end{tikzcd}
\]
\end{frame}

%% rest of subsection k-Algebroid
\begin{frame}
$\Bbbk$-Algebroid, $\Bbbk$-linear category and $\Bbbk$-Algebras with orthogonal idempotents are mutually inverse constructions.
\end{frame}

\subsection{The functor category $\HomAkmat$}
%% start of subsection HomAkmat
\begin{frame}[fragile]
\[
\begin{tikzcd}
|[visible on=<1->]|\mathrm{FinSets} \arrow[d, visible on=<1->]                                                        &                               &                                              \\
|[visible on=<1->]|{\text{finite concrete category}\,\mathcal{C}} \arrow[rd, visible on=<1->] \arrow[rrd,  visible on=<1->] \arrow[dd,  visible on=<1->]  &                               &                                              \\
                                                                                  & |[ visible on=<1->]|{\text{quiver}\,q} \arrow[end anchor={[xshift=1.25em, yshift=.75em]},  visible on=<1->]{ld} & |[ visible on=<1->]|{\text{relations}\,\mathtt{rel}} \arrow[end anchor={[xshift=3.00em, yshift=.75em]},  visible on=<1->]{lld} \\
|[ visible on=<1->]|{\mathclap{\text{finitely presented category}\,\mathrm{fp}\mathcal{C}}} \arrow[d,  visible on=<1->] &                               & |[ visible on=<1->]|{\text{field}\,\Bbbk} \arrow[d,  visible on=<1->] \arrow[lld,  visible on=<1->]  \\
|[ visible on=<1->]|{\Bbbk\text{-}\mathrm{Algebroid}\,\mathcal{A}} \arrow[end anchor={[xshift=-1.25em, yshift=.75em]},  visible on=<2>]{rd}                                    &                               & |[ visible on=<1->]|\Bbbk\text{-}\mathrm{Mat} \arrow[end anchor={[xshift=1.25em, yshift=.75em]},  visible on=<2>]{ld} \\
                                                                                  & |[ visible on=<2>]|{\mathclap{\text{category of $\Bbbk$-linear functors}\,\mathrm{Hom}_{\Bbbk}(\mathcal{A},\Bbbk\text{-}\mathrm{Mat})}}
\end{tikzcd}
\]
\end{frame}

%% juxtaposition kmat <-> HomAkmat
\begin{frame}[fragile]
\begin{tabular}{p{.28\textwidth}p{.33\textwidth}p{.33\textwidth}}
$\mathbf{Category}\,\mathcal{C}$ & $\HomAkmat$ & $\kmat$ \\
\hline \\
\textbf{Objects} $\mathcal{C}_{0}$ & \makecell[l]{$\Bbbk$-linear functors $F :$\\ $\mathcal{A} \rightarrow \kmat$ \\ $o \in \mathcal{A}_{0} \mapsto n \in \kmat_{0}$ \\ $\varphi \in \mathcal{A}_{1} \mapsto A \in \kmat_{1}$} & natural numbers $\mathbb{N}_{0}$ \\
\textbf{Morphisms} $\mathcal{C}_{1}$ & natural transformations & $m\times n$-matrices \\
\textbf{Composition}: & & \\
\makecell[l]{$\varphi : A \rightarrow B$, \\$\psi : B \rightarrow C$ \\ $\varphi\psi : A \rightarrow C$} & \makecell[l]{$\eta : F \Rightarrow G$,\\ $\varepsilon : G \Rightarrow H$ \\ $\eta\varepsilon : F \Rightarrow H$} & \makecell[l]{$\varphi : n \rightarrow m\qquad$ \\ $\psi : m \rightarrow l\qquad$ \\ $\varphi\psi : n \rightarrow l$}
\end{tabular}
%The category $\HomAkmat$ of $\Bbbk$-linear functors from a $\Bbbk$-Algebroid into the matrix category $\kmat$ is a $\Bbbk$-linear
%abelian category.
\end{frame}

\begin{frame}[fragile]
The functor category $\mathrm{Hom}_{\Bbbk}(\mathcal{A},\Bbbk\text{-}\mathrm{Mat})$ is
\begin{itemize}[<+->]
\item $\Bbbk$-linear
\item additive
\item pre-abelian
\item abelian
\end{itemize}
\end{frame}

% end section HomAkmat






\section{Direct sum decomposition}

\subsection{Morphisms between representations}

\begin{frame}
Calculating the External Hom between two representations.
\end{frame}

\section{Finding invariants for representations}

\subsection{Another Jupyter-Notebook example}

\begin{frame}
Representations with the same image on all objects.
\end{frame}




% Zusammenfassung



% Beispiele

Limitations: Categories with endomorphism monoids that are not explicitly cyclic
can not be represented by our method. This means we can't look at representations
for groups that are not explicitly cyclic, like the

Klein four group: V = {(), (1,2)(3,4), (1,3)(2,4), (1,4)(2,3) }

The first constructions pose no problems yet:

V := ConcreteCategoryForCAP( [ [2,1,4,3], [2,1,4,3], [3,4,1,2], [4,3,2,1] ] );
> A finite concrete category

qV := RightQuiverFromConcreteCategory( V );
> q(1)[a:1->1,b:1->1,c:1->1]

but we get an error as soon as we want to calculate the endomorphism relations:

rel := RelationsOfEndomorphisms( V );
> Error, we assume at most 1 generating endomorphism per vertex

And since these relations are essential going forward and other algorithms rely on it,
we can't go further with this example.

When we are representing groups, we therefore can only work with cyclic groups.

C4 := ConcreteCategoryForCAP( [ [2,3,4,1] ] );
> A finite concrete category

fpC := AsFpCategory( C4 );
> Monoid generated by the right quiver q(1)[a:1->1] with relations [ a*a*a*a = 1 ]


C6 := ConcreteCategoryForCAP( [ [2,3,4,5,6,1] ] );



\end{document}