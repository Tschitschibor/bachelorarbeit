\documentclass[12pt,compress]{beamer}
\setbeamertemplate{navigation symbols}{}

\usepackage[utf8]{inputenc}
\usepackage{tikz}
\usetikzlibrary{mindmap,trees,shadows}
\usepackage{tikz-cd}
\usepackage{mathtools}

\usepackage{ifthen}
\usetikzlibrary{calc}

%%% For proper underline
\usepackage{soul}
%\setuldepth{gjpqy}
%\setuldepth\strut
\setuldepth{-1}

%%% Macros for our recurring categories
\newcommand{\kmat}{\Bbbk\textnormal{-}\mathbf{mat}}
\newcommand{\kAlgebroid}{\Bbbk\textnormal{-}\mathrm{algebroid}}
\newcommand{\Rmat}{R\textnormal{-}\mathbf{mat}}
\newcommand{\HomAkmat}{\mathrm{Hom_{\Bbbk}}(\mathcal{A},\kmat)}
\newcommand{\HomARmat}{\mathrm{Hom_{R}}(\mathcal{A},\Rmat)}
\newcommand{\HomA}{\mathrm{Hom}_{\mathcal{A}}}
\newcommand{\FinSets}{\mathrm{FinSets}}
\newcommand{\Cat}{\mathrm{\textbf{Cat}}}
\newcommand{\Set}{\mathrm{\textbf{Set}}}
\newcommand{\Quiv}{\mathrm{\textbf{Quiv}}}
\newcommand{\kChat}{\widehat{\Bbbk\mathcal{C}}}
\newcommand{\fpC}{\mathrm{fp}\mathcal{C}}

\newcommand\overviewslide{1}

\newcounter{saveenumi}
\newcommand{\seti}{\setcounter{saveenumi}{\value{enumi}}}
\newcommand{\conti}{\setcounter{enumi}{\value{saveenumi}}}

\resetcounteronoverlays{saveenumi}


\newcommand{\visiblesub}[2]{
  \ifthenelse{\ifnum#1>#2}
  { 0; }
  { 1; }
}





% If you wish to uncover everything in a step-wise fashion, uncomment
% the following command: 

\beamerdefaultoverlayspecification{<+->}


\title[The category of representations]
{The category of representations of a concrete category as a functor category}

\author{Tibor Gr{\"u}n}

%\date{October 30, 2020} % date of printed version
%\date{November 3, 2020} % date of submission
\date{November 16, 2020} % date of seminar presentation
\begin{document}

  %% to uncover arrows and nodes one after another
  %% use: 
  %% |[visible on=<1->]|\mathrm{FinSets}
  % in front of node
  %% \arrow[rd, visible on=<0>]
  % inside arrow
  % 1- means from first slide on
  % 2 only on second slide
  % -3 on slides 1,2,3
  % 0 always invisible
    \tikzset{
      invisible/.style={opacity=0.3},
      visible on/.style={alt={#1{}{invisible}}},
      alt/.code args={<#1>#2#3}{%
        \alt<#1>{ \pgfkeysalso{#2} }{ \pgfkeysalso{#3} }%
      }
  }

  % Keys to support piece-wise uncovering of elements in TikZ pictures:
  % \node[visible on=<2->](foo){Foo}
  % \node[visible on=<{2,4}>](bar){Bar}   % put braces around comma expressions
  %
  % Internally works by setting opacity=0 when invisible, which has the 
  % adavantage (compared to \node<2->(foo){Foo} that the node is always there, hence
  % always consumes space plus that coordinate (foo) is always available.
  %
  % The actual command that implements the invisibility can be overriden
  % by altering the style invisible. For instance \tikzsset{invisible/.style={opacity=0.2}}
  % would dim the "invisible" parts. Alternatively, the color might be set to white, if the
  % output driver does not support transparencies (e.g., PS) 
  %

\begin{frame}
  \titlepage
\end{frame}

\section{How we get to $\HomAkmat$}
\subsection{$\FinSets$ and finite concrete categories}
%% start of subsection FinSets
\begin{frame}[fragile]
\[
\begin{tikzcd}
|[visible on=<1->]|\mathrm{FinSets} \arrow[d, visible on=<2>]                                                        &                               &                                              \\
|[visible on=<2>]|{\text{finite concrete category}\,\mathcal{C}} \arrow[rd, visible on=<0>] \arrow[rrd,  visible on=<0>] \arrow[dd,  visible on=<0>]  &                               &                                              \\
                                                                                  & |[ visible on=<0>]|{\text{quiver}\,q} \arrow[end anchor={[xshift=1.25em, yshift=.75em]},  visible on=<0>]{ld} & |[ visible on=<0>]|{\text{relations}\,\mathtt{rel}} \arrow[end anchor={[xshift=3.00em, yshift=.75em]},  visible on=<0>]{lld} \\
|[ visible on=<0>]|{\mathclap{\text{finitely presented category}\,\mathrm{fp}\mathcal{C}}} \arrow[d,  visible on=<0>] &                               & |[ visible on=<0>]|{\text{field}\,\Bbbk} \arrow[d,  visible on=<0>] \arrow[lld,  visible on=<0>]  \\
|[ visible on=<0>]|{\Bbbk\text{-}\mathrm{Algebroid}\,\mathcal{A}} \arrow[end anchor={[xshift=-1.25em, yshift=.75em]},  visible on=<0>]{rd}                                    &                               & |[ visible on=<0>]|\Bbbk\text{-}\mathrm{Mat} \arrow[end anchor={[xshift=1.25em, yshift=.75em]},  visible on=<0>]{ld} \\
                                                                                  & |[ visible on=<0>]|{\mathclap{\text{Functor category}\,\mathrm{Hom}_{\Bbbk}(\mathcal{A},\Bbbk\text{-}\mathrm{Mat})}}
\end{tikzcd}
\]
\end{frame}

%% rest of subsection FinSets
\begin{frame}
The category of finite sets. We call a finite subcategory of $\FinSets$, i.e. a category with only finitely many objects (and the objects still being finite sets)
a \ul{finite concrete category}.
\end{frame}

\begin{frame}[fragile]
An example of a finite concrete category.
\[
\begin{tikzcd}
{\{1,2,3\}} \arrow["\begin{pmatrix} 1\mapsto 2 \\ 2\mapsto 3\\ 3\mapsto 1\end{pmatrix}"', loop, distance=2em, in=125, out=55] \arrow[rr, "\begin{pmatrix} 1\mapsto 4 \\ 2\mapsto 5\\ 3\mapsto 6\end{pmatrix}"] &  & {\{4,5,6\}} \arrow["\begin{pmatrix} 4\mapsto 5 \\ 5\mapsto 6\\ 6\mapsto 4\end{pmatrix}"', loop, distance=2em, in=125, out=55]
\end{tikzcd}
\]
\end{frame}

\subsection{Quivers, relations, multiplication with association and identity all together define an abstract category}
%% start of subsection Quivers
\begin{frame}[fragile]
\[
\begin{tikzcd}
|[visible on=<1->]|\mathrm{FinSets} \arrow[d, visible on=<1->]                                                        &                               &                                              \\
|[visible on=<1->]|{\text{finite concrete category}\,\mathcal{C}} \arrow[rd, visible on=<2->] \arrow[rrd,  visible on=<3>] \arrow[dd,  visible on=<0>]  &                               &                                              \\
                                                                                  & |[ visible on=<2->]|{\text{quiver}\,q} \arrow[end anchor={[xshift=1.25em, yshift=.75em]},  visible on=<0>]{ld} & |[ visible on=<3>]|{\text{relations}\,\mathtt{rel}} \arrow[end anchor={[xshift=3.00em, yshift=.75em]},  visible on=<0>]{lld} \\
|[ visible on=<0>]|{\mathclap{\text{finitely presented category}\,\mathrm{fp}\mathcal{C}}} \arrow[d,  visible on=<0>] &                               & |[ visible on=<0>]|{\text{field}\,\Bbbk} \arrow[d,  visible on=<0>] \arrow[lld,  visible on=<0>]  \\
|[ visible on=<0>]|{\Bbbk\text{-}\mathrm{Algebroid}\,\mathcal{A}} \arrow[end anchor={[xshift=-1.25em, yshift=.75em]},  visible on=<0>]{rd}                                    &                               & |[ visible on=<0>]|\Bbbk\text{-}\mathrm{Mat} \arrow[end anchor={[xshift=1.25em, yshift=.75em]},  visible on=<0>]{ld} \\
                                                                                  & |[ visible on=<0>]|{\mathclap{\text{Functor category}\,\mathrm{Hom}_{\Bbbk}(\mathcal{A},\Bbbk\text{-}\mathrm{Mat})}}
\end{tikzcd}
\]
\end{frame}

%% rest of subsection Quivers
\begin{frame}[fragile]
The underlying quiver of the category. The objects are now denoted by numbers $1,2$ 
\[
\begin{tikzcd}
1 \arrow["a"', loop, distance=2em, in=305, out=235] \arrow[rr, "b"] &  & 2 \arrow["c"', loop, distance=2em, in=305, out=235]
\end{tikzcd}
\]
\end{frame}

\begin{frame}
We are now ``abstracting'' from the concrete category of sets and functions between those sets while retaining the
relations between those functions. $f, g, h \in \mathcal{C}$ with $fg = h$ then we want corresponding abstract morphisms
$a, b, c \in q$ with $ab = c$. This is accomplished by adding the relation $(ab, c)$ in our set of relations.
If at this stage, we introduce a field $\mathbb{k}$ we can write $(ab, c)$ as $ab - c = 0$. But at this stage we don't need
a field yet.
\end{frame}

\subsection{finitely presented category $\fpC$}
%% start of subsection fpC
\begin{frame}[fragile]
\[
\begin{tikzcd}
|[visible on=<1->]|\mathrm{FinSets} \arrow[d, visible on=<1->]                                                        &                               &                                              \\
|[visible on=<1->]|{\text{finite concrete category}\,\mathcal{C}} \arrow[rd, visible on=<1->] \arrow[rrd,  visible on=<1->] \arrow[dd,  visible on=<2->]  &                               &                                              \\
                                                                                  & |[ visible on=<1->]|{\text{quiver}\,q} \arrow[end anchor={[xshift=1.25em, yshift=.75em]},  visible on=<2->]{ld} & |[ visible on=<1->]|{\text{relations}\,\mathtt{rel}} \arrow[end anchor={[xshift=3.00em, yshift=.75em]},  visible on=<2->]{lld} \\
|[ visible on=<3>]|{\mathclap{\text{finitely presented category}\,\mathrm{fp}\mathcal{C}}} \arrow[d,  visible on=<0>] &                               & |[ visible on=<0>]|{\text{field}\,\Bbbk} \arrow[d,  visible on=<0>] \arrow[lld,  visible on=<0>]  \\
|[ visible on=<0>]|{\Bbbk\text{-}\mathrm{Algebroid}\,\mathcal{A}} \arrow[end anchor={[xshift=-1.25em, yshift=.75em]},  visible on=<0>]{rd}                                    &                               & |[ visible on=<0>]|\Bbbk\text{-}\mathrm{Mat} \arrow[end anchor={[xshift=1.25em, yshift=.75em]},  visible on=<0>]{ld} \\
                                                                                  & |[ visible on=<0>]|{\mathclap{\text{Functor category}\,\mathrm{Hom}_{\Bbbk}(\mathcal{A},\Bbbk\text{-}\mathrm{Mat})}}
\end{tikzcd}
\]
\end{frame}

%% rest of subsection fpC
\begin{frame}
This is a finite presentation of our concrete category.
\end{frame}

\subsection{field $\Bbbk$ and the matrix category}
%% start of subsection field k
\begin{frame}[fragile]
\[
\begin{tikzcd}
|[visible on=<1->]|\mathrm{FinSets} \arrow[d, visible on=<1->]                                                        &                               &                                              \\
|[visible on=<1->]|{\text{finite concrete category}\,\mathcal{C}} \arrow[rd, visible on=<1->] \arrow[rrd,  visible on=<1->] \arrow[dd,  visible on=<1->]  &                               &                                              \\
                                                                                  & |[ visible on=<1->]|{\text{quiver}\,q} \arrow[end anchor={[xshift=1.25em, yshift=.75em]},  visible on=<1->]{ld} & |[ visible on=<1->]|{\text{relations}\,\mathtt{rel}} \arrow[end anchor={[xshift=3.00em, yshift=.75em]},  visible on=<1->]{lld} \\
|[ visible on=<1->]|{\mathclap{\text{finitely presented category}\,\mathrm{fp}\mathcal{C}}} \arrow[d,  visible on=<0>] &                               & |[ visible on=<2->]|{\text{field}\,\Bbbk} \arrow[d,  visible on=<3>] \arrow[lld,  visible on=<0>]  \\
|[ visible on=<0>]|{\Bbbk\text{-}\mathrm{Algebroid}\,\mathcal{A}} \arrow[end anchor={[xshift=-1.25em, yshift=.75em]},  visible on=<0>]{rd}                                    &                               & |[ visible on=<3>]|\Bbbk\text{-}\mathrm{Mat} \arrow[end anchor={[xshift=1.25em, yshift=.75em]},  visible on=<0>]{ld} \\
                                                                                  & |[ visible on=<0>]|{\mathclap{\text{Functor category}\,\mathrm{Hom}_{\Bbbk}(\mathcal{A},\Bbbk\text{-}\mathrm{Mat})}}
\end{tikzcd}
\]
\end{frame}

%% rest of subsection field k
\begin{frame}
The matrix category $\kmat$ is an Abelian category.
\end{frame}

\subsection{$\Bbbk\text{-}\mathrm{Algebroid}$}
%% start of subsection k-Algebroid
\begin{frame}[fragile]
\[
\begin{tikzcd}
|[visible on=<1->]|\mathrm{FinSets} \arrow[d, visible on=<1->]                                                        &                               &                                              \\
|[visible on=<1->]|{\text{finite concrete category}\,\mathcal{C}} \arrow[rd, visible on=<1->] \arrow[rrd,  visible on=<1->] \arrow[dd,  visible on=<1->]  &                               &                                              \\
                                                                                  & |[ visible on=<1->]|{\text{quiver}\,q} \arrow[end anchor={[xshift=1.25em, yshift=.75em]},  visible on=<1->]{ld} & |[ visible on=<1->]|{\text{relations}\,\mathtt{rel}} \arrow[end anchor={[xshift=3.00em, yshift=.75em]},  visible on=<1->]{lld} \\
|[ visible on=<1->]|{\mathclap{\text{finitely presented category}\,\mathrm{fp}\mathcal{C}}} \arrow[d,  visible on=<2->] &                               & |[ visible on=<1->]|{\text{field}\,\Bbbk} \arrow[d,  visible on=<1->] \arrow[lld,  visible on=<2->]  \\
|[ visible on=<3>]|{\Bbbk\text{-}\mathrm{Algebroid}\,\mathcal{A}} \arrow[end anchor={[xshift=-1.25em, yshift=.75em]},  visible on=<0>]{rd}                                    &                               & |[ visible on=<1->]|\Bbbk\text{-}\mathrm{Mat} \arrow[end anchor={[xshift=1.25em, yshift=.75em]},  visible on=<0>]{ld} \\
                                                                                  & |[ visible on=<0>]|{\mathclap{\text{Functor category}\,\mathrm{Hom}_{\Bbbk}(\mathcal{A},\Bbbk\text{-}\mathrm{Mat})}}
\end{tikzcd}
\]
\end{frame}

%% rest of subsection k-Algebroid
\begin{frame}
$\Bbbk$-Algebroid, $\Bbbk$-linear category and $\Bbbk$-Algebras with orthogonal idempotents are mutually inverse constructions.
\end{frame}

\subsection{The functor category $\HomAkmat$}
%% start of subsection HomAkmat
\begin{frame}[fragile]
\[
\begin{tikzcd}
|[visible on=<1->]|\mathrm{FinSets} \arrow[d, visible on=<1->]                                                        &                               &                                              \\
|[visible on=<1->]|{\text{finite concrete category}\,\mathcal{C}} \arrow[rd, visible on=<1->] \arrow[rrd,  visible on=<1->] \arrow[dd,  visible on=<1->]  &                               &                                              \\
                                                                                  & |[ visible on=<1->]|{\text{quiver}\,q} \arrow[end anchor={[xshift=1.25em, yshift=.75em]},  visible on=<1->]{ld} & |[ visible on=<1->]|{\text{relations}\,\mathtt{rel}} \arrow[end anchor={[xshift=3.00em, yshift=.75em]},  visible on=<1->]{lld} \\
|[ visible on=<1->]|{\mathclap{\text{finitely presented category}\,\mathrm{fp}\mathcal{C}}} \arrow[d,  visible on=<1->] &                               & |[ visible on=<1->]|{\text{field}\,\Bbbk} \arrow[d,  visible on=<1->] \arrow[lld,  visible on=<1->]  \\
|[ visible on=<1->]|{\Bbbk\text{-}\mathrm{Algebroid}\,\mathcal{A}} \arrow[end anchor={[xshift=-1.25em, yshift=.75em]},  visible on=<2>]{rd}                                    &                               & |[ visible on=<1->]|\Bbbk\text{-}\mathrm{Mat} \arrow[end anchor={[xshift=1.25em, yshift=.75em]},  visible on=<2>]{ld} \\
                                                                                  & |[ visible on=<2>]|{\mathclap{\text{Functor category}\,\mathrm{Hom}_{\Bbbk}(\mathcal{A},\Bbbk\text{-}\mathrm{Mat})}}
\end{tikzcd}
\]
\end{frame}

%% rest of subsection HomAkmat
\begin{frame}
The functor category $\HomAkmat$ of $\Bbbk$-linear functors from a $\Bbbk$-Algebroid into the matrix category $\kmat$ is a $\Bbbk$-linear
abelian category.
\end{frame}

\begin{frame}[fragile]
The functor category $\mathrm{Hom}_{\Bbbk}(\mathcal{A},\Bbbk\text{-}\mathrm{Mat})$ is
\begin{itemize}[<+->]
\item $\Bbbk$-linear
\item additive
\item pre-abelian
\item abelian
\end{itemize}
\end{frame}

% end section HomAkmat





\section{Menschen}

Wie bin ich zu dieser Arbeit gekommen?

Sebastians entwickeln CAP.

Peter Webb entwickelt catreps in GAP. Findet CAP.

Meine Aufgabe ist es, die Funktionalität von Peter Webbs catreps in CAP
zu verwirklichen als Paket CatReps. Dabei wird hauptsächlich FunctorCategories
verwendet.

QPA2 mit Kamahl war bevor wir AsFpCategory geschrieben hatten die Schnittstelle zwischen
Konkreter Kategorie und abstraktem Algebroid: nämlich die Pfad-Algebra. Daraus ergab sich
überhaupt erst das Kapitel 4 zu Algebra und Algebroid.


\section{Direct sum decomposition}

\subsection{Morphisms between representations}

\begin{frame}
Calculating the External Hom between two representations.
\end{frame}

\section{Finding invariants for representations}

\subsection{Another Jupyter-Notebook example}

\begin{frame}
Representations with the same image on all objects.
\end{frame}



% Motivation
Wrap Peter Webb's catreps into a CAP package CatReps based on the package FunctorCategories.

Ohne Peter Webbs catreps gäbe es meine Arbeit nicht. Ebensowenig ohne CAP.

Meine Arbeit ist die Umsetzung von catreps in CAP und homalg. Insbesondere wird für
Matrizen und Ringe nicht mehr auf die internen Datenstrukturen von Gap zugegriffen, sondern
durch effizientere Datenstrukturen in homalg.

Dies ermöglicht es, 

% Zusammenfassung



% Beispiele

Limitations: Categories with endomorphism monoids that are not explicitly cyclic
can not be represented by our method. This means we can't look at representations
for groups that are not explicitly cyclic, like the

Klein four group: V = {(), (1,2)(3,4), (1,3)(2,4), (1,4)(2,3) }

The first constructions pose no problems yet:

V := ConcreteCategoryForCAP( [ [2,1,4,3], [2,1,4,3], [3,4,1,2], [4,3,2,1] ] );
> A finite concrete category

qV := RightQuiverFromConcreteCategory( V );
> q(1)[a:1->1,b:1->1,c:1->1]

but we get an error as soon as we want to calculate the endomorphism relations:

rel := RelationsOfEndomorphisms( V );
> Error, we assume at most 1 generating endomorphism per vertex

And since these relations are essential going forward and other algorithms rely on it,
we can't go further with this example.

When we are representing groups, we therefore can only work with cyclic groups.

C4 := ConcreteCategoryForCAP( [ [2,3,4,1] ] );
> A finite concrete category

fpC := AsFpCategory( C4 );
> Monoid generated by the right quiver q(1)[a:1->1] with relations [ a*a*a*a = 1 ]


C6 := ConcreteCategoryForCAP( [ [2,3,4,5,6,1] ] );



\end{document}