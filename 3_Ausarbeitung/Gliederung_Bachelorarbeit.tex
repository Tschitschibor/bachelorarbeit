\documentclass{article}

%% maybe some definitions are still here that I want in the definitions_packages.tex file
%% for now this preamble remains, in case I need some more definitions.

\usepackage[a4paper,%
            left=2.5cm,right=2.5cm,top=2.5cm,bottom=2.5cm,%
            footskip=.6cm]{geometry}

\def\changemargin#1#2{\list{}{\rightmargin#2\leftmargin#1}\item[]}
\let\endchangemargin=\endlist

\usepackage[utf8]{inputenc}
\usepackage{fancyvrb}

%%% For captions and references
\usepackage{hyperref}
\usepackage[figure]{hypcap}
\newcommand{\Algoref}[1]{%
	\hyperref[algo:#1]{Algorithm~\ref*{algo:#1}}%
}
\newcommand{\algoref}[1]{%
	\hyperref[algo:#1]{Algorithm~\ref*{algo:#1}}%
}
\newcommand{\Funcref}[1]{%
	\hyperref[func:#1]{Function~\ref*{func:#1}}%
}
\newcommand{\funcref}[1]{%
	\hyperref[func:#1]{\texttt{#1}}%
}

%%% For footnotes at end of text
\usepackage{endnotes}
%\let\footnote{\endnote}
%% Heading of endnotes section
\renewcommand*{\notesname}{Annotations}

\makeatletter
\def\enoteheading{\section{\notesname
\@mkboth{\MakeUppercase{\notesname}}{\MakeUppercase{\notesname}}}%
\mbox{}\par\vskip-\baselineskip}
\makeatother

%%% For switching languages in quotes
\usepackage[english]{babel}
%\usepackage[english, german]{babel} %% makes troubles

%%% For quotation
%% english guillemets have to be custom defined in /tex/latex/csquotes/csquotes.cfg
\usepackage[english = guillemets, autostyle = true,autopunct,csdisplay = true]{csquotes}

%%% For proper underline
\usepackage{soul}
%\setuldepth{gjpqy}
%\setuldepth\strut
\setuldepth{-1}

%%% Color
\usepackage{xcolor}
\usepackage{color}
\definecolor{FireBrick}{rgb}{0.5812,0.0074,0.0083}
\definecolor{RoyalBlue}{rgb}{0.0236,0.0894,0.6179}
\definecolor{RoyalGreen}{rgb}{0.0236,0.6179,0.0894}
\definecolor{RoyalRed}{rgb}{0.6179,0.0236,0.0894}
\definecolor{LightBlue}{rgb}{0.8544,0.9511,1.0000}
\definecolor{Black}{rgb}{0.0,0.0,0.0}

\definecolor{linkColor}{rgb}{0.0,0.0,0.554}
\definecolor{citeColor}{rgb}{0.0,0.0,0.554}
\definecolor{fileColor}{rgb}{0.0,0.0,0.554}
\definecolor{urlColor}{rgb}{0.0,0.0,0.554}
\definecolor{promptColor}{rgb}{0.0,0.0,0.589}
\definecolor{brkpromptColor}{rgb}{0.589,0.0,0.0}
\definecolor{gapinputColor}{rgb}{0.589,0.0,0.0}
\definecolor{gapoutputColor}{rgb}{0.0,0.0,0.0}

%%  for a long time these were red and blue by default,
%%  now black, but keep variables to overwrite
\definecolor{FuncColor}{rgb}{0.0,0.0,0.0}
%% strange name because of pdflatex bug:
\definecolor{Chapter }{rgb}{0.0,0.0,0.0}
\definecolor{DarkOlive}{rgb}{0.1047,0.2412,0.0064}

%% command for ColorPrompt style examples
\newcommand{\gapprompt}[1]{\color{promptColor}{\bfseries #1}}
\newcommand{\gapbrkprompt}[1]{\color{brkpromptColor}{\bfseries #1}}
\newcommand{\gapinput}[1]{\color{gapinputColor}{#1}}

%%% For source code listings
\usepackage{listings}[2013/08/05]
\input{pfad.tex}
%\lstloadlanguages{GAP} %% not needed in new listings version

%%% For algorithm styles
\usepackage[linesnumbered,ruled]{algorithm2e}

%%% Math theorem styles
\usepackage{amsthm}

\newtheorem{thm}{Theorem}[subsection]
\newtheorem{lemma}[thm]{Lemma}
\newtheorem{corollary}[thm]{Corollary}
\theoremstyle{definition}
\newtheorem{definition}[thm]{Definition}
\newtheorem{remark}[thm]{Remark}
\newtheorem{example}[thm]{Example}


%%% For Math
\usepackage{amsmath}
\usepackage{amsfonts}
\usepackage{amsbsy}
%%\usepackage{amsthm} (doubled)
\usepackage{amssymb}
\usepackage{mathtools}
\usepackage{commath}
\usepackage[sc,osf]{mathpazo}

%%% For arrows and categories
%\usepackage[all]{xy} %%not used anymore
\usepackage{tikz-cd}

%%% For calculations and loops inside tikz and latex
\usepackage{calc}
\usepackage{pgffor}

\newcounter{modresult}
\newcommand*{\themodulo}[2]{%
\setcounter{modresult}{%
#1-(#1/#2)*#2%
}%
#1 mod #2 = \themodresult\par
}

%%% For matrices
\let\ampersand =&

%%% Math operators bold
%\newcommand{\Category}{Category}
% just use /textup{#1} inside math environment instead of redefining every math operator.

%%% tikz
\usetikzlibrary{positioning}

%%% For dotted box around diagrams
\tikzcdset{
    boxedcd/.style={
        every matrix/.append style={
            draw=black,
            dotted,
            rounded corners,
            #1
        },
    },
}

%%% For dotted arrows in math and in text
%% dottedrightarrow
\makeatletter
\newbox\dottedrightarrow@box
\setbox\dottedrightarrow@box\hbox
  {%
    \begin{tikzpicture}
      \draw[dotted,->] (0,0) -- (1.5em,0);
    \end{tikzpicture}%
  }
\newcommand*\dottedrightarrow
  {\relax\ifmmode\expandafter\dottedrightarrow@m\else\expandafter\dottedrightarrow@t\fi}
\newcommand*\dottedrightarrow@t[1][1.5em]
  {\resizebox{#1}{!}{\raisebox{.5ex}{\usebox\dottedrightarrow@box}}}
\newcommand*\dottedrightarrow@m[1][]
  {%
    \if\relax\detokenize{#1}\relax
      \mathchoice% values are trial and error based\ldots
        {\dottedrightarrow@t}
        {\dottedrightarrow@t}
        {\dottedrightarrow@t[1.1em]}
        {\dottedrightarrow@t[0.9em]}%
    \else
      \dottedrightarrow@t[#1]%
    \fi
  }
\makeatother
\let\olddottedrightarrow\dottedrightarrow
\renewcommand{\dottedrightarrow}{\raisebox{-.2em}{\olddottedrightarrow}}

%% dottedleftarrow
\makeatletter
\newbox\dottedleftarrow@box
\setbox\dottedleftarrow@box\hbox
  {%
    \begin{tikzpicture}
      \draw[dotted,<-] (0,0) -- (1.5em,0);
    \end{tikzpicture}%
  }
\newcommand*\dottedleftarrow
  {\relax\ifmmode\expandafter\dottedleftarrow@m\else\expandafter\dottedleftarrow@t\fi}
\newcommand*\dottedleftarrow@t[1][1.5em]
  {\resizebox{#1}{!}{\raisebox{.5ex}{\usebox\dottedleftarrow@box}}}
\newcommand*\dottedleftarrow@m[1][]
  {%
    \if\relax\detokenize{#1}\relax
      \mathchoice% values are trial and error based\ldots
        {\dottedleftarrow@t}
        {\dottedleftarrow@t}
        {\dottedleftarrow@t[1.1em]}
        {\dottedleftarrow@t[0.9em]}%
    \else
      \dottedleftarrow@t[#1]%
    \fi
  }
\makeatother
\let\olddottedleftarrow\dottedleftarrow
\renewcommand{\dottedleftarrow}{\raisebox{-.2em}{\olddottedleftarrow}}

%%% For some big dots (they still don't look very big)
\makeatletter
\newcommand*{\bigcdot}{}% Check if undefined
\DeclareRobustCommand*{\bigcdot}{%
  \mathbin{\mathpalette\bigcdot@{}}%
}
\newcommand*{\bigcdot@scalefactor}{.5}
\newcommand*{\bigcdot@widthfactor}{1.15}
\newcommand*{\bigcdot@}[2]{%
  % #1: math style
  % #2: unused
  \sbox0{$#1\vcenter{}$}% math axis
  \sbox2{$#1\cdot\m@th$}%
  \hbox to \bigcdot@widthfactor\wd2{%
    \hfil
    \raise\ht0\hbox{%
      \scalebox{\bigcdot@scalefactor}{%
        \lower\ht0\hbox{$#1\bullet\m@th$}%
      }%
    }%
    \hfil
  }%
}
\makeatother

\title{Representations of a concrete category as objects in the functor category}

\author{Tibor Gr{\"u}n}

\begin{document}
	\pagenumbering{gobble}

	\maketitle

	\newpage

	\tableofcontents

	\newpage

	\pagenumbering{arabic}

%% mainfile: ../main.tex

\section{Introduction}

Die Aufgabe der vorliegenden Arbeit besteht darin, das GAP-Paket "catreps" von Peter Webb mit CAP zu re-organisieren. Dabei werden die Algorithmen, welche in catreps direkt implementiert sind, soweit wie möglich durch vorhandene Methoden von CAP ersetzt.
Insbesondere wird das CAP-Paket FunctorCategories Anwendung finden, weil ich zeigen werde, dass catreps, also die Kategorie der Darstellungen einer konkreten endlichen Kategorie, nichts anderes ist als eine Unterkategorie von FunctorCategories, also der Kategorie aller Funktoren zwischen Kategorien. 
Da catreps selbst bereits eine Verallgemeinerung der Darstellung endlicher Gruppen ist (eine Gruppe ist nichts anderes als eine Kategorie mit einem Objekt, in dem jeder Morphismus ein Isomorphismus ist), stellt FunctorCategories wohl den allgemeinsten Rahmen dar, den man sich vorstellen kann.

In this thesis I will define what a category is, then I go further in the doctrine of enriched categories, especially monoidal categories.
The morphisms in the category of categories are the functors between categories. I will treat the functor category where the functors themselves are
objects and natural transformations the morphisms between them. I will show that any representation of a category (and thus any representation
of a group) is a functor, so the category of representations (of a category) is a subcategory of the functor category.
I will show how the monoidal structure of the category of representations arises from the counit and the comultiplication on the Bialgebroid structure
on the category.

\noindent Throughout the thesis I will give proofs of existence by providing an algorithm that computes the object that exists. I will be using CAP, the gap package
developed by Sebastian Gutsche et al. Another purpose of this thesis is the translation of the work of Peter Webb, who used gap directly, into our CAP
framework. This includes his decomposition algorithm for a representation. As the category of representations is just a subcategory of the functor category,
most of the work will be done inside the package FunctorCategories by Prof. Mohamed Barakat. In this thesis I will also write the documentation for the
package FunctorCategories.


\section{Introduction}

\[
\mathbf{Quiv}\rightarrow^{CatClosure}\leftarrow_{U}\mathbf{Cats}
\rightarrow^{k-Algebroid}\leftarrow_{U}\mathbf{k-Cats}
\rightarrow^{AdditiveClosure}\leftarrow_{U}\mathbf{k-Cats^{\oplus}}
\]

% mainfile: ../main.tex

\section{A short overview of the tools used}

GAP, QPA / QPA2, Catreps, CAP, homalg\_project



\section{Introduction to quivers and category theory}
% mainfile: ../main.tex

This section serves two purposes: On the one hand, it is an introduction to quivers and category theory. On the other hand it introduces
concrete categories which we want to represent, and all the additional constructions that are needed to that goal.

\subsection{Quivers}
In this section, we first want to define the category \textbf{Quiv} and how it is the prototype for the category \textbf{Cats}.
In order to describe the category \textbf{Quiv} of quivers, we first have to define what a category is and for this we need
the definition of a quiver. Lateron we will revisit this definition as we can define quivers as the objects in the quiver category \textbf{Quiv}.

\begin{definition}{(Quiver)}\label{def:quiver}\\
A \ul{directed graph} or \ul{quiver} $q$ consists of a class of \ul{objects} (or \ul{vertices}) $q_{0} = \textup{Obj}\,q$ and
a class of \ul{morphisms} (or \ul{arrows}) $q_{1} = \textup{Mor}\,q$ together with two defining maps
\[
\begin{tikzcd}[column sep=small]
{s,t\colon q_{1}} \arrow[rr, shift left = 0.7ex] \arrow[rr, shift right = 0.7ex] & & q_{0}
\end{tikzcd}
\]
$s$ called \ul{source} and $t$ called \ul{target}.
\end{definition}

In the next definition we are giving a new characterization for $q_{1}$ by looking at all arrows between two fixed objects.

\begin{definition}{(Hom-set of a (locally) small quiver)}\label{def:hom-set}
\renewcommand{\labelenumi}{(\theenumi)}
\begin{enumerate}
\item Given two objects $M, N \in q_{0}$ we write $\textup{Hom}_{q}(M,N)$ or $q(M,N)$ for the fiber
$(s,t)^{-1} (\{(M,N)\})$ of the product map 
\begin{tikzcd}[column sep=small]
(s, t) : q_{1} \arrow[rr] &  & q_{0} \times q_{0} 
\end{tikzcd} over the pair $(M,N) \in q_{0} \times q_{0}$.
This is the class of all morphisms with source $= M$ and target $= N$.
We indicate this by writing
\begin{tikzcd}[column sep=small]
\varphi : M \arrow[rr] &  & N
\end{tikzcd} or 
\begin{tikzcd}[column sep=small]
M \arrow[rr,"\varphi"] &  & N.
\end{tikzcd} Hence $q_{1}$ is the disjoint union $\bigcup\limits^{\bigcdot}_{M,N \in q_{0}} \textup{Hom}_{q}(M,N) = q_{1}$.
As usual we define $\textup{End}_{q}(M):= \textup{Hom}_{q}(M,M)$.
\item If the class $\textup{Hom}_{q}(M,N)$ is a \ul{set} for all pairs $(M,N)$ then we call the quiver \ul{locally small}.
We therefore talk about \ul{Hom-sets}.
If additionally, $q_{0}$ is a set, then the quiver is called \ul{small}.
\item A quiver with a finite set of objects and a finite set of morphisms is called a \ul{finite} quiver.
\end{enumerate}
\end{definition}

When we don't assume the category to be locally small, but still talk about its hom-sets, we mean the class of morphisms,
if we don't explicitly use the fact that it's a set of morphisms.

\begin{example}\label{q(2)}{(Quiver with 2 objects and 3 morphisms)}\\
\[
\begin{tikzcd}
1 \arrow["a"', loop, distance=2em, in=305, out=235] \arrow[rr, "b"] &  & 2 \arrow["c"', loop, distance=2em, in=305, out=235]
\end{tikzcd}
\]
The objects of this quiver $q$ are $q_{0} = \{1, 2\}$, and the morphisms are $q_{1} = \{a, b, c\}$ with\\
$s (a) = 1 = t (a)$, $s (c) = 2 = t (c)$ and $s (b) = 1, t (b) = 2$.\\
\noindent Thus $\textup{End}_{q}(1) = \{a\}, \textup{End}_{q}(2) = \{c\}$ and $\textup{Hom}_{q}(1,2) = \{b\}$ whereas
$\textup{Hom}_{q}(2,1)=\emptyset$.\\

\noindent In \texttt{QPA} this quiver is encoded as \texttt{q(2)[a:1->1,b:1->2,c:2->2]} where the first \texttt{(2)} in parentheses stands for the total
number of objects.
\end{example}

\begin{definition}{(Composable arrows; path in a quiver)}\label{def:path}\\
Since we already have the source and target maps, we say two arrows $a, b \in q_{1}$ are \ul{composable} if $t(a) = s(b)$ or
$t(b) = s(a)$. In this case we can write a sequence of composable arrows $p = a_{1}a_{2}\cdots a_{n}$ where $t(a_{i}) = s(a_{i+1})$ for $i=1,\dots,n-1$.
We call this sequence a \ul{path} from $s(a_{1})$ to $t(a_{n})$ and the integer $n \in \mathbb{Z}_{\geq0}$ the \ul{length} $l(p)$ of the path $p$.
Although it's not an arrow, we can define the source and target of a path $p = a_{1}\cdots a_{n}$ as $s(p) := s(a_{1})$ and $t(p) := t(a_{n})$.
A path $p = a_{1}\cdots a_{n}$ with $s(a_{1}) = t(a_{n})$, i.e. $s(p) = t(p)$, is called \ul{cyclic}.\\
For an endomorphism $a \in \textup{End}_{q}(M)$ we write $a^{n}$ for $aa \cdots a$ ($n$ times). In the case of $n=0$ an \ul{empty path}
whose source and target are the vertex $i \in q_{0}$ is called the \ul{trivial path at $i$} and is denoted $e_{i}$. Note that the composition of paths
$e_{i}e_{i}$ has length zero starting at $i$ therefore $e_{i}^{2}=e_{i}$.
\end{definition}

\begin{lemma} Let Q be a quiver. If there is a path of length at least $\abs{Q_{0}}$, then there are cyclic paths,
and thus infinitely many paths.\cite{[leit4]}
\begin{proof}
Assume that there exists a path of length greater or equal to $\abs{Q_{0}}$. Then there exists a path of length $n = \abs{Q_{0}}$, say
$\alpha_{1}\cdots \alpha_{n}$. Consider the vertices $x_{i}=s(\alpha_{i})$ for $1 \leq i \leq n$ and $x_{n+1}=t(\alpha_{n})$. Then these
are $n+1$ vertices, thus there has to exist $i<j$ with $x_{i}=x_{j}$. Let $\omega=\alpha_{i}\cdots \alpha_{j-1}$, this is a path with source and target
$x_{i}=x_{j}$, thus a cyclic path. But then $\omega^{m}$ is a path for any natural number $m$. The path $\omega$ has length $j-i\geq1$, thus
$\omega^{m}$ has length $m(j-i)$. This shows that these paths are pairwise different.
\end{proof}
\end{lemma}

\begin{example}{(A quiver with no cycles)}\\
\[
\begin{tikzcd}
2 \arrow[rrrr, "\psi"] \arrow[rrrrddd, "\psi\rho", pos=0.3] &  &  &  &
3 \arrow[ddd, "\rho"] \\
 &  &  &  & \\
 &  &  &  & \\
1 \arrow[uuu, "\varphi"] \arrow[rrrruuu, "\varphi\psi", pos=0.3] \arrow[rrrr, "\varphi\psi\rho" '] &  &  &  & 4
\end{tikzcd}
\]
The longest path $1\rightarrow2\rightarrow3\rightarrow4$ has length 3. If after the object $4$ another arrow would go to either $1,2,3$ or $4$ itself,
we would have a cyclic path and thus infinitely many paths.
\end{example}

\subsection{Categories}

\begin{definition}{(Category)}\label{def:category}\\
\noindent A \ul{category} $\mathcal{C}$ is a quiver with two further maps:
\begin{enumerate}
\renewcommand{\labelenumi}{(id)}
\item The \ul{identity map} $1_{( )}$ mapping every object $X \in\mathcal{C}_{0}$ to its \ul{identity morphism} $1_{X}$:
\[
\begin{tikzcd}[column sep=small]
\mathcal{C}_{0} \arrow[rr,"1"] &  & \mathcal{C}_{1}
\end{tikzcd}
\]
\renewcommand{\labelenumi}{($\mu$)}
\item And for any two \ul{composable} morphisms $\varphi$ and $\psi \in \mathcal{C}_{1}$, i.e. with $t(\varphi) = s(\psi)$, the
\ul{composition map} $\mu$, which maps $\varphi, \psi \in \mathcal{C}_{1}\times\mathcal{C}_{1}$ to $\mu(\varphi,\psi) \in \mathcal{C}_{1}$ which
we also write as $\varphi\psi$. 
\[
\begin{tikzcd}[column sep=small]
\mathcal{C}_{1} \times \mathcal{C}_{1} \arrow[rr,"\mu"] &  & \mathcal{C}_{1}
\end{tikzcd}
\]
\end{enumerate}
\noindent The defining properties for $1$ and $\mu$ are:
\renewcommand{\labelenumi}{(\theenumi)}
\begin{enumerate}
\item $s(1_{M}) = M = t(1_{M})$, i.e.\\
$1_{M} \in \textup{End}_{\mathcal{C}} \forall M \in \mathcal{C}$.

\item $s(\varphi\psi) = s(\varphi)$ and\\
$t(\varphi\psi) = t(\psi)$\\
for all composable morphisms $\varphi, \psi \in \mathcal{C}$.
\[
\begin{tikzcd}[column sep=small]
\mu : \textup{Hom}_{\mathcal{C}}(M,L) \times \textup{Hom}_{\mathcal{C}}(L,N) \arrow[rr] &  & \textup{Hom}_{\mathcal{C}}(M,N)
\end{tikzcd}
\]
\item \label{associativity_of_composition} \begin{minipage}{.55\textwidth} $(\varphi\psi)\rho = \varphi(\psi\rho)$ \hfill{} [associativity of composition]\end{minipage}
\begin{minipage}{.45\textwidth}\phantom{}\end{minipage}
\item \label{unit_property} \begin{minipage}{.55\textwidth} $1_{s(\varphi)}\varphi = \varphi = \varphi1_{t(\varphi)}$ \hfill{} [unit property]\end{minipage}
\begin{minipage}{.45\textwidth}\phantom{}\end{minipage}\\
The identity is a left and right unit of the composition.
\end{enumerate}
\end{definition}

So with categories you always answer the four questions
\begin{itemize}
\item What are the objects? (which includes the question What are the identity morphisms?)
\item What are the morphisms?
\item How do you compose morphisms?
\item Why is the composition associative?
\end{itemize}

\subsection{Functors}

Categories are themselves objects in the category of categories, which leads to a question: What is a morphism between categories?

\begin{definition}{(Functor)}\label{def:functor}\\
\noindent A \ul{functor} $F : \mathcal{C} \rightarrow \mathcal{D}$, between categories $\mathcal{C}$ and $\mathcal{D}$, consists of the
following data:

\begin{itemize}
\item An object $Fc\in\mathcal{D}_{0}$, for each object $c \in \mathcal{C}_{0}$.
\item A morphism $Ff : Fc \rightarrow Fc' \in \mathcal{D}_{1}$, for each morphism $f : c \rightarrow c' \in \mathcal{C}_{1}$, so that the
source and target of $Ff$ are, respectively, equal to $F$ applied to the source or target of $f$, in other words,
$s(Ff) = Fs(f)$ and $t(Ff) = Ft(f)$.
\end{itemize}

\noindent The assignments are required to satisfy the following two \ul{functoriality axioms}:
\begin{itemize}\label{functoriality}
\item For any composable pair $f, g \in \mathcal{C}_{1}, Fg \cdot Ff = F(g \cdot f)$.
\item For each object $c \in \mathcal{C}_{0}, F(1_{c}) = 1_{Fc}$.
\end{itemize}

Put concisely, a functor consists of a mapping on objects and a mapping on morphisms that preserves all of the structure of a category,
namely domains and codomains, composition, and identities.
\end{definition}

So with functors you always answer the four questions
\begin{itemize}
\item How does it work on objects?
\item How does it work on morphisms?
\item Why does it respect composition?
\item Why does it respect identity morphisms?
\end{itemize}

We have already seen an example for a functor in definition \ref{def:hom-set} where we defined the hom-set $\textup{Hom}(M,N)$ between two
objects $M$ and $N$. There are two ways to leave blank one of the objects and thus define the 

\begin{example}{(partial Hom-functor)}\label{def:Hom_functor}
Let $\mathcal{C}$ be a category and $P \in \mathcal{C}_{0}$ any object. The \ul{Hom-functor}, also called \ul{partial Hom-functor},
\begin{enumerate}
\item $\textup{Hom}(P,-)$ is a functor from $\mathcal{C}$ to $\mathcal{C}_{1}$ where objects in $\mathcal{C}_{1}$ are the hom-sets 
$\textup{Hom}(P,N)$, and morphisms are maps from one hom-set to another.
$\textup{Hom}(P,-)$ works on objects by mapping the object $N \in \mathcal{C}_{0}$ to
the hom-set $\textup{Hom}(P,N) \in \mathcal{C}_{1}$.
$\textup{Hom}(P,-)$ works on morphisms by mapping the morphism $(f : M \rightarrow N ) \in \mathcal{C}_{1}$ to the transformation
$\textup{Hom}(P,f) : \textup{Hom}(P,M) \rightarrow \textup{Hom}(P,N); \varphi \mapsto \varphi f$, so for every morphism
$\varphi \in \textup{Hom}(P,M)$, you post-compose $f \in \textup{Hom}(M,N)$ to get a new morphism $\varphi f \in \textup{Hom}(P,N)$.

\item $\textup{Hom}(-,P)$ is a functor from $\mathcal{C}$ to $\mathcal{C}_{1}$ where objects in $\mathcal{C}_{1}$ are the hom-sets 
$\textup{Hom}(N,P)$, and morphisms are maps from one hom-set to another.
$\textup{Hom}(-,P)$ works on objects by mapping the object $N \in \mathcal{C}_{0}$ to
the hom-set $\textup{Hom}(N,P) \in \mathcal{C}_{1}$.
$\textup{Hom}(-,P)$ works on morphisms by mapping the morphism $(f : M \rightarrow N ) \in \mathcal{C}_{1}$ to the transformation
$\textup{Hom}(f,P) : \textup{Hom}(N,P) \rightarrow \textup{Hom}(M,P); \varphi \mapsto f\varphi$, so for every morphism
$\varphi \in \textup{Hom}(N,P)$, you pre-compose $f \in \textup{Hom}(M,N)$ to get a new morphism $f\varphi \in \textup{Hom}(M,P)$.
\end{enumerate}

The important difference between these two functors was how they worked on morphisms. If in both cases we take a morphism
$f : M \rightarrow N$ as given, then we have to arrange the source and target for $\textup{Hom}(P,f)$ and $\textup{Hom}(f,P)$
according to the post-composition and pre-composition. Thus if we wanted $\textup{Hom}(f,P)$ to be defined by pre-composition
$\varphi \mapsto f\varphi$, then we were forced to invert $M$ and $N$ as source and target to get 
$\textup{Hom}(f,P): \textup{Hom}(N,P) \rightarrow \textup{Hom}(M,P)$. 
This process of inverting source and target is caught in the following definition.
\end{example}

\begin{definition}{(covariant / contravariant functor)}\endnote{(Def 1.3.5. in \cite{[context]}, p. 17 (35/258))}\\
The way we defined a functor in definition \ref{def:functor} was in the \ul{covariant} way.\\
A \ul{contravariant} functor $F : \mathcal{C} \rightarrow \mathcal{D}$ works on objects the same way as a covariant one, i.e.
an object $Fc \in \mathcal{D}_{0}$ for each object $c \in \mathcal{C}_{0}$. For morphisms on the other hand, we have
a morphism $F f : Fc' \rightarrow Fc \in \mathcal{D}_{1}$ for each morphism $f : c \rightarrow c' \in \mathcal{C}_{1}$, so that
$s(F f) = F t(f)$ and $t(F f) = F s(f)$.
The \ul{functoriality axioms} are also inverted for a contravariant functor:
For any composable pair, $f, g \in \mathcal{C}_{1}$, $F f \cdot F g = F(g \cdot f)$.
For the identity morphisms, it is again the same as in the covariant case:
For each object $c \in \mathcal{C}_{0}$, $F(1_{c}) = 1_{Fc}$.
\end{definition}

In the following definition, we define different subclasses of functors. These adjectives often come in opposite pairs, so that you may be
tempted to think, duality lets you just swap all the adjectives for the opposite ones, but be careful there. E.g. when 
$\textup{Hom}(P,-)$ is a \ul{covariant}, \ul{left-exact} functor, the opposite $\textup{Hom}(-,P)$ is a \ul{contravariant}, but still \ul{left-exact} functor.
But their respective \ul{right-exactedness} is equivalent to opposite concepts concerning \ul{projective} and \ul{injective} objects.

\begin{definition}{(Exact functor)}\label{def:exact_functor}\endnote{(Def 4.5.9. in \cite{[context]}, p. 139 (157/258))}
A functor 
\end{definition}
left-/ right- exact

full

faithful functor

\subsection{Natural transformations}

With fixed categories $\mathcal{C}$ and $\mathcal{D}$ we can consider functors $F, G \in \textup{Hom}(\mathcal{C},\mathcal{D})$ themselves
as objects in the category $\textup{Hom}(\mathcal{C},\mathcal{D})$ of functors between $\mathcal{C}$ and $\mathcal{D}$. In this \ul{functor category},
the morphisms between two functors are called \ul{natural transformations}.

\begin{definition}{(Natural transformations)}\label{def:natural_transformation}\\
\noindent Given categories $\mathcal{C}$ and $\mathcal{D}$ and functors $F : \mathcal{C} \rightarrow \mathcal{D}$ and
$G : \mathcal{C} \rightarrow \mathcal{D}$, a \ul{natural transformation} $\alpha : F \Rightarrow G$ consists of:
\begin{itemize}
\item an arrow $\alpha_{c} : Fc \rightarrow Gc \in \mathcal{D}_{1}$ for each object $c \in \mathcal{C}_{0}$, the collection of which
define the \ul{components} of the natural transformation, so that, for any morphism $f : c \rightarrow c' \in \mathcal{C}_{1}$, the following
square of morphisms in $\mathcal{D}$
\[\begin{tikzcd}
Fc \arrow[rr, "\alpha_{c}"] \arrow[dd, "Ff"] &  & Gc \arrow[dd, "Gf"] \\
                                             &  &                     \\
Fc' \arrow[rr, "\alpha_{c'}"]                &  & Gc'                
\end{tikzcd}\]

\ul{commutes}, i.e., has a a common composite $Fc \rightarrow Gc' \in \mathcal{D}_{1}$.
\end{itemize}

\end{definition}
























% mainfile: ../main.tex

\section{Finite concrete categories}

\begin{definition}{(Finite and concrete categories)}
\renewcommand{\labelenumi}{(\theenumi)}
\begin{enumerate}
\item A \ul{finite} category is a category with a finite set of objects and a finite set of morphisms.
\item A \ul{concrete} category is a category whose objects have \ul{underlying sets} (or are themselves sets) and whose morphisms are
functions between these underlying sets. Otherwise it's called an \ul{abstract} category.
\end{enumerate}
\end{definition}

\noindent Clearly every finite concrete category is a small category.

\begin{remark}[Implementation]
Using the implementation of \texttt{FinSets} in %\cite{[BMZ20]} 
we implement a finite concrete category as a subcategory of FinSets.
The finite concrete category is generated by its set of generating morphisms $\{g_{1},\dots,g_{r}\}$.
\end{remark}

When our goal is representation of finite concrete categories, i.e. functors $\mathcal{C} \rightarrow k\text{-Mat}$, why are we not
defining the functor $\textup{Hom}(\mathcal{C},k\text{-Mat})$ but instead first define the $k$-Algebroid $kq$ of the\\
$\texttt{RightQuiverFromConcreteCategory}(\mathcal{C})$ and then the functor $\textup{Hom}(kq,k\textup{-Mat})$?

\begin{definition}
A \ul{congruence relation} $\sim$ on a category $\mathcal{C}$ is an equivalence relation on the set of morphisms $\mathcal{C}_{1}$ such that
for pre-composable $c$ and post-composable $d \in \mathcal{C}_{1}$:
\[ a \sim b => ca \sim cb \land ad \sim bd \]
The equivalence classes $[f]_{\sim}$ again form a set of morphisms $\mathcal{C}_{1}/\sim$.
\end{definition}

We can calculate the \texttt{RightQuiverFromConcreteCategory}. We can calculate the \texttt{CategoryClosure} of that quiver indirectly by first
calculating the \texttt{Algebroid}$( k, \mathcal{C} )$ and then the \texttt{UnderlyingCategory}.


As a subcategory of \textbf{FinSets}, our finite concrete category $\mathcal{C}$ does not have a pre-additive structure on it, i.e. for
two objects $M,N, \textup{Hom}_{\mathcal{C}}(M,N)$ does not have the structure of an abelian group.

The category \textbf{FinSets} as a subcategory of \textbf{Sets} does not have a zero object, since the empty set $\emptyset$ is the
unique initial object and every singleton is a terminal object which is different from the initial object.

\begin{example}
Forgetful functor / Category closure / k-Algebroid
\end{example}

When we want to calculate representations of our finite concrete categories, we make the Hom functor Hom( ccat, kMat ).
But functors from the concrete category directly are not useful when we know nothing about the relations of morphisms
in our category.
Instead we go an indirect route, first calculating the underlying quiver and from this the k-algebroid, i.e. the path algebra
with the endomorphism relations and such.

\begin{algorithm}\capstart
    \caption{\texttt{RightQuiverFromConcreteCategory}}\label{algo:RightQuiverFromConcreteCategory}
	\SetKwInput{Input}{Input~}
	\SetKwInput{Output}{Output~}
	\Input{~a finite concrete category $C$ with $n$ objects}
	\Output{~the right quiver $q(n)$}
	\BlankLine
	let $Obj$ be the set of objects of $C$\;
	let $n := Length(Obj)$\;
	let $gMor$ be the set of generating morphisms of $C$\;
	let $A$ be the empty set and let $i := 1$\;
	\ForEach{morphism $mor$ in $gMor$}{
	    let $A_{i,1}$ be the position of $Source( mor )$ in $Obj$\;
	    let $A_{i,2}$ be the position of $Range( mor )$ in $Obj$\;
	    let $i := i+1$\;
	}
	\BlankLine
	let $q$ be the right quiver with vertices $\{1,\dots,n\}$ and arrows $A$.
	\BlankLine
	\Return q\;
\end{algorithm}


% mainfile: ../main.tex

\subsection{Additional structure on the Hom-set of a category}

....

\begin{example}{(Group as a category)}\\
\noindent A group $\mathbf{G}$ defines a category $\mathcal{B}\mathbf{G}$ with a single object $\ast$ and the morphisms being the group elements. The group elements are its morphisms, which are
all automorphisms (i.e. bijective endomorphisms) of the single object. Composition of morphisms is defined by the binary group operation.
The identity element $e \in G$ acts as the identity morphism for the unique object in this category. The hom-set of that category is itself
a group.
$\Bbbk G$ group algebra, $\Bbbk \mathcal{C}$ category algebra
\end{example}

Our goal is to represent finite concrete categories, for this we need the source and target categories of our functors, which the
representations are.
As subcategories of $\textup{FinSets}$, our finite concrete categories only have definitions for their objects and their
morphisms, methods to check when two morphisms are congruent or equivalent, but not much else.


\begin{definition}{(Idempotent)}

\end{definition}

\begin{theorem}
Let $\mathcal{C}$ be a finite additive category.
\end{theorem}











A competing theory to category theory is that of quivers and path algebras. We already used their terminology in
\ref{def:path}, \ref{la:cyclic_paths} and \ref{def:path_algebra}, for instance when talking about the trivial path,
which in the language of category theory is nothing but the identity morphism, composition of arrows to a path is nothing but
composition of morphisms (if you make the path explicit by writing a new arrow for every path).

So what we called a path algebra in \ref{def:path_algebra} is a different data structure for a category. 
For one, the path algebra is an algebra, i.e. a vector space with additional structure, and thus a single set, comparable to the
class of morphisms $\mathcal{C}_{1}$ of a category $\mathcal{C}$.
But as it is an algebra, it not only contains the generating morphisms of the category, but also $\Bbbk$-linear combinations of
morphisms and paths. This is what our concrete categories lack, and what additional structure we have to give them in order
to represent them by matrices.

In practise, there is already developed software for \ul{q}uivers and \ul{p}ath \ul{a}lgebras, namely the \textsc{Gap} package
\textsc{QPA$2$}\endnote{(see \cite{[QPA2]})}.
What we are actually doing to represent finite concrete categories, is going from $\mathcal{C} \in \mathbf{Cats}$ to $q \in \mathbf{Quiv}$,
in theory by \ul{forgetting} (see \ref{ex:forgetful_functor}) the category concepts of identity morphism and composition, in practise by calculating the
underlying quiver $q$; and then for a commutative ring $\Bbbk$, constructing the path algebra $\Bbbk q$. In this step the path algebra
is infinite-dimensional, since there are infinitely many paths according to lemma \ref{la:cyclic_paths}, and \textsc{QPA$2$}'s function
\texttt{BasisPathsBetweenVertices} only works for finite-dimensional path algebras. Thus in a next step we have to provide
additional data in the form of generators of \ul{ideals of the path algebra}, by which we can divide and build the quotient path algebra,
which is then finite-dimensional. This is the purpose of \texttt{RelationsOfEndomorphisms}.

\subsection{Ideals of the path algebra}

\begin{definition}{(Ideal of a path algebra)}

\end{definition}

\begin{lemma}{($\Bbbk Q$ / rel is finite-dimensional)}

Let $\Bbbk Q$ be the path algebra from a small quiver $Q$ with $Q_{0}$ finite and for every $x \in Q_{0}$, $\mathrm{End}_{Q}(x) = \mathrm{Aut}_{Q}(x)$
is a finite group generated by one morphism $a_{x}$, with $n = n(x) \in \mathbb{N}$ such that $a_{x}^{n} = e_{x}$.



... defines an ideal $I$ of $\Bbbk Q$

... then the quotient algebra $\Bbbk Q / I$ is finite-dimensional.
\end{lemma}

Once we have a finite-dimensional path algebra $\Bbbk q$, we let \textsc{QPA$2$} calculate generators of the non-endomorphism relations,
and when we have a complete set of relations, that will be our definitive quotient quiver algebra $\Bbbk q$, which we then take it back into the category
theoretical context by constructing the $\Bbbk$-\textbf{Algebroid} $\mathcal{A}$ from the path algebra $\Bbbk q$.

An algebra with idempotents defines an algebroid.

The source category for our representation is then the $\Bbbk$-\textbf{Algebroid} $\mathcal{A}$ and not anymore our finite concrete
category $\mathcal{C}$, but it behaves in the same way regarding composition of morphisms and which morphisms are congruent.

The target category of our category representations will be $\Bbbk$-\textbf{Mat} which we will describe in the next section,
especially all the nice properties $\Bbbk$-\textbf{Mat} has, and how they get carried over to our functor category with $\Bbbk$-\textbf{Mat} as
target.\endnote{
In \cite{[Ab-Cat]}, Posur used the equivalence between categories $\textup{mat}_{\Bbbk} \cong \textup{vec}^{\text{fd}}_{\Bbbk}$,
as described in \cite{[context]}, \textsc{Example} 1.5.6 on page 30 (48/258), to justify that $\Bbbk$-\textbf{Mat} is a good
\textbf{computational model} to
\blockquote{transform otherwise inaccessible mathematical objects into computationally easily graspable entities},
which is what we are doing with \textbf{CatReps}.
}

With source and target categories defined, the category where our category representations lie in is \textbf{CatReps} for which we
show that it's a subcategory of the \textbf{Functor Category}. And even more in the next section.
\[
\textup{Hom}(\Bbbk\mathcal{C}, \kmat)
\]

\subsection{Generating morphisms of a category and the underlying quiver}

$\textup{gmorphisms} := \{g_{1},\dots,g_{r}\} \rightarrow$ concrete category with set of generating morphisms $\textup{gmorphisms}$.

This is the $\textup{Free}$ functor from $\mathbf{Quiv}$ to $\mathbf{Cat}$, taking a quiver and adding the missing structure of
identity morphisms and composition of arrows to that category. The result is a category.

The $\textup{forgetful}$ functor from $\mathbf{Cat}$ to $\mathbf{Quiv}$ is going the other way around and leaves all
morphisms that we now have in the category, but forgets their relations, what was identity, what was composition.

Given a field $\Bbbk$, we have the path algebra $\Bbbk q$ with all the arrows as a basis.

Given relations on endomorphisms and on the other morphisms, we make the quotient path algebra.

This is already a category, and now it has more structure.

\subsection{Ab-categories}


\begin{definition}{(semisimple)}
semisimple category
\end{definition}

\begin{theorem}{(Wedderburn)}
$\kmat$ of a semisimple ring k is a semisimple category.
\end{theorem}
TODO








\begin{definition}{($R$-linear category)}
Let $R$ be a commutative ring. An \ul{$R$-linear category} $\mathcal{A}$ is a category where every hom-set is an
$R$-module, and where for $x,y,z \in \mathcal{A}_{0}$ composition of morphisms
\[
\mu : \mathrm{Hom}_{\mathcal{A}}(x,y) \times \mathrm{Hom}_{\mathcal{A}}(y,z) \rightarrow \mathrm{Hom}_{\mathcal{A}}(x,z)
\]
is $R$-linear.

Note that this does imply that $\mathcal{A}$ is a pre-additive category, but it need not be additive.
\end{definition}

\begin{definition}{($R$-linear functor)}
Let $R$ be a commutative ring. A \ul{functor of $R$-linear categories} or an \ul{$R$-linear functor} is a functor
$F : \mathcal{A} \rightarrow \mathcal{B} $ where for all objects $x, y \in \mathcal{A}_{0}$, the map
$F : \mathrm{Hom}_{\mathcal{A}}(x,y) \rightarrow \mathrm{Hom}_{\mathcal{B}}(F(x), F(y))$ is a homomorphism
of $R$-modules.
\end{definition}

\begin{example}
Consider the following category $\mathcal{A}$ with four objects $\mathcal{A}_{0} = \{1,2,3,4\}$ and
identity morphisms $e_{1}, e_{2}, e_{3}, e_{4}$:
\[
\begin{tikzcd}
1 \arrow["e_{1}"', loop, distance=2em, in=215, out=145] \arrow[dd, "v"'] &  & 3 \arrow["x"', loop, distance=2em, in=35, out=325] \arrow["e_{3}"', loop, distance=2em, in=215, out=145] \arrow[dd, "w"', bend right=49] \arrow[dd, "xw" description, bend right=23] \arrow[dd, "wy" description, bend left=23] \arrow[dd, "xwy", bend left=49] \\
                                                                         &  &                                                                                                                                                                                                                                                           \\
2 \arrow["e_{2}"', loop, distance=2em, in=215, out=145]                  &  & 4 \arrow["y"', loop, distance=2em, in=5, out=295] \arrow["e_4"', loop, distance=2em, in=245, out=175]                                                                                                                                                   
\end{tikzcd}
\]
Note that the existence of $xw$ and $wy$ follows from the composition axioms of our category, but with the same argument,
we should also define $x^{2} = xx, x^{3} = xxx, ... $ and $y^{2} = yy, ... $ which are a priori all distinct morphisms.
Defining a set of relations of endomorphisms fixes this problem, so we define $x^{2} = e_{3}$ and $y^{2} = e_{4}$.
The identity axiom already implies the relations $e_{i}^{2} = e_{i}$ for $i = 1,2,3,4$.

If we take a field $\Bbbk$ for the commutative ring, the identity morphisms $e_{1}, e_{2}, e_{3}, e_{4}$ together with the other morphisms
\begin{align*}
v &: 1 \rightarrow 2 \\
x &: 3 \rightarrow 3 \\
xw, w, wy, xwy &: 3 \rightarrow 4 \\
y &: 4 \rightarrow 4
\end{align*}
form bases for the vector spaces
\begin{align*}
\HomA(1,1) &= \left< e_{1} \right> \\
\HomA(1,2) &= \left< v \right> \\
\HomA(2,2) &= \left< e_{2} \right> \\
\HomA(3,3) &= \left< e_{3}, x \right> \\
\HomA(3,4) &= \left< w, xw, wy, xwy \right> \\
\HomA(4,4) &= \left< e_{4}, y \right>.
\end{align*}
Linearity of composition necessitates that adding two arrows and then composing the result with another arrow
is the same as first composing each arrow and then adding them, e.g.
\begin{align*}
(x+x)w &= xw + xw \\
(e_{3} + e_{3})w &= (2\,e_{3})w = e_{3}(2w) = 2w
\end{align*}
The $\Bbbk$-dimensions of the vector spaces are $1, 1, 1, 2, 4, 2$ with a total of 11. Does that mean that
somewhere there exists an 11-dimensional $\Bbbk$-vector space that has these six vectorspaces as subspaces? Indeed it does:

The $\Bbbk$-vector space $\Bbbk\mathcal{A}$ with $\mathcal{A}_{1} = \{ e_{1},e_{2},e_{3},e_{4}, v,x,w,xw, wy, xwy, y \}$ as a basis.
The composition of arrows is already partially defined in the cases where two arrows are composable. To extend composition for all
arrows we first define for all arrows $\alpha, \beta \in \mathcal{A}_{1}$:
\begin{equation}
\alpha\beta :=\label{eq:cat_alg_mult} \begin{cases}
\alpha\cdot\beta & \text{ if $\alpha$ and $\beta$ can be composed} \\
0 & otherwise
\end{cases}
\end{equation}
and then extend this product to the whole of $\Bbbk\mathcal{A}$ using bilinearity of multiplication.
This construction turns $\Bbbk\mathcal{A}$ into not only a $\Bbbk$-vector space, but also an associative algebra.
\end{example}

\begin{definition}{(Category algebra)}
Let $\Bbbk$ be a commutative unital ring. For a category $\mathcal{C}$, we define the \ul{category algebra} $\Bbbk \mathcal{C}$
to be the free $\Bbbk$-module with $\mathcal{C}_{1}$ as a basis and the product of morphisms defined as in \eqref{eq:cat_alg_mult}.
\end{definition}

\begin{lemma}
If the set of objects $\mathcal{C}_{0}$ is finite, then the category algebra $\Bbbk\mathcal{C}$ has a unit element, namely
$\sum_{i\in \mathcal{C}_{0}} e_{i}$.
\end{lemma}
\begin{proof}
Let $e = \sum_{i\in \mathcal{C}_{0}} e_{i}$. We want to show for a general $w \in \Bbbk\mathcal{C}$ that $ew = w = we$.
\begin{enumerate}
\item Let $w = \sum_{l=1}^{N} \lambda_{l} b_{l}$ with $\{b_{l}\}_{1\leq l\leq N}$ a basis for $\mathrm{Hom}_{\mathcal{C}}(j,k)$ and
$\lambda_{l} \in \Bbbk$. Then $ew = e_{j}w + \sum_{i \neq j} e_{i} w = \sum_{l=1}^{N} \lambda_{l} (e_{j} b_{l})
+ \sum_{i \neq j} \sum_{l=1}^{N} \lambda_{l} (e_{i} b_{l})$ with each $e_{j} b_{l} = b_{l}$ in the first sum and
$e_{i} b_{l} = 0$ in the second sum, thus $ew = \sum_{l=1}^{N} \lambda_{l} b_{l} = w$ and similarly $we_{k} = w$ and $we_{i} = 0$ for
$i \neq k$.
\item 
\end{enumerate}
\end{proof}

\begin{definition}{(Algebra with enough idempotents)}
An \ul{algebra with enough idempotents} is an algebra $A$ which admits a family of nonzero orthogonal idempotents
$(e_{i})_{i\in I}$ such that $\oplus_{i\in I} e_{i} A = A = \oplus_{i\in I} Ae_{i}$.
This family $(e_{i})_{i\in I}$ is called \ul{complete} or \ul{distinguished family of orthogonal idempotents}.
\end{definition}



\begin{definition}
Once source and target categories $\mathcal{C}, \mathcal{D}$ are both $R$-linear categories we define the functor category
$\mathrm{Hom_{R}}(\mathcal{C},\mathcal{D})$ as the subcategory of $R$-linear functors.
\end{definition}





%% mainfile: ../main.tex

\subsection{Limit and colimit of a functor}

\begin{definition}{(Source of a functor)}
Let $D : \mathbf{I} \rightarrow \mathcal{C}$ be a functor. A \ul{source} of $D$ consists of the following data:
\begin{enumerate}
\renewcommand{\labelenumi}{(\theenumi)}
\item An object $S \in \mathcal{C}$.
\item A dependent function $s$ mapping an object $i \in \mathbf{I}$ to a morphism
$s(i) : S \rightarrow D(i)$ such that for all $i, j \in \mathbf{I}, \iota : i \rightarrow j$, we have $D(\iota) \cdot s(i) = s(j)$.
\end{enumerate}
\end{definition}

\begin{definition}{(Limit and colimit of a functor)}
Let $D : \mathbf{I} \rightarrow \mathcal{C}$ be a functor. A \ul{limit} of $D$ consists of the
following data:
\begin{enumerate}
\renewcommand{\labelenumi}{(\theenumi)}
\item A source of $D$ given by the data $(\mathrm{lim}\, D, (\lambda(i) : \mathrm{lim}\, D \rightarrow D(i))_{i\in\mathbf{I}})$.
\item A dependent function $u$ mapping every source $\tau = (T, (\tau(i) : T \rightarrow D(i))_{i \in \mathbf{I}})$ to a
morphism $u(\tau) : T \rightarrow \mathrm{lim}\, D$ such that $\lambda(i) \cdot u(\tau) = \tau(i)$ for all $i \in \mathbf{I}$.\label{itm:2}
\item For any other dependent function $v$ satisfying (\ref{itm:2}), we have $u = v$.
\end{enumerate}
A \ul{colimit} of $D$ is a limit of $D' : \mathbf{I} \rightarrow \mathcal{C}^{\mathrm{op}}$.
\end{definition}

\begin{definition}{(Limits of type \textbf{I})}
Let $\mathbf{I}$ be a category. We say a category $\mathcal{C}$ \ul{has limits of type} $\mathbf{I}$ if it is
equipped with a dependent function $\lambda$ mapping a functor $D : \mathbf{I} \rightarrow \mathcal{C}$ to a limit
$(\mathrm{lim}\, D, \lambda_{D}, u_{D})$ of $D$.
We say $\mathcal{C}$ \ul{has colimits of type} $\mathbf{I}$ if $\mathcal{C}^{\mathrm{op}}$ has limits of that type.
\end{definition}

\begin{example}\label{ex:limits}
Depending on \textbf{I} some limits and colimits have special names:
\begin{center}
\begin{tabular}{c|c|c}
\textbf{I} & limit & colimit \\
\hline
$\emptyset$ & terminal object & initial object \\
a set & direct product & coproduct \\
$\cdot \rightarrow \cdot \leftarrow \cdot$ & pullback & - \\
$\cdot \leftarrow \cdot \rightarrow \cdot$  & - & pushout \\
$ \cdot \rightrightarrows \cdot$ & equalizer & coequalizer
\end{tabular}
\end{center}
\end{example}

In the following section, we give explicit definitions for the limits in \ref{ex:limits} and examples for categories with such limits.

\subsection{Examples for limits and colimits}

\begin{definition}{(Product, coproduct)}\label{def:prod_coprod}
Let $I$ be an index set and $\{A_{i}\}_{i\in I}$ a family of objects in a category $\mathcal{C}$.
\begin{enumerate}
\renewcommand{\labelenumi}{(prod)}
\item The \ul{product} of the family $\{A_{i}\}_{i\in I}$ is an object $\invamalg A_{i}$ together with a family of morphisms
\[
\{ \pi_{i} : \invamalg A_{i} \twoheadrightarrow A_{i} \}
\]
called \ul{projections}, such that the following universal property is satisfied:\\
For any object $M \in \mathcal{C}_{0}$ and any family $\{ \varphi_{i} : M \rightarrow A_{i} \}_{i\in I}$ of morphisms, there exists
a unique morphism $\varphi : M \rightarrow \invamalg A_{i}$ called the \ul{product morphism} such that
\[
\varphi \pi_{i} = \varphi_{i} \, \forall i \in I.
\]
\begin{tikzcd}
                                                                                                                            &  &                                                          & A_{1} \\
M \arrow[rrru, "\varphi_{1}", bend left] \arrow[rrrd, "\varphi_{2}"', bend right] \arrow[rr, "\exists^{1} \varphi", dashed] &  & A_{1}\invamalg A_{2} \arrow[ru, "\pi_{1}"] \arrow[rd, "\pi_2"] &       \\
                                                                                                                            &  &                                                          & A_{2}
\end{tikzcd}
\renewcommand{\labelenumi}{(coprod)}
\item The dual notion to product is the \ul{coproduct} of the family $\{A_{i}\}_{i\in I}$, that is an object $\amalg A_{i}$ together with
a family of morphisms
\[
\{ \iota_{i} : A_{i} \hookrightarrow \amalg A_{i} \}
\]
called \ul{coprojections} or sometimes \ul{injections} or \ul{inclusions}, such that the following universal property is satisfied:\\
For any object $M \in \mathcal{C}$ and any family $\{ \psi_{i} : A_{i} \rightarrow M \}$ of morphisms, there exists a unique
morphism $\psi : \amalg A_{i} : M$ called the \ul{coproduct morphism} such that
\[
\iota_{i} \psi = \psi_{i} \, \forall i \in I.
\]
\begin{tikzcd}
  &  &                                                           & A_{1} \arrow[llld, "\psi_{1}"', bend right] \arrow[ld, "\iota_{1}"'] \\
M &  & A_{1}\amalg A_{2} \arrow[ll, "\exists^{1} \psi"', dashed] &                                                                      \\
  &  &                                                           & A_{2} \arrow[lllu, "\psi_{2}", bend left] \arrow[lu, "\iota_2"]     
\end{tikzcd}
\end{enumerate}
\end{definition}

\begin{definition}{(Terminal object, initial object, zero object)}\label{def:init_term_zero_object}
\renewcommand{\labelenumi}{(\theenumi)}
\begin{enumerate}
\item A \ul{terminal object} $T$ in a category $\mathcal{C}$ is an object such that $\textup{Hom}_{\mathcal{C}}(-,T)$ is a singleton.
\item An \ul{initial object} $I$ in a category $\mathcal{C}$ is an object such that $\textup{Hom}_{\mathcal{C}}(I,-)$ is a singleton.
\item An object is a \ul{zero object} if it is both initial and terminal.
\end{enumerate}
\end{definition}

\begin{definition}{(Zero morphism)}\label{def:zero_morphism}\\
A \ul{zero morphism} in a category with a zero object $Z$ is a morphism factoring over $Z$, i.e. $\varphi : M \rightarrow N$ is called a zero
morphism, if\\
\begin{minipage}{.35\textwidth}
\begin{tikzcd}
M \arrow[rr, "\varphi"] \arrow[rd, "\varphi_{1}"] &                              & N \\
                                                  & Z \arrow[ru, "\varphi_{2}"'] &  
\end{tikzcd}
\end{minipage}
\begin{minipage}{.65\textwidth}
$\exists \varphi_{1} : M \rightarrow Z, \varphi_{2} : Z \rightarrow N$\\
such that $\varphi = \varphi_{1}\varphi_{2}$.
\end{minipage}
\end{definition}

%%% insert definition of Ab-Category.


\begin{definition}{(Direct sum)}\label{def:direct_sum}
Let $\mathcal{C}$ be an Ab-category. Let $I$ be an index set and $\{A_{i}\}_{i\in I}$ a family of objects in $\mathcal{C}$.
A \ul{direct sum} consists of the following data:
\begin{itemize}
\item an object $S$,
\item a family of morphisms $\pi = \{ \pi_{i} : S \rightarrow S_{i} \}_{i\in I}$,
\item a family of morphisms $\iota = \{ \iota_{i} : S_{i} \rightarrow S \}_{i\in I}$,
\item a dependent function $u_{\text{in}}$ mapping every family $\tau = \{ \tau_{i} : T \rightarrow S_{i} \}_{i\in I}$ to a morphism
$u_{\text{in}}(\tau) : T \rightarrow S$ such that $u_{\text{in}}(\tau) \pi_{i} \sim \tau_{i}$ for all $i \in I$,
\item a dependent function $u_{\text{out}}$ mapping every family $\rho = \{ \rho_{i} : S_{i} \rightarrow R \}_{i\in I}$ to a morphism
$u_{\text{out}}(\rho) : S \rightarrow R$ such that $\iota_{i} u_{\text{out}}(\rho) \sim \rho_{i}$ for all $i \in I$,
\end{itemize}
such that
\begin{itemize}
\item $\sum_{i\in I} \iota_{i} \pi_{i} \sim 1_{S}$,
\item $\pi_{j} \iota_{i} \sim \delta_{i, j} =  \begin{cases}
            1_{S_{i}} & \text{ if } i = j  \\
            0_{ij} & \text{ if } i \neq j
        \end{cases}$,
\end{itemize}
where $\delta_{i, j} \in \mathrm{Hom}(S_{i}, S_{j})$ is the identity if $i = j$, and the zero morphism $0_{ij} = 0_{S_{i}, S_{j}}$ otherwise.
\end{definition}

\begin{example}{($\kmat$ has direct sums)}\label{ex:kmat_has_direct_sum}
The matrix category $\kmat$ is an Ab-category. The direct sum of two ($I = \{1,2\}$) natural numbers $S_{1} = m, S_{2} = n \in \kmat_{0}$ is
\begin{itemize}
\item the object $S = m+n$,
\item the two morphisms $\pi_{1} : m+n \rightarrow m$, $\pi_{2} : m+n \rightarrow n$ which are $(m+n) \times m$- and $(m+n) \times n$-matrices.
\item the two morphisms $\iota_{1} : m \rightarrow m+n$, $\iota_{2} : n \rightarrow m+n$ which are $m \times (m+n)$- and $n \times (m+n)$-matrices.
\item a family of two morphisms $\tau = \{ \tau_{1} : t \rightarrow m, \tau_{2} : t \rightarrow n\}$ gets mapped to a $t \times (m+n)$-matrix
$u_{\text{in}}(\tau) : t \rightarrow m+n$ with $u_{\text{in}}(\tau) \pi_{1} = \tau_{1}$ and $u_{\text{in}}(\tau) \pi_{2} = \tau_{2}$.
\item a family of two morphisms $\rho = \{ \rho_{1} : m \rightarrow r, \rho_{2} : n \rightarrow r \}$ gets mapped to a $(m+n) \times r$-matrix
$u_{\text{out}}(\rho) : m+n \rightarrow r$ with $\iota_{1} u_{\text{out}} = \rho_{1}$ and $\iota_{2} u_{\text{out}} = \rho_{2}$.
\end{itemize}
\end{example}

\begin{example}{(number example)}
$m = 3, n = 5$
\begin{align*}
\pi_{1} = \begin{pmatrix}
1 \ampersand 0 \ampersand 0 \\
0 \ampersand 1 \ampersand 0 \\
0 \ampersand 0 \ampersand 1 \\
0 \ampersand 0 \ampersand 0 \\
0 \ampersand 0 \ampersand 0 \\
0 \ampersand 0 \ampersand 0 \\
0 \ampersand 0 \ampersand 0 \\
0 \ampersand 0 \ampersand 0
\end{pmatrix},
\pi_{2} = \begin{pmatrix}
0 \ampersand 0 \ampersand 0 \ampersand 0 \ampersand 0 \\
0 \ampersand 0 \ampersand 0 \ampersand 0 \ampersand 0 \\
0 \ampersand 0 \ampersand 0 \ampersand 0 \ampersand 0 \\
1 \ampersand 0 \ampersand 0 \ampersand 0 \ampersand 0 \\
0 \ampersand 1 \ampersand 0 \ampersand 0 \ampersand 0 \\
0 \ampersand 0 \ampersand 1 \ampersand 0 \ampersand 0 \\
0 \ampersand 0 \ampersand 0 \ampersand 1 \ampersand 0 \\
0 \ampersand 0 \ampersand 0 \ampersand 0 \ampersand 1
\end{pmatrix}, 
\begin{array}{rr}
\iota_{1} &= \begin{pmatrix}
1 \ampersand 0 \ampersand 0 \ampersand 0 \ampersand 0 \ampersand 0 \ampersand 0 \ampersand 0 \\
0 \ampersand 1 \ampersand 0 \ampersand 0 \ampersand 0 \ampersand 0 \ampersand 0 \ampersand 0 \\
0 \ampersand 0 \ampersand 1 \ampersand 0 \ampersand 0 \ampersand 0 \ampersand 0 \ampersand 0
\end{pmatrix} \\
\\
\iota_{2} &= \begin{pmatrix}
0 \ampersand 0 \ampersand 0 \ampersand 1 \ampersand 0 \ampersand 0 \ampersand 0 \ampersand 0 \\
0 \ampersand 0 \ampersand 0 \ampersand 0 \ampersand 1 \ampersand 0 \ampersand 0 \ampersand 0 \\
0 \ampersand 0 \ampersand 0 \ampersand 0 \ampersand 0 \ampersand 1 \ampersand 0 \ampersand 0 \\
0 \ampersand 0 \ampersand 0 \ampersand 0 \ampersand 0 \ampersand 0 \ampersand 1 \ampersand 0 \\
0 \ampersand 0 \ampersand 0 \ampersand 0 \ampersand 0 \ampersand 0 \ampersand 0 \ampersand 1
\end{pmatrix}
\end{array}
\end{align*}
\begin{minipage}[t]{.5\textwidth}
and for $t = 4$, $\tau = \{\tau_{1}, \tau_{2}\}$ defined as
\begin{align*}
\tau_{1} = \begin{pmatrix}
1 \ampersand 2 \ampersand 2 \\
4 \ampersand 3 \ampersand 1 \\
0 \ampersand 1 \ampersand 0 \\
1 \ampersand 2 \ampersand 1
\end{pmatrix},
\tau_{2} = \begin{pmatrix}
1 \ampersand 1 \ampersand 2 \ampersand 2 \ampersand 3 \\
3 \ampersand 4 \ampersand 4 \ampersand 5 \ampersand 5 \\
6 \ampersand 6 \ampersand 7 \ampersand 7 \ampersand 8 \\
8 \ampersand 9 \ampersand 9 \ampersand 10 \ampersand 10
\end{pmatrix}
\end{align*}
we get the matrix
\begin{align*}
u_{\text{in}}(\tau) = \begin{pmatrix}
1 \ampersand 2 \ampersand 2 \ampersand 1 \ampersand 1 \ampersand 2 \ampersand 2 \ampersand 3 \\
4 \ampersand 3 \ampersand 1 \ampersand 3 \ampersand 4 \ampersand 4 \ampersand 5 \ampersand 5 \\
0 \ampersand 1 \ampersand 0 \ampersand 6 \ampersand 6 \ampersand 7 \ampersand 7 \ampersand 8 \\
1 \ampersand 2 \ampersand 1 \ampersand 8 \ampersand 9 \ampersand 9 \ampersand 10 \ampersand 10
\end{pmatrix}
\end{align*}
\end{minipage}
\begin{minipage}[t]{.5\textwidth}
and for $r = 2$, $\rho = \{\rho_{1}, \rho_{2}\}$ we get the matrix
\begin{align*}
\begin{array}{rr}
\rho_{1} &= \begin{pmatrix}
0 \ampersand 1 \\
1 \ampersand 1 \\
2 \ampersand 2
\end{pmatrix} \\
\\
\rho_{2} &= \begin{pmatrix}
4 \ampersand 5 \\
-7 \ampersand 0 \\
0 \ampersand 5 \\
0 \ampersand 0 \\
1 \ampersand 1
\end{pmatrix}
\end{array}
u_{\text{in}}(\rho) = \begin{pmatrix}
0 \ampersand 1 \\
1 \ampersand 1 \\
2 \ampersand 2 \\
4 \ampersand 5 \\
-7 \ampersand 0 \\
0 \ampersand 5 \\
0 \ampersand 0 \\
1 \ampersand 1
\end{pmatrix}
\end{align*}
\end{minipage}
\end{example}


\subsection{Monomorphisms and epimorphisms}

\subsection{Kernel and cokernel; image and coimage}

\subsection{Direct sum and direct product}

%% mainfile: ../main.tex

\section{Adjunctions}

\subsection{Universal objects}

\subsection{Forgetting the forgetful functor: Free constructions}

\section{Yoneda's Lemma: Completion and cocompletion of a category}
% mainfile: ../main.tex

\section{Yoneda's Lemma: Completion and cocompletion of a category}

\subsection{Embedding categories}

\begin{lemma}{(Yoneda's Lemma)}

\begin{proof}

\end{proof}
\end{lemma}
\[
\begin{pmatrix}1 \ampersand 2 \ampersand 3 \ampersand 4\end{pmatrix}
\begin{pmatrix} 1\to5, \ampersand 2\to6 \\ 3\to7, \ampersand 4\to8 \end{pmatrix}
\begin{pmatrix}6 \ampersand 8\end{pmatrix}
\begin{pmatrix}5\to9\\6\to10\\7\to11\\8\to12\end{pmatrix}
\begin{pmatrix}9\ampersand10\end{pmatrix}\begin{pmatrix}11\ampersand12\end{pmatrix}
\begin{pmatrix}9\to13, \ampersand 10\to14\\11\to15 \ampersand 12\to16\end{pmatrix}
\textup{id}
\]

\begin{tikzcd}
{\{1,2,3,4\}} \arrow["{(1,2,3,4)}"', loop, distance=2em, in=125, out=55] \arrow[rr, "{\begin{pmatrix} 1\to5, 2\to6, \\ 3\to7, 4\to8 \end{pmatrix}}"] \arrow[dd] \arrow[rrdd] &  & {\{5,6,7,8\}} \arrow["{(6,8)}"', loop, distance=2em, in=125, out=55] \arrow[dd, " \begin{pmatrix}5\to9\\6\to10\\7\to11\\8\to12\end{pmatrix}"', bend left] \arrow[lldd] \\
                                                                                                                                                                             &  &                                                                                                                                                                        \\
{\{13,14,15,16\}} \arrow["\textup{id}"', loop, distance=2em, in=305, out=235]                                                                                                &  & {\{9,10,11,12\}} \arrow["{(9,10)(11,12)}"', loop, distance=2em, in=305, out=235] \arrow[ll, "{\begin{pmatrix}9\to13, 10\to14,\\ 11\to 15, 12\to16\end{pmatrix}}"]     
\end{tikzcd}

Dimension of the (quotient of the) path algebra is 43.
Sum of all dimensions of the yoneda projectives on each objects is 43.

% mainfile: ../main.tex

\section{Functors and natural transformations}

\subsection{Functors act on objects and morphisms of a category}

\subsection{Natural transformations are morphisms between functors}

\subsection{Representations are Functors into a matrix category}

%% mainfile: ../main.tex

\section{The example CatReps}
\[
\begin{tabular}{r|cccccc}
   & 1 & 2 & 3 & 4 & 5 & 6 \\
\hline
$e_{1}$ & 1 & 2 & 3 &   &   &  \\
$e_{2}$ &   &   &   & 4 & 5 & 6 \\
\hline
$a$ & 2 & 3 & 1 &   &   &  \\
$b$ & 4 & 5 & 6 &   &   &  \\
$c$ &   &   &   & 5 & 6 & 4 \\
\hline
\rule{0pt}{0.01pt} \\
$a^{2}$ & 3 & 1 & 2 &   &   &  \\
$ab$ & 5 & 6 & 4 &   &   &  \\
$bc$ & 5 & 6 & 4 &   &   &  \\
$c^{2}$ &   &   &   & 6 & 4 & 5 \\
\hline
\rule{0pt}{1pt} \\
$a^{3}$ & 1 & 2 & 3 &   &   &  \\
$a^{2}b$ & 6 & 4 & 5 &   &   &  \\
$abc$ & 6 & 4 & 5 &   &   &  \\
$bc^{2}$ & 6 & 4 & 5 &   &   &  \\
$c^{3}$ &   &   &   & 4 & 5 & 6 \\
\hline
\hline
\end{tabular}
\]

\[
\begin{tikzcd}
\{1,2,3\} \arrow["a"', loop, distance=2em, in=305, out=235] \arrow[rr, "b"] &  & \{4,5,6\} \arrow["c"', loop, distance=2em, in=305, out=235]
\end{tikzcd}
\]
\[
\begin{tikzcd}
1 \arrow["a"', loop, distance=2em, in=305, out=235] \arrow["e_{1}", loop, distance=2em, in=55, out=125] \arrow[rr, "b"] & 
 & 2 \arrow["c"', loop, distance=2em, in=305, out=235] \arrow["e_{2}", loop, distance=2em, in=55, out=125]
\end{tikzcd}
\]


% mainfile: ../main.tex

\section{Algorithms}

\begin{algorithm}\capstart
   \caption{\texttt{ConvertToMapOfFinSets}}\label{algo:ConvertToMapOfFinSets}
      \SetKwInput{Input}{Input~}
      \SetKwInput{Output}{Output~}
      \Input{~a list $objects$ of objects in FinSets and a morphism $gen$ given as a list of images in the convention of catreps}
      \Output{~the corresponding map of finite sets from source $S$ to target $T$}
      \BlankLine
      let $T$ be the first object $O \in objects$ such that $mor \cap O \not= \emptyset$\;
      \If{$mor \cap O = \emptyset \, \forall O \in objects$}{
         Error "unable to find target set"
      }
      let $S$ be the list of positions $i$ such that $gen[i]$ is bound\;
      let $S$ be the first object $O \in objects$ such that $O = S$\;
      \If{$S \not= O \, \forall O \in objects$}{
          Error "unable to find source set"
      }
      \BlankLine
      let $G$ be the list of pairs $[ i, gen[i] ], i \in S$;
      \BlankLine
      \Return MapOfFinSets( S, G, T );
\end{algorithm}

We can now create finite concrete categories with objects not starting from 1, to demonstrate that
\texttt{ConcreteCategoryForCAP( [ [,,,5,6,4], [,,,7,8,9], [,,,,,,8,9,7] ] )} and\\
\texttt{ConcreteCategoryForCAP( [ [2,3,1], [4,5,6], [,,,5,6,4] ] )} yield
equivalent categories, i.e. their underlying quivers are the same and they give the same category of representations.

\begin{algorithm}\capstart
    \caption{\texttt{RightQuiverFromConcreteCategory}}\label{algo:RightQuiverFromConcreteCategory}
	\SetKwInput{Input}{Input~}
	\SetKwInput{Output}{Output~}
	\Input{~a finite concrete category $C$ with $n$ objects}
	\Output{~the right quiver $q(n)$}
	\BlankLine
	let $Obj$ be the set of objects of $C$\;
	let $n := Length(Obj)$\;
	let $gMor$ be the set of generating morphisms of $C$\;
	let $A$ be the empty set and let $i := 1$\;
	\ForEach{morphism $mor$ in $gMor$}{
	    let $A_{i,1}$ be the position of $Source( mor )$ in $Obj$\;
	    let $A_{i,2}$ be the position of $Range( mor )$ in $Obj$\;
	    let $i := i+1$\;
	}
	\BlankLine
	let $q$ be the right quiver with vertices $\{1,\dots,n\}$ and arrows $A$.
	\BlankLine
	\Return q\;
\end{algorithm}

\begin{algorithm}\capstart
    \caption{\texttt{RelationsOfEndomorphisms}}\label{algo:RelationsOfEndomorphisms}
	\SetKwInput{Input}{~Input}
	\SetKwInput{Output}{~Output}
	\Input{~a commutative ring $k$ and a finite concrete category $C$}
	\Output{~the endomorphism relations of the category $C$}
	\BlankLine
	let $q := \texttt{RightQuiverFromConcreteCategory}(C)$\;
	let $kq$ be the path algebra generated by $k$ and $q$\;
	let $gMor$ be the set of generating morphisms of $C$\;
	let $A := Arrows(q)$\;
	let $relsEndo$ be the empty set\;
	\ForEach{$i = 1, \dots, Length(gMor)$}{
	    let $mor := gMor_i$\\
	    \If{$mor$ is not an endomorphism}{
		continue\;
	    }
	    let $m := 0$ and let $powers$ be the empty set\;
	    let $foundEqual$ be false\;
	    \While{$mor^{m}\nin powers$}{
		let $n := 1$\;
		\While{$\neg foundEqual$}{
		    \If{$mor^{(m+n)} = mor^{m}$}{
		    	Add the relation $kq.(A_{i})^{(m+n)}-kq.(A_{i})^{m}$ to relsEndo\;
		    	foundEqual := true\;
		    }
		    n := n+1\;
		}
		Add $mor^{m}$ to powers\;
		m := m+1\;
	    }
	}
	\Return{relsEndo}\;
\end{algorithm}

\begin{algorithm}\capstart
   \caption{\texttt{Algebroid}}\label{algo:Algebroid}
      \SetKwInput{Input}{~Input}
      \SetKwInput{Output}{~Output}
      \Input{~a commutative ring $k$ and a finite concrete category $C$}
      \Output{~the $k$-linear closure of the category $C$ over the commutative ring $k$}
      \BlankLine
      
      \Return{}\;
\end{algorithm}

Yonedas Einbettungs-Lemma: Fehlende Limiten bzw. Kolimiten exitieren nach der Einbettung.

Einbettung in Kategorien, die mehr Limiten haben als die Zielkategorie.

"(Ko-)Vervollständigung" der Kategorie (Completion / Cocompletion)

Quiver = unvollständige Struktur einer Kategorie
Erzeugendensystem einer Kategorie.

K-linearer Abschluss einer Kategorie

Pfadalgebra = Kategorien-Algebra
path algebra = 1 Object, welches eine Algebra ist. Dabei verliert man wieder die Informationen über die
mehreren Objekte.

So wie Menge ein Erz-system eines Monoid.

% mainfile: ../main.tex

\section{Relations of the Algebroid}

\subsection{Relations of endomorphisms}

% A forest on a cycle
\begin{tikzpicture}[x=0.5cm,y=0.5cm]
\tikzstyle{cblack}=[circle, fill=black, scale=0.5]

%Nodes
\foreach \place/\x in {{(0,0)/0}, {(-4.5,0)/1}, {(-7,-3)/2}, {(-4.5,-6)/3},
  {(0,-6)/4}, {(2.5,-3)/5},
  {(-4.5,3)/6}, {(-7.5,6)/7}, {(-4.5,6)/8},
  {(0,3)/9}, {(0,6)/10}, {(0,9)/11},
  {(3,3)/12}, {(3,6)/13}, {(3,9)/14},
  {(7.5,6)/15}, {(7.5,9)/16}, {(7.5,12)/17}, {(10.5,9)/18}}
\node[cblack] (a\x) at \place {};

%Arrows
\foreach \i in {0,1,2,3,4,5}
{
  \pgfmathtruncatemacro\result{Mod(\i+1,6)}%
  \draw[->] (a\i) -> (a\result);
}
\path[->] (a7) edge (a6); 
\path[->] (a8) edge (a6) edge (a1);
\path[->] (a11) edge (a10) edge (a9) edge (a0);
\path[->] (a14) edge (a13) edge (a12); 
\path[->] (a12) edge (a0);
\path[->] (a18) edge (a15) (a15) edge (a12);
\path[->] (a17) edge (a16) edge (a15);
%\path[->] (a\x) edge (a\y);

\end{tikzpicture}
%\cite{facchini_2019}
\begin{lemma}[$\sigma$-Lemma]
Let $\mathcal{C}$ be a finite concrete category. Then for each object $M \in \mathcal{C}_{0}$ the set
$\textup{End}_{\mathcal{C}}(M)$ is a monoid and for each endomorphism $f \in \textup{End}_{\mathcal{C}}(M)$
there exist $m,n \in \mathbb{N}$ such that $f^{(m+n)}=f^{m}$. If $m = 0$ and $n \geq 1$ then $f$ is bijective with $f^{-1} = f^{n-1}$.
\begin{proof}
The properties of a monoid are precisely the associativity of composition and the unit property from \ref{associativity_of_composition} and \ref{unit_property}.
Since $\abs{\textup{End}_{\mathcal{C}}(M)}<\infty$ there are only finitely many endomorphisms $f_{1},\dots, f_{N} \in \textup{End}_{\mathcal{C}}(M)$.
Let $\{f^{k} | k \in \mathbb{N} \} \subset \textup{End}_{\mathcal{C}}(M)$, i.e. there is a function 
$\{f^{k} | k \in \mathbb{N}\} \rightarrow \{f_{j} | j \in \{1,\dots,N\}\}; f^{k} \mapsto f_{j}$ not necessarily surjective and 
by the pigeonhole principle highly non injective, since $\abs{\mathbb{N}}>\abs{\textup{End}_{\mathcal{C}}(M)}$.
Let $m := Min \{ k \in \mathbb{N}| f^{k} =  f_{j} \}$

\begin{minipage}{.45\textwidth}\phantom{}\end{minipage}
\end{proof}
\end{lemma}

Beschreibung der Algorithmen

WeakDirectSumDecomposition <-- Tiefensuche.
Objekte (Funktoren) in indecomposable Functors.




\section{$\mathbb{K}$-linear Category (Algebroid)}

Group: Category with one object.

Groupoid: A small category in which every morphism is an isomorphism.

Algebroid

EmbeddingOfSumOfImages

What is an Algebroid? Bialgebroid?

\section{Additive Category}

\section{Abelian Category}

\section{The Category of Categories}

\section{The Categories of Functors}

\section{The Representation of a Category}

\section{Representation}

Grundidee von FunctorCategory

Standard-Monoidale Struktur von der Zielkategorie z.B. TensorUnit(C)

\section{Algorithms}

\begin{algorithm}\capstart
    \caption{\texttt{RightQuiverFromConcreteCategory}}\label{algo:RightQuiverFromConcreteCategory}
	\SetKwInput{Input}{Input~}
	\SetKwInput{Output}{Output~}
	\Input{~a finite concrete category $C$ with $n$ objects}
	\Output{~the right quiver $q(n)$}
	\BlankLine
	let $Obj$ be the set of objects of $C$\;
	let $n := Length(Obj)$\;
	let $gMor$ be the set of generating morphisms of $C$\;
	let $A$ be the empty set and let $i := 1$\;
	\ForEach{morphism $mor$ in $gMor$}{
	    let $A_{i,1}$ be the position of $Source( mor )$ in $Obj$\;
	    let $A_{i,2}$ be the position of $Range( mor )$ in $Obj$\;
	    let $i := i+1$\;
	}
	\BlankLine
	let $q$ be the right quiver with vertices $\{1,\dots,n\}$ and arrows $A$.
	\BlankLine
	\Return q\;
\end{algorithm}

We want the endomorphism relations so that the path algebra is finite-dimensional and we
get a finite Gröbner basis.

\begin{algorithm}\capstart
    \caption{\texttt{RelationsOfEndomorphisms}}\label{algo:RelationsOfEndomorphisms}
	\SetKwInput{Input}{~Input}
	\SetKwInput{Output}{~Output}
	\Input{~a commutative ring $k$ and a finite concrete category $C$}
	\Output{~the endomorphism relations of the category $C$}
	\BlankLine
	let $q := \texttt{RightQuiverFromConcreteCategory}(C)$\;
	let $kq$ be the path algebra generated by $k$ and $q$\;
	let $gMor$ be the set of generating morphisms of $C$\;
	let $A := Arrows(q)$\;
	let $relsEndo$ be the empty set\;
	\ForEach{$i = 1, \dots, Length(gMor)$}{
	    let $mor := gMor_i$
	    \If{$mor$ is not an endomorphism}{
		continue\;
	    }
	    let $m := 0$ and let $powers$ be the empty set\;
	    let $foundEqual$ be false\;
	    \While{$mor^{m}\notin powers$}{
		let $n := 1$\;
		\While{$\neg foundEqual$}{
		    \If{$mor^{(m+n)} = mor^{m}$}{
		    	Add the relation $kq.(A_{i})^{(m+n)}-kq.(A_{i})^{m}$ to relsEndo\;
		    	foundEqual := true\;
		    }
		    n := n+1\;
		}
		Add $mor^{m}$ to powers\;
		m := m+1\;
	    }
	}
	\Return{relsEndo}\;
\end{algorithm}

Proof that algorithm is correct
Proof that it terminates.

Wir haben BasisOfExternalHom benutzt um Decompose in CAP umzusetzen um EmbeddingOfSubRepresentation umzusetzen um
WeakDirectSumDecomposition umzusetzen.

%%% insert endnotes in some way
\begingroup
     \parindent 0pt
     \parskip 2ex
     \def\enotesize{\normalsize}
     \theendnotes
\endgroup 

\input{bib/sources.bib}

\end{document}