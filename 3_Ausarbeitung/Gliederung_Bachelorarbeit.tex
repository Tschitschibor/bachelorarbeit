\documentclass{article}

\usepackage[a4paper,%
            left=2.5cm,right=2.5cm,top=2.5cm,bottom=2.5cm,%
            footskip=.6cm]{geometry}

\def\changemargin#1#2{\list{}{\rightmargin#2\leftmargin#1}\item[]}
\let\endchangemargin=\endlist


%%% For Math
\usepackage{amsmath}
\usepackage{amsfonts}
\usepackage{amsbsy}
\usepackage{amsthm}
\usepackage{amssymb}

\usepackage{mathtools}
\usepackage{commath}
\usepackage[sc,osf]{mathpazo}
\usepackage{mnsymbol}

%%% For calculations inside tikz and latex
\usepackage{calc}

\newcounter{modresult}

\newcommand*{\themodulo}[2]{%
\setcounter{modresult}{%
#1-(#1/#2)*#2%
}%
#1 mod #2 = \themodresult\par
}

%%% For algorithm styles
\usepackage[linesnumbered,ruled]{algorithm2e}


%%% Math theorem styles
\newtheorem{thm}{Theorem}[subsection]
\newtheorem{lemma}[thm]{Lemma}
\theoremstyle{definition}
\newtheorem{definition}[thm]{Definition}
\newtheorem{rmk}[thm]{Remark}
\newtheorem{example}[thm]{Example}

%%% Math operators bold
%\newcommand{\Category}{Category}

%%% For arrows and categories
\usepackage[all]{xy}
\usepackage{tikz-cd}

%%% For matrices
\let\ampersand =&

%%% tikz
\usetikzlibrary{positioning}

%%% For dotted box around diagrams
\tikzcdset{
    boxedcd/.style={
        every matrix/.append style={
            draw=black,
            dotted,
            rounded corners,
            #1
        },
    },
}

%%% For some big dots
\makeatletter
\newcommand*{\bigcdot}{}% Check if undefined
\DeclareRobustCommand*{\bigcdot}{%
  \mathbin{\mathpalette\bigcdot@{}}%
}
\newcommand*{\bigcdot@scalefactor}{.5}
\newcommand*{\bigcdot@widthfactor}{1.15}
\newcommand*{\bigcdot@}[2]{%
  % #1: math style
  % #2: unused
  \sbox0{$#1\vcenter{}$}% math axis
  \sbox2{$#1\cdot\m@th$}%
  \hbox to \bigcdot@widthfactor\wd2{%
    \hfil
    \raise\ht0\hbox{%
      \scalebox{\bigcdot@scalefactor}{%
        \lower\ht0\hbox{$#1\bullet\m@th$}%
      }%
    }%
    \hfil
  }%
}
\makeatother

%%% For horizontal times


%%% For proper underline
\usepackage{soul}
%\setuldepth{gjpqy}
%\setuldepth\strut
\setuldepth{-1}

\usepackage{pgffor}

%%% For source code listings
\usepackage{listings}
\def \pkgpath {C:/Users/Tibor/AppData/Local/Packages/CanonicalGroupLimited.UbuntuonWindows_79rhkp1fndgsc/LocalState/rootfs/home/user/.gap/pkg}
%%% For footnotes at end of text
\usepackage{endnotes}
\let\footnote=\endnote

%%% For captions
\usepackage{hyperref}
\usepackage[figure]{hypcap}

\title{Representations of a concrete category as objects in the functor category}

\author{Tibor Gr{\"u}n}

\begin{document}
	\pagenumbering{gobble}

	\maketitle

	\newpage

	\tableofcontents

	\newpage

	\pagenumbering{arabic}

%% mainfile: ../main.tex

\section{Introduction}

Die Aufgabe der vorliegenden Arbeit besteht darin, das GAP-Paket "catreps" von Peter Webb mit CAP zu re-organisieren. Dabei werden die Algorithmen, welche in catreps direkt implementiert sind, soweit wie möglich durch vorhandene Methoden von CAP ersetzt.
Insbesondere wird das CAP-Paket FunctorCategories Anwendung finden, weil ich zeigen werde, dass catreps, also die Kategorie der Darstellungen einer konkreten endlichen Kategorie, nichts anderes ist als eine Unterkategorie von FunctorCategories, also der Kategorie aller Funktoren zwischen Kategorien. 
Da catreps selbst bereits eine Verallgemeinerung der Darstellung endlicher Gruppen ist (eine Gruppe ist nichts anderes als eine Kategorie mit einem Objekt, in dem jeder Morphismus ein Isomorphismus ist), stellt FunctorCategories wohl den allgemeinsten Rahmen dar, den man sich vorstellen kann.

In this thesis I will define what a category is, then I go further in the doctrine of enriched categories, especially monoidal categories.
The morphisms in the category of categories are the functors between categories. I will treat the functor category where the functors themselves are
objects and natural transformations the morphisms between them. I will show that any representation of a category (and thus any representation
of a group) is a functor, so the category of representations (of a category) is a subcategory of the functor category.
I will show how the monoidal structure of the category of representations arises from the counit and the comultiplication on the Bialgebroid structure
on the category.

\noindent Throughout the thesis I will give proofs of existence by providing an algorithm that computes the object that exists. I will be using CAP, the gap package
developed by Sebastian Gutsche et al. Another purpose of this thesis is the translation of the work of Peter Webb, who used gap directly, into our CAP
framework. This includes his decomposition algorithm for a representation. As the category of representations is just a subcategory of the functor category,
most of the work will be done inside the package FunctorCategories by Prof. Mohamed Barakat. In this thesis I will also write the documentation for the
package FunctorCategories.


\section{Introduction}

\[
\mathbf{Quiv}\rightarrow^{CatClosure}\leftarrow_{U}\mathbf{Cats}
\rightarrow^{k-Algebroid}\leftarrow_{U}\mathbf{k-Cats}
\rightarrow^{AdditiveClosure}\leftarrow_{U}\mathbf{k-Cats^{\oplus}}
\]

% mainfile: ../main.tex

\section{A short overview of the tools used}

GAP, QPA / QPA2, Catreps, CAP, homalg\_project



% mainfile: ../main.tex

\section{Quivers, categories and k-algebroids}

\begin{definition}{(Quiver)}\\
A \ul{directed graph} or \ul{quiver} $q$ consists of a class of \ul{objects} (or \ul{vertices}) $q_{0} = \textup{Obj}\,q$ and
a class of \ul{morphisms} (or \ul{arrows}) $q_{1} = \textup{Mor}\,q$ together with two defining maps
\[
\begin{tikzcd}[column sep=small]
{s,t\colon q_{1}} \arrow[rr, shift left = 0.7ex] \arrow[rr, shift right = 0.7ex] & & q_{0}
\end{tikzcd}
\]
$s$ called \ul{source} and $t$ called \ul{target}.
\end{definition}

\noindent In \texttt{QPA}, the objects are coded by natural numbers, so the first object is $1$, the second $2$ and so on. The arrows are denoted by
small letters $a, b, c$ and so on. There is a difference between \texttt{RightQuiver} and \texttt{LeftQuiver} in that the right quiver is \ul{right-oriented}
(that is, the convention for order in multiplication of paths is the opposite of that used for \ul{left-oriented} quivers).

\begin{definition}{(Hom-set of a (locally) small quiver)}
\renewcommand{\labelenumi}{(\theenumi)}
\begin{enumerate}
\item Given two objects $M, N \in q_{0}$ we write $\textup{Hom}_{q}(M,N)$ or $q(M,N)$ for the fiber
$(s,t)^{-1} (\{(M,N)\})$ of the product map 
\begin{tikzcd}[column sep=small]
(s, t) : q_{1} \arrow[rr] &  & q_{0} \times q_{0} 
\end{tikzcd} over the pair $(M,N) \in q_{0} \times q_{0}$.
This is the class of all morphisms with source $= M$ and target $= N$.
We indicate this by writing
\begin{tikzcd}[column sep=small]
\varphi : M \arrow[rr] &  & N
\end{tikzcd} or 
\begin{tikzcd}[column sep=small]
M \arrow[rr,"\varphi"] &  & N.
\end{tikzcd} Hence $q_{1}$ is the disjoint union $\bigcup\limits^{\bigcdot}_{M,N \in q_{0}} \textup{Hom}_{q}(M,N) = q_{1}$.
As usual we define $\textup{End}_{q}(M):= \textup{Hom}_{q}(M,M)$.
\item If the class $\textup{Hom}_{q}(M,N)$ is a \ul{set} for all pairs $(M,N)$ then we call the quiver \ul{locally small}.
We therefore talk about \ul{Hom-sets}.
If additionally, $q_{0}$ is a set, then the quiver is called \ul{small}.
\end{enumerate}
\end{definition}

\begin{example}\label{q(2)}{(Quiver with 2 objects and 3 morphisms)}\\
\[
\begin{tikzcd}
1 \arrow["a"', loop, distance=2em, in=305, out=235] \arrow[rr, "b"] &  & 2 \arrow["c"', loop, distance=2em, in=305, out=235]
\end{tikzcd}
\]
The objects of this quiver $q$ are $q_{0} = \{1, 2\}$, and the morphisms are $q_{1} = \{a, b, c\}$ with\\
$s (a) = 1 = t (a)$, $s (c) = 2 = t (c)$ and $s (b) = 1, t (b) = 2$.\\
\noindent Thus $\textup{End}_{q}(1) = \{a\}, \textup{End}_{q}(2) = \{c\}$ and $\textup{Hom}_{q}(1,2) = \{b\}$ whereas
$\textup{Hom}_{q}(2,1)=\emptyset$.\\

\noindent In \texttt{QPA} this quiver is encoded as \texttt{q(2)[a:1->1,b:1->2,c:2->2]} where the first \texttt{(2)} in parentheses stands for the total
number of objects.
\end{example}

\begin{definition}{(Composable arrows; path in a quiver)}\\
Since we already have the source and target maps, we say two arrows $a, b \in q_{1}$ are \ul{composable} if $t(a) = s(b)$ or
$t(b) = s(a)$. In this case we can write a sequence of composable arrows $p = a_{1}a_{2}\cdots a_{n}$ where $t(a_{i}) = s(a_{i+1})$ for $i=1,\dots,n-1$.
We call this sequence a \ul{path} from $s(a_{1})$ to $t(a_{n})$ and the integer $n \in \mathbb{Z}_{\geq0}$ the \ul{length} $l(p)$ of the path $p$.
Although it's not an arrow, we can define the source and target of a path $p = a_{1}\cdots a_{n}$ as $s(p) := s(a_{1})$ and $t(p) := t(a_{n})$.
A path $p = a_{1}\cdots a_{n}$ with $s(a_{1}) = t(a_{n})$, i.e. $s(p) = t(p)$, is called \ul{cyclic}.\\
For an endomorphism $a \in \textup{End}_{q}(M)$ we write $a^{n}$ for $aa \cdots a$ ($n$ times). In the case of $n=0$ an \ul{empty path}
whose source and target are the vertex $i \in q_{0}$ is called the \ul{trivial path at $i$} and is denoted $e_{i}$. Note that the composition of paths
$e_{i}e_{i}$ has length zero starting at $i$ therefore $e_{i}^{2}=e_{i}$.
\end{definition}

\begin{lemma} Let Q be a quiver. If there is a path of length at least $\abs{Q_{0}}$, then there are cyclic paths,
and thus infinitely many paths.
\begin{proof}
Assume that there exists a path of length greater or equal to $\abs{Q_{0}}$. Then there exists a path of length $n = \abs{Q_{0}}$, say
$\alpha_{1}\cdots \alpha_{n}$. Consider the vertices $x_{i}=s(\alpha_{i})$ for $1 \leq i \leq n$ and $x_{n+1}=t(\alpha_{n})$. Then these
are $n+1$ vertices, thus there has to exist $i<j$ with $x_{i}=x_{j}$. Let $\omega=\alpha_{i}\cdots \alpha_{j-1}$, this is a path with source and target
$x_{i}=x_{j}$, thus a cyclic path. But then $\omega^{m}$ is a path for any natural number $m$. The path $\omega$ has length $j-i\geq1$, thus
$\omega^{m}$ has length $m(j-i)$. This shows that these paths are pairwise different.
\end{proof}
\end{lemma}

\begin{example}{(A quiver with no cycles)}\\
\[
\begin{tikzcd}
2 \arrow[rrrr, "\psi"] \arrow[rrrrddd, "\psi\rho", pos=0.3] &  &  &  &
3 \arrow[ddd, "\rho"] \\
 &  &  &  & \\
 &  &  &  & \\
1 \arrow[uuu, "\varphi"] \arrow[rrrruuu, "\varphi\psi", pos=0.3] \arrow[rrrr, "\varphi\psi\rho" '] &  &  &  & 4
\end{tikzcd}
\]
\end{example}

\begin{definition}{(Category)}\\

\noindent A \ul{category} $\mathcal{C}$ is a quiver with two further maps:

\begin{enumerate}
\renewcommand{\labelenumi}{(id)}
\item The \ul{identity map} $1_{( )}$ mapping every object $X \in\mathcal{C}_{0}$ to its \ul{identity morphism} $1_{X}$:
\[
\begin{tikzcd}[column sep=small]
\mathcal{C}_{0} \arrow[rr,"1"] &  & \mathcal{C}_{1}
\end{tikzcd}
\]
\renewcommand{\labelenumi}{($\mu$)}
\item And for any two \ul{composable} morphisms $\varphi$ and $\psi \in \mathcal{C}_{1}$, i.e. with $t(\varphi) = s(\psi)$, the
\ul{composition map} $\mu$, which maps $\varphi, \psi \in \mathcal{C}_{1}\times\mathcal{C}_{1}$ to $\mu(\varphi,\psi) \in \mathcal{C}_{1}$ which
we also write as $\varphi\psi$. 
\[
\begin{tikzcd}[column sep=small]
\mathcal{C}_{1} \times \mathcal{C}_{1} \arrow[rr,"\mu"] &  & \mathcal{C}_{1}
\end{tikzcd}
\]
\end{enumerate}

\noindent The defining properties for $1$ and $\mu$ are:
\renewcommand{\labelenumi}{(\theenumi)}
\begin{enumerate}
\item $s(1_{M}) = M = t(1_{M})$, i.e.\\
$1_{M} \in \textup{End}_{\mathcal{C}} \forall M \in \mathcal{C}$.

\item $s(\varphi\psi) = s(\varphi)$ and\\
$t(\varphi\psi) = t(\psi)$\\
for all composable morphisms $\varphi, \psi \in \mathcal{C}$.
\[
\begin{tikzcd}[column sep=small]
\mu : \textup{Hom}_{\mathcal{C}}(M,L) \times \textup{Hom}_{\mathcal{C}}(L,N) \arrow[rr] &  & \textup{Hom}_{\mathcal{C}}(M,N)
\end{tikzcd}
\]
\item \begin{minipage}{.55\textwidth} $(\varphi\psi)\rho = \varphi(\psi\rho)$ \hfill{} [associativity of composition]\end{minipage}
\begin{minipage}{.45\textwidth}\phantom{}\end{minipage}
\item \begin{minipage}{.55\textwidth} $1_{s(\varphi)}\varphi = \varphi = \varphi1_{t(\varphi)}$ \hfill{} [unit property]\end{minipage}
\begin{minipage}{.45\textwidth}\phantom{}\end{minipage}\\
The identity is a left and right \ul{unit} of the composition.
\end{enumerate}
\end{definition}

\noindent As we have seen, every category is a quiver, but in general, to become a category, a quiver is lacking identity morphisms
and the composition of morphisms. To be more precise, there is a \ul{functor} $U$ from the \ul{category of categories} $\textup{CAT}$ to the
\ul{category of quivers} $\textup{Quiv}$, called the \ul{underlying quiver} or \ul{forgetful functor}.
\[
\begin{tikzcd}
\textup{CAT} \arrow[rr,"U"] &  & \textup{Quiv}
\end{tikzcd}
\]
mapping every object $M \in \mathcal{C}_{0}$ to the same objects in $q_{0}$, mapping every arrow $\varphi \in \mathcal{C}_{1}$ to 
an arrow $a \in q_{1}$, respecting source and target, but forgetting the special role of the identity morphisms and of the composition morphisms.

\newpage
\begin{example}{(Underlying quiver)}\\

\noindent\begin{minipage}{.08\textwidth}
\phantom{}
\end{minipage}
\begin{minipage}{.37\textwidth}
\begin{tikzcd}[boxedcd={inner xsep=1.5em, inner ysep=3em}]
B \arrow[rrrr, "\psi"] \arrow[rrrrddd, "\psi\rho", pos=0.3] \arrow["1_{B}"', loop, distance=2em, in=125, out=55] &  &  &  &
C \arrow[ddd, "\rho"] \arrow["1_{C}"', loop, distance=2em, in=125, out=55]\\
 &  &  &  & \\
 &  &  &  & \\
A \arrow[uuu, "\varphi"] \arrow[rrrruuu, "\varphi\psi", pos=0.3] \arrow[rrrr, bend left, "(\varphi\psi)\rho" ', shift right=2]
\arrow[rrrr, "\varphi(\psi\rho)", bend right] \arrow["1_{A}"', loop, distance=2em, in=305, out=235] &  &  &  &
D \arrow["1_{D}"', loop, distance=2em, in=305, out=235]
\end{tikzcd}
\end{minipage}
%
\begin{minipage}{.10\textwidth}
\begin{tikzcd}
{} \arrow[r, "U", Rightarrow] & {}
\end{tikzcd}
\end{minipage}
%
\begin{minipage}{.37\textwidth}
\begin{tikzcd}[boxedcd={inner xsep=1.5em, inner ysep=3em}]
2 \arrow[rrrr, "b"] \arrow[rrrrddd, "e", pos=0.3] \arrow["h"', loop, distance=2em, in=125, out=55] &  &  &  &
3 \arrow[ddd, "c"] \arrow["i"', loop, distance=2em, in=125, out=55]\\
 &  &  &  & \\
 &  &  &  & \\
1 \arrow[uuu, "a"] \arrow[rrrruuu, "d", pos=0.3] \arrow[rrrr, bend left, "f" ', shift right=2]
\arrow[rrrr, "f", bend right] \arrow["g"', loop, distance=2em, in=305, out=235] &  &  &  &
4 \arrow["j"', loop, distance=2em, in=305, out=235]
\end{tikzcd}
\end{minipage}
\begin{minipage}{.08\textwidth}
\phantom{}
\end{minipage}\\

\noindent In the category on the left, associativity of composition guaranteed that $(\varphi\psi)\rho = \varphi(\psi\rho)$, so those two arrows
were already the same, so they are mapped to the same arrow $f = U((\varphi\psi)\rho) = U(\varphi(\psi\rho))$ in the quiver on the right.
We didn't have to draw both arrows for $f$, but since they are equal, there is still only one arrow in the hom-set $\textup{Hom}_{q}(1,4)=\{f,f\} = \{f\}$.\\
All the other identities are not preserved under the forgetful functor, e.g. $d$ doesn't know what it has to do with $a$ and $b$ apart from
$s(d) = s(a)$ and $t(d) = t(b)$. Especially the former identity arrows are now just endomorphisms with no defining property.\\
The paths $g^{2}f, gf$ and $fj^{3}$ are all different, while in the category, they all simplify to
$1_{A}1_{A}(\varphi\psi)\rho = 1_{A}(\varphi\psi)\rho = (\varphi\psi)\rho1_{D}1_{D}1_{D} =  (\varphi\psi)\rho$ due to the unit property and associativity.
\end{example}


\begin{definition}{(Ab-category)}
An \ul{Ab-category} is a category in which all homomorphism sets are abelian groups, and composition distributes over addition.\\
In other words,
A category $\mathcal{C}$ is an \ul{Ab-category} if for every pair of objects $M,N \in \mathcal{C}_{0}$, $( \textup{Hom}_{\mathcal{C}}(M,N), + )$ is
an abelian group (with the neutral element called \ul{zero morphism}), and for all morphisms $\gamma, \delta \in \textup{Hom}_{\mathcal{C}}(M,N),
\alpha, \beta \in \textup{Hom}_{\mathcal{C}}(N,L)$
\[
(\gamma + \delta)\alpha = \gamma\alpha + \delta\alpha \textup{ and }\]\[
\gamma(\alpha+\beta) = \gamma\alpha + \gamma\beta.
\]
Note that every hom-set has its own unique zero morphism. E.g. in $\textup{Mat}_{\mathbb{Q}}$ the 2-by-3 zero-matrix $0 \in \textup{Hom}(2,3)$ is different from
the 4-by-4 zero-matrix $0 \in \textup{Hom}(4,4)$.
\end{definition}

\begin{definition}{(Initial object, terminal object, zero object)}

\end{definition}

\begin{example}{}

\end{example}

\begin{definition}{(Kernel of a morphism}

\end{definition}

\begin{definition}{(Abelian category)}

\end{definition}
\begin{definition}{(k-linear category)}

\end{definition}


Quiver -> CAT: U: forget 1, forget composition

search $U^{-1}$

Beispiel für Adjunktion


Path Algebra:

% mainfile: ../main.tex

\subsection{Limit and colimit of a functor}

\begin{definition}{(Source of a functor)}
Let $D : \mathbf{I} \rightarrow \mathcal{C}$ be a functor. A \ul{source} of $D$ consists of the following data:
\begin{enumerate}
\renewcommand{\labelenumi}{(\theenumi)}
\item An object $S \in \mathcal{C}$.
\item A dependent function $s$ mapping an object $i \in \mathbf{I}$ to a morphism
$s(i) : S \rightarrow D(i)$ such that for all $i, j \in \mathbf{I}, \iota : i \rightarrow j$, we have $D(\iota) \cdot s(i) = s(j)$.
\end{enumerate}
\end{definition}

\begin{definition}{(Limit and colimit of a functor)}
Let $D : \mathbf{I} \rightarrow \mathcal{C}$ be a functor. A \ul{limit} of $D$ consists of the
following data:
\begin{enumerate}
\renewcommand{\labelenumi}{(\theenumi)}
\item A source of $D$ given by the data $(\mathrm{lim}\, D, (\lambda(i) : \mathrm{lim}\, D \rightarrow D(i))_{i\in\mathbf{I}})$.
\item A dependent function $u$ mapping every source $\tau = (T, (\tau(i) : T \rightarrow D(i))_{i \in \mathbf{I}})$ to a
morphism $u(\tau) : T \rightarrow \mathrm{lim}\, D$ such that $\lambda(i) \cdot u(\tau) = \tau(i)$ for all $i \in \mathbf{I}$.\label{itm:2}
\item For any other dependent function $v$ satisfying (\ref{itm:2}), we have $u = v$.
\end{enumerate}
A \ul{colimit} of $D$ is a limit of $D' : \mathbf{I} \rightarrow \mathcal{C}^{\mathrm{op}}$.
\end{definition}

\begin{definition}{(Limits of type \textbf{I})}
Let $\mathbf{I}$ be a category. We say a category $\mathcal{C}$ \ul{has limits of type} $\mathbf{I}$ if it is
equipped with a dependent function $\lambda$ mapping a functor $D : \mathbf{I} \rightarrow \mathcal{C}$ to a limit
$(\mathrm{lim}\, D, \lambda_{D}, u_{D})$ of $D$.
We say $\mathcal{C}$ \ul{has colimits of type} $\mathbf{I}$ if $\mathcal{C}^{\mathrm{op}}$ has limits of that type.
\end{definition}

\begin{example}\label{ex:limits}
Depending on \textbf{I} some limits and colimits have special names:
\begin{center}
\begin{tabular}{c|c|c}
\textbf{I} & limit & colimit \\
\hline
$\emptyset$ & terminal object & initial object \\
a set & direct product & coproduct \\
$\cdot \rightarrow \cdot \leftarrow \cdot$ & pullback & - \\
$\cdot \leftarrow \cdot \rightarrow \cdot$  & - & pushout \\
$ \cdot \rightrightarrows \cdot$ & equalizer & coequalizer
\end{tabular}
\end{center}
\end{example}

In the following section, we give explicit definitions for the limits in \ref{ex:limits} and examples for categories with such limits.

\subsection{Examples for limits and colimits}

\begin{definition}{(Product, coproduct)}\label{def:prod_coprod}
Let $I$ be an index set and $\{A_{i}\}_{i\in I}$ a family of objects in a category $\mathcal{C}$.
\begin{enumerate}
\renewcommand{\labelenumi}{(prod)}
\item The \ul{product} of the family $\{A_{i}\}_{i\in I}$ is an object $\invamalg A_{i}$ together with a family of morphisms
\[
\{ \pi_{i} : \invamalg A_{i} \twoheadrightarrow A_{i} \}
\]
called \ul{projections}, such that the following universal property is satisfied:\\
For any object $M \in \mathcal{C}_{0}$ and any family $\{ \varphi_{i} : M \rightarrow A_{i} \}_{i\in I}$ of morphisms, there exists
a unique morphism $\varphi : M \rightarrow \invamalg A_{i}$ called the \ul{product morphism} such that
\[
\varphi \pi_{i} = \varphi_{i} \, \forall i \in I.
\]
\begin{tikzcd}
                                                                                                                            &  &                                                          & A_{1} \\
M \arrow[rrru, "\varphi_{1}", bend left] \arrow[rrrd, "\varphi_{2}"', bend right] \arrow[rr, "\exists^{1} \varphi", dashed] &  & A_{1}\invamalg A_{2} \arrow[ru, "\pi_{1}"] \arrow[rd, "\pi_2"] &       \\
                                                                                                                            &  &                                                          & A_{2}
\end{tikzcd}
\renewcommand{\labelenumi}{(coprod)}
\item The dual notion to product is the \ul{coproduct} of the family $\{A_{i}\}_{i\in I}$, that is an object $\amalg A_{i}$ together with
a family of morphisms
\[
\{ \iota_{i} : A_{i} \hookrightarrow \amalg A_{i} \}
\]
called \ul{coprojections} or sometimes \ul{injections} or \ul{inclusions}, such that the following universal property is satisfied:\\
For any object $M \in \mathcal{C}$ and any family $\{ \psi_{i} : A_{i} \rightarrow M \}$ of morphisms, there exists a unique
morphism $\psi : \amalg A_{i} : M$ called the \ul{coproduct morphism} such that
\[
\iota_{i} \psi = \psi_{i} \, \forall i \in I.
\]
\begin{tikzcd}
  &  &                                                           & A_{1} \arrow[llld, "\psi_{1}"', bend right] \arrow[ld, "\iota_{1}"'] \\
M &  & A_{1}\amalg A_{2} \arrow[ll, "\exists^{1} \psi"', dashed] &                                                                      \\
  &  &                                                           & A_{2} \arrow[lllu, "\psi_{2}", bend left] \arrow[lu, "\iota_2"]     
\end{tikzcd}
\end{enumerate}
\end{definition}

\begin{definition}{(Terminal object, initial object, zero object)}\label{def:init_term_zero_object}
\renewcommand{\labelenumi}{(\theenumi)}
\begin{enumerate}
\item A \ul{terminal object} $T$ in a category $\mathcal{C}$ is an object such that $\textup{Hom}_{\mathcal{C}}(-,T)$ is a singleton.
\item An \ul{initial object} $I$ in a category $\mathcal{C}$ is an object such that $\textup{Hom}_{\mathcal{C}}(I,-)$ is a singleton.
\item An object is a \ul{zero object} if it is both initial and terminal.
\end{enumerate}
\end{definition}

\begin{definition}{(Zero morphism)}\label{def:zero_morphism}\\
A \ul{zero morphism} in a category with a zero object $Z$ is a morphism factoring over $Z$, i.e. $\varphi : M \rightarrow N$ is called a zero
morphism, if\\
\begin{minipage}{.35\textwidth}
\begin{tikzcd}
M \arrow[rr, "\varphi"] \arrow[rd, "\varphi_{1}"] &                              & N \\
                                                  & Z \arrow[ru, "\varphi_{2}"'] &  
\end{tikzcd}
\end{minipage}
\begin{minipage}{.65\textwidth}
$\exists \varphi_{1} : M \rightarrow Z, \varphi_{2} : Z \rightarrow N$\\
such that $\varphi = \varphi_{1}\varphi_{2}$.
\end{minipage}
\end{definition}

%%% insert definition of Ab-Category.


\begin{definition}{(Direct sum)}\label{def:direct_sum}
Let $\mathcal{C}$ be an Ab-category. Let $I$ be an index set and $\{A_{i}\}_{i\in I}$ a family of objects in $\mathcal{C}$.
A \ul{direct sum} consists of the following data:
\begin{itemize}
\item an object $S$,
\item a family of morphisms $\pi = \{ \pi_{i} : S \rightarrow S_{i} \}_{i\in I}$,
\item a family of morphisms $\iota = \{ \iota_{i} : S_{i} \rightarrow S \}_{i\in I}$,
\item a dependent function $u_{\text{in}}$ mapping every family $\tau = \{ \tau_{i} : T \rightarrow S_{i} \}_{i\in I}$ to a morphism
$u_{\text{in}}(\tau) : T \rightarrow S$ such that $u_{\text{in}}(\tau) \pi_{i} \sim \tau_{i}$ for all $i \in I$,
\item a dependent function $u_{\text{out}}$ mapping every family $\rho = \{ \rho_{i} : S_{i} \rightarrow R \}_{i\in I}$ to a morphism
$u_{\text{out}}(\rho) : S \rightarrow R$ such that $\iota_{i} u_{\text{out}}(\rho) \sim \rho_{i}$ for all $i \in I$,
\end{itemize}
such that
\begin{itemize}
\item $\sum_{i\in I} \iota_{i} \pi_{i} \sim 1_{S}$,
\item $\pi_{j} \iota_{i} \sim \delta_{i, j} =  \begin{cases}
            1_{S_{i}} & \text{ if } i = j  \\
            0_{ij} & \text{ if } i \neq j
        \end{cases}$,
\end{itemize}
where $\delta_{i, j} \in \mathrm{Hom}(S_{i}, S_{j})$ is the identity if $i = j$, and the zero morphism $0_{ij} = 0_{S_{i}, S_{j}}$ otherwise.
\end{definition}

\begin{example}{($\kmat$ has direct sums)}\label{ex:kmat_has_direct_sum}
The matrix category $\kmat$ is an Ab-category. The direct sum of two ($I = \{1,2\}$) natural numbers $S_{1} = m, S_{2} = n \in \kmat_{0}$ is
\begin{itemize}
\item the object $S = m+n$,
\item the two morphisms $\pi_{1} : m+n \rightarrow m$, $\pi_{2} : m+n \rightarrow n$ which are $(m+n) \times m$- and $(m+n) \times n$-matrices.
\item the two morphisms $\iota_{1} : m \rightarrow m+n$, $\iota_{2} : n \rightarrow m+n$ which are $m \times (m+n)$- and $n \times (m+n)$-matrices.
\item a family of two morphisms $\tau = \{ \tau_{1} : t \rightarrow m, \tau_{2} : t \rightarrow n\}$ gets mapped to a $t \times (m+n)$-matrix
$u_{\text{in}}(\tau) : t \rightarrow m+n$ with $u_{\text{in}}(\tau) \pi_{1} = \tau_{1}$ and $u_{\text{in}}(\tau) \pi_{2} = \tau_{2}$.
\item a family of two morphisms $\rho = \{ \rho_{1} : m \rightarrow r, \rho_{2} : n \rightarrow r \}$ gets mapped to a $(m+n) \times r$-matrix
$u_{\text{out}}(\rho) : m+n \rightarrow r$ with $\iota_{1} u_{\text{out}} = \rho_{1}$ and $\iota_{2} u_{\text{out}} = \rho_{2}$.
\end{itemize}
\end{example}

\begin{example}{(number example)}
$m = 3, n = 5$
\begin{align*}
\pi_{1} = \begin{pmatrix}
1 \ampersand 0 \ampersand 0 \\
0 \ampersand 1 \ampersand 0 \\
0 \ampersand 0 \ampersand 1 \\
0 \ampersand 0 \ampersand 0 \\
0 \ampersand 0 \ampersand 0 \\
0 \ampersand 0 \ampersand 0 \\
0 \ampersand 0 \ampersand 0 \\
0 \ampersand 0 \ampersand 0
\end{pmatrix},
\pi_{2} = \begin{pmatrix}
0 \ampersand 0 \ampersand 0 \ampersand 0 \ampersand 0 \\
0 \ampersand 0 \ampersand 0 \ampersand 0 \ampersand 0 \\
0 \ampersand 0 \ampersand 0 \ampersand 0 \ampersand 0 \\
1 \ampersand 0 \ampersand 0 \ampersand 0 \ampersand 0 \\
0 \ampersand 1 \ampersand 0 \ampersand 0 \ampersand 0 \\
0 \ampersand 0 \ampersand 1 \ampersand 0 \ampersand 0 \\
0 \ampersand 0 \ampersand 0 \ampersand 1 \ampersand 0 \\
0 \ampersand 0 \ampersand 0 \ampersand 0 \ampersand 1
\end{pmatrix}, 
\begin{array}{rr}
\iota_{1} &= \begin{pmatrix}
1 \ampersand 0 \ampersand 0 \ampersand 0 \ampersand 0 \ampersand 0 \ampersand 0 \ampersand 0 \\
0 \ampersand 1 \ampersand 0 \ampersand 0 \ampersand 0 \ampersand 0 \ampersand 0 \ampersand 0 \\
0 \ampersand 0 \ampersand 1 \ampersand 0 \ampersand 0 \ampersand 0 \ampersand 0 \ampersand 0
\end{pmatrix} \\
\\
\iota_{2} &= \begin{pmatrix}
0 \ampersand 0 \ampersand 0 \ampersand 1 \ampersand 0 \ampersand 0 \ampersand 0 \ampersand 0 \\
0 \ampersand 0 \ampersand 0 \ampersand 0 \ampersand 1 \ampersand 0 \ampersand 0 \ampersand 0 \\
0 \ampersand 0 \ampersand 0 \ampersand 0 \ampersand 0 \ampersand 1 \ampersand 0 \ampersand 0 \\
0 \ampersand 0 \ampersand 0 \ampersand 0 \ampersand 0 \ampersand 0 \ampersand 1 \ampersand 0 \\
0 \ampersand 0 \ampersand 0 \ampersand 0 \ampersand 0 \ampersand 0 \ampersand 0 \ampersand 1
\end{pmatrix}
\end{array}
\end{align*}
\begin{minipage}[t]{.5\textwidth}
and for $t = 4$, $\tau = \{\tau_{1}, \tau_{2}\}$ defined as
\begin{align*}
\tau_{1} = \begin{pmatrix}
1 \ampersand 2 \ampersand 2 \\
4 \ampersand 3 \ampersand 1 \\
0 \ampersand 1 \ampersand 0 \\
1 \ampersand 2 \ampersand 1
\end{pmatrix},
\tau_{2} = \begin{pmatrix}
1 \ampersand 1 \ampersand 2 \ampersand 2 \ampersand 3 \\
3 \ampersand 4 \ampersand 4 \ampersand 5 \ampersand 5 \\
6 \ampersand 6 \ampersand 7 \ampersand 7 \ampersand 8 \\
8 \ampersand 9 \ampersand 9 \ampersand 10 \ampersand 10
\end{pmatrix}
\end{align*}
we get the matrix
\begin{align*}
u_{\text{in}}(\tau) = \begin{pmatrix}
1 \ampersand 2 \ampersand 2 \ampersand 1 \ampersand 1 \ampersand 2 \ampersand 2 \ampersand 3 \\
4 \ampersand 3 \ampersand 1 \ampersand 3 \ampersand 4 \ampersand 4 \ampersand 5 \ampersand 5 \\
0 \ampersand 1 \ampersand 0 \ampersand 6 \ampersand 6 \ampersand 7 \ampersand 7 \ampersand 8 \\
1 \ampersand 2 \ampersand 1 \ampersand 8 \ampersand 9 \ampersand 9 \ampersand 10 \ampersand 10
\end{pmatrix}
\end{align*}
\end{minipage}
\begin{minipage}[t]{.5\textwidth}
and for $r = 2$, $\rho = \{\rho_{1}, \rho_{2}\}$ we get the matrix
\begin{align*}
\begin{array}{rr}
\rho_{1} &= \begin{pmatrix}
0 \ampersand 1 \\
1 \ampersand 1 \\
2 \ampersand 2
\end{pmatrix} \\
\\
\rho_{2} &= \begin{pmatrix}
4 \ampersand 5 \\
-7 \ampersand 0 \\
0 \ampersand 5 \\
0 \ampersand 0 \\
1 \ampersand 1
\end{pmatrix}
\end{array}
u_{\text{in}}(\rho) = \begin{pmatrix}
0 \ampersand 1 \\
1 \ampersand 1 \\
2 \ampersand 2 \\
4 \ampersand 5 \\
-7 \ampersand 0 \\
0 \ampersand 5 \\
0 \ampersand 0 \\
1 \ampersand 1
\end{pmatrix}
\end{align*}
\end{minipage}
\end{example}


\subsection{Monomorphisms and epimorphisms}

\subsection{Kernel and cokernel; image and coimage}

\subsection{Direct sum and direct product}

% mainfile: ../main.tex

\section{Functors and natural transformations}

\subsection{Functors map one category to another}

\begin{example}{(Identity Functor)}
\end{example}
\begin{example}{(Forgetful functor)}
\end{example}

\begin{definition}{(full functor; faithful functor)}

\end{definition}

\subsection{Natural transformations are morphisms between functors}

% mainfile: ../main.tex

\section{Adjunctions}

\subsection{Universal objects}

\subsection{Forgetting the forgetful functor: Free constructions}

% mainfile: ../main.tex

\section{Yoneda's Lemma: Completion and cocompletion of a category}

\subsection{Embedding categories}

\begin{lemma}{(Yoneda's Lemma)}

\begin{proof}

\end{proof}
\end{lemma}
\[
\begin{pmatrix}1 \ampersand 2 \ampersand 3 \ampersand 4\end{pmatrix}
\begin{pmatrix} 1\to5, \ampersand 2\to6 \\ 3\to7, \ampersand 4\to8 \end{pmatrix}
\begin{pmatrix}6 \ampersand 8\end{pmatrix}
\begin{pmatrix}5\to9\\6\to10\\7\to11\\8\to12\end{pmatrix}
\begin{pmatrix}9\ampersand10\end{pmatrix}\begin{pmatrix}11\ampersand12\end{pmatrix}
\begin{pmatrix}9\to13, \ampersand 10\to14\\11\to15 \ampersand 12\to16\end{pmatrix}
\textup{id}
\]

\begin{tikzcd}
{\{1,2,3,4\}} \arrow["{(1,2,3,4)}"', loop, distance=2em, in=125, out=55] \arrow[rr, "{\begin{pmatrix} 1\to5, 2\to6, \\ 3\to7, 4\to8 \end{pmatrix}}"] \arrow[dd] \arrow[rrdd] &  & {\{5,6,7,8\}} \arrow["{(6,8)}"', loop, distance=2em, in=125, out=55] \arrow[dd, " \begin{pmatrix}5\to9\\6\to10\\7\to11\\8\to12\end{pmatrix}"', bend left] \arrow[lldd] \\
                                                                                                                                                                             &  &                                                                                                                                                                        \\
{\{13,14,15,16\}} \arrow["\textup{id}"', loop, distance=2em, in=305, out=235]                                                                                                &  & {\{9,10,11,12\}} \arrow["{(9,10)(11,12)}"', loop, distance=2em, in=305, out=235] \arrow[ll, "{\begin{pmatrix}9\to13, 10\to14,\\ 11\to 15, 12\to16\end{pmatrix}}"]     
\end{tikzcd}

Dimension of the (quotient of the) path algebra is 43.
Sum of all dimensions of the yoneda projectives on each objects is 43.

% mainfile: ../main.tex

\section{Functors and natural transformations}

\subsection{Functors act on objects and morphisms of a category}

\subsection{Natural transformations are morphisms between functors}

\subsection{Representations are Functors into a matrix category}

Yonedas Einbettungs-Lemma: Fehlende Limiten bzw. Kolimiten exitieren nach der Einbettung.

Einbettung in Kategorien, die mehr Limiten haben als die Zielkategorie.

"(Ko-)Vervollständigung" der Kategorie (Completion / Cocompletion)

Quiver = unvollständige Struktur einer Kategorie
Erzeugendensystem einer Kategorie.

K-linearer Abschluss einer Kategorie

Pfadalgebra = Kategorien-Algebra
path algebra = 1 Object, welches eine Algebra ist. Dabei verliert man wieder die Informationen über die
mehreren Objekte.

So wie Menge ein Erz-system eines Monoid.

% mainfile: ../main.tex

\section{Relations of the Algebroid}

\subsection{Relations of endomorphisms}

% A forest on a cycle
\begin{tikzpicture}[x=0.5cm,y=0.5cm]
\tikzstyle{cblack}=[circle, fill=black, scale=0.5]

%Nodes
\foreach \place/\x in {{(0,0)/0}, {(-4.5,0)/1}, {(-7,-3)/2}, {(-4.5,-6)/3},
  {(0,-6)/4}, {(2.5,-3)/5},
  {(-4.5,3)/6}, {(-7.5,6)/7}, {(-4.5,6)/8},
  {(0,3)/9}, {(0,6)/10}, {(0,9)/11},
  {(3,3)/12}, {(3,6)/13}, {(3,9)/14},
  {(7.5,6)/15}, {(7.5,9)/16}, {(7.5,12)/17}, {(10.5,9)/18}}
\node[cblack] (a\x) at \place {};

%Arrows
\foreach \i in {0,1,2,3,4,5}
{
  \pgfmathtruncatemacro\result{Mod(\i+1,6)}%
  \draw[->] (a\i) -> (a\result);
}
\path[->] (a7) edge (a6); 
\path[->] (a8) edge (a6) edge (a1);
\path[->] (a11) edge (a10) edge (a9) edge (a0);
\path[->] (a14) edge (a13) edge (a12); 
\path[->] (a12) edge (a0);
\path[->] (a18) edge (a15) (a15) edge (a12);
\path[->] (a17) edge (a16) edge (a15);
%\path[->] (a\x) edge (a\y);

\end{tikzpicture}
%\cite{facchini_2019}
\begin{lemma}[$\sigma$-Lemma]
Let $\mathcal{C}$ be a finite concrete category. Then for each object $M \in \mathcal{C}_{0}$ the set
$\textup{End}_{\mathcal{C}}(M)$ is a monoid and for each endomorphism $f \in \textup{End}_{\mathcal{C}}(M)$
there exist $m,n \in \mathbb{N}$ such that $f^{(m+n)}=f^{m}$. If $m = 0$ and $n \geq 1$ then $f$ is bijective with $f^{-1} = f^{n-1}$.
\begin{proof}
The properties of a monoid are precisely the associativity of composition and the unit property from \ref{associativity_of_composition} and \ref{unit_property}.
Since $\abs{\textup{End}_{\mathcal{C}}(M)}<\infty$ there are only finitely many endomorphisms $f_{1},\dots, f_{N} \in \textup{End}_{\mathcal{C}}(M)$.
Let $\{f^{k} | k \in \mathbb{N} \} \subset \textup{End}_{\mathcal{C}}(M)$, i.e. there is a function 
$\{f^{k} | k \in \mathbb{N}\} \rightarrow \{f_{j} | j \in \{1,\dots,N\}\}; f^{k} \mapsto f_{j}$ not necessarily surjective and 
by the pigeonhole principle highly non injective, since $\abs{\mathbb{N}}>\abs{\textup{End}_{\mathcal{C}}(M)}$.
Let $m := Min \{ k \in \mathbb{N}| f^{k} =  f_{j} \}$

\begin{minipage}{.45\textwidth}\phantom{}\end{minipage}
\end{proof}
\end{lemma}

Beschreibung der Algorithmen

WeakDirectSumDecomposition <-- Tiefensuche.
Objekte (Funktoren) in indecomposable Functors.


\section{Category}

\begin{definition}{(Quiver)}\\
A \ul{quiver} $A$ consists of a class of  \ul{objects} (or vertices) $A_{0} = \textup{Obj} A$ and 
a class of \ul{morphisms} (or arrows) $A_{1} = \textup{Mor} A$ together with two defining maps
\[
\begin{tikzcd}[column sep=small]
{s,t\colon A_{1}} \arrow[rr, shift left = 0.7ex] \arrow[rr, shift right = 0.7ex] & & A_{0}
\end{tikzcd}
\]
$s$ called \ul{source} and $t$ called \ul{target}.\footnote{Some authors use maps $t, h$ for $tail$ and $head$ instead of source and target, defining the arrows to go from the tail to the head. This use of $t$ as the starting point instead of the end target as in our definition can lead to some confusion.}\\
\noindent We write $\textup{Hom}_A (M,N)$ (sometimes also $A(M,N)$) for the fiber $(s,t)^{-1} (\{(M,N)\})$ of the product
map $(s,t) : A_{1} \rightarrow A_{0} \times A_{0}$ over the pair $(M,N) \in A_{0} \times A_{0}$.\\
This is the class of all morphisms with source $= M$ and target $= N$.\\
For a morphism $\varphi \in \textup{Hom}_{A}(M,N)$ we write
\[
\begin{tikzcd}[column sep=small]
\varphi : M \arrow[rr] & & N
\end{tikzcd}
or
\begin{tikzcd}[column sep=small]
M \arrow[rr, "\varphi"] & & N
\end{tikzcd}
\]
Clearly $A_{1}$ is the disjoint union $\bigcup\limits^{\bigcdot}_{M,N \in A_{0}} \textup{Hom}_{A}(M,N) = A_{1}$. As usual we define 
$\textup{End}_{A}(M):= \textup{Hom}_{A}(M,M)$.
\end{definition}



\begin{definition}{(Category)}\\
A \ul{category} $\mathfrak{A}$ is a quiver with two further defining maps
\[
\begin{tikzcd}[column sep=small]
A_{0} \arrow[rr,"1"] &  & A_{1} &  & A_{1} \times_{s,A_{0},t} A_{1} \arrow[ll,"\mu"]
\end{tikzcd}
\]
\end{definition}

\begin{example}\label{representation}{(Representation of a concrete category)}\\
\begin{center}
\begin{tikzcd}[boxedcd={inner sep=1pt}]
                                                                                           &  &  &  & \\
                                                                                              &  &                                                                       \\
                                                                                              &  &                                                                       \\
                                                                                              &  &                                                                       \\
&  & 5 \arrow[rr, "{\begin{pmatrix} 
0\ampersand1\ampersand0\ampersand0\\
0\ampersand0\ampersand1\ampersand0\\
0\ampersand0\ampersand0\ampersand0\\
0\ampersand1\ampersand0\ampersand1\\
0\ampersand0\ampersand1\ampersand0
\end{pmatrix}}"]
\arrow["{\begin{pmatrix} 
1\ampersand 1\ampersand 0\ampersand 0\ampersand 0\\
0\ampersand 1\ampersand 1\ampersand 0\ampersand 0\\
0\ampersand 0\ampersand 1\ampersand 0\ampersand 0\\
0\ampersand 0\ampersand 0\ampersand 1\ampersand 1\\
0\ampersand 0\ampersand 0\ampersand 0\ampersand 1 
\end{pmatrix}}"', loop, distance=2em, in=305, out=235]             &  & 
4 \arrow["{\begin{pmatrix}
1\ampersand1\ampersand0\ampersand0\\
0\ampersand1\ampersand1\ampersand0\\
0\ampersand0\ampersand1\ampersand0\\
0\ampersand0\ampersand0\ampersand1
\end{pmatrix}}"', loop, distance=2em, in=305, out=235]  &  &         \\
                                                                                              &  &                                                                       \\
                                                                                              &  &                                                                       \\
                                                                                              &  &                                                                       \\   
                                                                                              &  &                                                                       \\   
\end{tikzcd}
\end{center}
\begin{center}
\begin{tikzcd}
                                                                                              & {} &                                                                       \\
                                                                                              & {} &                                                                       \\
                                                                                              & {} \arrow["nine",u, Rightarrow] &                                                                       \\
\end{tikzcd}
\end{center}
\begin{center}
\begin{tikzcd}[boxedcd={inner sep=1pt}]
                                                                                              &  &                                                                       \\
&  1 \arrow["a"', loop, distance=2em, in=305, out=235] \arrow[rr, "b"] \arrow[rr] \arrow[rr]     &  & 
2 \arrow["c"', loop, distance=2em, in=305, out=235]  &                   \\
                                                                                              &  &                                                                       \\
\end{tikzcd}
\end{center}
\begin{center}
\begin{tikzcd}[boxedcd={inner sep=1pt}]
                                                                                              &  &                                                                       \\
&  {\{1,2,3\}} \arrow["{(2,1,3)}"', loop, distance=2em, in=305, out=235] 
\arrow[rr, "{(4,5,6)}"] &  & 
{\{4,5,6\}} \arrow["{(5,6,4)}"', loop, distance=2em, in=305, out=235]  & \\
                                                                                              &  &                                                                       \\
\end{tikzcd}
\end{center}

\[
F(a) \eta_{1} = \eta_{1} G(a)\\
F(b) \eta_{2} = \eta_{1} G(b)
\]
\end{example}

\section{$\mathbb{K}$-linear Category (Algebroid)}

Group: Category with one object.

Groupoid: A small category in which every morphism is an isomorphism.

Algebroid

EmbeddingOfSumOfImages

What is an Algebroid? Bialgebroid?

\section{Additive Category}



\section{Abelian Category}

\section{The Category of Categories}

\section{The Categories of Functors}

\section{The Representation of a Category}

\section{Representation}

Grundidee von FunctorCategory

Standard-Monoidale Struktur von der Zielkategorie z.B. TensorUnit(C)

\section{Algorithms}
\lstinputlisting[numbers=left,firstnumber=60,firstline=60,lastline=142]{\pkgpath/catreps/gap/CatRepsWithCAP.gi}

\begin{algorithm}\capstart
    \caption{\texttt{RightQuiverFromConcreteCategory}}\label{algo:RightQuiverFromConcreteCategory}
	\SetKwInput{Input}{Input~}
	\SetKwInput{Output}{Output~}
	\Input{~a finite concrete category $C$ with $n$ objects}
	\Output{~the right quiver $q(n)$}
	\BlankLine
	let $Obj$ be the set of objects of $C$\;
	let $n := Length(Obj)$\;
	let $gMor$ be the set of generating morphisms of $C$\;
	let $A$ be the empty set and let $i := 1$\;
	\ForEach{morphism $mor$ in $gMor$}{
	    let $A_{i,1}$ be the position of $Source( mor )$ in $Obj$\;
	    let $A_{i,2}$ be the position of $Range( mor )$ in $Obj$\;
	    let $i := i+1$\;
	}
	\BlankLine
	let $q$ be the right quiver with vertices $\{1,\dots,n\}$ and arrows $A$.
	\BlankLine
	\Return q\;
\end{algorithm}

We want the endomorphism relations so that the path algebra is finite-dimensional and we
get a finite Gröbner basis.

\begin{algorithm}\capstart
    \caption{\texttt{RelationsOfEndomorphisms}}\label{algo:RelationsOfEndomorphisms}
	\SetKwInput{Input}{~Input}
	\SetKwInput{Output}{~Output}
	\Input{~a commutative ring $k$ and a finite concrete category $C$}
	\Output{~the endomorphism relations of the category $C$}
	\BlankLine
	let $q := \texttt{RightQuiverFromConcreteCategory}(C)$\;
	let $kq$ be the path algebra generated by $k$ and $q$\;
	let $gMor$ be the set of generating morphisms of $C$\;
	let $A := Arrows(q)$\;
	let $relsEndo$ be the empty set\;
	\ForEach{$i = 1, \dots, Length(gMor)$}{
	    let $mor := gMor_i$
	    \If{$mor$ is not an endomorphism}{
		continue\;
	    }
	    let $m := 0$ and let $powers$ be the empty set\;
	    let $foundEqual$ be false\;
	    \While{$mor^{m}\nin powers$}{
		let $n := 1$\;
		\While{$\neg foundEqual$}{
		    \If{$mor^{(m+n)} = mor^{m}$}{
		    	Add the relation $kq.(A_{i})^{(m+n)}-kq.(A_{i})^{m}$ to relsEndo\;
		    	foundEqual := true\;
		    }
		    n := n+1\;
		}
		Add $mor^{m}$ to powers\;
		m := m+1\;
	    }
	}
	\Return{relsEndo}\;
\end{algorithm}

Proof that algorithm is correct
Proof that it terminates.

Wir haben BasisOfExternalHom benutzt um Decompose in CAP umzusetzen um EmbeddingOfSubRepresentation umzusetzen um
WeakDirectSumDecomposition umzusetzen.

\begingroup
     \parindent 0pt
     \parskip 2ex
     \def\enotesize{\normalsize}
     \theendnotes
\endgroup 

\input{bib/sources.bib}

\end{document}