\documentclass{article}

%% maybe some definitions are still here that I want in the definitions_packages.tex file
%% for now this preamble remains, in case I need some more definitions.

\usepackage[a4paper,%
            left=2.5cm,right=2.5cm,top=2.5cm,bottom=2.5cm,%
            footskip=.6cm]{geometry}

\def\changemargin#1#2{\list{}{\rightmargin#2\leftmargin#1}\item[]}
\let\endchangemargin=\endlist

\usepackage[utf8]{inputenc}
\usepackage{fancyvrb}

%%% For captions and references
\usepackage{hyperref}
\usepackage[figure]{hypcap}
\newcommand{\Algoref}[1]{%
	\hyperref[algo:#1]{Algorithm~\ref*{algo:#1}}%
}
\newcommand{\algoref}[1]{%
	\hyperref[algo:#1]{Algorithm~\ref*{algo:#1}}%
}
\newcommand{\Funcref}[1]{%
	\hyperref[func:#1]{Function~\ref*{func:#1}}%
}
\newcommand{\funcref}[1]{%
	\hyperref[func:#1]{\texttt{#1}}%
}

%%% For footnotes at end of text
\usepackage{endnotes}
%\let\footnote{\endnote}
%% Heading of endnotes section
\renewcommand*{\notesname}{Annotations}

\makeatletter
\def\enoteheading{\section{\notesname
\@mkboth{\MakeUppercase{\notesname}}{\MakeUppercase{\notesname}}}%
\mbox{}\par\vskip-\baselineskip}
\makeatother

%%% For switching languages in quotes
\usepackage[english]{babel}
%\usepackage[english, german]{babel} %% makes troubles

%%% For quotation
%% english guillemets have to be custom defined in /tex/latex/csquotes/csquotes.cfg
\usepackage[english = guillemets, autostyle = true,autopunct,csdisplay = true]{csquotes}

%%% For proper underline
\usepackage{soul}
%\setuldepth{gjpqy}
%\setuldepth\strut
\setuldepth{-1}

%%% Color
\usepackage{xcolor}
\usepackage{color}
\definecolor{FireBrick}{rgb}{0.5812,0.0074,0.0083}
\definecolor{RoyalBlue}{rgb}{0.0236,0.0894,0.6179}
\definecolor{RoyalGreen}{rgb}{0.0236,0.6179,0.0894}
\definecolor{RoyalRed}{rgb}{0.6179,0.0236,0.0894}
\definecolor{LightBlue}{rgb}{0.8544,0.9511,1.0000}
\definecolor{Black}{rgb}{0.0,0.0,0.0}

\definecolor{linkColor}{rgb}{0.0,0.0,0.554}
\definecolor{citeColor}{rgb}{0.0,0.0,0.554}
\definecolor{fileColor}{rgb}{0.0,0.0,0.554}
\definecolor{urlColor}{rgb}{0.0,0.0,0.554}
\definecolor{promptColor}{rgb}{0.0,0.0,0.589}
\definecolor{brkpromptColor}{rgb}{0.589,0.0,0.0}
\definecolor{gapinputColor}{rgb}{0.589,0.0,0.0}
\definecolor{gapoutputColor}{rgb}{0.0,0.0,0.0}

%%  for a long time these were red and blue by default,
%%  now black, but keep variables to overwrite
\definecolor{FuncColor}{rgb}{0.0,0.0,0.0}
%% strange name because of pdflatex bug:
\definecolor{Chapter }{rgb}{0.0,0.0,0.0}
\definecolor{DarkOlive}{rgb}{0.1047,0.2412,0.0064}

%% command for ColorPrompt style examples
\newcommand{\gapprompt}[1]{\color{promptColor}{\bfseries #1}}
\newcommand{\gapbrkprompt}[1]{\color{brkpromptColor}{\bfseries #1}}
\newcommand{\gapinput}[1]{\color{gapinputColor}{#1}}

%%% For source code listings
\usepackage{listings}[2013/08/05]
\input{pfad.tex}
%\lstloadlanguages{GAP} %% not needed in new listings version

%%% For algorithm styles
\usepackage[linesnumbered,ruled]{algorithm2e}

%%% Math theorem styles
\usepackage{amsthm}

\newtheorem{thm}{Theorem}[subsection]
\newtheorem{lemma}[thm]{Lemma}
\theoremstyle{definition}
\newtheorem{definition}[thm]{Definition}
\newtheorem{remark}[thm]{Remark}
\newtheorem{example}[thm]{Example}


%%% For Math
\usepackage{amsmath}
\usepackage{amsfonts}
\usepackage{amsbsy}
%%\usepackage{amsthm} (doubled)
\usepackage{amssymb}
\usepackage{mathtools}
\usepackage{commath}
\usepackage[sc,osf]{mathpazo}

%%% For arrows and categories
%\usepackage[all]{xy} %%not used anymore
\usepackage{tikz-cd}

%%% For calculations and loops inside tikz and latex
\usepackage{calc}
\usepackage{pgffor}

\newcounter{modresult}
\newcommand*{\themodulo}[2]{%
\setcounter{modresult}{%
#1-(#1/#2)*#2%
}%
#1 mod #2 = \themodresult\par
}

%%% For matrices
\let\ampersand =&

%%% Math operators bold
%\newcommand{\Category}{Category}
% just use /textup{#1} inside math environment instead of redefining every math operator.

%%% tikz
\usetikzlibrary{positioning}

%%% For dotted box around diagrams
\tikzcdset{
    boxedcd/.style={
        every matrix/.append style={
            draw=black,
            dotted,
            rounded corners,
            #1
        },
    },
}

%%% For dotted arrows in math and in text
%% dottedrightarrow
\makeatletter
\newbox\dottedrightarrow@box
\setbox\dottedrightarrow@box\hbox
  {%
    \begin{tikzpicture}
      \draw[dotted,->] (0,0) -- (1.5em,0);
    \end{tikzpicture}%
  }
\newcommand*\dottedrightarrow
  {\relax\ifmmode\expandafter\dottedrightarrow@m\else\expandafter\dottedrightarrow@t\fi}
\newcommand*\dottedrightarrow@t[1][1.5em]
  {\resizebox{#1}{!}{\raisebox{.5ex}{\usebox\dottedrightarrow@box}}}
\newcommand*\dottedrightarrow@m[1][]
  {%
    \if\relax\detokenize{#1}\relax
      \mathchoice% values are trial and error based\ldots
        {\dottedrightarrow@t}
        {\dottedrightarrow@t}
        {\dottedrightarrow@t[1.1em]}
        {\dottedrightarrow@t[0.9em]}%
    \else
      \dottedrightarrow@t[#1]%
    \fi
  }
\makeatother
\let\olddottedrightarrow\dottedrightarrow
\renewcommand{\dottedrightarrow}{\raisebox{-.2em}{\olddottedrightarrow}}

%% dottedleftarrow
\makeatletter
\newbox\dottedleftarrow@box
\setbox\dottedleftarrow@box\hbox
  {%
    \begin{tikzpicture}
      \draw[dotted,<-] (0,0) -- (1.5em,0);
    \end{tikzpicture}%
  }
\newcommand*\dottedleftarrow
  {\relax\ifmmode\expandafter\dottedleftarrow@m\else\expandafter\dottedleftarrow@t\fi}
\newcommand*\dottedleftarrow@t[1][1.5em]
  {\resizebox{#1}{!}{\raisebox{.5ex}{\usebox\dottedleftarrow@box}}}
\newcommand*\dottedleftarrow@m[1][]
  {%
    \if\relax\detokenize{#1}\relax
      \mathchoice% values are trial and error based\ldots
        {\dottedleftarrow@t}
        {\dottedleftarrow@t}
        {\dottedleftarrow@t[1.1em]}
        {\dottedleftarrow@t[0.9em]}%
    \else
      \dottedleftarrow@t[#1]%
    \fi
  }
\makeatother
\let\olddottedleftarrow\dottedleftarrow
\renewcommand{\dottedleftarrow}{\raisebox{-.2em}{\olddottedleftarrow}}

%%% For some big dots (they still don't look very big)
\makeatletter
\newcommand*{\bigcdot}{}% Check if undefined
\DeclareRobustCommand*{\bigcdot}{%
  \mathbin{\mathpalette\bigcdot@{}}%
}
\newcommand*{\bigcdot@scalefactor}{.5}
\newcommand*{\bigcdot@widthfactor}{1.15}
\newcommand*{\bigcdot@}[2]{%
  % #1: math style
  % #2: unused
  \sbox0{$#1\vcenter{}$}% math axis
  \sbox2{$#1\cdot\m@th$}%
  \hbox to \bigcdot@widthfactor\wd2{%
    \hfil
    \raise\ht0\hbox{%
      \scalebox{\bigcdot@scalefactor}{%
        \lower\ht0\hbox{$#1\bullet\m@th$}%
      }%
    }%
    \hfil
  }%
}
\makeatother

\title{Representations of a concrete category as objects in the functor category}

\author{Tibor Gr{\"u}n}

\begin{document}
	\pagenumbering{gobble}

	\maketitle

	\newpage

	\tableofcontents

	\newpage

	\pagenumbering{arabic}

%% mainfile: ../main.tex

\section{Introduction}

Die Aufgabe der vorliegenden Arbeit besteht darin, das GAP-Paket "catreps" von Peter Webb mit CAP zu re-organisieren. Dabei werden die Algorithmen, welche in catreps direkt implementiert sind, soweit wie möglich durch vorhandene Methoden von CAP ersetzt.
Insbesondere wird das CAP-Paket FunctorCategories Anwendung finden, weil ich zeigen werde, dass catreps, also die Kategorie der Darstellungen einer konkreten endlichen Kategorie, nichts anderes ist als eine Unterkategorie von FunctorCategories, also der Kategorie aller Funktoren zwischen Kategorien. 
Da catreps selbst bereits eine Verallgemeinerung der Darstellung endlicher Gruppen ist (eine Gruppe ist nichts anderes als eine Kategorie mit einem Objekt, in dem jeder Morphismus ein Isomorphismus ist), stellt FunctorCategories wohl den allgemeinsten Rahmen dar, den man sich vorstellen kann.

In this thesis I will define what a category is, then I go further in the doctrine of enriched categories, especially monoidal categories.
The morphisms in the category of categories are the functors between categories. I will treat the functor category where the functors themselves are
objects and natural transformations the morphisms between them. I will show that any representation of a category (and thus any representation
of a group) is a functor, so the category of representations (of a category) is a subcategory of the functor category.
I will show how the monoidal structure of the category of representations arises from the counit and the comultiplication on the Bialgebroid structure
on the category.

\noindent Throughout the thesis I will give proofs of existence by providing an algorithm that computes the object that exists. I will be using CAP, the gap package
developed by Sebastian Gutsche et al. Another purpose of this thesis is the translation of the work of Peter Webb, who used gap directly, into our CAP
framework. This includes his decomposition algorithm for a representation. As the category of representations is just a subcategory of the functor category,
most of the work will be done inside the package FunctorCategories by Prof. Mohamed Barakat. In this thesis I will also write the documentation for the
package FunctorCategories.


\section{Introduction}

\[
\mathbf{Quiv}\rightarrow^{CatClosure}\leftarrow_{U}\mathbf{Cats}
\rightarrow^{k-Algebroid}\leftarrow_{U}\mathbf{k-Cats}
\rightarrow^{AdditiveClosure}\leftarrow_{U}\mathbf{k-Cats^{\oplus}}
\]

% mainfile: ../main.tex

\section{A short overview of the tools used}

\textsc{Gap}, \textsc{QPA$2$}, \texttt{catreps}, \textsc{Cap}, \texttt{homalg\_project}

\section{Introduction to quivers and category theory}
% mainfile: ../main.tex

This section serves two purposes: On the one hand, it is an introduction to quivers and category theory. On the other hand it introduces
concrete categories which we want to represent, and all the additional constructions that are needed to that goal.

\subsection{Quivers}
In this section, we first want to define the category \textbf{Quiv} and how it is the prototype for the category \textbf{Cats}.
In order to describe the category \textbf{Quiv} of quivers, we first have to define what a category is and for this we need
the definition of a quiver. Lateron we will revisit this definition as we can define quivers as the objects in the quiver category \textbf{Quiv}.

\begin{definition}{(Quiver)}\label{def:quiver}\\
A \ul{directed graph} or \ul{quiver} $q$ consists of a class of \ul{objects} (or \ul{vertices}) $q_{0} = \textup{Obj}\,q$ and
a class of \ul{morphisms} (or \ul{arrows}) $q_{1} = \textup{Mor}\,q$ together with two defining maps
\[
\begin{tikzcd}[column sep=small]
{s,t\colon q_{1}} \arrow[rr, shift left = 0.7ex] \arrow[rr, shift right = 0.7ex] & & q_{0}
\end{tikzcd}
\]
$s$ called \ul{source} and $t$ called \ul{target}.
\end{definition}

In the next definition we are giving a new characterization for $q_{1}$ by looking at all arrows between two fixed objects.

\begin{definition}{(Hom-set of a (locally) small quiver)}\label{def:hom_set}
\renewcommand{\labelenumi}{(\theenumi)}
\begin{enumerate}
\item Given two objects $M, N \in q_{0}$ we write $\textup{Hom}_{q}(M,N)$ or $q(M,N)$ for the fiber
$(s,t)^{-1} (\{(M,N)\})$ of the product map 
\begin{tikzcd}[column sep=small]
(s, t) : q_{1} \arrow[rr] &  & q_{0} \times q_{0} 
\end{tikzcd} over the pair $(M,N) \in q_{0} \times q_{0}$.
This is the class of all morphisms with source $= M$ and target $= N$.
We indicate this by writing
\begin{tikzcd}[column sep=small]
\varphi : M \arrow[rr] &  & N
\end{tikzcd} or 
\begin{tikzcd}[column sep=small]
M \arrow[rr,"\varphi"] &  & N.
\end{tikzcd} Hence $q_{1}$ is the disjoint union $\bigcup\limits^{\bigcdot}_{M,N \in q_{0}} \textup{Hom}_{q}(M,N) = q_{1}$.
As usual we define $\textup{End}_{q}(M):= \textup{Hom}_{q}(M,M)$.
\item If the class $\textup{Hom}_{q}(M,N)$ is a \ul{set} for all pairs $(M,N)$ then we call the quiver \ul{locally small}.
We therefore talk about \ul{Hom-sets}.
If additionally, $q_{0}$ is a set, then the quiver is called \ul{small}.
\item A quiver with a finite set of objects and a finite set of morphisms is called a \ul{finite} quiver.
\end{enumerate}
\end{definition}

When we don't assume the category to be locally small, but still talk about its hom-sets, we mean the class of morphisms,
if we don't explicitly use the fact that it's a set of morphisms.

\begin{example}\label{q(2)}{(Quiver with 2 objects and 3 morphisms)}\\
\[
\begin{tikzcd}
1 \arrow["a"', loop, distance=2em, in=305, out=235] \arrow[rr, "b"] &  & 2 \arrow["c"', loop, distance=2em, in=305, out=235]
\end{tikzcd}
\]
The objects of this quiver $q$ are $q_{0} = \{1, 2\}$, and the morphisms are $q_{1} = \{a, b, c\}$ with\\
$s (a) = 1 = t (a)$, $s (c) = 2 = t (c)$ and $s (b) = 1, t (b) = 2$.\\
\noindent Thus $\textup{End}_{q}(1) = \{a\}, \textup{End}_{q}(2) = \{c\}$ and $\textup{Hom}_{q}(1,2) = \{b\}$ whereas
$\textup{Hom}_{q}(2,1)=\emptyset$.\\

\noindent In \texttt{QPA} this quiver is encoded as \texttt{q(2)[a:1->1,b:1->2,c:2->2]} where the first \texttt{(2)} in parentheses stands for the total
number of objects.
\end{example}

\begin{definition}{(Composable arrows; path in a quiver)}\label{def:path}\endnote{(ref. \ref{[leit4]} 4.1)}
Let $q$ be a quiver.
\begin{enumerate}
\renewcommand{\labelenumi}{(\theenumi)}
\item We say two arrows $a, b \in q_{1}$ are \ul{composable} if $t(a) = s(b)$ or $t(b) = s(a)$. In this case we can write a
sequence of composable arrows $p = a_{1}a_{2}\cdots a_{n}$ where $t(a_{i}) = s(a_{i+1})$ for $i=1,\dots,n-1$.
We call this sequence a \ul{path} from $s(a_{1})$ to $t(a_{n})$ and the integer $n \in \mathbb{Z}_{\geq0}$ the \ul{length} $l(p)$ of the path $p$.
Although it may not be an arrow, we can define the source and target of a path $p = a_{1}\cdots a_{n}$ as $s(p) := s(a_{1})$ and $t(p) := t(a_{n})$.
Then again we define two paths $p$ and $q$ as composable, if $t(p) = s(q)$ (or $t(q) = s(p)$) and we call $pq$ (or $qp$) the \ul{concatenation} or
\ul{composition} of the two paths. We can identify each arrow again as a path of length 1.
A path $p = a_{1}\cdots a_{n}$ with $s(a_{1}) = t(a_{n})$, i.e. $s(p) = t(p)$, is called \ul{cyclic}.
\item For an endomorphism $a \in \textup{End}_{q}(M)$ we write $a^{n}$ for $aa \cdots a$ ($n$ times).
\item In the case of $n=0$ an \ul{empty path} whose source and target are the vertex $i \in q_{0}$ is called the \ul{trivial path at $i$} and
is denoted $e_{i}$. Note that the composition of paths $e_{i}e_{i}$ has length zero starting at $i$ therefore $e_{i}^{2}=e_{i}$,
in other words, each $e_{i}$ is an \ul{idempotent}.
\end{enumerate}
\end{definition}

\begin{lemma}\label{la:cyclic_paths}
Let Q be a quiver. If there is a path of length at least $\abs{Q_{0}}$, then there are cyclic paths,
and thus infinitely many paths.\cite{[leit4]}
\end{lemma}
\begin{proof}
Assume that there exists a path of length greater or equal to $\abs{Q_{0}}$. Then there exists a path of length $n = \abs{Q_{0}}$, say
$\alpha_{1}\cdots \alpha_{n}$. Consider the vertices $x_{i}=s(\alpha_{i})$ for $1 \leq i \leq n$ and $x_{n+1}=t(\alpha_{n})$. Then these
are $n+1$ vertices, thus there has to exist $i<j$ with $x_{i}=x_{j}$. Let $\omega=\alpha_{i}\cdots \alpha_{j-1}$, this is a path with source and target
$x_{i}=x_{j}$, thus a cyclic path. But then $\omega^{m}$ is a path for any natural number $m$. The path $\omega$ has length $j-i\geq1$, thus
$\omega^{m}$ has length $m(j-i)$. This shows that these paths are pairwise different.
\end{proof}

\begin{example}{(A quiver with no cycles)}\\
\[
\begin{tikzcd}
2 \arrow[rrrr, "\psi"] \arrow[rrrrddd, "\psi\rho", pos=0.3] &  &  &  &
3 \arrow[ddd, "\rho"] \\
 &  &  &  & \\
 &  &  &  & \\
1 \arrow[uuu, "\varphi"] \arrow[rrrruuu, "\varphi\psi", pos=0.3] \arrow[rrrr, "\varphi\psi\rho" '] &  &  &  & 4
\end{tikzcd}
\]
The longest path $1\rightarrow2\rightarrow3\rightarrow4$ has length 3. If after the object $4$ another arrow would go to either $1,2,3$ or $4$ itself,
we would have a cyclic path and thus infinitely many paths.
\end{example}

\begin{definition}{(Path algebra of a quiver)}\label{def:path_algebra}\endnote{(from \ref{[leit4]} 4.1 )}
Let $\Bbbk$ be a field. For a quiver $Q$ let $\Bbbk Q$ be the vector space with basis the set of all paths in $Q$, together with the
following multiplication: if $w, w'$ are paths, let $ww'$ be the concatenation of $w$ and $w'$ if they are composable, and the zero vector
otherwise, and extend this multiplication bilinearly to $\Bbbk Q$. We call $\Bbbk Q$ the \ul{path algebra} of the quiver $Q$.
\end{definition}

Note that the addition of two paths $w + w'$ doesn't necessarily yield a path as result, but instead an abstract element of the
path algebra, that you can't easily see in the quiver.

\begin{lemma}\label{la:path_algebra_is_ass_algebra}\endnote{(from \ref{[leit4]} 4.1 )}
For a quiver $Q$ and a field $\Bbbk$, the path algebra $\Bbbk Q$ is an associative $\Bbbk$-algebra.
\end{lemma}
\begin{proof}
Let $w, w', w''$ be paths. Then both $(ww')w''$ and $w(w'w'')$ are the concatenation of $w$ on the left,
$w'$ in the middle and $w''$ on the right, in case both conditions $t(w) = s(w')$ and $t(w') = s(w'')$ are satisfied, and
otherwise the zero element (since $(ww')0 = 0, 0(w'w'') = 0$, according to bilinearity).\\
Since the multiplication was defined on a basis and extended bilinearly, the axioms of an algebra are clearly satisfied.
\end{proof}

\begin{lemma}\label{la:unit_in_path_algebra}
If the set of vertices of a quiver $Q_{0}$ is finite, then $\Bbbk Q$ has a unit element $\sum_{x\in Q_{0}} e_{x}$. In this case, $\Bbbk Q$ is a unital ring.
\end{lemma}
\begin{proof}
Let $e := \sum_{x\in Q_{0}} e_{x}$. Let $w$ be a path with $s(w) = x$ and $t(w) = y$, then $e_{x}w = w$ and $e_{z}w = 0$ for all $z \neq x$,
thus $ew = e_{x}w + \sum_{z\neq x} e_{z}w = w + 0 = w$. Similarly, $we_{y} = w$ and $we_{z} = 0$ for $z \neq x$.
\end{proof}

\subsection{Categories}

\begin{definition}{(Category)}\label{def:category}\\
\noindent A \ul{category} $\mathcal{C}$ is a quiver with two further maps:
\begin{enumerate}
\renewcommand{\labelenumi}{(id)}
\item The \ul{identity map} $1_{( )}$ mapping every object $X \in\mathcal{C}_{0}$ to its \ul{identity morphism} $1_{X}$:
\[
\begin{tikzcd}[column sep=small]
\mathcal{C}_{0} \arrow[rr,"1"] &  & \mathcal{C}_{1}
\end{tikzcd}
\]
\renewcommand{\labelenumi}{($\mu$)}
\item And for any two \ul{composable} morphisms $\varphi$ and $\psi \in \mathcal{C}_{1}$, i.e. with $t(\varphi) = s(\psi)$, the
\ul{composition map} $\mu$, which maps $\varphi, \psi \in \mathcal{C}_{1}\times\mathcal{C}_{1}$ to $\mu(\varphi,\psi) \in \mathcal{C}_{1}$ which
we also write as $\varphi\psi$. 
\[
\begin{tikzcd}[column sep=small]
\mathcal{C}_{1} \times \mathcal{C}_{1} \arrow[rr,"\mu"] &  & \mathcal{C}_{1}
\end{tikzcd}
\]
\end{enumerate}
\noindent The defining properties for $1$ and $\mu$ are:
\renewcommand{\labelenumi}{(\theenumi)}
\begin{enumerate}
\item $s(1_{M}) = M = t(1_{M})$, i.e.\\
$1_{M} \in \textup{End}_{\mathcal{C}} \forall M \in \mathcal{C}$.

\item $s(\varphi\psi) = s(\varphi)$ and\\
$t(\varphi\psi) = t(\psi)$\\
for all composable morphisms $\varphi, \psi \in \mathcal{C}$.
\[
\begin{tikzcd}[column sep=small]
\mu : \textup{Hom}_{\mathcal{C}}(M,L) \times \textup{Hom}_{\mathcal{C}}(L,N) \arrow[rr] &  & \textup{Hom}_{\mathcal{C}}(M,N)
\end{tikzcd}
\]
\item \label{associativity_of_composition} \begin{minipage}{.55\textwidth} $(\varphi\psi)\rho = \varphi(\psi\rho)$ \hfill{} [associativity of composition]\end{minipage}
\begin{minipage}{.45\textwidth}\phantom{}\end{minipage}
\item \label{unit_property} \begin{minipage}{.55\textwidth} $1_{s(\varphi)}\varphi = \varphi = \varphi1_{t(\varphi)}$ \hfill{} [unit property]\end{minipage}
\begin{minipage}{.45\textwidth}\phantom{}\end{minipage}\\
The identity is a left and right unit of the composition.
\end{enumerate}
\end{definition}

\noindent So with categories you always answer the four questions
\begin{itemize}\label{category_questions}
\item What are the objects? (which includes the question What are the identity morphisms?)
\item What are the morphisms?
\item How do you compose morphisms?
\item Why is the composition associative?
\end{itemize}

\subsection{Functors}

Categories are themselves objects in the category of categories, which leads to a question: What is a morphism between categories?

\begin{definition}{(Functor)}\label{def:functor}\\
\noindent A \ul{functor} $F : \mathcal{C} \rightarrow \mathcal{D}$, between categories $\mathcal{C}$ and $\mathcal{D}$, consists of the
following data:

\begin{itemize}
\item An object $Fc\in\mathcal{D}_{0}$, for each object $c \in \mathcal{C}_{0}$.
\item A function $Ff : Fc \rightarrow Fc' \in \mathcal{D}_{1}$, for each morphism $f : c \rightarrow c' \in \mathcal{C}_{1}$, so that the
source and target of $Ff$ are, respectively, equal to $F$ applied to the source or target of $f$, in other words,
$s(Ff) = Fs(f)$ and $t(Ff) = Ft(f)$.
\end{itemize}

\noindent The assignments are required to satisfy the following two \ul{functoriality axioms}:
\begin{itemize}\label{functoriality}
\item For any composable pair $f, g \in \mathcal{C}_{1}, Fg \cdot Ff = F(g \cdot f)$.
\item For each object $c \in \mathcal{C}_{0}, F(1_{c}) = 1_{Fc}$.
\end{itemize}

Put concisely, a functor consists of a mapping on objects and a mapping on morphisms that preserves all of the structure of a category,
namely domains and codomains, composition, and identities.
\end{definition}

\noindent So with functors you always answer the four questions
\begin{itemize}\label{four_functor_questions}
\item How does it work on objects?
\item How does it work on morphisms?
\item Why does it respect composition?
\item Why does it respect identity morphisms?
\end{itemize}

We have already seen an example for a functor in definition \ref{def:hom_set} where we defined the hom-set $\textup{Hom}(M,N)$ between two
objects $M$ and $N$. There are two ways to leave blank one of the objects and thus define the 

\begin{example}{(partial Hom-functor)}\label{ex:hom_functor}
Let $\mathcal{C}$ be a category and $P \in \mathcal{C}_{0}$ any object. The \ul{Hom-functor}, also called \ul{partial Hom-functor},
\begin{enumerate}
\item $\textup{Hom}(P,-)$ is a functor from $\mathcal{C}$ to $\mathcal{C}_{1}$ where objects in $\mathcal{C}_{1}$ are the hom-sets 
$\textup{Hom}(P,N)$, and morphisms are maps from one hom-set to another.
$\textup{Hom}(P,-)$ works on objects by mapping the object $N \in \mathcal{C}_{0}$ to
the hom-set $\textup{Hom}(P,N) \in \mathcal{C}_{1}$.
$\textup{Hom}(P,-)$ works on morphisms by mapping the morphism $(f : M \rightarrow N ) \in \mathcal{C}_{1}$ to the transformation
$\textup{Hom}(P,f) : \textup{Hom}(P,M) \rightarrow \textup{Hom}(P,N); \varphi \mapsto \varphi f$, so for every morphism
$\varphi \in \textup{Hom}(P,M)$, you post-compose $f \in \textup{Hom}(M,N)$ to get a new morphism $\varphi f \in \textup{Hom}(P,N)$.

\item $\textup{Hom}(-,P)$ is a functor from $\mathcal{C}$ to $\mathcal{C}_{1}$ where objects in $\mathcal{C}_{1}$ are the hom-sets 
$\textup{Hom}(N,P)$, and morphisms are maps from one hom-set to another.
$\textup{Hom}(-,P)$ works on objects by mapping the object $N \in \mathcal{C}_{0}$ to
the hom-set $\textup{Hom}(N,P) \in \mathcal{C}_{1}$.
$\textup{Hom}(-,P)$ works on morphisms by mapping the morphism $(f : M \rightarrow N ) \in \mathcal{C}_{1}$ to the transformation
$\textup{Hom}(f,P) : \textup{Hom}(N,P) \rightarrow \textup{Hom}(M,P); \varphi \mapsto f\varphi$, so for every morphism
$\varphi \in \textup{Hom}(N,P)$, you pre-compose $f \in \textup{Hom}(M,N)$ to get a new morphism $f\varphi \in \textup{Hom}(M,P)$.
\end{enumerate}

The important difference between these two functors was how they worked on morphisms. If in both cases we take a morphism
$f : M \rightarrow N$ as given, then we have to arrange the source and target for $\textup{Hom}(P,f)$ and $\textup{Hom}(f,P)$
according to the post-composition and pre-composition. Thus if we wanted $\textup{Hom}(f,P)$ to be defined by pre-composition
$\varphi \mapsto f\varphi$, then we were forced to invert $M$ and $N$ as source and target to get 
$\textup{Hom}(f,P): \textup{Hom}(N,P) \rightarrow \textup{Hom}(M,P)$. 
This process of inverting source and target is caught in the following definition.
\end{example}

\begin{definition}{(covariant / contravariant functor)}\endnote{(Def 1.3.5. in \cite{[context]}, p. 17 (35/258))}\\
The way we defined a functor in definition \ref{def:functor} was in the \ul{covariant} way.\\
A \ul{contravariant} functor $F : \mathcal{C} \rightarrow \mathcal{D}$ works on objects the same way as a covariant one, i.e.
an object $Fc \in \mathcal{D}_{0}$ for each object $c \in \mathcal{C}_{0}$. For morphisms on the other hand, we have
a morphism $F f : Fc' \rightarrow Fc \in \mathcal{D}_{1}$ for each morphism $f : c \rightarrow c' \in \mathcal{C}_{1}$, so that
$s(F f) = F t(f)$ and $t(F f) = F s(f)$.
The \ul{functoriality axioms} are also inverted for a contravariant functor:
For any composable pair, $f, g \in \mathcal{C}_{1}$, $F f \cdot F g = F(g \cdot f)$.
For the identity morphisms, it is again the same as in the covariant case:
For each object $c \in \mathcal{C}_{0}$, $F(1_{c}) = 1_{Fc}$.
\end{definition}

In the following definitions, we define different subclasses of functors. These adjectives often come in opposite pairs, so that you may be
tempted to think, duality lets you just swap all the adjectives for the opposite ones, but be careful there. E.g. when 
$\textup{Hom}(P,-)$ is a \ul{covariant}, \ul{left-exact} functor, the opposite $\textup{Hom}(-,P)$ is a \ul{contravariant}, but still \ul{left-exact} functor.
But their respective \ul{right-exactedness} is equivalent to dual concepts concerning \ul{projective} and \ul{injective} objects.

Limiten 

(mit Beispielen / dual)
Kernel

Pullback

Terminal object

Equalizer

\begin{definition}{(Exact functor)}\label{def:exact_functor}\endnote{(Def 4.5.9. in \cite{[context]}, p. 139 (157/258))}\\
A functor is \ul{right exact} or \ul{finitely cocontinuous} if it preserves finite colimits, and \ul{left exact} or \ul{finitely continuous} if it preserves finite limits.
\end{definition}

\begin{remark}
Without going into the details of defining what a limit and a colimit is, and with \ul{pullbacks} and \ul{pushouts} as specific kinds of
finite limits or colimits, and with the following proposition characterizing monomorphisms and epimorphisms,
we can give a definition for exact functor that is useful enough for our purposes.\endnote{(For a more on exact functors see above footnote,
on limits and colimits the same \cite{[context]}, chapter 3, pages 73 (91/258) onward, on pullback and pushout Def 3.1.15 p. 78 / Ex. 3.1.22, p. 80 f)}
\end{remark}

\begin{lemma}\label{prop:mono_pullback}
A morphism $f : a \rightarrow b$ is a monomorphism if and only if
the pullback of $f$ and $f$ exists and is $a$, together with the identity maps $1_{a} : a \rightarrow a$.
In other words, $f : a \rightarrow b$ is a monomorphism if and only if the commutative square
\[
\begin{tikzcd}
a \arrow[r, "1_{a}"] \arrow[d, "1_{a}"'] & a \arrow[d, "f"] \\
a \arrow[r, "f"]                         & b               
\end{tikzcd}
\]
is a pullback square.\endnote{(Cited from \cite{[Annoying Precision]})}

A dual statement exists for epimorphisms and pushouts, which are finite colimits.
\end{lemma}

\begin{corollary}{(from \ref{prop:mono_pullback})}\label{cor:preserve_mono_epi}

\begin{enumerate}
\item Being a monomorphism is a “limit property”: more precisely, any functor which preserves pullbacks
(in particular any functor which preserves finite limits, in particular any functor which preserves all limits)
preserves monomorphisms.
\item Being an epimorphism is a “colimit property”: more precisely, any functor which preserves pushouts
(in particular any functor which preserves finite colimits, in particular any functor which preserves all colimits)
preserves epimorphisms.\endnote{(Cited from \cite{[Annoying Precision]},
after pullback square and pushout square respectively)}
\end{enumerate}
\end{corollary}

\begin{lemma}
For functors between Abelian categories, left/right exactness is equivalent to preserving monos/epis.
\end{lemma}

\begin{lemma}\label{la:hom_functor_left_exact}
The hom functors $\textup{Hom}(P,-)$ and $\textup{Hom}(-,P)$ from \ref{ex:hom_functor} are left exact, i.e. respect monos.
\begin{proof}
Let $f : M \rightarrow N \in \mathcal{C}_{1}$ be a monomorphism, and let $O \in \mathcal{C}_{0}$ be any object.
Let $\mathfrak{g} : \textup{Hom}(P,N) \rightarrow \textup{Hom}(P,O); \varphi \mapsto \mathfrak{g}(\varphi)$
and $\mathfrak{h} : \textup{Hom}(P,N) \rightarrow \textup{Hom}(P,O); \varphi \mapsto \mathfrak{h}(\varphi)$
such that $\textup{Hom}(P,f) \cdot \mathfrak{g} : \textup{Hom}(P,M) \rightarrow \textup{Hom}(P,O); \psi \mapsto \mathfrak{g}(\psi f)$
and  $\textup{Hom}(P,f) \cdot \mathfrak{h} : \textup{Hom}(P,M) \rightarrow \textup{Hom}(P,O); \psi \mapsto \mathfrak{h}(\psi f)$
yield the same morphism, i.e. $\forall \psi \in \textup{Hom}(P,M), \mathfrak{g}(\psi f) = \mathfrak{h}(\psi f)$.
We want to show that - under the assumption that $f : M \rightarrow N$ was a monomorphism, already $\mathfrak{g} = \mathfrak{h}$.
TODO
\end{proof}
\end{lemma}

\begin{definition}{(Full functor)}\label{def:full_functor}\endnote{(Def 1.5.7. in \cite{[context]}, p. 30 (48/258))}\\
A functor $F : \mathcal{C} \rightarrow \mathcal{D}$ is \ul{full} if
$\forall x, y \in \mathcal{C}_{0}$, the map $\mathcal{C}(x, y) \rightarrow \mathcal{D}(Fx, Fy)$ is surjective.
\end{definition}

\begin{definition}{(Faithful functor)}\label{def:faithful_functor}\endnote{(ebd.)}\\
A functor $F$ as in \ref{def:full_functor} is \ul{faithful} if
$\forall x, y \in \mathcal{C}_{0}$, the map $\mathcal{C}(x, y) \rightarrow \mathcal{D}(Fx, Fy)$ is injective.
\end{definition}

\begin{definition}{(Essentially surjective on objects)}\label{def:ess_surj_o_o}\endnote{(ebd.)}\\
A functor $F$ as in \ref{def:full_functor} is \ul{essentially surjective on objects} if for every object $d \in \mathcal{D}_{0}$ there
is some $c \in \mathcal{C}_{0}$ such that $d$ is isomorphic to $Fc$.
\end{definition}

\begin{definition}{(Embedding)}\label{def:embedding}\endnote{(Rmk 1.5.8. in \cite{[context]}, p. 31 (49/258))}\\
A faithful functor that is injective on objects is called an \ul{embedding} and identifies the source category
as a subcategory of the target. In this case, faithfulness implies that the functor is (globally) injective on arrows.
\end{definition}

\begin{definition}{(Full embedding / full subcategory)}\label{def:full_fully}\endnote{(ebd.)}\\
A full and faithful functor, called \ul{fully faithful} for short, that is injective on objects defines a \ul{full embedding} of the
source category into the target category. The source then defines a \ul{full subcategory} of the target category.
\end{definition}

% cut-pasted from k-Algebroid.tex
\noindent As we have seen, every category is a quiver, but in general, to become a category, a quiver is lacking identity morphisms
and the composition of morphisms. To be more precise, there is a \ul{functor} $U$ from the \ul{category of categories} $\textup{CAT}$ to the
\ul{category of quivers} $\textup{Quiv}$, called the \ul{underlying quiver} or \ul{forgetful functor}.
\[
\begin{tikzcd}
\textup{Cat} \arrow[rr,"U"] &  & \textup{Quiv}
\end{tikzcd}
\]
mapping every object $M \in \mathcal{C}_{0}$ to the same objects in $q_{0}$, mapping every arrow $\varphi \in \mathcal{C}_{1}$ to 
an arrow $a \in q_{1}$, respecting source and target, but forgetting the special role of the identity morphisms and of the composition morphisms.

\begin{example}{(Free / Forgetful functor)}\label{ex:forgetful_functor}\\
TODO

$Free : \mathbf{Quiv} \rightarrow \mathbf{Cat}$

$U : \mathbf{Cat} \rightarrow \mathbf{Quiv}$
\end{example}

% Was bisher bei Category Closure geschah...
\begin{example}{(Category closure)}\\

\noindent\begin{minipage}{.08\textwidth}
\phantom{}
\end{minipage}
\begin{minipage}{.37\textwidth}
\begin{tikzcd}[boxedcd={inner xsep=1.5em, inner ysep=3em}]
B \arrow[rrrr, "\psi"] &  &  &  & C \arrow[ddd, "\rho"] \\
 &  &  &  & \\
 &  &  &  & \\
A \arrow[uuu, "\varphi"] &  &  &  & D
\end{tikzcd}
\end{minipage}
%
\begin{minipage}{.10\textwidth}
$\xrightarrow{\text{  Free }}$
\end{minipage}
%
\begin{minipage}{.37\textwidth}
\begin{tikzcd}[boxedcd={inner xsep=1.5em, inner ysep=3em}]
B \arrow[rrrr, "\psi"] \arrow[rrrrddd, "\psi\rho", pos=0.3] \arrow["1_{B}"', loop, distance=2em, in=125, out=55] &  &  &  &
C \arrow[ddd, "\rho"] \arrow["1_{C}"', loop, distance=2em, in=125, out=55]\\
 &  &  &  & \\
 &  &  &  & \\
A \arrow[uuu, "\varphi"] \arrow[rrrruuu, "\varphi\psi", pos=0.3] \arrow[rrrr, bend left, "(\varphi\psi)\rho" ', shift right=2]
\arrow[rrrr, "\varphi(\psi\rho)", bend right] \arrow["1_{A}"', loop, distance=2em, in=305, out=235] &  &  &  &
D \arrow["1_{D}"', loop, distance=2em, in=305, out=235]
\end{tikzcd}
\end{minipage}
\begin{minipage}{.08\textwidth}
\phantom{}
\end{minipage}\\

We can think of a quiver as a prototype for a category. That means we can construct the missing data for a category
from a quiver by adding the identity morphisms and the composed arrows.
\end{example}

% to be seen how useful this example is...
\begin{example}{(Underlying quiver)}\\

\noindent\begin{minipage}{.08\textwidth}
\phantom{}
\end{minipage}
\begin{minipage}{.37\textwidth}
\begin{tikzcd}[boxedcd={inner xsep=1.5em, inner ysep=3em}]
2 \arrow[rrrr, "b"] \arrow[rrrrddd, "e", pos=0.3] \arrow["h"', loop, distance=2em, in=125, out=55] &  &  &  &
3 \arrow[ddd, "c"] \arrow["i"', loop, distance=2em, in=125, out=55]\\
 &  &  &  & \\
 &  &  &  & \\
1 \arrow[uuu, "a"] \arrow[rrrruuu, "d", pos=0.3] \arrow[rrrr, bend left, "f" ', shift right=2]
\arrow[rrrr, "f", bend right] \arrow["g"', loop, distance=2em, in=305, out=235] &  &  &  &
4 \arrow["j"', loop, distance=2em, in=305, out=235]
\end{tikzcd}
\end{minipage}
%
\begin{minipage}{.10\textwidth}
$\xleftarrow{\text{   U   }}$
\end{minipage}
%
\begin{minipage}{.37\textwidth}
\begin{tikzcd}[boxedcd={inner xsep=1.5em, inner ysep=3em}]
B \arrow[rrrr, "\psi"] \arrow[rrrrddd, "\psi\rho", pos=0.3] \arrow["1_{B}"', loop, distance=2em, in=125, out=55] &  &  &  &
C \arrow[ddd, "\rho"] \arrow["1_{C}"', loop, distance=2em, in=125, out=55]\\
 &  &  &  & \\
 &  &  &  & \\
A \arrow[uuu, "\varphi"] \arrow[rrrruuu, "\varphi\psi", pos=0.3] \arrow[rrrr, bend left, "(\varphi\psi)\rho" ', shift right=2]
\arrow[rrrr, "\varphi(\psi\rho)", bend right] \arrow["1_{A}"', loop, distance=2em, in=305, out=235] &  &  &  &
D \arrow["1_{D}"', loop, distance=2em, in=305, out=235]
\end{tikzcd}
\end{minipage}
\begin{minipage}{.08\textwidth}
\phantom{}
\end{minipage}\\

\noindent In the category on the left, associativity of composition guaranteed that $(\varphi\psi)\rho = \varphi(\psi\rho)$, so those two arrows
were already the same, so they are mapped to the same arrow $f = U((\varphi\psi)\rho) = U(\varphi(\psi\rho))$ in the quiver on the right.
We didn't have to draw both arrows for $f$, but since they are equal, there is still only one arrow in the hom-set $\textup{Hom}_{q}(1,4)=\{f,f\} = \{f\}$.\\
All the other identities are not preserved under the forgetful functor, e.g. $d$ doesn't know what it has to do with $a$ and $b$ apart from
$s(d) = s(a)$ and $t(d) = t(b)$. Especially the former identity arrows are now just endomorphisms with no defining property.\\
The paths $g^{2}f, gf$ and $fj^{3}$ are all different, while in the category, they all simplify to
$1_{A}1_{A}(\varphi\psi)\rho = 1_{A}(\varphi\psi)\rho = (\varphi\psi)\rho1_{D}1_{D}1_{D} =  (\varphi\psi)\rho$ due to the unit property and associativity.
\end{example}


\subsection{Natural transformations}

With fixed categories $\mathcal{C}$ and $\mathcal{D}$ we can consider functors $F, G \in \textup{Hom}(\mathcal{C},\mathcal{D})$ themselves
as objects in the category $\textup{Hom}(\mathcal{C},\mathcal{D})$ of functors between $\mathcal{C}$ and $\mathcal{D}$. In this \ul{functor category},
the morphisms between two functors are called \ul{natural transformations}.

\begin{definition}{(Natural transformations)}\label{def:natural_transformation}\\
\noindent Given categories $\mathcal{C}$ and $\mathcal{D}$ and functors $F : \mathcal{C} \rightarrow \mathcal{D}$ and
$G : \mathcal{C} \rightarrow \mathcal{D}$, a \ul{natural transformation} $\alpha : F \Rightarrow G$ consists of:
\begin{itemize}
\item a morphism $\alpha_{c} : Fc \rightarrow Gc \in \mathcal{D}_{1}$ for each object $c \in \mathcal{C}_{0}$, the collection of which
define the \ul{components} of the natural transformation, so that, for any morphism $f : c \rightarrow c' \in \mathcal{C}_{1}$, the following
square of morphisms in $\mathcal{D}$
\[\begin{tikzcd}
Fc \arrow[rr, "\alpha_{c}"] \arrow[dd, "Ff"] &  & Gc \arrow[dd, "Gf"] \\
                                             &  &                     \\
Fc' \arrow[rr, "\alpha_{c'}"]                &  & Gc'                
\end{tikzcd}\]

\ul{commutes}, i.e., has a a common composite $Fc \rightarrow Gc' \in \mathcal{D}_{1}$.
\end{itemize}
When each component $\alpha_{c}$ is an isomorphism, we call $\alpha$ a \ul{natural isomorphism}.
\end{definition}

\subsection{The functor category}

\begin{definition}{(The functor category)}\label{def:functor_category}\endnote{(cited from ncatlab \cite{[ncatlab_functor_category]})}\\
Given categories $\mathcal{C}$ and $\mathcal{D}$, the \ul{functor category} - written $\mathcal{D}^{\mathcal{C}}$, $\textup{Hom}(\mathcal{C},
\mathcal{D})$ or $[\mathcal{C}, \mathcal{D}]$ -
is the category whose
\begin{itemize}
\item objects are functors $F : \mathcal{C} \rightarrow \mathcal{D}$
\item morphisms are natural transformations between these functors.
\end{itemize}
Main usage of functor categories is as $\textup{Hom}$ categories in place of hom-sets (comp. \ref{def:hom_set} and \ref{ex:hom_functor}) where
we have much more than a set, namely a whole category of morphisms between two objects (together with the morphisms between morphisms).
\end{definition}


% mainfile: ../main.tex

\section{Finite concrete categories}

\begin{definition}{(Finite and concrete categories)}
\renewcommand{\labelenumi}{(\theenumi)}
\begin{enumerate}
\item A \ul{finite} category is a category with a finite set of objects and a finite set of morphisms.
\item A \ul{concrete} category is a category whose objects have \ul{underlying sets} and whose morphisms are functions between these
underlying sets. Otherwise it's called an \ul{abstract} category.
\end{enumerate}
\end{definition}

Clearly every finite category is a small category.

As we have seen in the previous section, a quiver $q$ with a path of length greater than $\abs{q_{0}}$ must have loops and is thus infinite.
We will construct finite concrete categories by paying attention that the arrows between different objects are only one-directional, thus we
have a partial order on the set of objects.

The following algorithm takes two integers n and m as arguments and gives a finite concrete category as output with n objects which are each
FinSets with m elements. The generating endomorphisms are each a permutation of order m, while the non-endomorphisms are bijective
mappings from each object $c \in \mathcal{C}_{0}$ to all later objects $c' \in \mathcal{C}, c' > c$ with the obvious order.

\begin{verbatim}
##
InstallMethod( ConcreteCategoryForCAP,
        "for two integers",
        [ IsInt, IsInt ],
        
  function( n, m )
	local objects, gmorphisms, permute, j, k, list, C;
  objects := [];
  for j in [1..n] do
    objects[j] := FinSet([1+(j-1)*m..j*m]);
  od;
  gmorphisms := [];
  permute := function(o, j, m)
    local r;
    r := RemInt( o+1, m );
    if r > 0 then
      return (r+(j-1)*m);
    else
      return j*m;
    fi;
  end;
  for j in [1..n] do
    for k in [j..n] do
		if j = k then
		    Add( gmorphisms, MapOfFinSets( objects[j], 
				List( objects[j], o-> [o, permute(o,j,m) ] ),
				objects[k]) );
		else # k > j
			Add( gmorphisms, MapOfFinSets( objects[j],
				List( objects[j], o -> [o, o+(k-j)*m] ),
				objects[k]) );
		fi;
	od;
  od;
  
    DeactivateCachingOfCategory( FinSets );
    CapCategorySwitchLogicOff( FinSets );
    DisableSanityChecks( FinSets );
    
    C := Subcategory( FinSets, "A finite concrete category" : overhead := false, FinalizeCategory := false );
	
	DeactivateCachingOfCategory( C );
    CapCategorySwitchLogicOff( C );
    DisableSanityChecks( C );
	
	SetFilterObj( C, IsFiniteConcreteCategory );
	
	AddIsAutomorphism( C,
      function( alpha )
        return IsAutomorphism( UnderlyingCell( alpha ) );
    end );
	
	AddInverse( C,
      function( alpha )
        return Inverse( UnderlyingCell( alpha ) ) / CapCategory( alpha );
    end );
	
	SetSetOfObjects( C, List( objects, o-> o / C ) );
	SetSetOfGeneratingMorphisms( C, List( gmorphisms, g-> g / C ) );
	
    Finalize( C );
    
    return C;
end );
\end{verbatim}

% mainfile: ../main.tex

\subsection{Additional structure on the Hom-set of a category}

....

\begin{example}{(Group as a category)}\\
\noindent A group $\mathbf{G}$ defines a category $\mathcal{B}\mathbf{G}$ with a single object $\ast$. The group elements are its morphisms, which are
all automorphisms (i.e. bijective endomorphisms) of the single object. Composition of morphisms is defined by the binary group operation.
The identity element $e \in G$ acts as the identity morphism for the unique object in this category. The hom-set of that category is itself
a group.
\end{example}

This example can be generalized to categories where the hom-set is a ring or an R-algebra. But for this we need a commutative ring R.

Our goal is to represent finite concrete categories, for this we need the source and target categories of our functors, which the
representations are.
As subcategories of $\textup{FinSets}$, our finite concrete categories only have definitions for their objects and their
morphisms, methods to check when two morphisms are congruent or equivalent, but not much else.
A competing theory to category theory is that of quivers and path algebras. We already used their terminology in
\ref{def:path}, \ref{la:cyclic_paths} and \ref{def:path_algebra}, for instance when talking about the trivial path,
which in the language of category theory is nothing but the identity morphism, composition of arrows to a path is nothing but
composition of morphisms (if you make the path explicit by writing a new arrow for every path).

So what we called a path algebra in \ref{def:path_algebra} is a different data structure for a category. 
For one, the path algebra is an algebra, i.e. a vector space with additional structure, and thus a single set, comparable to the
class of morphisms $\mathcal{C}_{1}$ of a category $\mathcal{C}$.
But as it is an algebra, it not only contains the generating morphisms of the category, but also $\Bbbk$-linear combinations of
morphisms and paths. This is what our concrete categories lack, and what additional structure we have to give them in order
to represent them by matrices.

In practise, there is already developed software for \ul{q}uivers and \ul{p}ath \ul{a}lgebras, namely the \textsc{Gap} package
\textsc{QPA$2$}\endnote{(see \cite{[QPA2]})}.
What we are actually doing to represent finite concrete categories, is going from $\mathcal{C} \in \mathbf{Cats}$ to $q \in \mathbf{Quiv}$,
in theory by \ul{forgetting} (see \ref{ex:forgetful_functor}) the category concepts of identity morphism and composition, in practise by calculating the
underlying quiver $q$, and then for a commutative ring $\Bbbk$, constructing the path algebra $\Bbbk q$. In this step the path algebra
is infinite-dimensional, since there are infinitely many paths according to lemma \ref{la:cyclic_paths}, and \textsc{QPA$2$}'s function
\texttt{BasisPathsBetweenVertices} only works for finite-dimensional path algebras. Thus in a next step we have to provide
additional data in the form of generators of ideals of the path algebra, by which we can divide and build the quotient path algebra,
which is then finite-dimensional. This is the purpose of \texttt{RelationsOfEndomorphisms}.

Once we have a finite-dimensional path algebra $\Bbbk q$, we let \textsc{QPA$2$} calculate generators of the non-endomorphism relations,
and when we have a complete set of relations, that will be our definitive quotient quiver algebra $\Bbbk q$, which we then take it back into the category
theoretical context by constructing the $\Bbbk$-\textbf{Algebroid} $\mathcal{A}$ from the path algebra $\Bbbk q$.

The source category for our representation is then the $\Bbbk$-\textbf{Algebroid} $\mathcal{A}$ and not anymore our finite concrete
category $\mathcal{C}$, but it behaves in the same way regarding composition of morphisms and which morphisms are congruent.

The target category of our category representations will be $\Bbbk$-\textbf{Mat} which we will describe in the next section,
especially all the nice properties $\Bbbk$-\textbf{Mat} has, and how they get carried over to our functor category with $\Bbbk$-\textbf{Mat} as
target.\endnote{
In \cite{[Ab-Cat]}, Posur used the equivalence between categories $\textup{mat}_{\Bbbk} \cong \textup{vec}^{\text{fd}}_{\Bbbk}$,
as described in \cite{[context]}, \textsc{Example} 1.5.6 on page 30 (48/258), to justify that $\Bbbk$-\textbf{Mat} is a good
\textbf{computational model} to
%\setquotestyle[guillemets]{english} don't do that!
\blockquote{transform otherwise inaccessible mathematical objects into computationally easily graspable entities}
\setquotestyle{default}, which is what we are doing with \textbf{CatReps}.
}

With source and target categories defined, the category where our category representations lie in is \textbf{CatReps} for which we
show that it's a subcategory of the \textbf{Functor Category}. And even more in the next section.
\[
\mathbf{CatReps_{\mathcal{C}}} = \textup{Hom}(\Bbbk\mathbf{-Algebroid_{\mathcal{C}}}, \kmat)
\]

\subsection{Generating morphisms of a category and the underlying quiver}

$\textup{gmorphisms} := \{g_{1},\dots,g_{r}\} \rightarrow$ concrete category with set of generating morphisms $\textup{gmorphisms}$.

This is the $\textup{Free}$ functor from $\mathbf{Quiv}$ to $\mathbf{Cat}$, taking a quiver and adding the missing structure of
identity morphisms and composition of arrows to that category. The result is a category.

The $\textup{forgetful}$ functor from $\mathbf{Cat}$ to $\mathbf{Quiv}$ is going the other way around and leaves all
morphisms that we now have in the category, but forgets their relations, what was identity, what was composition.

Given a field $\Bbbk$, we have the path algebra $\Bbbk q$ with all the arrows as a basis.

Given relations on endomorphisms and on the other morphisms, we make the quotient path algebra.

This is already a category, and now it has more structure.

\subsection{Ab-categories}

\begin{definition}{(Ab-category)}
An \ul{Ab-category} is a category in which all homomorphism sets are abelian groups, and composition distributes over addition.\\
In other words, a category $\mathcal{C}$ is an \ul{Ab-category} if for every pair of objects $M,N \in \mathcal{C}_{0}$,
$( \textup{Hom}_{\mathcal{C}}(M,N), + )$ is an abelian group (with the neutral element called \ul{zero morphism}),
and for all morphisms $\gamma, \delta \in \textup{Hom}_{\mathcal{C}}(M,N),
\alpha, \beta \in \textup{Hom}_{\mathcal{C}}(N,L)$
\begin{align}\label{eq:dist}
(\gamma + \delta)\alpha &= \gamma\alpha + \delta\alpha \textup{ and }\\
\gamma(\alpha+\beta) &= \gamma\alpha + \gamma\beta.
\end{align}
Note that every hom-set has its own unique zero morphism. E.g. in $\textup{Mat}_{\mathbb{Q}}$ the $2 \times 3$ zero-matrix
$\mathbf{0} \in \textup{Hom}(2,3)$ is different from the $4 \times 4$ zero-matrix $\mathbf{0} \in \textup{Hom}(4,4)$.
\end{definition}

\begin{definition}{(semisimple)}
A ring R is semisimple if ...
\end{definition}

\begin{example}{(The matrix category $\Rmat$ over a commutative ring $R$)}\label{ex:matrix_category}
\begin{itemize}
\item Objects are natural numbers $\textup{Obj}(\textup{Mat}_{R}) = \mathbb{N} = \mathbb{N}_{0} = \{0,1,2,\dots\}$
\item Morphisms $\textup{Mor}(\textup{Mat}_{R}) \ni (m \rightarrow n)$ are $m \times n$ matrices over $R$.
We write the set of morphisms between $m$ and $n$, as $R^{m\times n} := \textup{Hom}(m,n)$. Identity morphisms are the
identity matrices.
\item Composition is matrix multiplication (associative).
\item It is a skeletal category, i.e. $m$ is isomorphic to $n \Rightarrow m = n$. Only quadratic matrices ($m = n$) can be
isomorphisms.
\end{itemize}
In this category, the number $0$ is \ul{the} zero object.\\
A zero matrix (zero morphism) is a matrix factoring through the zero object $0$.\\
\begin{minipage}{.2\textwidth}\phantom{ }\end{minipage}
\begin{minipage}{.25\textwidth}
Matrix $R^{m\times n} \ni A = 0$
\end{minipage}
\begin{minipage}{.08\textwidth}
$\Longleftrightarrow$
\end{minipage}
\begin{minipage}{.32\textwidth}
\begin{tikzcd}
m \arrow[rr, "A"] \arrow[rd, "(m \times 0)"'] &                               & n \\
                                              & 0 \arrow[ru, "(0 \times n)"'] &  
\end{tikzcd}\\
$\Rightarrow A = (m \times 0) \cdot (0 \times n)$.
\end{minipage}
\begin{minipage}{.15\textwidth}\phantom{ }\end{minipage}\\
\noindent The ``matrices'' $(m \times 0)$ and $(0 \times n)$ have zero columns or zero rows respectively, but it is
important to note that for each $m \in \textup{Obj}(\textup{Mat}_{R})$ there is exactly one such matrix $(m \times 0)$ and $(0 \times m)$
(that's what initial and terminal object means), and for different $m$, these morphisms are different.
\end{example}


\begin{example}{($\kmat$ is an Ab-category)}
For two natural numbers $m,n \in {\kmat}_{0} = \mathbb{N} = \mathbb{N}_{0}$, the set of morphisms with source $m$ and target $n$ is
$\Bbbk^{m\times n}$, the set of $m \times n$-matrices. This is an abelian group:
\begin{itemize}
\item The neutral element of the addition is the $m \times n$ zero matrix $\mathbf{0}$.
\item Addition of matrices is associative and commutative, so it's an abelian group.
\end{itemize}
The distributive laws \eqref{eq:dist} for composable morphisms hold.
\end{example}









\begin{definition}{(Abelian category)}\endnote{(From \cite{[context]}, appendix E.5, Def. E.5.1)}
A category $\mathcal{C}$ is \ul{abelian} if
\begin{itemize}
\item it has a \ul{zero object} $0$, that is both initial and terminal,
\item it has all \ul{binary products} and \ul{binary coproducts},
\item it has all \ul{kernels} and \ul{cokernels}, defined repsectively to be the \ul{equalizer} and
\ul{coequalizer} of a map $f : A \rightarrow B$ with the zero map $A \rightarrow 0 \rightarrow B$, and
\item all monomorphisms and epimorphisms arise as kernels or cokernels, respectively.
\end{itemize}
\end{definition}

\begin{definition}{($R$-linear category)}
Let $R$ be a commutative ring.
\end{definition}

For $R = \mathbb{Z}$ an $R$-linear category is nothing but an Ab-category.


\begin{definition}
Once source and target categories $\mathcal{C}, \mathcal{D}$ are both $R$-linear categories we define the functor category
$\mathrm{Hom_{R}}(\mathcal{C},\mathcal{D})$ the subcategory of $R$-linear functors.
\end{definition}





%% mainfile: ../main.tex

\subsection{Limit and colimit of a functor}

\begin{definition}{(Source of a functor)}
Let $D : \mathbf{I} \rightarrow \mathcal{C}$ be a functor. A \ul{source} of $D$ consists of the following data:
\begin{enumerate}
\renewcommand{\labelenumi}{(\theenumi)}
\item An object $S \in \mathcal{C}$.
\item A dependent function $s$ mapping an object $i \in \mathbf{I}_{0}$ to a morphism
$s(i) : S \rightarrow D(i)$ such that for all $i, j \in \mathbf{I}, \iota : i \rightarrow j$, we have $D(\iota) \cdot s(i) = s(j)$.
\end{enumerate}
\end{definition}

\begin{definition}{(Limit and colimit of a functor)}\label{def:limit}
Let $D : \mathbf{I} \rightarrow \mathcal{C}$ be a functor. A \ul{limit} of $D$ consists of the
following data:
\begin{enumerate}
\renewcommand{\labelenumi}{(\theenumi)}
\item A source of $D$ given by the data $(\mathrm{lim}\, D, (\lambda(i) : \mathrm{lim}\, D \rightarrow D(i))_{i\in\mathbf{I}_{0}})$.
\item A dependent function $u$, called the \ul{lift}, mapping every source $\tau = (T, (\tau(i) : T \rightarrow D(i))_{i \in \mathbf{I}})$ to a
morphism $u(\tau) : T \rightarrow \mathrm{lim}\, D$ such that $\lambda(i) \cdot u(\tau) = \tau(i)$ for all $i \in \mathbf{I}$.
This dependent function $u$ is unique with this property.
\end{enumerate}
A \ul{colimit} in $\mathcal{C}$ is a limit in $\mathcal{C}^{\mathrm{op}}$.
\end{definition}

\begin{definition}[Limits of type \textbf{I}]
Let $\mathbf{I}$ be a category. We say a category $\mathcal{C}$ \ul{has limits of type} $\mathbf{I}$ if it is
equipped with a dependent function $\lambda$ mapping a functor $D : \mathbf{I} \rightarrow \mathcal{C}$ to a limit
$(\mathrm{lim}\, D, \lambda_{D}, u_{D})$ of $D$.
We say $\mathcal{C}$ \ul{has colimits of type} $\mathbf{I}$ if $\mathcal{C}^{\mathrm{op}}$ has limits of that type.
\end{definition}

\begin{example}\label{ex:limits}
Depending on \textbf{I} some limits and colimits have special names:
\begin{center}
\begin{tabular}{c|c|c}
generating quiver of $\mathbf{I}$ & limit & colimit \\
\hline
$\emptyset$ & terminal object & initial object \\
$\cdot \text{\phantom{$\rightarrow$}} \cdot$ & binary product & binary coproduct \\
$\cdot \rightarrow \cdot \leftarrow \cdot$ & binary pullback & - \\
$\cdot \leftarrow \cdot \rightarrow \cdot$  & - & binary pushout \\
$ \cdot \rightrightarrows \cdot$ & binary equalizer & binary coequalizer
\end{tabular}
\end{center}
\end{example}

A category having or lacking limits and colimits of a certain type formalizes the notion of what one can or cannot \textit{do} in a category.
Since we are \textit{doing} constructive category theory, this all boils down to which limits and colimits we can \ul{compute}, i.e. for which we
have algorithms, and what needs to be true for the category in order for those algorithms to terminate with a correct output.
Just as in algebra words like ``field'' or ``ring'' or ``abelian group'' are established names and adjectives for sets with additional structure,
in category theory we give special names for categories which have certain limits. This is summarized under the vaguely defined
notion of ``doctrine''.

We say a category is of a certain \ul{categorical doctrine}, if it has all categorical operations appearing in the
defining axioms of the doctrine, in particular if it has all limits and colimits of a required set of types.
In this thesis, we will define a categorical doctrine by specifying a set of algorithms for all existential quantifyers and
disjunctions appearing in the defining axioms of the doctrine, in particular by specifying all algorithms needed to
compute the required (co-)limits.

Being a skeletal category ($\mathtt{IsSkeletalCategory}$) is not a categorical doctrine. Still, it is of high computational benefit
if we can represent a category by a skeletal model.

\subsection{A hierarchy of categorical doctrines}

In this subsection, we give explicit definitions for the (co-)limits in \ref{ex:limits} and define the
doctrines for our categories with such (co-)limits. The doctrine we are interested in is that of an Abelian category with
enough projectives and enough injectives. For now we will describe the doctrines up to and including Abelian categories, and
leave projective and injective objects for section 5. In section 3 we will work on the doctrine of $\Bbbk$-algebroids or $\Bbbk$-linear
categories.

\begin{doctrine}[Category]\label{doc:category}
The doctrine $\mathtt{IsCategory}$ involves the two algorithms
\begin{itemize}
\item $\mathtt{Source}$\endnote{The two algorithms $\mathtt{Source}$ and $\mathtt{Range}$ are implicit since
each morphism in \CAP has to be defined with source and target.},
\item $\mathtt{Range}$,
\item $\mathtt{PreCompose}$,
\item $\mathtt{IdentityMorphism}$.
\end{itemize}
\end{doctrine}


\subsubsection{Ab-categories}
We are starting simple with Ab-categories. All we need is an abelian group structure on the hom-set between two objects.

\begin{definition}[Zero morphism]\label{def:zero_morphism}\phantom{}\\
A \ul{zero morphism} from $M$ to $N$ is the neutral element of the abelian group $\mathrm{Hom}_{\mathcal{C}}(M,N)$.
\end{definition}
Note that every hom-set has its own unique zero morphism. E.g. in $\kmat$ the $2 \times 3$ zero-matrix
$0_{2,3} \in \textup{Hom}_{\kmat}(2,3)$ is different from the $4 \times 4$ zero-matrix $0_{4,4} \in \textup{Hom}_{\kmat}(4,4)$.

\begin{definition}[Ab-category]
An \ul{Ab-category} (also called \ul{pre-additive category}) is a category in which all homomorphism sets are abelian groups,
and composition distributes over addition.\\
In other words, a category $\mathcal{C}$ is an Ab-category if for every pair of objects $M,N \in \mathcal{C}_{0}$,
$( \mathrm{Hom}_{\mathcal{C}}(M,N), + )$ is an abelian group (with the zero morphism $0_{M,N}$ as the neutral element),
and for all morphisms $\gamma, \delta \in \mathrm{Hom}_{\mathcal{C}}(M,N),
\alpha, \beta \in \mathrm{Hom}_{\mathcal{C}}(N,L)$
\begin{align}
(\gamma + \delta)\alpha &=\label{eq:dist1} \gamma\alpha + \delta\alpha\quad \mathrm{ and }\\
\gamma(\alpha+\beta) &=\label{eq:dist2} \gamma\alpha + \gamma\beta.
\end{align}
\end{definition}

\begin{doctrine}[Ab-category]\label{doc:ab-category}
The doctrine $\mathtt{IsAbCategory}$ therefore involves algorithms for $\mathtt{IsCategory}$ and
\begin{itemize}
\item $\mathtt{AdditionForMorphisms}$,
\item $\mathtt{AdditiveInverseForMorphisms}$,
\item ($\mathtt{SubtractionForMorphisms}$),
\item $\mathtt{ZeroMorphism}$,
\item $\mathtt{IsZeroForMorphisms}$.
\end{itemize}
\end{doctrine}

\begin{definition}[Ab-functor]
A functor between Ab-categories is called an \ul{Ab-functor} if in the functor definition \ref{def:functor} the function $Ff$ for each
morphism $f$ is a homomorphism of abelian groups, i.e. $F(f+g) = Ff + Fg$.
\end{definition}

\subsubsection{Categories with a zero object}

\begin{remark}[Terminal object, initial object, zero object]\label{def:init_term_zero_object}
The limit / colimit of $\emptyset$.
\renewcommand{\labelenumi}{(\theenumi)}
\begin{enumerate}
\item A \ul{terminal object} $T$ in a category $\mathcal{C}$ is an object such that $\textup{Hom}_{\mathcal{C}}(-,T)$ is a singleton.
\item An \ul{initial object} $I$ in a category $\mathcal{C}$ is an object such that $\textup{Hom}_{\mathcal{C}}(I,-)$ is a singleton.
\item An object $Z$ or $0$ is a \ul{zero object} if it is both initial and terminal. (Bilimit of $\emptyset$).
\end{enumerate}
\end{remark}

\begin{definition}[Zero morphism]\label{def:zero_morphism}\phantom{}\\
A \ul{zero morphism} in a category with a zero object $0$ is a morphism factoring over $0$, i.e. $\varphi : M \rightarrow N$ is called a zero
morphism, if\\
\begin{minipage}{.35\textwidth}
\begin{tikzcd}
M \arrow[rr, "\varphi"] \arrow[rd, "\varphi_{1}"] &                              & N \\
                                                  & 0 \arrow[ru, "\varphi_{2}"'] &  
\end{tikzcd}
\end{minipage}
\begin{minipage}{.65\textwidth}
$\exists \varphi_{1} : M \rightarrow 0, \varphi_{2} : 0 \rightarrow N$\\
such that $\varphi = \varphi_{1}\varphi_{2}$.
\end{minipage}
Since the zero object $0$ is both initial and terminal, a zero morphism is uniquely defined by its source and target, thus we can
talk about \textit{the} zero morphism from $M$ to $N$, which we denote by $0_{M,N}$.\endnote{We could define the zero morphism without
the zero object, just as the neutral element of the abelian group $\mathrm{Hom}_{\mathcal{C}}(M,N)$. Therefore $\mathtt{ZeroMorphism}(M,N)$
is needed in the doctrine of $\mathtt{IsAbCategory}$, but the $\mathtt{ZeroObject}$ is not. Eventually at additive categories the
zero object arrises naturally as the direct sum of $\emptyset$, which is why $\mathtt{ZeroObject}$ is listed only there.}
\end{definition}

\begin{doctrine}[Category with zero object]
The doctrine $\mathtt{IsCategoryWithZeroObject}$ therefore involves algorithms for
\begin{itemize}
 \item $\mathtt{ZeroObject}$,
 \item $\mathtt{UniversalMorphismFromZeroObject}$,
 \item $\mathtt{UniversalMorphismIntoZeroObject}$.
\end{itemize}
\end{doctrine}

If an Ab-category has a zero object, then both notions of a zero morphism coincide.

\subsubsection{Additive categories}
The definition of the binary operation $+$ in a pre-additive category came as arbitrary outside data and could be defined
in multiple ways. A category with the following limits is additive in at most one way, warrenting the name additive category.

\begin{definition}[Product, coproduct]\label{def:prod_coprod}
The limit / colimit of a set of objects $\,\cdot \text{\phantom{$\rightarrow$}} \cdot$\\
Let $I$ be an index set and $\{A_{i}\}_{i\in I}$ a family of objects in a category $\mathcal{C}$.
\setlist[description]{font=\normalfont}
\begin{description}
\item[(prod)] The \ul{product} of the family $\{A_{i}\}_{i\in I}$ is an object $\invamalg A_{i}$ together with a family of morphisms
\[
\{ \pi_{i} : \invamalg A_{i} \rightarrow A_{i} \}
\]
called \ul{projections}, such that the following universal property is satisfied:\\
For any object $M \in \mathcal{C}_{0}$ and any family $\{ \varphi_{i} : M \rightarrow A_{i} \}_{i\in I}$ of morphisms, there exists
a unique morphism $\varphi : M \rightarrow \invamalg A_{i}$ called the \ul{product morphism} such that
\[
\varphi \pi_{i} = \varphi_{i} \, \forall i \in I.
\]
\begin{tikzcd}
                                                                                                                            &  &                                                          & A_{1} \\
M \arrow[rrru, "\varphi_{1}", bend left] \arrow[rrrd, "\varphi_{2}"', bend right] \arrow[rr, "\exists^{1} \varphi", dashed] &  & A_{1}\invamalg A_{2} \arrow[ru, "\pi_{1}"] \arrow[rd, "\pi_2"] &       \\
                                                                                                                            &  &                                                          & A_{2}
\end{tikzcd}
\item[(coprod)] The dual notion to product is the \ul{coproduct} of the family $\{A_{i}\}_{i\in I}$, that is an object $\amalg A_{i}$ together with
a family of morphisms
\[
\{ \iota_{i} : A_{i} \rightarrow \amalg A_{i} \}
\]
called \ul{coprojections} or sometimes \ul{injections}, \ul{inclusions} or \ul{embeddings}, such that the following universal property is satisfied:\\
For any object $M \in \mathcal{C}$ and any family $\{ \psi_{i} : A_{i} \rightarrow M \}$ of morphisms, there exists a unique
morphism $\psi : \amalg A_{i} \rightarrow M$ called the \ul{coproduct morphism} such that
\[
\iota_{i} \psi = \psi_{i} \, \forall i \in I.
\]
\begin{tikzcd}
  &  &                                                           & A_{1} \arrow[llld, "\psi_{1}"', bend right] \arrow[ld, "\iota_{1}"'] \\
M &  & A_{1}\amalg A_{2} \arrow[ll, "\exists^{1} \psi"', dashed] &                                                                      \\
  &  &                                                           & A_{2} \arrow[lllu, "\psi_{2}", bend left] \arrow[lu, "\iota_2"]     
\end{tikzcd}
\end{description}
\end{definition}

\begin{definition}[Biproduct]\label{def:biproduct}
Let $I$ be an index set and $\{S_{i}\}_{i\in I}$ a family of objects in a category $\mathcal{C}$.
A \ul{biproduct} is a product and a coproduct simultaneously, i.e. consists of the following data:
\begin{itemize}
\item an object $S \in \mathcal{C}_{0}$,
\item a family of morphisms $\pi = \{ \pi_{i} : S \rightarrow S_{i} \}_{i\in I}$,
\item a family of morphisms $\iota = \{ \iota_{i} : S_{i} \rightarrow S \}_{i\in I}$,
\item a dependent function $u_{\text{in}}$ mapping every family $\tau = \{ \tau_{i} : T \rightarrow S_{i} \}_{i\in I}$ of morphisms
with the same source $T$ to a morphism
$u_{\text{in}}(\tau) : T \rightarrow S$ such that $u_{\text{in}}(\tau) \pi_{i} \sim \tau_{i}$ for all $i \in I$,
\item a dependent function $u_{\text{out}}$ mapping every family $\rho = \{ \rho_{i} : S_{i} \rightarrow R \}_{i\in I}$ of morphisms
with the same target $R$ to a morphism
$u_{\text{out}}(\rho) : S \rightarrow R$ such that $\iota_{i} u_{\text{out}}(\rho) \sim \rho_{i}$ for all $i \in I$,
\end{itemize}
\end{definition}

\begin{definition}{(Direct sum)}\label{def:direct_sum}
A \ul{direct sum} is a biproduct of objects in an Ab-category such that
\begin{itemize}
\item $\sum_{i\in I}  \pi_{i} \iota_{i} \sim 1_{S}$,
\item $ \iota_{i} \pi_{j} \sim \delta_{i, j} =  \begin{cases}
            1_{S_{i}} & \text{ if } i = j  \\
            0_{ij} & \text{ if } i \neq j
        \end{cases}$,
\end{itemize}
where $\delta_{i, j} \in \mathrm{Hom}(S_{i}, S_{j})$ is the identity if $i = j$, and the zero morphism $0_{ij} := 0_{S_{i}, S_{j}}$ otherwise.
\end{definition}

\begin{example}[The direct sum of $\emptyset$]\label{ex:sum_of_empty}
For the index set $I = \emptyset$ we get an empty family of objects in a category $\mathcal{C}$. Its direct sum is
\begin{itemize}
\item an object $Z \in \mathcal{C}_{0}$,
\item empty morphism sets $\pi$ and $\iota$,
\item a dependent function $u_{\text{in}}$ maps an empty collection $\tau$ of morphisms with same source $T$ to a unique
morphism $u_{\text{in}}(\tau) : T \rightarrow Z$.
\item a dependent function $u_{\text{out}}$ maps an empty collection $\rho$ of morphisms with same target $R$ to a unique
morphism $u_{\text{out}}(\rho) : Z \rightarrow R$.
\end{itemize}
The unique morphisms $T \rightarrow Z$ and $Z \rightarrow T$ for any object $T$ (the empty family $\tau$ imposing no further condition)
suggests that our object $Z = \mathrm{DirectSum}(\emptyset)$ is in fact the zero object from \ref{def:init_term_zero_object}, and the
unique morphisms are the unique zero morphisms in $\mathrm{Hom}(Z,T)$ and $\mathrm{Hom}(T,Z)$ from \ref{def:zero_morphism}.
\end{example}

\begin{definition}\label{def:additive_category}
An \ul{additive category} is a pre-additive category $\mathcal{C}$ together with a dependent function $\oplus^{\mathcal{C}}$ mapping
a finite set $I$ and a family $(A_{i})_{i\in I}$ of objects in $\mathcal{C}$ to a corresponding direct sum $(\oplus_{i\in I}^{\mathcal{C}} A_{i},
(\pi_{i})_{i\in I}, (\iota_{i})_{i\in I})$.
\end{definition}

\begin{remark}[Addition of morphisms]\label{rmk:addition_derived_from_direct_sum}
In an additive category the abelian group structure on $\mathrm{Hom}_{\mathcal{C}}(M,N)$ can be derived from the direct sum:
For $\rho_{1}, \rho_{2} \in \mathrm{Hom}_{\mathcal{C}}(M,N)$
\[
\rho_{1} + \rho_{2} = u_{\mathrm{in}}(1_{M},1_{M}) u_{\mathrm{out}}(\rho_{1},\rho_{2}) 
= u_{\mathrm{in}}(\rho_{1},\rho_{2}) u_{\mathrm{out}}(1_{N},1_{N})
\]
The above equation illustrates that an additive category is
pre-additive in at most one way, i.e. we don't have a choice how we define the abelian group structure on the hom-sets.
\endnote{This result is already implemented in \textsc{Cap} as a derivation of \texttt{AdditionForMorphisms} from the four morphisms
\texttt{UniversalMorphismIntoDirectSum}, \texttt{IdentityMorphism}, \texttt{UniversalMorphismFromDirectSum} and \texttt{PreCompose}.
See 
\url{https://github.com/homalg-project/CAP_project/blob/v2019.06.06/CAP/gap/DerivedMethods.gi\#L1024}}
\end{remark}

\begin{doctrine}[Additive category]
The doctrine $\mathtt{IsAdditiveCategory}$ therefore involves algorithms of $\mathtt{IsAbCategory}$ and
$\mathtt{IsCategoryWithZeroObject}$ together with algorithms for
\begin{itemize}
 \item $\mathtt{DirectSum}$,
 \item $\mathtt{ProjectionInFactorOfDirectSum}$,
 \item $\mathtt{InjectionOfCofactorOfDirectSum}$,
 \item $\mathtt{UniversalMorphismIntoDirectSum}$,
 \item $\mathtt{UniversalMorphismFromDirectSum}$.
\end{itemize}
\end{doctrine}

\subsubsection{Pre-abelian categories}

A pre-abelian category is an additive category with kernels and cokernels, and hence images and coimages. To define
kernels and cokernels we need the following definition.

\begin{definition}[binary equalizer]
The limit of two parallel morphisms $\,\cdot \rightrightarrows \cdot$\\
If it exists in a category $\mathcal{C}$, the \ul{equalizer} of two morphisms $f, g : A \rightrightarrows B \in \mathcal{C}_{1}$
consists of the data
\begin{itemize}
\item an object $E := \mathrm{Eq}(f,g) \in \mathcal{C}_{0}$
\item a morphism $\iota := E \hookrightarrow A$ such that pulled back to $E$, both morphisms are equal $\iota\,f = \iota\,g$:
\begin{align*}
&E \xrightarrow{\iota} A \xrightarrow{f} B \\
=\, &E \xrightarrow{\iota} A \xrightarrow{g} B
\end{align*}
\item a dependent function $u$ such that for any other morphism $\tau : T \rightarrow A$ with
\begin{align*}
&T \xrightarrow{\tau} A \xrightarrow{f} B \\
=\, &T \xrightarrow{\tau} A \xrightarrow{g} B
\end{align*}
we have a unique morphism $u( \tau ) : T \rightarrow E$ such that $u( \tau ) \iota = \tau$.
\[
\begin{tikzcd}
E \arrow[r, "\iota", hook]                              & A \arrow[r, "f", shift left] \arrow[r, "g"', shift right] & B \\
T \arrow[ru, "\tau"] \arrow[u, "u(\tau)", dashed] &                                                           &  
\end{tikzcd}
\]
\end{itemize}
The dual concept is that of a \ul{coequalizer}, which is the colimit of $\,\cdot \rightrightarrows \cdot$.
\end{definition}

The following definition is a special case of an equalizer where $g = 0_{A,B}$. We will write it all out explicitly with their
own names for objects, morphisms and (co-)lifts.

\begin{definition}[Kernel]\label{def:kernel}\phantom{}\\
In an additive category $\mathcal{C}$, the \ul{kernel} of a morphism $f : A \rightarrow B \in \mathcal{C}_{1}$ is the equalizer
of $f$ and $0_{A,B}$, i.e. consists of the data
\begin{enumerate}
\renewcommand{\labelenumi}{(\theenumi)}
\item An object $K = \mathrm{Ker}(f)$
\item A morphism $\mathrm{KernelEmbedding}(f) := \kappa : K \rightarrow A$ such that $\kappa\,f = \kappa\,0_{A,B} = 0_{K,B}$:
\begin{align*}
\begin{tikzcd}[
  ampersand replacement=\&,
  row sep=1em,
]
{\phantom{=\, }K} \arrow[r, "\kappa"]                 \& A \arrow[r, "f"] \& B \\
{=\, K} \arrow[r, "\kappa"]   \& A \arrow[r, "0_{A,B}"] \& B \\
{= \, K} \arrow[rr, "{0_{K,B}}"] \&                            \& B
\end{tikzcd}
\end{align*}
\item A unique dependent function $\mathrm{KernelLift}(f,-) := ( - /\kappa)$ mapping a morphism $\tau : T \rightarrow A$ with $\tau f = 0_{T,B}$ to a
morphism $\mathrm{KernelLift}(f,\tau) = (\tau / \kappa) : T \rightarrow K$ such that
\[
\tau = (\tau / \kappa)\, \kappa.
\]
\end{enumerate}
\[
\begin{tikzcd}
K \arrow[r, "\kappa"]                            & A \arrow[r, "f"] & B \\
T \arrow[ru, "\tau"'] \arrow[u, "(\tau/\kappa)", dashed] &                  &  
\end{tikzcd}
\]
The last property means that the kernel embedding $\kappa$ dominates every such $\tau$, and that the kernel lift
$(\tau/\kappa)$ is the unique lift of $\tau$ along $\kappa$ in the sense of \ref{def:lift_colift_codominate}(1).
\end{definition}

\begin{definition}[Cokernel]
In an additive category $\mathcal{C}$, the \ul{cokernel} of a morphism $f : A \rightarrow B \in \mathcal{C}_{1}$ is the coequalizer of
$f$ and $0_{A,B}$, i.e. consists of the data
\begin{enumerate}
\renewcommand{\labelenumi}{(\theenumi)}
\item An object $C = \mathrm{Coker}(f)$
\item A morphism $\mathrm{CokernelProjection}(f) := \varepsilon : B \rightarrow C$ such that $f\,\varepsilon = 0_{A,B}\,\varepsilon = 0_{A,C}$:
\begin{align*}
\begin{tikzcd}[
  ampersand replacement=\&,
  row sep=1em,
]
{\phantom{=\, }A} \arrow[r, "f"]                 \& B \arrow[r, "\varepsilon"] \& C \\
{=\, A} \arrow[r, "{0_{A,B}}"]   \& B \arrow[r, "\varepsilon"] \& C \\
{= \, A} \arrow[rr, "{0_{A,C}}"] \&                            \& C
\end{tikzcd}
\end{align*}
\item A unique dependent function $\mathrm{CokernelColift}(f,-) := ( \varepsilon \backslash -)$ mapping a morphism $\tau : B \rightarrow T$ with
$f \tau  = 0_{A,T}$ to a morphism $\mathrm{CokernelColift}(f,\tau) = ( \varepsilon \backslash \tau) : C \rightarrow T$ such that
\[
\tau =\label{eq:cokernel_colift} \varepsilon \, (\varepsilon \backslash \tau).
\]
\end{enumerate}
\[
\begin{tikzcd}
A \arrow[r, "f"] & B \arrow[r, "\varepsilon", two heads] \arrow[rd, "\tau"] & C \arrow[d, "(\varepsilon\backslash\tau)", dashed] \\
                 &                                                          & T                                                 
\end{tikzcd}
\]
The last property means that the cokernel projection $\varepsilon$ codominates every such $\tau$, and that the cokernel colift
$(\varepsilon \backslash \tau)$ is the unique colift of $\tau$ along $\varepsilon$ in the sense of \ref{def:lift_colift_codominate}(2).
\end{definition}

\begin{definition}[Image]
In an additive category $\mathcal{C}$ we define the \ul{image} of a morphism\\
$f : A \rightarrow B \in \mathcal{C}_{1}$ as the kernel of its cokernel:\\
\begin{minipage}{.06\textwidth} \phantom{} \end{minipage}
\begin{minipage}{.39\textwidth}
\[
\begin{tikzcd}[
  ampersand replacement=\&,
]
A \arrow[r, "f"] \arrow[d, "(f/\kappa_{\varepsilon})"', dashed]         \& B \arrow[r, "\varepsilon", shift left, two heads]
\arrow[r, "{0_{B,C}}"', shift right] \& C \\
K \arrow[ru, "\kappa_{\varepsilon}", hook]                                    \&                            \&   \\
T \arrow[u, "(\tau/\kappa_{\varepsilon})", dashed] \arrow[ruu, "\tau"'] \&                            \&  
\end{tikzcd}
\]
\end{minipage}
\begin{minipage}{.49\textwidth}
Since $\varepsilon$ is the cokernel projection of $f$, the morphism $f : A \rightarrow B$ with $f\,\varepsilon = 0_{A,C}$ plays the
same role as any $\tau : T \rightarrow B$ with $\tau\,\varepsilon = 0_{T,C}$ in that it factors over the kernel object $K$ in a
unique way: 
\[
f = (f/\kappa_{\varepsilon})\,\kappa_{\varepsilon}
\]
\end{minipage}
\begin{minipage}{.06\textwidth} \phantom{} \end{minipage}

Thus, the image is
\begin{itemize}
\item An object $\mathrm{Im}(f) := \mathrm{Ker}(\varepsilon)$ where $\varepsilon = \mathrm{CokernelProjection}(f)$,
\item A morphism $\mathrm{ImageEmbedding}(f) := \mathrm{KernelEmbedding}(\mathrm{CokernelProjection}(f))$
\end{itemize}
To differentiate the image of $f$ (which is a kernel, but not the kernel of $f$) from the kernel of $f$,
we are using $I$ for $\mathrm{Im}(f)$ and $\iota$ for $\mathrm{ImageEmbedding}(f)$.
\end{definition}

\begin{definition}[Coimage]
In an additive category $\mathcal{C}$ we define the \ul{coimage} of a morphism $f : A \rightarrow B \in \mathcal{C}_{1}$ as the 
cokernel of its kernel:\\
\begin{minipage}{.06\textwidth} \phantom{} \end{minipage}
\begin{minipage}{.39\textwidth}
\[
\begin{tikzcd}[
  ampersand replacement=\&,
]
K \arrow[r, "\kappa", hook, shift left] \arrow[r, "{0_{K,A}}"', shift right] \& A \arrow[r, "f"] \arrow[rd, "{\varepsilon_{\kappa}}", two heads] \arrow[rdd, "\tau"'] \& B                                                                                                   \\
                                                                             \&                                                                            \& C \arrow[u, "({\varepsilon_{\kappa}}\backslash f)"', dashed] \arrow[d, "({\varepsilon_{\kappa}}\backslash \tau)", dashed] \\
                                                                             \&                                                                            \& T                                                                                                  
\end{tikzcd}
\]
\end{minipage}
\begin{minipage}{.49\textwidth}
Since $\kappa$ is the kernel embedding of $f$, the morphism $f : A \rightarrow B$ with $\kappa\,f = 0_{K,B}$ plays the
same role as any $\tau : A \rightarrow T$ with $\kappa\,\tau = 0_{K,T}$ in that it factors over the cokernel object $C$ in a
unique way:
\[
f = \varepsilon_{\kappa}\,(\varepsilon_{\kappa}\backslash f)
\]
\end{minipage}
\begin{minipage}{.06\textwidth} \phantom{} \end{minipage}

Thus, the coimage is
\begin{itemize}
\item An object $\mathrm{Coim}(f) := \mathrm{Coker}(\kappa)$ where $\kappa = \mathrm{KernelEmbedding}(f)$,
\item A morphism $\varepsilon_{\kappa} := \mathrm{CoimageProjection}(f) := \mathrm{CokernelProjection}(\mathrm{KernelEmbedding}(f))$
\end{itemize}
Since it was reserved for cokernel, we are not using $C$ to denote the coimage object.
\end{definition}

\begin{definition}{(Pre-Abelian category)}
A \ul{pre-abelian category} consists of the following data:
\begin{enumerate}
\renewcommand{\labelenumi}{(\theenumi)}
\item An additive category $\mathcal{C}$.
\item A dependent function mapping every morphism $f : A \rightarrow B$ for $A, B \in \mathcal{C}_{0}$ to a
kernel of $f$.
\item A dependent function mapping every morphism $f : A \rightarrow B$ for $A, B \in \mathcal{C}_{0}$ to a
cokernel of $f$.
\end{enumerate}
\end{definition}

\begin{doctrine}[Pre-abelian category]\label{doc:pre-abelian}
The doctrine $\mathtt{IsPreAbelianCategory}$ therefore involves algorithms of $\mathtt{IsAdditiveCategory}$ together with algorithms for
\begin{itemize}
\item $\mathtt{KernelObject}$,
\item $\mathtt{KernelEmbedding}$,
\item $\mathtt{KernelLift}$,
\item $\mathtt{CokernelObject}$,
\item $\mathtt{CokernelProjection}$,
\item $\mathtt{CokernelColift}$
\end{itemize}
\end{doctrine}

\begin{remark}[Kernel lift of cokernel colift $=$ cokernel colift of kernel lift]\label{rmk:kerl_col_col_kerl}\phantom{}\\
In categories with kernels and cokernels we can compute images and coimages together with an induced morphism
\begin{align}
\overline{\varphi} : \mathrm{Coim}(\varphi) \rightarrow \mathrm{Im}(\varphi).
\end{align}
This morphism being an isomorphism is one of the defining axioms of Abelian categories.
\end{remark}
\begin{proof}
Given a morphism $\varphi : M \rightarrow N$ in a pre-Abelian category $\mathcal{C}$, we can compute its kernel embedding
$\kappa : \mathrm{Ker}(\varphi) \hookrightarrow M$ and its cokernel projection
$\varepsilon : N \twoheadrightarrow \mathrm{Coker}(\varphi)$. They are the equalizer and coequalizer of $\varphi$ and $0_{M,N}$,
in particular we have $\kappa \, \varphi = 0_{\mathrm{Ker}(\varphi),N}$ and
$\varphi \, \varepsilon = 0_{M,\mathrm{Coker}(\varphi)}$ (the two $0$-arrows on the top).

For the kernel embedding $\kappa$ we can compute its cokernel projection
$\varepsilon_{\kappa} : M \twoheadrightarrow \mathrm{Coim}(\varphi)$ which we called the coimage projection of $\varphi$.
Keeping in mind, that it is the cokernel of $\kappa$, i.e. the coequalizer of $\kappa$ and $0_{\mathrm{Ker}(\varphi),M}$, we get
the zero morphism $0_{\mathrm{Ker}(\varphi),\mathrm{Coim}(\varphi)}$ (the bottom left $0$-arrow).

Dually for the cokernel projection $\varepsilon$ we can compute its kernel embedding
$\kappa_{\varepsilon} : \mathrm{Im} \hookrightarrow N$ which we called the image embedding of $\varphi$. Again we see
that we have the zero morphism $0_{\mathrm{Im}(\varphi),\mathrm{Coker}(\varphi)}$ (the bottom right $0$-arrow).

The two morphisms $\varphi$ and its coimage projection $\varepsilon_{\kappa}$ have the same source $M$ and
composed with $\kappa$, both yield zero morphisms. Since $\varepsilon_{\kappa}$ codominates $\varphi$, we have
a unique cokernel colift\\
$(\varepsilon_{\kappa}\backslash \varphi) : \mathrm{Coim}(\varphi) \rightarrow N$, which is one of the two diagonals
in the square diagram, and for which\\
$\varepsilon_{\kappa} \cdot (\varepsilon_{\kappa}\backslash \varphi) = \varphi$.

Now $\kappa_{\varepsilon}$ and the above colift $(\varepsilon_{\kappa}\backslash \varphi)$ both have the same target $N$.
For $\kappa_{\varepsilon}$ we already know that composed with $\varepsilon$ we get the zero morphism.
If we also proved that
$(\varepsilon_{\kappa}\backslash \varphi) \cdot \varepsilon = 0_{\mathrm{Coim}(\varphi),\mathrm{Coker}(\varphi)}$, i.e.
the red $0$-arrow in the picture, then since $\kappa_{\varepsilon}$ dominates $(\varepsilon_{\kappa}\backslash \varphi)$
we get the existence of the kernel lift of the cokernel colift
\[
(\varepsilon_{\kappa}\backslash \varphi) / \kappa_{\varepsilon} : \mathrm{Coim}(\varphi) \rightarrow \mathrm{Im}(\varphi)
\]

\[
\begin{tikzcd}
\mathrm{Ker}(\varphi) \arrow[rd, "\kappa", hook] \arrow[rddd, "0"', bend right] \arrow[rrrd, "0", pos=0.3, bend left] &                                                                                                                                                                                             &  &                                                                                               & \mathrm{Coker}(\varphi) \\
                                                                                                             & M \arrow[rrru, "0", pos=0.7, bend left] \arrow[rr, "\varphi"] \arrow[dd, "\varepsilon_{\kappa}"', two heads]                                                                                         &  & N \arrow[ru, "\varepsilon", two heads]                                                        &                         \\
                                                                                                             &                                                                                                                                                                                             &  &                                                                                               &                         \\
                                                                                                             & \mathrm{Coim}(\varphi) \arrow[rruu, "(\varepsilon_{\kappa}\backslash \varphi)"'] \arrow[rr, "(\varepsilon_{\kappa}\backslash\varphi)/\kappa_{\varepsilon}"'] \arrow[rrruuu, "0", pos=0.7, color=red, bend left] &  & \mathrm{Im}(\varphi) \arrow[uu, "\kappa_{\varepsilon}"', hook] \arrow[ruuu, "0"', bend right] &                        
\end{tikzcd}
\]

\begin{subproof}[Proof that $(\varepsilon_{\kappa}\backslash \varphi) \cdot \varepsilon = 0_{\mathrm{Coim}(\varphi),\mathrm{Coker}(\varphi)}$]\phantom{}\\
We have $\varepsilon_{\kappa} \cdot (\varepsilon_{\kappa}\backslash \varphi) = \varphi$, i.e. the top left triangle
of the square commutes. But then since composition with a zero morphism gives a zero morphism we have
\begin{align*}
&\varepsilon_{\kappa} \cdot 0_{\mathrm{Coim}(\varphi),\mathrm{Coker}(\varphi)} : M \rightarrow \mathrm{Coker}(\varphi) \\
=\, &0_{M,\mathrm{Coker}(\varphi)} \\
=\, &\varphi \cdot \varepsilon \\
=\, &\varepsilon_{\kappa} \cdot (\varepsilon_{\kappa}\backslash \varphi) \cdot \varepsilon
\end{align*}
And pre-cancellation of the epimorphism $\varepsilon_{\kappa}$ gives
\begin{align*}
0_{\mathrm{Coim}(\varphi),\mathrm{Coker}(\varphi)} = (\varepsilon_{\kappa}\backslash \varphi) \cdot \varepsilon.
\end{align*}
\end{subproof}

Dual to the above picture, the two morphisms $\varphi$ and its image embedding $\kappa_{\varepsilon}$ both have the same
target $N$ and composed with $\varepsilon$, both yield zero morphisms. Since $\kappa_{\varepsilon}$ dominates $\varphi$,
we have a unique kernel lift
$(\varphi / \kappa_{\varepsilon}) : M \rightarrow \mathrm{Im}(\varphi)$, which is the other diagonal in the square diagram,
and for which $(\varphi / \kappa_{\varepsilon})\cdot \kappa_{\varepsilon} = \varphi$.

Now $\varepsilon_{\kappa}$ and $(\varphi / \kappa_{\varepsilon})$ both have the same source $M$. For $\varepsilon_{\kappa}$
we already know that composed with $\kappa$ we get the zero morphism. If we also proved that
$\kappa \cdot (\varphi / \kappa_{\varepsilon}) = 0_{\mathrm{Ker}(\varphi),\mathrm{Im}(\varphi)}$, i.e. the red $0$-arrow in the
picture, then, since $\varepsilon_{\kappa}$ codominates $(\varphi / \kappa_{\varepsilon})$ we get the existence of the
cokernel colift of the kernel lift
\[
\varepsilon_{\kappa}\backslash(\varphi/\kappa_{\varepsilon}) : \mathrm{Coim}(\varphi) \rightarrow \mathrm{Im}(\varphi)
\]

\[
\begin{tikzcd}
\mathrm{Ker}(\varphi) \arrow[rd, "\kappa", hook] \arrow[rddd, "0"', bend right] \arrow[rrrd, "0", pos=0.3, bend left] \arrow[rrrddd, "0"', pos=0.25, color=red, shift left=2, bend right] &                                                                                                                                                    &  &                                                                                               & \mathrm{Coker}(\varphi) \\
                                                                                                                                             & M \arrow[rrru, "0", pos=0.7, bend left] \arrow[rr, "\varphi"] \arrow[dd, "\varepsilon_{\kappa}"', two heads] \arrow[rrdd, "(\varphi/\kappa_{\varepsilon})"] &  & N \arrow[ru, "\varepsilon", two heads]                                                        &                         \\
                                                                                                                                             &                                                                                                                                                    &  &                                                                                               &                         \\
                                                                                                                                             & \mathrm{Coim}(\varphi) \arrow[rr, "\varepsilon_{\kappa}\backslash(\varphi/\kappa_{\varepsilon})"']                                                 &  & \mathrm{Im}(\varphi) \arrow[uu, "\kappa_{\varepsilon}"', hook] \arrow[ruuu, "0"', bend right] &                        
\end{tikzcd}
\]

\begin{subproof}[Proof that $\kappa \cdot (\varphi / \kappa_{\varepsilon}) = 0_{\mathrm{Ker}(\varphi),\mathrm{Im}(\varphi)}$]\phantom{}\\
We have $(\varphi / \kappa_{\varepsilon}) \cdot \kappa_{\varepsilon} = \varphi$, i.e. the top right triangle
of the square commutes. But then since composition with a zero morphism gives a zero morphism we have
\begin{align*}
&0_{\mathrm{Ker}(\varphi),\mathrm{Im}(\varphi)} \cdot \kappa_{\varepsilon} : \mathrm{Ker}(\varphi) \rightarrow N \\
=\, &0_{\mathrm{Ker}(\varphi),N} \\
=\, &\kappa \cdot \varphi \\
=\, &\kappa \cdot (\varphi / \kappa_{\varepsilon}) \cdot \kappa_{\varepsilon}
\end{align*}
And post-cancellation of the monomorphism $\kappa_{\varepsilon}$ gives
\begin{align*}
0_{\mathrm{Ker}(\varphi),\mathrm{Im}(\varphi)} = \kappa \cdot (\varphi / \kappa_{\varepsilon}).
\end{align*}
\end{subproof}

So we have two morphisms from $\mathrm{Coim}(\varphi)$ to $\mathrm{Im}(\varphi)$:
\begin{alignat*}{3}
\underbrace{(\varepsilon_{\kappa}\backslash \varphi) / \kappa_{\varepsilon}}_{\alpha} & \quad\text{and}\quad
&& \underbrace{\varepsilon_{\kappa}\backslash(\varphi/\kappa_{\varepsilon})}_{\beta}.
\end{alignat*}
In fact they are equal since the two commutative traingles give commutative squares.
\begin{align*}
\varepsilon_{\kappa}\, \alpha \, \kappa_{\varepsilon} = \varphi = \varepsilon_{\kappa}\, \beta \, \kappa_{\varepsilon} 
\end{align*}
Since the epi $\varepsilon_{\kappa}$ is pre-cancelable and the mono $\kappa_{\varepsilon}$ is post-cancelable, we get $\alpha = \beta$, i.e.
\begin{align*}
(\varepsilon_{\kappa}\backslash \varphi) / \kappa_{\varepsilon} = \varepsilon_{\kappa}\backslash(\varphi/\kappa_{\varepsilon})
\end{align*}
This justifies the notation
\begin{align}\label{eq:natural_morphism}
\overline{\varphi} := \varepsilon_{\kappa}\backslash \varphi / \kappa_{\varepsilon} : \mathrm{Coim}(\varphi) \rightarrow \mathrm{Im}(\varphi).
\end{align}
In this context associativity holds, i.e. we can leave out the parentheses of the lift and colift.
\end{proof}






\newpage
\subsubsection{Abelian categories}

In this section we give three equivalent definitions for Abelian categories, starting abstractly from the above diagrams
and ending with the existence of two algorithms needed for the doctrine $\mathtt{IsAbelianCategory}$.

\begin{definition}[Abelian category]
A pre-Abelian category $\mathcal{C}$ is Abelian if for every morphism $\varphi \in \mathcal{C}_{1}$, the natural morphism
in \eqref{eq:natural_morphism}
\[
\overline{\varphi} := \varepsilon_{\kappa}\backslash \varphi/\kappa_{\varepsilon} : \mathrm{Coim}(\varphi)
\xrightarrow{\sim} \mathrm{Im}(\varphi)
\]
is an isomorphism.
\end{definition}
Therefore in the doctrine of Abelian categories we will not need to distinguish between coimages and images\endnote{Especially
in a skeletal category such as $\kmat$.}, and
the diagrams from \ref{rmk:kerl_col_col_kerl} simplify into an epi-mono factorization diagram.

\begin{corollary}[Epi-mono-factorization]\label{cor:epi_mono_factorization}
Every morphism $\varphi : M \rightarrow N$ in an abelian category $\mathcal{C}$ can be factored as the
composition of an epimorphism $\pi : M \twoheadrightarrow I$ and a monomorphism $\iota : I \hookrightarrow N$\\
where $I \cong \mathrm{Im}(\varphi) \cong \mathrm{Coim}(\varphi)$.
\[
\begin{tikzcd}
K \arrow[rd, "\kappa", hook] &                                                       &                             &                                        & C \\
                             & M \arrow[rr, "\varphi"] \arrow[rd, "\pi"', two heads] &                             & N \arrow[ru, "\varepsilon", two heads] &   \\
                             &                                                       & I \arrow[ru, "\iota", hook] &                                        &  
\end{tikzcd}
\]
The factorization is unique up to unique isomorphism.
\end{corollary}

\begin{definition}[Abelian category]
An \ul{abelian category} is
\begin{itemize}
\item a pre-abelian category $\mathcal{C}$ where
\item every monomorphism $\kappa$ is a kernel-embedding of its cokernel-projection and
\item every epimorphism $\varepsilon$ is a cokernel-projection of  its kernel-embedding.
\end{itemize}
\end{definition}

\begin{remark}
The above definition, that we can regard
\begin{itemize}
\item every monomorphism $\kappa : K \hookrightarrow A$ as a kernel-embedding
of its cokernel-projection $\varepsilon_{\kappa} : A \twoheadrightarrow C$, with $K$ being the kernel object
$\mathrm{Ker}(\varepsilon_{\kappa}) = s(\kappa)$,
\item and every epimorphism
$\varepsilon : B \twoheadrightarrow C$ as a cokernel-projection of its kernel-embedding $\kappa_{\varepsilon} : K \hookrightarrow B$,
with $C$ being the cokernel object $\mathrm{Coker}(\kappa_{\varepsilon}) = t(\varepsilon)$
\end{itemize}
implies that we also have the third ingredient
for kernels and cokernels, namely the dependent functions
\begin{itemize}
\item kernel lift $(-/\kappa) : \underline{\phantom{T}} \rightarrow K$ mapping a morphism $\tau : T \rightarrow A$ with same
target $t(\tau) = t(\kappa) = A$ and same cokernel $\varepsilon_{\kappa} = \varepsilon_{\tau}$ (i.e. $\tau\varepsilon_{\kappa} = 0_{T,C}$)
as $\kappa$ to a unique kernel lift $(\tau / \kappa) : T \rightarrow K$
\[
\begin{tikzcd}
K \arrow[r, "\kappa", hook] \arrow[r, "{0_{K,A}}", bend left, shift left=2]                                & A \arrow[r, "\varepsilon_{\kappa}", two heads] \arrow[r, "\varepsilon_{\tau}"', two heads] & C \\
T \arrow[ru, "{0_{T,A}}"', bend right, shift right=2] \arrow[ru, "\tau"] \arrow[u, "(\tau/\kappa)", dashed] &                                                                                            &  
\end{tikzcd}
\]
\item cokernel colift $(\varepsilon \backslash - ) : C \rightarrow \underline{\phantom{T}}$ mapping a morphism $\tau : B \rightarrow T$ with
same source $s(\tau) = s(\varepsilon) = B$ and same kernel $\kappa_{\varepsilon} = \kappa_{\tau}$
(i.e. $\kappa_{\varepsilon}\tau = 0_{K,T}$)  as $\varepsilon$ to a unique cokernel
colift $(\varepsilon \backslash \tau) : C \rightarrow T$.
\[
\begin{tikzcd}
K \arrow[r, "\kappa_{\varepsilon}", hook] \arrow[r, "\kappa_{\tau}"', hook] & B \arrow[r, "\varepsilon", two heads] \arrow[r, "{0_{B,C}}", bend left, shift left=2] \arrow[rd, "\tau"] \arrow[rd, "{0_{B,T}}"', bend right, shift right=2] & C \arrow[d, "(\varepsilon\backslash\tau)", dashed] \\
                                                                            &                                                                                                                                                   & T                                                 
\end{tikzcd}
\]
\end{itemize}
\end{remark}

The existence of \ul{lifts along monos} and \ul{colifts along epis} (in the sense of the above remark)
in a pre-abelian category is therefore an equivalent definition of an abelian category.

\begin{definition}[Abelian category]\label{def:abelian_category}
An \ul{Abelian category} consists of the following data:
\begin{enumerate}
\renewcommand{\labelenumi}{(\theenumi)}
\item A pre-abelian category $\mathcal{C}$.
\item A dependent function $( - / - )$ mapping a monomorphism $\kappa : K \hookrightarrow A$ and a morphism $\tau : T \rightarrow A$ with
the same target $t(\tau) = t(\kappa)$ and the same cokernel projection $\varepsilon_{\tau} = \varepsilon_{\kappa}$ to a lift $(\tau / \kappa)$ of
$\tau$ along $\kappa$.
\item A dependent function $( - \backslash - )$ mapping an epimorphism $\varepsilon : B \twoheadrightarrow C$ and a morphism
$\tau : B \rightarrow T$ with same source $s(\tau) = s(\varepsilon)$ and the same kernel embedding $\kappa_{\tau} = \kappa_{\varepsilon}$
to a colift $(\varepsilon \backslash \tau)$ of $\tau$ along $\varepsilon$.
\end{enumerate}
\end{definition}

\begin{doctrine}[Abelian category]
The doctrine $\mathtt{IsAbelianCategory}$ therefore involves algorithms of $\mathtt{IsPreAbelianCategory}$ together
with two additional algorithms
\begin{itemize}
\item $\mathtt{LiftAlongMonomorphism}$,
\item $\mathtt{ColiftAlongEpimorphism}$,
\end{itemize}
fulfilling the specification of definition \ref{def:abelian_category}
\end{doctrine}

\newpage
\subsection{The matrix category $\kmat$ is an abelian category}

The following is an example of a category which has all the limits and colimits mentioned so far, and has them implemented constructively.
We will check the four doctrines $\mathtt{IsAbCategory}$, $\mathtt{IsAdditiveCategory}$, $\mathtt{IsPreAbelianCategory}$ and
$\mathtt{IsAbelianCategory}$ by providing the needed algorithms.

\begin{example}{(The matrix category $\kmat$ over a commutative ring $\Bbbk$)}\label{ex:kmat_skeletal}
\begin{itemize}
\item Objects are natural numbers $\kmat_{0} = \mathbb{N} = \mathbb{N}_{0} = \{0,1,2,\dots\}$ for wich we use small latin letters
($m, n, k,\dots$).
\item Morphisms $(m \rightarrow n) \in \kmat_{1}$ are $m \times n$ matrices over $\Bbbk$.
We write the set of morphisms between $m$ and $n$, as $\Bbbk^{m\times n} := \textup{Hom}_{\kmat}(m,n)$. 
For variables that are Matrices we use small greek letters ($\varphi, \psi,\dots$) or capital latin letters ($A, B, C,\dots$). When only source and target are relevant,
we write $(m \times n)$.
\item $s(\varphi) = \mathtt{Source}(\varphi) := \mathtt{NrRows}(\varphi)$
\item $t(\varphi) = \mathtt{Range}(\varphi) := \mathtt{NrColumns}(\varphi)$
\item Identity morphisms are the identity matrices.
\[
1_{m} = \mathtt{IdentityMorphism}(m) := \mathtt{IdentityMat}(m,\Bbbk).
\]
\item Composition is matrix multiplication which is associative.
\[
\varphi\psi = \mathtt{PreCompose}(\varphi,\psi) := \mathtt{MatMul}(\varphi,\psi).
\]
\item It is a skeletal category, i.e. $m \cong n \Rightarrow m = n$. Only quadratic matrices ($m = n$) can be
isomorphisms.
\end{itemize}
\end{example}

\begin{example}[$\kmat$ is an Ab-category]\label{ex:kmat_pre-additive}
In $\kmat$, the number $0$ is the zero object. $\mathtt{ZeroObject := 0}$\\
A zero matrix (zero morphism) is a matrix factoring through the zero object $0$.\\
\begin{minipage}{.2\textwidth}\phantom{ }\end{minipage}
\begin{minipage}{.25\textwidth}
$\Bbbk^{m\times n} \ni A = 0_{m,n}$
\end{minipage}
\begin{minipage}{.08\textwidth}
$\Longleftrightarrow$
\end{minipage}
\begin{minipage}{.32\textwidth}
\begin{tikzcd}
m \arrow[rr, "A"] \arrow[rd, "(m \times 0)"'] &                               & n \\
                                              & 0 \arrow[ru, "(0 \times n)"'] &  
\end{tikzcd}\\
$\Rightarrow A = (m \times 0) \cdot (0 \times n)$.
\end{minipage}
\begin{minipage}{.15\textwidth}\phantom{ }\end{minipage}\\

\noindent The matrices $(m \times 0)$ and $(0 \times n)$ have zero columns or zero rows respectively, but it is
important to note that for each $m \in \kmat_{0}$ there is exactly one such matrix $(m \times 0)$ and $(0 \times m)$
(that's what initial and terminal object means), and for different $m$, these morphisms are different.\endnote{It is
challenging to decide between different types of zero matrices with zero rows or zero columns, if they
are represented by lists of lists with their entries. How many rows does the $(3 \times 0)$ matrix have, if you represent it
by the empty list \texttt{[ ]}? That's why the implenentation in \homalgProject uses a special function
\texttt{HomalgZeroMatrix}, and why the morphisms in \CAP are always implemented with their source and target defined.}

\begin{itemize}
\item For the matrix $(m \times 0)$ we have\\
$\mathtt{UniversalMorphismIntoZeroObject (m) := ZeroMorphism(m, 0) := ZeroMatrix( m, 0 )}$,
\item For the matrix $(0 \times n)$ we have\\
$\mathtt{UniversalMorphismFromZeroObject (n) := ZeroMorphism(0, n) := ZeroMatrix( 0, n )}$,
\item For zero matrices $(m \times n)$ we write $\mathtt{ZeroMorphism(m, n) := ZeroMatrix( m, n )}$.
\end{itemize}

For two natural numbers $m,n \in {\kmat}_{0} = \mathbb{N} = \mathbb{N}_{0}$, the set of morphisms with source $m$ and target $n$ is
$\Bbbk^{m\times n}$, the set of $m \times n$-matrices. This is an abelian group:
\begin{itemize}
\item The neutral element of the addition is the $m \times n$ zero matrix $0_{m,n}$.
\item Addition of matrices $\mathtt{AdditionForMorphisms( phi, psi ) := phi + psi}$ is associative and commutative.
\item For every matrix $A \in \Bbbk^{m\times n}$ there is a negative matrix $-A \in \Bbbk^{m \times n}$ such that $A + (-A) = 0_{m,n}$.
\end{itemize}
Composition of matrices is defined as matrix multiplication, which is bilinear, i.e. satisfies the distributive laws \eqref{eq:dist1} and
\eqref{eq:dist2}.\\
It is an easy exercise to provide the algorithms for an Ab-category mentioned in doctrine \ref{doc:ab-category}.
\end{example}

\begin{example}[$\kmat$ is an additive category]\label{ex:kmat_additive}
Let for $I = \{1,\dots,N\},$ the set $\{n_{1},\dots,n_{N}\}$ be a family of objects in $\kmat_{0}$. Their direct sum is
\begin{itemize}
\item the object $n := \bigoplus_{i=1}^{N} n_{i} := \sum_{i=1}^{N} n_{i} = \mathtt{Sum}$
\item For $i \in I$ we have as identity morphism $1_{n_{i}}$ of the object $n_{i}$ the $n_{i} \times n_{i}$ identity matrix.
Define
\[
n_{<i} := \sum_{j=1}^{i-1} n_{j}\quad \text{ and }\quad n_{>i} := \sum_{j=i+1}^{N} n_{j}.
\]
Then we have
\item The projection $\pi_{i} : n \rightarrow n_{i}$ is an $n \times n_{i}$ matrix that is a stacked matrix of the $n_{j}\times n_{i}$
zero matrices not including $0_{n_{i},n_{i}}$ and the identity matrix $1_{n_{i}}$.
\begin{align}
\pi_{i} = \label{eq:projection_direct_sum_matrix}
\begin{pmatrix}
0_{n_{1},\,n_{i}} \\
0_{n_{2},\,n_{i}} \\
\vdots \\
0_{n_{i-1},\,n_{i}} \\
1_{n_{i}} \\
0_{n_{i+1},\,n_{i}} \\
\vdots \\
0_{n_{N},\,n_{i}}
\end{pmatrix}
=
\begin{pmatrix}
0_{n_{<i},\,n_{i}} \\
1_{n_{i}} \\
0_{n_{>i},\,n_{i}}
\end{pmatrix}
\end{align}
\item The coprojection $\iota_{i} : n_{i} \rightarrow n$ is the transposed matrix $\iota_{i} = \pi_{i}^{T}$, i.e. an $n_{i} \times n$ matrix of
the $n_{i} \times n_{j}$ zero matrices not including $0_{n_{i},n_{i}}$ and the identity matrix $1_{n_{i}}$ lined up next to each other.
\begin{align}
\iota_{i} = \label{eq:coprojection_direct_sum_matrix}
\begingroup
\setlength\arraycolsep{2pt}
\begin{pmatrix}
0_{n_{i},n_{1}} & 0_{n_{i},n_{2}} & \dots & 0_{n_{i},n_{i-1}} & 1_{n_{i}} & 0_{n_{i},n_{i+1}} & \dots & 0_{n_{i},n_{N}}
\end{pmatrix}
= \begin{pmatrix}
0_{n_{i},\,n_{<i}} & 1_{n_{i}} & 0_{n_{i},\,n_{>i}}
\end{pmatrix} \endgroup
\end{align}

\item For a family $\tau = (\tau_{i} : t \rightarrow n_{i})_{i\in I}$ we have the morphism $u_{\text{in}}(\tau)$ which is a $t \times n$ block matrix of
the $\tau_{i}$:
\begin{align}
u_{\text{in}}(\tau) = \label{eq:u_in_direct_sum_matrix}
\begin{pmatrix}
\tau_{1} & \cdots & \tau_{N}
\end{pmatrix}
\end{align}
with $u_{\text{in}}(\tau) \pi_{i} = \tau_{i}$.
\item For a family $\tau = (\tau_{i} : n_{i} \rightarrow t)_{i \in I}$ we have the morphism $u_{\text{out}}(\tau)$ which is an $n \times t$ block matrix
of the $\tau_{i}$:
\begin{align}
u_{\text{out}}(\tau) = \label{eq:u_out_direct_sum_matrix}
\begin{pmatrix}
\tau_{1} \\
\vdots \\
\tau_{N}
\end{pmatrix}
\end{align}
with $\iota_{i} u_{\text{out}}(\tau) = \tau_{i}$.
\end{itemize}
One can easily verify the conditions for $\pi$, $\iota$, $u_{\text{in}}$ and $u_{\text{out}}$ in Definitions \ref{def:biproduct} and \ref{def:direct_sum}.
\end{example}

\begin{computation} 
If we assume algorithms for adding natural numbers
\begin{alignat*}{3}
&\mathtt{Sum}( [ m, n &&] ) = m + n, \\
&\mathtt{Sum}( [  &&] ) = 0,
\end{alignat*}
for stacking two matrices with the same number of columns (same target) on top of each other
\begin{align*}
\varphi &: k \rightarrow n \\
\psi &: m \rightarrow n\\
\mathtt{StackMatrix}( \varphi, \psi ) =
\begin{pmatrix}
\varphi \\
\psi
\end{pmatrix} &: k + m \rightarrow n,
\end{align*}
and for aligning two matrices with the same number of rows (same source) next to each other
\begin{align*}
\varphi &: m \rightarrow k \\
\psi &: m \rightarrow n\\
\mathtt{AugmentMatrix}( \varphi, \psi ) =
\begin{pmatrix}
\varphi & \psi
\end{pmatrix} &: m \rightarrow k + n.
\end{align*}
Together with the algorithms $\mathtt{IdentityMorphism( n )}$ and $\mathtt{ZeroMorphism( m, n )}$ from $\mathtt{IsAbCategory}$
we can then calculate all the algorithms in the doctrine $\mathtt{IsAdditiveCategory}$:

The direct sum of the objects $\mathtt{D := [ V1, V2, \dots, VN ]}$ in $\kmat$ is defined as
\begin{itemize}
\item \texttt{DirectSum ( D ) := VectorSpaceObject( Sum( List( D, V $\mapsto$ Dimension( V ) ) ), k )}
\item \texttt{ProjectionInFactorOfDirectSum( D, i ) := StackMatrix( [ \\
\phantom{x}\hspace{3ex}List( D[1, \dots, (i-1)], V $\mapsto$ ZeroMorphism( V, D[i] ) ), \\
\phantom{x}\hspace{3ex}IdentityMorphism( D[i] ), \\
\phantom{x}\hspace{3ex}List( D[(i+1), \dots, N], V $\mapsto$ ZeroMorphism( V, D[i] ) ) ] )}
\item \texttt{InjectionOfCofactorOfDirectSum( D, i ) := AugmentMatrix( [ \\
\phantom{x}\hspace{3ex}List( D[1, \dots, (i-1)], V $\mapsto$ ZeroMorphism( D[i], V ) ), \\
\phantom{x}\hspace{3ex}IdentityMorphism( D[i] ), \\
\phantom{x}\hspace{3ex}List( D[(i+1), \dots, N], V $\mapsto$ ZeroMorphism( D[i], V ) ) ] )}
\item \texttt{UniversalMorphismIntoDirectSum( [ phi, psi ] ) := StackMatrix( [ phi, psi ] )}
\item \texttt{UniversalMorphismFromDirectSum( [ phi, psi ] ) := AugmentMatrix( [ phi, psi ] )}
\end{itemize}

In the following \Gap{} session, we demonstrate the algorithms of $\mathtt{IsAdditiveCategory}$ and verify in
an example the two axioms of the direct sum in \ref{def:direct_sum}. To verify two these properties for a general case
of a direct sum of objects in the matrix category is left as an exercise.

The implementation of objects in the matrix category $\kmat$ in \textsc{Cap} is slightly different than simply natural numbers, since
we always have to mention the commutative ring $\Bbbk$ for the objects. We are using the finite field
$\Bbbk := \mathbb{F}_{3}$ for our calculations below.
\begin{Verbatim}[commandchars=!@|,fontsize=\small,frame=single,label=Example]
  !gapprompt@gap>| !gapinput@LoadPackage("LinearAlgebraForCAP");|
  true
  !gapprompt@gap>| !gapinput@GF3 := HomalgRingOfIntegers( 3 );|
  GF(3)
  !gapprompt@gap>| !gapinput@V3 := VectorSpaceObject( 3, GF3 );|
  <A vector space object over GF(3) of dimension 3>
  !gapprompt@gap>| !gapinput@V5 := VectorSpaceObject( 5, GF3 );|
  <A vector space object over GF(3) of dimension 5>
  !gapprompt@gap>| !gapinput@V2 := VectorSpaceObject( 2, GF3 );|
  <A vector space object over GF(3) of dimension 2>
  !gapprompt@gap>| !gapinput@D := [ V3, V5, V2 ];|
  [ <A vector space object over GF(3) of dimension 3>,
    <A vector space object over GF(3) of dimension 5>,
    <A vector space object over GF(3) of dimension 2> ]
  !gapprompt@gap>| !gapinput@S := DirectSum( D );|
  <A vector space object over GF(3) of dimension 10>
  !gapprompt@gap>| !gapinput@zero35 := ZeroMorphism( V3, V5 );|
  <A zero morphism in Category of matrices over GF(3)>
  !gapprompt@gap>| !gapinput@Display( zero35 );|
   . . . . .
   . . . . .
   . . . . .
   
   A zero morphism in Category of matrices over GF(3)
  !gapprompt@gap>| !gapinput@one5 := IdentityMorphism( V5 );|
  <An identity morphism in Category of matrices over GF(3)>
  !gapprompt@gap>| !gapinput@Display( one5 );|
   1 . . . .
   . 1 . . .
   . . 1 . .
   . . . 1 .
   . . . . 1
   
   An identity morphism in Category of matrices over GF(3)
  !gapprompt@gap>| !gapinput@zero25 := ZeroMorphism( V2, V5 );|
  <A zero morphism in Category of matrices over GF(3)>
  !gapprompt@gap>| !gapinput@Display( zero25 );|
   . . . . .
   . . . . .
   
   A zero morphism in Category of matrices over GF(3)
  !gapprompt@gap>| !gapinput@pi2 := ProjectionInFactorOfDirectSum( D, 2 );|
  <A morphism in Category of matrices over GF(3)>
  !gapprompt@gap>| !gapinput@Display( pi2 );|
   . . . . .
   . . . . .
   . . . . .
   1 . . . .
   . 1 . . .
   . . 1 . .
   . . . 1 .
   . . . . 1
   . . . . .
   . . . . .
  
  A morphism in Category of matrices over GF(3)
  !gapprompt@gap>| !gapinput@iota1 := InjectionOfCofactorOfDirectSum( D, 1 );|
  <A morphism in Category of matrices over GF(3)>
  !gapprompt@gap>| !gapinput@Display( iota1 );|
   1 . . . . . . . . .
   . 1 . . . . . . . .
   . . 1 . . . . . . .
   
   A morphism in Category of matrices over GF(3)
  !gapprompt@gap>| !gapinput@Display( PreCompose( iota1, pi2 ) );|
   . . . . .
   . . . . .
   . . . . .
   
   A morphism in Category of matrices over GF(3)
  !gapprompt@gap>| !gapinput@IsEqualForMorphisms( PreCompose( iota1, pi2 ), ZeroMorphism( V3, V5 ) );|
  true
  !gapprompt@gap>| !gapinput@iota2 := InjectionOfCofactorOfDirectSum( D, 2 );;|
  !gapprompt@gap>| !gapinput@Display( iota2 );|
   . . . 1 . . . . . .
   . . . . 1 . . . . .
   . . . . . 1 . . . .
   . . . . . . 1 . . .
   . . . . . . . 1 . .
   
   A morphism in Category of matrices over GF(3)
  !gapprompt@gap>| !gapinput@Display( PreCompose( iota2, pi2 ) );|
   1 . . . .
   . 1 . . .
   . . 1 . .
   . . . 1 .
   . . . . 1
   
   A morphism in Category of matrices over GF(3)
  !gapprompt@gap>| !gapinput@IsEqualForMorphisms( PreCompose( iota2, pi2 ), IdentityMorphism( V5 ) );|
  true
  !gapprompt@gap>| !gapinput@Display( PreCompose( pi2, iota2 ) );|
   . . . . . . . . . .
   . . . . . . . . . .
   . . . . . . . . . .
   . . . 1 . . . . . .
   . . . . 1 . . . . .
   . . . . . 1 . . . .
   . . . . . . 1 . . .
   . . . . . . . 1 . .
   . . . . . . . . . .
   . . . . . . . . . .
   
   A morphism in Category of matrices over GF(3)
  !gapprompt@gap>| !gapinput@iota3 := InjectionOfCofactorOfDirectSum( D, 3 );;|
  !gapprompt@gap>| !gapinput@pi1 := ProjectionInFactorOfDirectSum( D, 1 );;|
  !gapprompt@gap>| !gapinput@pi3 := ProjectionInFactorOfDirectSum( D, 3 );;|
  !gapprompt@gap>| !gapinput@Display( PreCompose( pi1, iota1 ) + PreCompose( pi2, iota2 )|
  !gapprompt@>| !gapinput@   + PreCompose( pi3, iota3 ) );|
   1 . . . . . . . . .
   . 1 . . . . . . . .
   . . 1 . . . . . . .
   . . . 1 . . . . . .
   . . . . 1 . . . . .
   . . . . . 1 . . . .
   . . . . . . 1 . . .
   . . . . . . . 1 . .
   . . . . . . . . 1 .
   . . . . . . . . . 1
   
   A morphism in Category of matrices over GF(3)
  !gapprompt@gap>| !gapinput@IsEqualForMorphisms(|
  !gapprompt@>| !gapinput@     PreCompose( pi1, iota1 )|
  !gapprompt@>| !gapinput@   + PreCompose( pi2, iota2 )|
  !gapprompt@>| !gapinput@   + PreCompose( pi3, iota3 ),|
  !gapprompt@>| !gapinput@   IdentityMorphism( S ) );|
  true
\end{Verbatim}

We can also verify the result in \ref{rmk:addition_derived_from_direct_sum} that the abelian group operation
$\mathtt{AdditionForMorphisms}$ can be derived from $\mathtt{UniversalMorphismIntoDirectSum}$,
$\mathtt{UniversalMorphismFromDirectSum}$,\\
$\mathtt{IdentityMorphism}$ and
$\mathtt{PreCompose}$. As an example we add two $\mathrm{GF}_{3}$-matrices from $\mathtt{V2}$ to $\mathtt{V3}$ in three
different ways.

\begin{Verbatim}[commandchars=!@|,fontsize=\small,frame=single,label=Example]
  !gapprompt@gap>| !gapinput@mat1 := HomalgMatrix( [ 0, 1, 2, 1, 1, 2 ], 2, 3, GF3 );|
  <A 2 x 3 matrix over an internal ring>
  !gapprompt@gap>| !gapinput@mat2 := HomalgMatrix( [ 1, 1, 1, 1, 1, 1 ], 2, 3, GF3 );|
  <A 2 x 3 matrix over an internal ring>
  !gapprompt@>| !gapinput@mor1 := VectorSpaceMorphism( V2, mat1, V3 );|
  <A morphism in Category of matrices over GF(3)>
  !gapprompt@gap>| !gapinput@Display( mor1 );|
   . 1 2
   1 1 2
   
   A morphism in Category of matrices over GF(3)
  !gapprompt@>| !gapinput@mor2 := VectorSpaceMorphism( V2, mat2, V3 );|
  <A morphism in Category of matrices over GF(3)>
  !gapprompt@gap>| !gapinput@Display( mor2 );|
   1 1 1
   1 1 1
   
   A morphism in Category of matrices over GF(3)
  !gapprompt@>| !gapinput@result1 := mor1 + mor2;|
  <A morphism in Category of matrices over GF(3)>
  !gapprompt@gap>| !gapinput@Display( result1 );|
   1 2 .
   2 2 .
   
   A morphism in Category of matrices over GF(3)
  !gapprompt@gap>| !gapinput@one2 := IdentityMorphism( V2 );|
  <An identity morphism in Category of matrices over GF(3)>
  !gapprompt@gap>| !gapinput@result2 := PreCompose( UniversalMorphismIntoDirectSum( [ one2, one2 ] ),|
  !gapprompt@>| !gapinput@   UniversalMorphismFromDirectSum( [ mor1, mor2 ] ) );|
  <A morphism in Category of matrices over GF(3)>
  !gapprompt@gap>| !gapinput@Display( result2 );|
   1 2 .
   2 2 .
   
   A morphism in Category of matrices over GF(3)
  !gapprompt@gap>| !gapinput@result1 = result2;|
  true
  !gapprompt@gap>| !gapinput@one3 := IdentityMorphism( V3 );|
  <An identity morphism in Category of matrices over GF(3)>
  !gapprompt@gap>| !gapinput@result3 := PreCompose( UniversalMorphismIntoDirectSum( [ mor1, mor2 ] ),|
  !gapprompt@>| !gapinput@  UniversalMorphismFromDirectSum( [ one3, one3 ] ) );|
  <A morphism in Category of matrices over GF(3)>
  !gapprompt@gap>| !gapinput@Display( result3 );|
   1 2 .
   2 2 .
   
   A morphism in Category of matrices over GF(3)
  !gapprompt@gap>| !gapinput@result3 = result2;|
  true
\end{Verbatim}
\end{computation}

Next we provide the algorithms from \ref{doc:pre-abelian} that make $\kmat$ into a pre-abelian category.
They are all based on the well-known $\mathtt{Gauss}$ algorithm that gives us the row echolon form (REF) and
the column echolon form (CEF) of a matrix.

\begin{computation}\label{comp:gauss-algorithms}
Let $\varphi : m \rightarrow n \in \kmat_{1}$ be a matrix. We assume algorithms for 
\begin{itemize}
\item The rank of a matrix, $r := \mathtt{Rank( phi )}$, defined as $\Bbbk$-dimension of the column space
$\mathrm{dim_{Col}}\,( \varphi ) = r$ which is the
same\endnote{The result ``row-rank = column-rank'' whose importance a first-year student might not understand
right away, is a very nice property of matrices.}
as the $\Bbbk$-dimension of the row space $\mathrm{dim_{Row}}( \varphi ) = r$,
\item The left nullspace of a matrix $\varphi$ is a matrix $x = \mathtt{LeftNullSpace( phi )}$ satisfying $x\, \varphi = 0$ and
for each matrix $y$ with $y\,\varphi = 0$ there exists a matrix $z$ with $zx = y$.
\item The right nullspace of a matrix $\varphi$ is a matrix $x = \mathtt{RightNullSpace( phi )}$ satisfying $\varphi \,x= 0$ and
for each matrix $y$ with $\varphi \,y= 0$ there exists a matrix $z$ with $xz = y$.
\item As the standardized form to represent these subspaces, the $\mathtt{Gauss}$-algorithm can calculate the
row-echolon-form $\mathtt{REF( LeftNullSpace( phi ) )}$ and the\\
column-echolon-form $\mathtt{CEF( RightNullSpace( phi ) )}$.
\end{itemize}
\end{computation}

\begin{example}[$\kmat$ is a pre-abelian category]\label{ex:kmat_pre-abelian}
With the algorithms in \ref{comp:gauss-algorithms} taken as given, we now give all the algorithms needed for the doctrine
$\mathtt{IsPreAbelianCategory}$ in \ref{doc:pre-abelian}:
\begin{itemize}
\item $\mathtt{KernelObject( phi ) := NrRows( phi ) - Rank( phi )}$
\item $\mathtt{KernelEmbedding( phi ) := REF( LeftNullSpace( phi ) )}$
\item $\mathtt{KernelLift( phi, tau ) := Solve( x \cdot REF( LeftNullSpace( phi ) ) = tau )}$
\item $\mathtt{CokernelObject( phi ) := NrColumns( phi ) - Rank( phi )}$
\item $\mathtt{CokernelProjection( phi ) := CEF( RightNullSpace( phi ) )}$
\item $\mathtt{CokernelColift( phi, tau ) := Solve( CEF( RightNullSpace( phi ) ) \cdot x = tau )}$
\end{itemize}
With these constructions, $\kmat$ becomes a pre-abelian category.
\end{example}

Note that the right-hand side $B$ in the equation 
\[
A \cdot x = B
\]
can be more than a single column vector, but a matrix with multiple columns, as long as they have the same number of rows as $A$.
This corresponds to solving the system of equations simultaneously for different right-hand sides. In case that it is solvable,
we get a particular solution as a matrix $x = \mathtt{LeftDivide( A, B )}$.

Dually for $B$ and $A$ matrices having the same number of columns, for the equation
\[
x \cdot A = B
\]
we get a particular solution as a matrix $x = \mathtt{RightDivide( B, A )}$, if it exists.

\begin{example}[$\kmat$ is an Abelian category]\phantom{}\\
\begin{enumerate}
\renewcommand{\labelenumi}{(\theenumi)}
\item Let $\kappa : K \hookrightarrow A \in \kmat_{1}$ and $\tau : T \rightarrow A \in \kmat_{1}$ be as in
\ref{def:abelian_category}(2).
Then with the algorithms from \ref{comp:gauss-algorithms} we define
\begin{itemize}
\item $\mathtt{LiftAlongMonomorphism( kappa, tau ) := Solve( x \cdot kappa = tau )}$\\
$\mathtt{ := RightDivide( tau, kappa )}$.
\end{itemize}
\item Let $\varepsilon : B \twoheadrightarrow C \in \kmat_{1}$ and $\tau : B \rightarrow T \in \kmat_{1}$ be as in
\ref{def:abelian_category}(3).
Then with the algorithms from \ref{comp:gauss-algorithms} we define
\begin{itemize}
\item $\mathtt{ColiftAlongEpimorphism( epsilon, tau ) := Solve( epsilon \cdot x = tau )}$\\
$\mathtt{ := LeftDivide( epsilon, tau )}$
\end{itemize}
\end{enumerate}

Since a matrix $\kappa : m \rightarrow n$ is a monomorphism iff its kernel is $0$ iff it has full row rank $\mathtt{Rank(kappa) = m}$,
the lift along monomorphism $\mathtt{RightDivide( tau, kappa )}$ always exists.

Since a matrix $\varepsilon : m \rightarrow n$ is an epimorphism iff its image is $n$ iff it has full column rank
$\mathtt{Rank(kappa) = n}$, the colift along epimorphism $\mathtt{LeftDivide( epsilon, tau )}$ always exists.

With these algorithms, $\kmat$ becomes an abelian category. In particular we have
\begin{align*}
\mathrm{Coim}(\varphi) &\cong \mathrm{Im}(\varphi)\quad\text{and since $\kmat$ is skeletal}\\
\Rightarrow\, \mathrm{Coim}(\varphi) &=\mathrm{Im}(\varphi).
\end{align*}
\end{example}


The following situation where we have a family of matrices $\{a_{i} : m_{i} \rightarrow n_{i}\}_{i\in I}$,
i.e. a family $\{m_{i}\}_{i\in I}$ of sources and $\{n_{i}\}_{i\in I}$ of targets, is useful to understand. There are two
different direct sums involved, one of the $m_{i}$'s and one of the $n_{i}$'s. We will need this construction in section 4
for the direct sum of functors, and in section 6 for the Sylvester equations.

\begin{example}[Block-Diagonal matrices]\phantom{}\label{ex:block_diagonal_matrix}\\
In a situation with an index set $I = \{1,\dots,N\}$, two families of objects $\{m_{i}\}_{i\in I}, \{n_{i}\}_{i\in I}$ and a family of
morphisms $\{a_{i} : m_{i} \rightarrow n_{i}\}_{i\in I}$ in $\kmat$, we have the two direct sums
\begin{alignat}{4}
m &:= \bigoplus_{i\in I} m_{i},\quad &&(\pi_{i})_{m} : m \rightarrow m_{i},\quad &&(\iota_{i})_{m} : m_{i} \rightarrow m, \\
n &:= \bigoplus_{i\in I} n_{i},\quad &&(\pi_{i})_{n} : n \rightarrow n_{i},\quad &&(\iota_{i})_{n} : n_{i} \rightarrow n.
\end{alignat}
This situation is displayed in the following diagram
\[
\begin{tikzcd}
m \arrow[dd, "(\pi_{i})_{m}", shift left=2] \arrow[rr, "a"]           &  & n \arrow[dd, "(\pi_{i})_{n}", shift left=2]       \\
                                                                      &  &                                                   \\
m_{i} \arrow[uu, "(\iota_{i})_{m}", shift left=2] \arrow[rr, "a_{i}"] &  & n_{i} \arrow[uu, "(\iota_{i})_{n}", shift left=2]
\end{tikzcd}
\]
The morphism $a : m \rightarrow n$ defined as
\begin{align}
a = \sum_{i \in I} (\pi_{i})_{m} a_{i} (\iota_{i})_{n}
\end{align}
satisfies
\begin{align}
(\iota_{i})_{m}\, a &= a_{i}\, (\iota_{i})_{n}, \\
a\, (\pi_{i})_{n} &= (\pi_{i})_{m}\, a_{i}\,\text{ and }\\
(\iota_{i})_{m}\, a\, (\pi_{i})_{n} &= a_{i}.
\end{align}
This an be interpreted in two ways:
\begin{enumerate}
\renewcommand{\labelenumi}{(\theenumi)}
\item For the family $\iota_{m} a := \{ (\iota_{i})_{m} a : m_{i} \rightarrow n \} := \{ a_{i}\,(\iota_{i})_{n} : m_{i} \rightarrow n \}$
of morphisms with same target $n$, we have the morphism
$u_{\text{out}}(\iota_{m} a) : m \rightarrow n$ such that\\
$(\iota_{i})_{m} u_{\text{out}}(\iota_{m} a) = (\iota_{i})_{m} a = a_{i}\,(\iota_{i})_{n}$, and
\item For the family $a \pi_{n} := \{ a (\pi_{i})_{n} : m \rightarrow n_{i} \} := \{ (\pi_{i})_{m}\,a_{i} : m \rightarrow n_{i} \}$ of
morphisms with same source $m$, we have the morphism
$u_{\text{in}}(a \pi_{n}) : m \rightarrow n$ such that\\
$u_{\text{in}}(a \pi_{n}) (\pi_{i})_{n} = a (\pi_{i})_{n} = (\pi_{i})_{m}\,a_{i}$.
\end{enumerate}
So we have
\begin{alignat}{3}
(\iota_{i})_{m} u_{\text{out}}(\iota_{m} a) (\pi_{i})_{n} &= a_{i} &&= (\iota_{i})_{m} u_{\text{in}}(a \pi_{n}) (\pi_{i})_{n}\,
\text{ and }\\
u_{\text{out}}(\iota_{m} a) &= a &&= u_{\text{in}}(a \pi_{n})
\end{alignat}
\end{example}

%%% functors between abelian categories.

\begin{theorem}
The functor category has all limits, colimits and bilimits which exist in the target category.
\end{theorem}

Instead of proving this in general, we prove this as part of \ref{thm:functor_category_abelian} for the direct sum, from which
the procedure of the general proof becomes apparent.



%% mainfile: ../main.tex

\section{Adjunctions}

\subsection{Universal objects}

\subsection{Forgetting the forgetful functor: Free constructions}

\section{Yoneda's Lemma: Completion and cocompletion of a category}

\subsection{The category of presheaves}

\begin{definition}{(The category of presheaves)}\endnote{(cited from ncatlab \cite{[ncatlab_presheaves]})}\\
For $\mathcal{C}$ a small category, its \ul{category of presheaves} is the functor category
\[ \mathrm{PSh}(\mathcal{C}) := \mathrm{Hom}(\mathcal{C}^{\text{op}}, \Set) \]
from the opposite category of $\mathcal{C}$ to $\Set$.
An object in this category is a \ul{presheaf}.\\
\noindent Taking $\mathcal{C}^{\text{op}}$ instead of $\mathcal{C}$ (and with $(\mathcal{C}^{\text{op}})^{\text{op}} = \mathcal{C}$)
we get the functor category as in \ref{def:functor_category}
\[
\mathrm{Hom}(\mathcal{C}, \Set) = \mathrm{PSh}(\mathcal{C}^{\text{op}})
\]
\end{definition}

\begin{remark}{(General properties of presheaves)}\\
The category of presheaves $\mathrm{PSh}(\mathcal{C})$ is called the \ul{free cocompletion} of $\mathcal{C}$.
\end{remark}

\begin{definition}{(Representable functor)}\label{def:repres_functor}\endnote{(from ncatlab \ref{[ncatlab_repres_functor]})}
\begin{enumerate}
\renewcommand{\labelenumi}{(\theenumi)}
\item A functor from a locally small category $\mathcal{C}$ to $\Set$ is \ul{representable} if there is an object $c \in \mathcal{C}$ and a
natural isomorphism between $F$ and $\mathrm{Hom}(c,-)$ (for a covariant $F$, otherwise $\mathrm{Hom}(-,c)$ for a contravariant $F$),
in which case one says thet the functor $F$ is \ul{represented by} the object $c$.
\item A \ul{representation} for a functor $F$ is a choice of object $c \in \mathcal{C}$ together with a specified natural isomorphism
$\mathrm{Hom}(c,-) \cong F$ (for a covariant $F$, or $\mathrm{Hom}(-,c) \cong F$ for a contravariant $F$).
\end{enumerate}
\end{definition}

\begin{lemma}{(Yoneda's Lemma)}\endnote{(Statement of Yoneda's lemma from \cite{[context]}, Lemma 2.2.4)}
For any functor $F : \mathcal{C} \rightarrow \mathrm{Set}$, whose source $\mathcal{C}$ is locally small and any
object $c \in \mathcal{C}_{0}$, there is a bijection
\[
\mathrm{Hom}(\mathrm{Hom}_{\mathcal{C}}(c,-), F) \cong Fc
\]
that associates a natural transformation $\alpha : \mathrm{Hom}_{\mathcal{C}}(c,-) \Rightarrow F$ to the element $\alpha_{c}(1_{c}) \in Fc$.
Moreover, this correspondence is natural in both $c$ and $F$.
\end{lemma}
As $\mathcal{C}$ is locally small but not necessarily small, a priori the collection of natural transformations
$\mathrm{Hom}(\mathrm{Hom}_{\mathcal{C}}(c,-),F)$ might be large. However, the bijection in the Yoneda lemma proves that this particular
collection of natural transformations indeed forms a set.
\begin{proof}
A proof of Yoneda's lemma can be found in many books on category theory, e.g. \cite{[context]}, Lemma 2.2.4, pages 57-59.
\end{proof}

\begin{definition}{(Yoneda embedding)}\label{def:yoneda_embedding}\endnote{(cited from ncatlab \cite{[ncatlab_yoneda_emb]})}\\
The \ul{Yoneda embedding} for a locally small category $\mathcal{C}$ is the functor
\[
Y : \mathcal{C} \hookrightarrow \mathrm{Hom}(\mathcal{C}^{\text{op}}, \mathrm{Set})
\]
from $\mathcal{C}$ to the category of presheaves over $\mathcal{C}$ which is the image of the hom-functor
\[
\mathrm{Hom} : \mathcal{C}^{\text{op}}\times\mathcal{C} \rightarrow \mathrm{Set}
\]
under the $\mathrm{Hom}$ adjunction
\[
\mathrm{Hom}(\mathcal{C}^{\text{op}}\times\mathcal{C}, \mathrm{Set}) \simeq
\mathrm{Hom}(\mathcal{C},\mathrm{Hom}(\mathcal{C}^{\text{op}}, \mathrm{Set}))
\]
in the closed symmetric monoidal category $\mathrm{Cat}$.
If instead we have the opposite category $\mathcal{C}^{\text{op}}$, then we get the embedding into the functor category:
\[
Y^{\text{op}} : \mathcal{C}^{\text{op}} \hookrightarrow \mathrm{Hom}(\mathcal{C},\mathrm{Set})
\]
\end{definition}

\begin{remark}[Our $\Bbbk$-linear version of Yoneda's embedding]
Let $\mathcal{A}$ be a $\Bbbk$-linear category with finite-dimensional $\Bbbk$-vector spaces as hom-sets. Then we get
$\Bbbk$-linear versions of Yoneda's lemma and Yondeda's embedding:

For any $\Bbbk$-linear functor $F : \mathcal{A} \rightarrow \kmat$ and any object $i \in \mathcal{A}_{0}$, there is a bijection
\[
\mathrm{Hom}_{\HomAkmat}(\mathrm{Hom}_{\mathcal{A}}(-,i), F) \cong F(i)
\]

\[
Y : \mathcal{A} \hookrightarrow \mathrm{Hom_{\Bbbk}}(\mathcal{A^{\text{op}}},\kmat)
\]

\[
Y^{\text{op}} : \mathcal{A}^{\text{op}} \hookrightarrow \HomAkmat
\]

\end{remark}

$\HomAkmat$ is Abelian and therefore finitely complete and finitely cocomplete.

\subsection{Projective objects and the Yoneda projective}\label{sec:projective_objects}

\begin{lemma}\label{la:Hom_exact_proj_Lift_along_epis}
Let $\mathcal{C}$ be a locally small category. For an object $P \in \mathcal{C}_{0}$ the following are equivalent:
\begin{itemize}
\item The covariant functor $\mathrm{Hom}(P,-)$ is exact.
\item For all epimorphisms $\varphi : M \twoheadrightarrow N$ and morphisms $\theta : P \rightarrow N$, there exists a
projective lift $\psi : P\dottedrightarrow M$ such that $\theta = \psi\varphi$.\\
\begin{tikzcd}
M \arrow[r, "\varphi", two heads] & N \\
	& P \arrow[u, "\theta", "=\,\psi\varphi"'] \arrow[lu, "\psi", dotted]
\end{tikzcd}
\end{itemize}
\begin{proof}
For an object $L\in \mathcal{C}_{0}$, $\mathrm{Hom}(L,-)$ is always a covariant left-exact functor, i.e. respects monos.\\
\setlist[description]{font=\normalfont}
\begin{description}
\item[``$\Leftarrow$:''] Prove that $\mathrm{Hom}(P,-)$  is right exact, i.e. respects epis.\\
For this, let $M, N \in \mathcal{C}_{0}$ and $\varphi : M \twoheadrightarrow N$ be an epi. The Hom-functor works on morphisms
by mapping the Hom-sets of the source and target objects of the morphism, i.e.
$\mathrm{Hom}(P,\varphi) : \mathrm{Hom}(P,M) \rightarrow \mathrm{Hom}(P,N)$, given by $\rho \mapsto \rho\varphi\, \forall \rho \in \mathrm{Hom}(P,M)$.
Now given that $\varphi$ is an epi, we want to show that $\mathrm{Hom}(P,\varphi)$ is also an epi.\\
Let $O \in \mathcal{C}_{0}$,  $\gamma : N \rightarrow O$ and $\varepsilon : N \rightarrow O$ such that
$\mathrm{Hom}(P,\gamma) : \mathrm{Hom}(P,N) \rightarrow \mathrm{Hom}(P,O);\, \theta \mapsto \theta\gamma$ and
$\mathrm{Hom}(P,\varepsilon) : \mathrm{Hom}(P,N) \rightarrow \mathrm{Hom}(P,O);\, \theta \mapsto \theta\varepsilon$ and
$\mathrm{Hom}(P,\varphi)\mathrm{Hom}(P,\gamma) = \mathrm{Hom}(P,\varphi)\mathrm{Hom}(P,\varepsilon)$. 
From the functoriality axioms (ref. definition \ref{def:functor} of a functor) it follows that $\mathrm{Hom}(P,\varphi\gamma) = \mathrm{Hom}(P,\varphi\varepsilon)$. This implies
\begin{equation}\label{eqn:Hom_functoriality}\rho(\varphi\gamma) = \rho(\varphi\varepsilon)\, \forall \rho \in \mathrm{Hom}(P,M)\end{equation}. 

\begin{tikzcd}
M \arrow[r, "\varphi", shift right, two heads] & N \arrow[r, "\gamma"'] \arrow[r, shift left=2, "\varepsilon"] & O \\
	& P \arrow[lu, "\rho"] \arrow[u, "\theta"] \arrow[ru, outer sep=2, pos=.55, "\rho(\varphi\gamma) = \rho(\varphi\varepsilon)"'] \arrow[ru, shift right=2]
\end{tikzcd}

We want to show that the parallel morphisms $\mathrm{Hom}(P,\gamma)$ and $\mathrm{Hom}(P,\varepsilon)$ are the same, i.e. for all
$\theta \in \mathrm{Hom}(P,N), \theta\gamma = \theta\varepsilon$. Our assumtion that there exists a projective lift helps us in this situation:
$\forall \theta \in \mathrm{Hom}(P,N)\, \exists\, \rho \in \mathrm{Hom}(P,M)$ such that $\theta = \rho\varphi$ and therefore with the above 
equation \eqref{eqn:Hom_functoriality},
$\theta\gamma = (\rho\varphi)\gamma = \rho(\varphi\gamma) = \rho(\varphi\varepsilon) = (\rho\varphi)\varepsilon = \theta\varepsilon$
and therefore $\mathrm{Hom}(P,\gamma) = \mathrm{Hom}(P,\varepsilon)$, i.e. $\mathrm{Hom}(P,\varphi)$ is epi.\\

\item[``$\Rightarrow$:''] Let $\mathrm{Hom}(P,-)$ be right exact. Let $M, N \in \mathcal{C}_{0}$, the morphism
$\varphi : M \twoheadrightarrow N$ be an epi and $\theta : P \rightarrow N$ any morphism.
We want to show the existence of a morphism $\psi : P \dottedrightarrow\, M$ such that $\theta = \psi\varphi$.
With $\mathrm{Hom}(P,-)$ being exact, we have that $\mathrm{Hom}(P,\varphi) : \mathrm{Hom}(P,M) \twoheadrightarrow \mathrm{Hom}(P,N)$ is
an epi, and is given by $\mathrm{Hom}(P,M) \ni \rho \mapsto \rho\varphi \in \mathrm{Hom}(P,N)$.\\
$\mathcal{C}$ is locally small, i.e. for the two objects $P, N \in \mathcal{C}_{0},$ there is a \ul{set} $\mathrm{Hom}(P,N)$
of morphisms between them. The $\mathrm{Hom}$-functor moves the morphisms from a general categorical context 
in $\mathcal{C}$ into the category of sets, i.e. $\mathrm{Hom}(P,\varphi)$ is a function in the category of sets.
And for those it's true that every epimorphism is surjective. Thus $\forall \theta \in \mathrm{Hom}(P,N)\, \exists \rho \in \mathrm{Hom}(P,M)$ such
that $\theta = (\mathrm{Hom}(P,\varphi))(\rho) = \rho\varphi$. This $\rho$ is the projective lift $\psi := \rho$ we were looking for.
\end{description}
\end{proof}
\end{lemma}

\begin{definition}{(Projective object)}\label{def:proj_object}\\
An object $P$ in a category $\mathcal{C}$ that satisfies one (and thus both) of the equivalent properties in Lemma
 \ref{la:Hom_exact_proj_Lift_along_epis} is called a \ul{projective object}.
\end{definition}

The dual statement to Lemma \ref{la:Hom_exact_proj_Lift_along_epis} is
\begin{lemma}\label{la:dual_Hom_exact_proj_colift}
Let $\mathcal{C}$ be a category. For an object $P \in \mathcal{C}_{0}$ the following are equivalent:
\begin{itemize}
\item The contravariant functor $\mathrm{Hom}(-,P)$ is exact.
\item For all monomorphisms $\varphi : M \hookleftarrow N$ and morphisms $\theta : P \leftarrow N$, there exists a
projective colift $\psi : P \dottedleftarrow M$ such that $\theta = \varphi\psi$.\\
\begin{tikzcd}
M \arrow[rd, "\psi"', dotted] & N \arrow[l, "\varphi", hook] \arrow[d, "=\,\varphi\psi", "\theta"'] \\
	& P 
\end{tikzcd}
\end{itemize}
\end{lemma}

\begin{definition}{(Injective object)}\label{def:inj_object}\\
An object $P$ in a category $\mathcal{C}$ that satisfies one (and thus both) of the equivalent properties in Lemma
 \ref{la:dual_Hom_exact_proj_colift} is called an \ul{injective object}.
\end{definition}

\begin{definition}{(Yoneda projective)}
Let $\mathcal{A}$ be a $\Bbbk$-algebroid with finitely many objects and finite-dimensional hom-sets over $\Bbbk$.
With the Yoneda embedding 
\[
Y^{\text{op}} : \begin{cases}\mathcal{A}^{\text{op}} \hookrightarrow \HomAkmat \\
i \mapsto \mathrm{Hom}_{\mathcal{A}}(-,i)
\end{cases}
\]
we see that the image
$Y^{\text{op}}(i)$ of the object $i \in \mathcal{A}^{\text{op}}$ is the representable functor $\mathrm{Hom}_{\mathcal{A}}(-,i)$ in $\HomAkmat$. 
It is called the \ul{$i$-th Yoneda projective}.
\end{definition}

Analogous to Prop. \ref{prop:Alg-Alg-Correspondence} one can easily prove that a $\Bbbk$-representation $F$ of $\mathcal{A}$ corresponds to a 
module $\bigoplus_{i \in \mathcal{A}_{0}} F(i)$ over the algebra $\mathbf{A} = \mathrm{Algebra}(\mathcal{A})$.
In particular the Yoneda projective $Y^{\text{op}}(i) = \mathrm{Hom}_{\mathcal{A}}(-,i)$ corresponds to the $\mathbf{A}$-module
$M_{i} = \bigoplus_{j \in \mathcal{A}_{0}} \mathrm{Hom}_{\mathcal{A}}(j,i)$.

\begin{lemma}
Yoneda projectives are projective objects in $\HomAkmat$.
\end{lemma}
\begin{proof}\phantom{}\\
We want to show that $\mathrm{Hom}_{\HomAkmat}(Y^{\text{op}}(i),-)$ is right exact, i.e. finitely cocontinuous.

\noindent For this, let
\[
\begin{tikzcd}
0 \arrow[r] & F_{1} \arrow[r, "\eta"] & F_{2} \arrow[r,"\rho"] & F_{3} \arrow[r] & 0
\end{tikzcd}
\]
be a short exact sequence in $\HomAkmat$, i.e.
\begin{enumerate}
\item $\eta$ is a monomorphism,
\item $\rho$ is an epimorphism and
\item the image of $\eta$ equals the kernel of $\rho$.
\end{enumerate}

\noindent Then applying the functor $\mathrm{Hom}_{\HomAkmat}(Y^{\text{op}}(i),-)$ on the sequence and simplifying with Yoneda's lemma

\begin{align*}
\mathrm{Hom}_{\HomAkmat}(\mathrm{Hom}_{\mathcal{A}}(-,i),F) &\simeq F(i) \\
\mathrm{Hom}_{\HomAkmat}(\mathrm{Hom}_{\mathcal{A}}(-,i),\rho) &\simeq \rho_{i}
\end{align*}

We only have to measure exactness on the components
\[
\begin{tikzcd}
0 \arrow[r] & F_{1}(i) \arrow[r, "\eta_{i}"] &
F_{2}(i) \arrow[r,"\rho_{i}"] & F_{3}(i) \arrow[r] & 0\mbox{,}
\end{tikzcd}
\]
but this coincides with the definition of the exactness of functors. So the proof is a tautology by Yoneda's lemma.
\end{proof}

\subsubsection{Abelian categories with enough projective objects (constructively)}

\begin{definition}[Enough projective objects]\label{def:enough_projectives}
A category $\mathcal{C}$ is said to have \ul{enough projective objects} if every object admits an epimorphism from a projective object,
i.e. for any object $A \in \mathcal{C}_{0}$ there exists an epimorphism $P \twoheadrightarrow A$, where $P$ is projective.
\end{definition}

We state without a proof the following fact about our functor category:

\begin{theorem}[$\HomAkmat$ has enough projectives]\phantom{}\\
Let $\mathcal{A}$ be a finite-dimensional algebroid over some field $\Bbbk$. The functor category $\HomAkmat$ has sufficiently many
projectives.
\end{theorem}
\begin{proof}
(no proof)
\end{proof}

\begin{doctrine}[Abelian category with enough projective objects]\phantom{}\\
The doctrine $\mathtt{IsAbelianCategoryWithEnoughProjectives}$ therefore involves algorithms of\\
$\mathtt{IsAbelianCategory}$ together with algorithms for
\begin{itemize}
\item $\mathtt{EpimorphismFromSomeProjectiveObject}$,
\item $\mathtt{ProjectiveLift}$,
\end{itemize}
\end{doctrine}

\subsubsection{Abelian categories with enough injective objects}
Dually to \ref{def:enough_projectives} we define

\begin{definition}[Enough injective objects]\label{def:enough_injectives}
A category $\mathcal{C}$ is said to have \ul{enough injective objects} if every object admits a monomorphism into an injective object,
i.e. for any object $A \in \mathcal{C}_{0}$ there exists a monomorphism $A \hookrightarrow J$, where $J$ is injective.
\end{definition}

\begin{doctrine}[Abelian category with enough injective objects]\phantom{}\\
The doctrine $\mathtt{IsAbelianCategoryWithEnoughInjectives}$ therefore involves algorithms of\\
$\mathtt{IsAbelianCategory}$ together with algorithms for
\begin{itemize}
\item $\mathtt{MonomorphismIntoSomeInjectiveObject}$,
\item $\mathtt{InjectiveColift}$,
\end{itemize}
\end{doctrine}


% mainfile: ../main.tex

\section{Functors and natural transformations}

\subsection{Functors act on objects and morphisms of a category}

\subsection{Natural transformations are morphisms between functors}

\subsection{Representations are Functors into a matrix category}

%% mainfile: ../main.tex

\section{The example CatReps}
\[
\begin{tabular}{r|cccccc}
   & 1 & 2 & 3 & 4 & 5 & 6 \\
\hline
$e_{1}$ & 1 & 2 & 3 &   &   &  \\
$e_{2}$ &   &   &   & 4 & 5 & 6 \\
\hline
$a$ & 2 & 3 & 1 &   &   &  \\
$b$ & 4 & 5 & 6 &   &   &  \\
$c$ &   &   &   & 5 & 6 & 4 \\
\hline
\rule{0pt}{0.01pt} \\
$a^{2}$ & 3 & 1 & 2 &   &   &  \\
$ab$ & 5 & 6 & 4 &   &   &  \\
$bc$ & 5 & 6 & 4 &   &   &  \\
$c^{2}$ &   &   &   & 6 & 4 & 5 \\
\hline
\rule{0pt}{1pt} \\
$a^{3}$ & 1 & 2 & 3 &   &   &  \\
$a^{2}b$ & 6 & 4 & 5 &   &   &  \\
$abc$ & 6 & 4 & 5 &   &   &  \\
$bc^{2}$ & 6 & 4 & 5 &   &   &  \\
$c^{3}$ &   &   &   & 4 & 5 & 6 \\
\hline
\hline
\end{tabular}
\]

\[
\begin{tikzcd}
\{1,2,3\} \arrow["a"', loop, distance=2em, in=305, out=235] \arrow[rr, "b"] &  & \{4,5,6\} \arrow["c"', loop, distance=2em, in=305, out=235]
\end{tikzcd}
\]
\[
\begin{tikzcd}
1 \arrow["a"', loop, distance=2em, in=305, out=235] \arrow["e_{1}", loop, distance=2em, in=55, out=125] \arrow[rr, "b"] & 
 & 2 \arrow["c"', loop, distance=2em, in=305, out=235] \arrow["e_{2}", loop, distance=2em, in=55, out=125]
\end{tikzcd}
\]


% mainfile: ../main.tex

\section{Algorithms}

\begin{algorithm}\capstart
   \caption{\texttt{ConvertToMapOfFinSets}}\label{algo:ConvertToMapOfFinSets}
      \SetKwInput{Input}{Input~}
      \SetKwInput{Output}{Output~}
      \Input{~a list $objects$ of objects in FinSets and a morphism $gen$ given as a list of images in the convention of catreps}
      \Output{~the corresponding map of finite sets from source $S$ to target $T$}
      \BlankLine
      let $T$ be the first object $O \in objects$ such that $gen \cap O \not= \emptyset$\;
      \If{$gen \cap O = \emptyset \, \forall O \in objects$}{
         Error "unable to find target set"
      }
      let $fl$ be the flattening of $objects$ as a list\;
      let $S$ be the sublist of $fl$ according to positions $i$ such that $gen[i]$ is bound\;
      set $S$ to be the first object $O \in objects$ such that $O = S$\;
      \If{$S \not= O \, \forall O \in objects$}{
          Error "unable to find source set"
      }
      \BlankLine
      let $G$ be the list of pairs $[ i, gen[i] ], i \in S$;
      \BlankLine
      \Return MapOfFinSets( S, G, T );
\end{algorithm}

We can now create finite concrete categories with objects not starting from 1, to demonstrate that
\texttt{ConcreteCategoryForCAP( [ [,,,5,6,4], [,,,7,8,9], [,,,,,,8,9,7] ] )} and\\
\texttt{ConcreteCategoryForCAP( [ [2,3,1], [4,5,6], [,,,5,6,4] ] )} yield
equivalent categories, i.e. their underlying quivers are the same and they give the same category of representations.

\begin{algorithm}\capstart
    \caption{\texttt{RightQuiverFromConcreteCategory}}\label{algo:RightQuiverFromConcreteCategory}
	\SetKwInput{Input}{Input~}
	\SetKwInput{Output}{Output~}
	\Input{~a finite concrete category $C$ with $n$ objects}
	\Output{~the right quiver $q(n)$}
	\BlankLine
	let $Obj$ be the set of objects of $C$\;
	let $n := Length(Obj)$\;
	let $gMor$ be the set of generating morphisms of $C$\;
	let $A$ be the empty set and let $i := 1$\;
	\ForEach{morphism $mor$ in $gMor$}{
	    let $A_{i,1}$ be the position of $Source( mor )$ in $Obj$\;
	    let $A_{i,2}$ be the position of $Range( mor )$ in $Obj$\;
	    let $i := i+1$\;
	}
	\BlankLine
	let $q$ be the right quiver with vertices $\{1,\dots,n\}$ and arrows $A$.
	\BlankLine
	\Return q\;
\end{algorithm}

\begin{algorithm}\capstart
    \caption{\texttt{RelationsOfEndomorphisms}}\label{algo:RelationsOfEndomorphisms}
	\SetKwInput{Input}{~Input}
	\SetKwInput{Output}{~Output}
	\Input{~a commutative ring $k$ and a finite concrete category $C$}
	\Output{~the endomorphism relations of the category $C$}
	\BlankLine
	let $q := \texttt{RightQuiverFromConcreteCategory}(C)$\;
	let $kq$ be the path algebra generated by $k$ and $q$\;
	let $gMor$ be the set of generating morphisms of $C$\;
	let $A := Arrows(q)$\;
	let $relsEndo$ be the empty set\;
	\ForEach{$i = 1, \dots, Length(gMor)$}{
	    let $mor := gMor_i$\\
	    \If{$mor$ is not an endomorphism}{
		continue\;
	    }
	    let $m := 0$ and let $powers$ be the empty set\;
	    let $foundEqual$ be false\;
	    \While{$mor^{m}\notin powers$}{
		let $n := 1$\;
		\While{$\neg foundEqual$}{
		    \If{$mor^{(m+n)} = mor^{m}$}{
		    	Add the relation $kq.(A_{i})^{(m+n)}-kq.(A_{i})^{m}$ to relsEndo\;
		    	foundEqual := true\;
		    }
		    n := n+1\;
		}
		Add $mor^{m}$ to powers\;
		m := m+1\;
	    }
	}
	\Return{relsEndo}\;
\end{algorithm}

\begin{algorithm}\capstart
   \caption{\texttt{Algebroid}}\label{algo:Algebroid}
      \SetKwInput{Input}{~Input}
      \SetKwInput{Output}{~Output}
      \Input{~a commutative ring $k$ and a finite concrete category $C$}
      \Output{~the $k$-linear closure of the category $C$ over the commutative ring $k$}
      \BlankLine
      
      \Return{}\;
\end{algorithm}

Yonedas Einbettungs-Lemma: Fehlende Limiten bzw. Kolimiten exitieren nach der Einbettung.

Einbettung in Kategorien, die mehr Limiten haben als die Zielkategorie.

"(Ko-)Vervollständigung" der Kategorie (Completion / Cocompletion)

Quiver = unvollständige Struktur einer Kategorie
Erzeugendensystem einer Kategorie.

K-linearer Abschluss einer Kategorie

Pfadalgebra = Kategorien-Algebra
path algebra = 1 Object, welches eine Algebra ist. Dabei verliert man wieder die Informationen über die
mehreren Objekte.

So wie Menge ein Erz-system eines Monoid.

% mainfile: ../main.tex

\section{Relations of the Algebroid}

\subsection{Relations of endomorphisms}

% A forest on a cycle
\begin{tikzpicture}[x=0.5cm,y=0.5cm]
\tikzstyle{cblack}=[circle, fill=black, scale=0.5]

%Nodes
\foreach \place/\x in {{(0,0)/0}, {(-4.5,0)/1}, {(-7,-3)/2}, {(-4.5,-6)/3},
  {(0,-6)/4}, {(2.5,-3)/5},
  {(-4.5,3)/6}, {(-7.5,6)/7}, {(-4.5,6)/8},
  {(0,3)/9}, {(0,6)/10}, {(0,9)/11},
  {(3,3)/12}, {(3,6)/13}, {(3,9)/14},
  {(7.5,6)/15}, {(7.5,9)/16}, {(7.5,12)/17}, {(10.5,9)/18}}
\node[cblack] (a\x) at \place {};

%Arrows
\foreach \i in {0,1,2,3,4,5}
{
  \pgfmathtruncatemacro\result{Mod(\i+1,6)}%
  \draw[->] (a\i) -> (a\result);
}
\path[->] (a7) edge (a6); 
\path[->] (a8) edge (a6) edge (a1);
\path[->] (a11) edge (a10) edge (a9) edge (a0);
\path[->] (a14) edge (a13) edge (a12); 
\path[->] (a12) edge (a0);
\path[->] (a18) edge (a15) (a15) edge (a12);
\path[->] (a17) edge (a16) edge (a15);
%\path[->] (a\x) edge (a\y);

\end{tikzpicture}
%\cite{facchini_2019}
\begin{lemma}[$\sigma$-Lemma]
Let $\mathcal{C}$ be a finite concrete category. Then for each object $M \in \mathcal{C}_{0}$ the set
$\textup{End}_{\mathcal{C}}(M)$ is a monoid and for each endomorphism $f \in \textup{End}_{\mathcal{C}}(M)$
there exist $m,n \in \mathbb{N}$ such that $f^{(m+n)}=f^{m}$. If $m = 0$ and $n \geq 1$ then $f$ is bijective with $f^{-1} = f^{n-1}$.
\begin{proof}
The properties of a monoid are precisely the associativity of composition and the unit property from \ref{associativity_of_composition} and \ref{unit_property}.
Since $\abs{\textup{End}_{\mathcal{C}}(M)}<\infty$ there are only finitely many endomorphisms $f_{1},\dots, f_{N} \in \textup{End}_{\mathcal{C}}(M)$.
Let $\{f^{k} | k \in \mathbb{N} \} \subset \textup{End}_{\mathcal{C}}(M)$, i.e. there is a function 
$\{f^{k} | k \in \mathbb{N}\} \rightarrow \{f_{j} | j \in \{1,\dots,N\}\}; f^{k} \mapsto f_{j}$ not necessarily surjective and 
by the pigeonhole principle highly non injective, since $\abs{\mathbb{N}}>\abs{\textup{End}_{\mathcal{C}}(M)}$.
Let $m := Min \{ k \in \mathbb{N}| f^{k} =  f_{j} \}$

\begin{minipage}{.45\textwidth}\phantom{}\end{minipage}
\end{proof}
\end{lemma}

\begin{lemma}
Algorithm \ref{algo:RelationsOfEndomorphisms} terminates and yields the correct result.
\begin{proof}
Since $\mathcal{C}$ is a finite concrete category, for each object $M \in \mathcal{C}_{0}$ the endomorphism set
$\textup{End}_{\mathcal{C}}(M)$ is finite. If in step $i$, $f := gMor_{i} \in \textup{End}_{\mathcal{C}}(M)$ is an endomorphism,
then $\{f, f^{2}, f^{3}, \dots\}$ is a subset of the finite set $\textup{End}_{\mathcal{C}}(M)$, therefore $\exists N \in \mathbb{N}$ such that
$f^{k} \in \{f, f^{2}, f^{3}, \dots, f^{N}\} \, \forall k \geq N$.
This proves that the sets $mpowers$ and $npowers$, which contain increasing powers of $f$, will be finite and thus at some point
already contain $f^{m}$ or $f^{m+n}$ respectively, causing the while loops to terminate.
The if clause makes sure that the set $relsEndo$ only contains the desired relations. It remains to be shown that those are all
the endomorphism relations.
\end{proof}
\end{lemma}

\begin{algorithm}\capstart
    \caption{\texttt{RelationsOfEndomorphisms}}\label{algo:RelationsOfEndomorphisms}
	\SetKwInput{Input}{~Input}
	\SetKwInput{Output}{~Output}
	\Input{~a commutative ring $k$ and a finite concrete category $\mathcal{C}$}
	\Output{~the endomorphism relations of the category $\mathcal{C}$}
	\BlankLine
	let $q := \texttt{RightQuiverFromConcreteCategory}(\mathcal{C})$\;
	let $kq$ be the path algebra generated by $k$ and $q$\;
	let $gMor$ be the set of generating morphisms of $\mathcal{C}$\;
	let $A := Arrows(q)$\;
	set $relsEndo := \emptyset$\;
	\ForEach{$i = 1, \dots, Length(gMor)$}{
	    let $f := gMor_i$\\
	    \If{$f$ is not an endomorphism}{
		continue\;
	    }
	    let $m := 0$\;
	    set $mpowers := \emptyset$\;
	    let $foundEqual$ be false\;
	    \While{ $f^{m}\nin mpowers$ }{
		let $n := 1$\;
	        set $npowers := \emptyset$\;
		\While{ $\neg foundEqual$ and $f^{m+n} \nin npowers$ }{
		    \If{ $f^{m+n} = f^{m}$ }{
		    	Add the relation $kq.(A_{i})^{m+n}-kq.(A_{i})^{m}$ to relsEndo\;
		    	foundEqual := true\;
		    }
		    Add $f^{m+n}$ to npowers\;
		    n := n+1\;
		}
		Add $f^{m}$ to mpowers\;
		m := m+1\;
	    }
	}
	\Return{relsEndo}\;
\end{algorithm}

Beschreibung der Algorithmen

WeakDirectSumDecomposition <-- Tiefensuche.
Objekte (Funktoren) in indecomposable Functors.


\begin{example}\label{representation}{(Representation of a concrete category)}\\
\begin{center}
\begin{tikzcd}[boxedcd={inner sep=1pt}]
                                                                                           &  &  &  & \\
                                                                                              &  &                                                                       \\
                                                                                              &  &                                                                       \\
                                                                                              &  &                                                                       \\
&  & 5 \arrow[rr, "{\begin{pmatrix} 
0\ampersand1\ampersand0\ampersand0\\
0\ampersand0\ampersand1\ampersand0\\
0\ampersand0\ampersand0\ampersand0\\
0\ampersand1\ampersand0\ampersand1\\
0\ampersand0\ampersand1\ampersand0
\end{pmatrix}}"]
\arrow["{\begin{pmatrix} 
1\ampersand 1\ampersand 0\ampersand 0\ampersand 0\\
0\ampersand 1\ampersand 1\ampersand 0\ampersand 0\\
0\ampersand 0\ampersand 1\ampersand 0\ampersand 0\\
0\ampersand 0\ampersand 0\ampersand 1\ampersand 1\\
0\ampersand 0\ampersand 0\ampersand 0\ampersand 1 
\end{pmatrix}}"', loop, distance=2em, in=305, out=235]             &  & 
4 \arrow["{\begin{pmatrix}
1\ampersand1\ampersand0\ampersand0\\
0\ampersand1\ampersand1\ampersand0\\
0\ampersand0\ampersand1\ampersand0\\
0\ampersand0\ampersand0\ampersand1
\end{pmatrix}}"', loop, distance=2em, in=305, out=235]  &  &         \\
                                                                                              &  &                                                                       \\
                                                                                              &  &                                                                       \\
                                                                                              &  &                                                                       \\   
                                                                                              &  &                                                                       \\   
\end{tikzcd}
\end{center}
\begin{center}
\begin{tikzcd}
                                                                                              & {} &                                                                       \\
                                                                                              & {} &                                                                       \\
                                                                                              & {} \arrow["nine",u, Rightarrow] &                                                                       \\
\end{tikzcd}
\end{center}
\begin{center}
\begin{tikzcd}[boxedcd={inner sep=1pt}]
                                                                                              &  &                                                                       \\
&  1 \arrow["a"', loop, distance=2em, in=305, out=235] \arrow[rr, "b"] \arrow[rr] \arrow[rr]     &  & 
2 \arrow["c"', loop, distance=2em, in=305, out=235]  &                   \\
                                                                                              &  &                                                                       \\
\end{tikzcd}
\end{center}
\begin{center}
\begin{tikzcd}[boxedcd={inner sep=1pt}]
                                                                                              &  &                                                                       \\
&  {\{1,2,3\}} \arrow["{(2,1,3)}"', loop, distance=2em, in=305, out=235] 
\arrow[rr, "{(4,5,6)}"] &  & 
{\{4,5,6\}} \arrow["{(5,6,4)}"', loop, distance=2em, in=305, out=235]  & \\
                                                                                              &  &                                                                       \\
\end{tikzcd}
\end{center}


$F(a) \eta_{1} = \eta_{1} G(a)$,\\
$F(b) \eta_{2} = \eta_{1} G(b)$
\end{example}

\section{$\mathbb{K}$-linear Category (Algebroid)}

Group: Category with one object.

Groupoid: A small category in which every morphism is an isomorphism.

Algebroid

EmbeddingOfSumOfImages

What is an Algebroid? Bialgebroid?

\section{Additive Category}

\section{Abelian Category}

\section{The Category of Categories}

\section{The Categories of Functors}

\section{The Representation of a Category}

\section{Representation}

Grundidee von FunctorCategory

Standard-Monoidale Struktur von der Zielkategorie z.B. TensorUnit(C)

\section{Algorithms}

\begin{algorithm}\capstart
    \caption{\texttt{RightQuiverFromConcreteCategory}}\label{algo:RightQuiverFromConcreteCategory}
	\SetKwInput{Input}{Input~}
	\SetKwInput{Output}{Output~}
	\Input{~a finite concrete category $C$ with $n$ objects}
	\Output{~the right quiver $q(n)$}
	\BlankLine
	let $Obj$ be the set of objects of $C$\;
	let $n := Length(Obj)$\;
	let $gMor$ be the set of generating morphisms of $C$\;
	let $A$ be the empty set and let $i := 1$\;
	\ForEach{morphism $mor$ in $gMor$}{
	    let $A_{i,1}$ be the position of $Source( mor )$ in $Obj$\;
	    let $A_{i,2}$ be the position of $Range( mor )$ in $Obj$\;
	    let $i := i+1$\;
	}
	\BlankLine
	let $q$ be the right quiver with vertices $\{1,\dots,n\}$ and arrows $A$.
	\BlankLine
	\Return q\;
\end{algorithm}

We want the endomorphism relations so that the path algebra is finite-dimensional and we
get a finite Gröbner basis.

\begin{algorithm}\capstart
    \caption{\texttt{RelationsOfEndomorphisms}}\label{algo:RelationsOfEndomorphisms}
	\SetKwInput{Input}{~Input}
	\SetKwInput{Output}{~Output}
	\Input{~a commutative ring $k$ and a finite concrete category $C$}
	\Output{~the endomorphism relations of the category $C$}
	\BlankLine
	let $q := \texttt{RightQuiverFromConcreteCategory}(C)$\;
	let $kq$ be the path algebra generated by $k$ and $q$\;
	let $gMor$ be the set of generating morphisms of $C$\;
	let $A := Arrows(q)$\;
	let $relsEndo$ be the empty set\;
	\ForEach{$i = 1, \dots, Length(gMor)$}{
	    let $mor := gMor_i$
	    \If{$mor$ is not an endomorphism}{
		continue\;
	    }
	    let $m := 0$ and let $powers$ be the empty set\;
	    let $foundEqual$ be false\;
	    \While{$mor^{m}\notin powers$}{
		let $n := 1$\;
		\While{$\neg foundEqual$}{
		    \If{$mor^{(m+n)} = mor^{m}$}{
		    	Add the relation $kq.(A_{i})^{(m+n)}-kq.(A_{i})^{m}$ to relsEndo\;
		    	foundEqual := true\;
		    }
		    n := n+1\;
		}
		Add $mor^{m}$ to powers\;
		m := m+1\;
	    }
	}
	\Return{relsEndo}\;
\end{algorithm}

Proof that algorithm is correct
Proof that it terminates.

Wir haben BasisOfExternalHom benutzt um Decompose in CAP umzusetzen um EmbeddingOfSubRepresentation umzusetzen um
WeakDirectSumDecomposition umzusetzen.

%%% insert endnotes in some way
\begingroup
     \parindent 0pt
     \parskip 2ex
     \def\enotesize{\normalsize}
     \theendnotes
\endgroup 

\input{bib/sources.bib}

\end{document}