\documentclass{article}

\usepackage[a4paper,%
            left=2.5cm,right=2.5cm,top=2.5cm,bottom=2.5cm,%
            footskip=.6cm]{geometry}

\def\changemargin#1#2{\list{}{\rightmargin#2\leftmargin#1}\item[]}
\let\endchangemargin=\endlist


%%% For Math
\usepackage{amsmath}
\usepackage{amsfonts}
\usepackage{amsbsy}
\usepackage{amsthm}
\usepackage{amssymb}

\usepackage{mathtools}
\usepackage{commath}
\usepackage[sc,osf]{mathpazo}
\usepackage{mnsymbol}

%%% For calculations inside tikz and latex
\usepackage{calc}

\newcounter{modresult}

\newcommand*{\themodulo}[2]{%
\setcounter{modresult}{%
#1-(#1/#2)*#2%
}%
#1 mod #2 = \themodresult\par
}

%%% For algorithm styles
\usepackage[linesnumbered,ruled]{algorithm2e}


%%% Math theorem styles
\newtheorem{thm}{Theorem}[section]
\newtheorem{lemma}[thm]{Lemma}
\theoremstyle{definition}
\newtheorem{definition}[thm]{Definition}
\newtheorem{rmk}[thm]{Remark}
\newtheorem{example}[thm]{Example}

%%% Math operators bold
%\newcommand{\Category}{Category}

%%% For arrows and categories
\usepackage[all]{xy}
\usepackage{tikz-cd}

%%% For matrices
\let\ampersand =&

%%% tikz
\usetikzlibrary{positioning}

%%% For dotted box around diagrams
\tikzcdset{
    boxedcd/.style={
        every matrix/.append style={
            draw=black,
            dotted,
            rounded corners,
            #1
        },
    },
}

%%% For some big dots
\makeatletter
\newcommand*{\bigcdot}{}% Check if undefined
\DeclareRobustCommand*{\bigcdot}{%
  \mathbin{\mathpalette\bigcdot@{}}%
}
\newcommand*{\bigcdot@scalefactor}{.5}
\newcommand*{\bigcdot@widthfactor}{1.15}
\newcommand*{\bigcdot@}[2]{%
  % #1: math style
  % #2: unused
  \sbox0{$#1\vcenter{}$}% math axis
  \sbox2{$#1\cdot\m@th$}%
  \hbox to \bigcdot@widthfactor\wd2{%
    \hfil
    \raise\ht0\hbox{%
      \scalebox{\bigcdot@scalefactor}{%
        \lower\ht0\hbox{$#1\bullet\m@th$}%
      }%
    }%
    \hfil
  }%
}
\makeatother

%%% For horizontal times


%%% For proper underline
\usepackage{soul}
%\setuldepth{gjpqy}
%\setuldepth\strut
\setuldepth{-1}

\usepackage{pgffor}

%%% For source code listings
\usepackage{listings}
\def \pkgpath {C:/Users/Tibor/AppData/Local/Packages/CanonicalGroupLimited.UbuntuonWindows_79rhkp1fndgsc/LocalState/rootfs/home/user/.gap/pkg}
%%% For footnotes at end of text
\usepackage{endnotes}
\let\footnote=\endnote

%%% For captions
\usepackage{hyperref}
\usepackage[figure]{hypcap}

\title{Bachelorarbeit Representations of Concrete Categories as Objects in the Functor Category}

\author{Tibor Gr{\"u}n}

\begin{document}
	\pagenumbering{gobble}

	\maketitle

	\newpage

	\tableofcontents

	\newpage

	\pagenumbering{arabic}

%% mainfile: ../main.tex

\section{Introduction}

\subsection{Purpose of this thesis}

Die Aufgabe der vorliegenden Arbeit besteht darin, das GAP-Paket "catreps" von Peter Webb mit CAP zu re-organisieren und Peter dabei ziemlich alt aussehen zu lassen. Dabei werden die Algorithmen, welche in catreps direkt implementiert sind, soweit wie möglich durch vorhandene Methoden von CAP ersetzt.
Insbesondere wird das CAP-Paket FunctorCategories Anwendung finden, weil ich zeigen werde, dass catreps, also die Kategorie der Darstellungen einer konkreten endlichen Kategorie, nichts anderes ist als eine Unterkategorie von FunctorCategories, also der Kategorie aller Funktoren zwischen Kategorien. 
Da catreps selbst bereits eine Verallgemeinerung der Darstellung endlicher Gruppen ist (eine Gruppe ist nichts anderes als eine Kategorie mit einem Objekt, in dem jeder Morphismus ein Isomorphismus ist), stellt FunctorCategories wohl den allgemeinsten Rahmen dar, den man sich vorstellen kann.

In this thesis I will define what a category is, then I go further in the doctrine of enriched categories, especially monoidal categories.
The morphisms in the category of categories are the functors between categories. I will treat the functor category where the functors themselves are
objects and natural transformations the morphisms between them. I will show that any representation of a category (and thus any representation
of a group) is a functor, so the category of representations (of a category) is a subcategory of the functor category.
I will show how the monoidal structure of the category of representations arises from the counit and the comultiplication on the Bialgebroid structure
on the category.

\noindent Throughout the thesis I will give proofs of existence by providing an algorithm that computes the object that exists. I will be using CAP, the gap package
developed by Sebastian Gutsche et al. Another purpose of this thesis is the translation of the work of Peter Webb, who used gap directly, into our CAP
framework. This includes his decomposition algorithm for a representation. As the category of representations is just a subcategory of the functor category,
most of the work will be done inside the package FunctorCategories by Prof. Mohamed Barakat. In this thesis I will also write the documentation for the
package FunctorCategories.

% mainfile: ../main.tex

\section{Introduction}

\subsection{Purpose of this thesis}

Die Aufgabe der vorliegenden Arbeit besteht darin, das GAP-Paket "catreps" von Peter Webb mit CAP zu re-organisieren und Peter dabei ziemlich alt aussehen zu lassen. Dabei werden die Algorithmen, welche in catreps direkt implementiert sind, soweit wie möglich durch vorhandene Methoden von CAP ersetzt.
Insbesondere wird das CAP-Paket FunctorCategories Anwendung finden, weil ich zeigen werde, dass catreps, also die Kategorie der Darstellungen einer konkreten endlichen Kategorie, nichts anderes ist als eine Unterkategorie von FunctorCategories, also der Kategorie aller Funktoren zwischen Kategorien. 
Da catreps selbst bereits eine Verallgemeinerung der Darstellung endlicher Gruppen ist (eine Gruppe ist nichts anderes als eine Kategorie mit einem Objekt, in dem jeder Morphismus ein Isomorphismus ist), stellt FunctorCategories wohl den allgemeinsten Rahmen dar, den man sich vorstellen kann.

In this thesis I will define what a category is, then I go further in the doctrine of enriched categories, especially monoidal categories.
The morphisms in the category of categories are the functors between categories. I will treat the functor category where the functors themselves are
objects and natural transformations the morphisms between them. I will show that any representation of a category (and thus any representation
of a group) is a functor, so the category of representations (of a category) is a subcategory of the functor category.
I will show how the monoidal structure of the category of representations arises from the counit and the comultiplication on the Bialgebroid structure
on the category.

\noindent Throughout the thesis I will give proofs of existence by providing an algorithm that computes the object that exists. I will be using CAP, the gap package
developed by Sebastian Gutsche et al. Another purpose of this thesis is the translation of the work of Peter Webb, who used gap directly, into our CAP
framework. This includes his decomposition algorithm for a representation. As the category of representations is just a subcategory of the functor category,
most of the work will be done inside the package FunctorCategories by Prof. Mohamed Barakat. In this thesis I will also write the documentation for the
package FunctorCategories.


\[
\mathbf{Quiv}\rightarrow^{CatClosure}\leftarrow_{U}\mathbf{Cats}
\rightarrow^{k-Algebroid}\leftarrow_{U}\mathbf{k-Cats}
\rightarrow^{AdditiveClosure}\leftarrow_{U}\mathbf{k-Cats^{\oplus}}
\]

% mainfile: ../main.tex

\section{Directed Quiver, Path Algebra and the Algebroid}

\begin{definition}{(Quiver)}\\
A \ul{directed quiver} $q$ consists of a class of \ul{objects} (or \ul{vertices}) $q_{0} = \textup{Obj}\,q$ and
a class of \ul{morphisms} (or \ul{arrows}) $q_{1} = \textup{Mor}\,q$ together with two defining maps
\[
\begin{tikzcd}[column sep=small]
{s,t\colon q_{1}} \arrow[rr, shift left = 0.7ex] \arrow[rr, shift right = 0.7ex] & & q_{0}
\end{tikzcd}
\]
$s$ called \ul{source} and $t$ called \ul{target}.
\end{definition}

\noindent In \texttt{QPA}, the objects are coded by natural numbers, so the first object is $1$, the second $2$ and so on. The arrows are denoted by
small letters $a, b, c$ and so on. There is a difference between \texttt{RightQuiver} and \texttt{LeftQuiver} in that the right quiver is \ul{right-oriented}
(that is, the convention for order in multiplication of paths is the opposite of that used for left-oriented quivers).

\begin{example}\label{q(2)}{(Quiver with 2 objects and 3 morphisms)}\\
\[
\begin{tikzcd}
1 \arrow["a"', loop, distance=2em, in=305, out=235] \arrow[rr, "b"] &  & 2 \arrow["c"', loop, distance=2em, in=305, out=235]
\end{tikzcd}
\]
The objects of this quiver $q$ are $q_{0} = \{1, 2\}$, and the morphisms are $q_{1} = \{a, b, c\}$ with\\
$s (a) = 1 = t (a)$, $s (c) = 2 = t (c)$ and $s (b) = 1, t (b) = 2$.\\

\noindent In \texttt{QPA} this quiver is encoded as \texttt{q(2)[a:1->1,b:1->2,c:2->2]} where the first \texttt{(2)} in parentheses stands for the total
number of objects.

Category closure of quiver

\end{example}

Quiver -> CAT: U: forget 1, forget composition

search U^(-1)

Beispiel für Adjunktion

\begin{lemma} Let Q be a quiver. If there is a path of length at least $\abs{Q_{0}}$, then there are cyclic paths,
and thus infinitely many paths.
\end{lemma}
\begin{proof}
Assume that there exists a path of length greater or equal to $\abs{Q_{0}}$. Then there exists a path of length |$Q_{0}$|, say
$alpha_{n}\cdots alpha_{1}$. Consider the vertices $x_{i}=s(alpha_{i})$ for $1 \leq i \leq n$ and $x_{n+1}=t(alpha_{n})$. Then these
are $n+1$ vertices, thus there has to exist $i<j$ with $x_{i}=x_{j}$. Let $\omega=alpha_{j-1}\cdots alpha_{i}$, this is a path with target and source
$x_{i}=x_{j}$, thus a cyclic path. But then $\omega^{m}$ is a path for any natural number $m$. The path $\omega$ has length $j-i\geq1$, thus
$\omega^{m}$ has length $m(j-i)$. This shows that these paths are pairwise different.
\end{proof}
Path Algebra:
%% mainfile: ../main.tex

\section{Directed Quiver, Path Algebra and the Algebroid}

\begin{definition}{(Quiver)}\\
A \ul{directed quiver} $q$ consists of a class of \ul{objects} (or \ul{vertices}) $q_{0} = \textup{Obj}\,q$ and
a class of \ul{morphisms} (or \ul{arrows}) $q_{1} = \textup{Mor}\,q$ together with two defining maps
\[
\begin{tikzcd}[column sep=small]
{s,t\colon q_{1}} \arrow[rr, shift left = 0.7ex] \arrow[rr, shift right = 0.7ex] & & q_{0}
\end{tikzcd}
\]
$s$ called \ul{source} and $t$ called \ul{target}.
\end{definition}

\noindent In \texttt{QPA}, the objects are coded by natural numbers, so the first object is $1$, the second $2$ and so on. The arrows are denoted by
small letters $a, b, c$ and so on. There is a difference between \texttt{RightQuiver} and \texttt{LeftQuiver} in that the right quiver is \ul{right-oriented}
(that is, the convention for order in multiplication of paths is the opposite of that used for left-oriented quivers).

\begin{example}\label{q(2)}{(Quiver with 2 objects and 3 morphisms)}\\
\[
\begin{tikzcd}
1 \arrow["a"', loop, distance=2em, in=305, out=235] \arrow[rr, "b"] &  & 2 \arrow["c"', loop, distance=2em, in=305, out=235]
\end{tikzcd}
\]
The objects of this quiver $q$ are $q_{0} = \{1, 2\}$, and the morphisms are $q_{1} = \{a, b, c\}$ with\\
$s (a) = 1 = t (a)$, $s (c) = 2 = t (c)$ and $s (b) = 1, t (b) = 2$.\\

\noindent In \texttt{QPA} this quiver is encoded as \texttt{q(2)[a:1->1,b:1->2,c:2->2]} where the first \texttt{(2)} in parentheses stands for the total
number of objects.

Category closure of quiver

\end{example}

Quiver -> CAT: U: forget 1, forget composition

search U^(-1)

Beispiel für Adjunktion

\begin{lemma} Let Q be a quiver. If there is a path of length at least $\abs{Q_{0}}$, then there are cyclic paths,
and thus infinitely many paths.
\end{lemma}
\begin{proof}
Assume that there exists a path of length greater or equal to $\abs{Q_{0}}$. Then there exists a path of length |$Q_{0}$|, say
$alpha_{n}\cdots alpha_{1}$. Consider the vertices $x_{i}=s(alpha_{i})$ for $1 \leq i \leq n$ and $x_{n+1}=t(alpha_{n})$. Then these
are $n+1$ vertices, thus there has to exist $i<j$ with $x_{i}=x_{j}$. Let $\omega=alpha_{j-1}\cdots alpha_{i}$, this is a path with target and source
$x_{i}=x_{j}$, thus a cyclic path. But then $\omega^{m}$ is a path for any natural number $m$. The path $\omega$ has length $j-i\geq1$, thus
$\omega^{m}$ has length $m(j-i)$. This shows that these paths are pairwise different.
\end{proof}
Path Algebra:


Quiver = unvollständige Struktur einer Kategorie
Erzeugendensystem einer Kategorie.

K-linearer Abschluss einer Kategorie

Pfadalgebra = Kategorien-Algebra

So wie Menge ein Erz-system eines Monoid.

% mainfile: ../main.tex

\section{Relations of the Algebroid}

\subsection{Relations of endomorphisms}

% A forest on a cycle
\begin{tikzpicture}[x=0.5cm,y=0.5cm]
\tikzstyle{cblack}=[circle, fill=black, scale=0.5]

%Nodes
\foreach \place/\x in {{(0,0)/0}, {(-4.5,0)/1}, {(-7,-3)/2}, {(-4.5,-6)/3},
  {(0,-6)/4}, {(2.5,-3)/5},
  {(-4.5,3)/6}, {(-7.5,6)/7}, {(-4.5,6)/8},
  {(0,3)/9}, {(0,6)/10}, {(0,9)/11},
  {(3,3)/12}, {(3,6)/13}, {(3,9)/14},
  {(7.5,6)/15}, {(7.5,9)/16}, {(7.5,12)/17}, {(10.5,9)/18}}
\node[cblack] (a\x) at \place {};

%Arrows
\foreach \i in {0,1,2,3,4,5}
{
  \pgfmathtruncatemacro\result{Mod(\i+1,6)}%
  \draw[->] (a\i) -> (a\result);
}
\path[->] (a7) edge (a6); 
\path[->] (a8) edge (a6) edge (a1);
\path[->] (a11) edge (a10) edge (a9) edge (a0);
\path[->] (a14) edge (a13) edge (a12); 
\path[->] (a12) edge (a0);
\path[->] (a18) edge (a15) (a15) edge (a12);
\path[->] (a17) edge (a16) edge (a15);
%\path[->] (a\x) edge (a\y);

\end{tikzpicture}
%\cite{facchini_2019}
\begin{lemma}[$\sigma$-Lemma]
For each endomorphism $f$ in a finite concrete category $\mathcal{C}$ there exist $m,n\in\mathbb{N}$
such that $f^{(m+n)}=f^{m}$.
\end{lemma}
%% mainfile: ../main.tex

\section{Relations of the Algebroid}

\subsection{Relations of endomorphisms}

% A forest on a cycle
\begin{tikzpicture}[x=0.5cm,y=0.5cm]
\tikzstyle{cblack}=[circle, fill=black, scale=0.5]

%Nodes
\foreach \place/\x in {{(0,0)/0}, {(-4.5,0)/1}, {(-7,-3)/2}, {(-4.5,-6)/3},
  {(0,-6)/4}, {(2.5,-3)/5},
  {(-4.5,3)/6}, {(-7.5,6)/7}, {(-4.5,6)/8},
  {(0,3)/9}, {(0,6)/10}, {(0,9)/11},
  {(3,3)/12}, {(3,6)/13}, {(3,9)/14},
  {(7.5,6)/15}, {(7.5,9)/16}, {(7.5,12)/17}, {(10.5,9)/18}}
\node[cblack] (a\x) at \place {};

%Arrows
\foreach \i in {0,1,2,3,4,5}
{
  \pgfmathtruncatemacro\result{Mod(\i+1,6)}%
  \draw[->] (a\i) -> (a\result);
}
\path[->] (a7) edge (a6); 
\path[->] (a8) edge (a6) edge (a1);
\path[->] (a11) edge (a10) edge (a9) edge (a0);
\path[->] (a14) edge (a13) edge (a12); 
\path[->] (a12) edge (a0);
\path[->] (a18) edge (a15) (a15) edge (a12);
\path[->] (a17) edge (a16) edge (a15);
%\path[->] (a\x) edge (a\y);

\end{tikzpicture}
%\cite{facchini_2019}
\begin{lemma}[$\sigma$-Lemma]
For each endomorphism $f$ in a finite concrete category $\mathcal{C}$ there exist $m,n\in\mathbb{N}$
such that $f^{(m+n)}=f^{m}$.
\end{lemma}



\section{Category}

\begin{definition}{(Quiver)}\\
A \ul{quiver} $A$ consists of a class of  \ul{objects} (or vertices) $A_{0} = \textup{Obj} A$ and 
a class of \ul{morphisms} (or arrows) $A_{1} = \textup{Mor} A$ together with two defining maps
\[
\begin{tikzcd}[column sep=small]
{s,t\colon A_{1}} \arrow[rr, shift left = 0.7ex] \arrow[rr, shift right = 0.7ex] & & A_{0}
\end{tikzcd}
\]
$s$ called \ul{source} and $t$ called \ul{target}.\footnote{Some authors use maps $t, h$ for $tail$ and $head$ instead of source and target, defining the arrows to go from the tail to the head. This use of $t$ as the starting point instead of the end target as in our definition can lead to some confusion.}\\
\noindent We write $\textup{Hom}_A (M,N)$ (sometimes also $A(M,N)$) for the fiber $(s,t)^{-1} (\{(M,N)\})$ of the product
map $(s,t) : A_{1} \rightarrow A_{0} \times A_{0}$ over the pair $(M,N) \in A_{0} \times A_{0}$.\\
This is the class of all morphisms with source $= M$ and target $= N$.\\
For a morphism $\varphi \in \textup{Hom}_{A}(M,N)$ we write
\[
\begin{tikzcd}[column sep=small]
\varphi : M \arrow[rr] & & N
\end{tikzcd}
or
\begin{tikzcd}[column sep=small]
M \arrow[rr, "\varphi"] & & N
\end{tikzcd}
\]
Clearly $A_{1}$ is the disjoint union $\bigcup\limits^{\bigcdot}_{M,N \in A_{0}} \textup{Hom}_{A}(M,N) = A_{1}$. As usual we define 
$\textup{End}_{A}(M):= \textup{Hom}_{A}(M,M)$.
\end{definition}



\begin{definition}{(Category)}\\
A \ul{category} $\mathfrak{A}$ is a quiver with two further defining maps
\[
\begin{tikzcd}[column sep=small]
A_{0} \arrow[rr,"1"] &  & A_{1} &  & A_{1} \times_{s,A_{0},t} A_{1} \arrow[ll,"\mu"]
\end{tikzcd}
\]
\end{definition}

\begin{example}\label{representation}{(Representation of a concrete category)}\\
\begin{center}
\begin{tikzcd}[boxedcd={inner sep=1pt}]
                                                                                           &  &  &  & \\
                                                                                              &  &                                                                       \\
                                                                                              &  &                                                                       \\
                                                                                              &  &                                                                       \\
&  & 5 \arrow[rr, "{\begin{pmatrix} 
0\ampersand1\ampersand0\ampersand0\\
0\ampersand0\ampersand1\ampersand0\\
0\ampersand0\ampersand0\ampersand0\\
0\ampersand1\ampersand0\ampersand1\\
0\ampersand0\ampersand1\ampersand0
\end{pmatrix}}"]
\arrow["{\begin{pmatrix} 
1\ampersand 1\ampersand 0\ampersand 0\ampersand 0\\
0\ampersand 1\ampersand 1\ampersand 0\ampersand 0\\
0\ampersand 0\ampersand 1\ampersand 0\ampersand 0\\
0\ampersand 0\ampersand 0\ampersand 1\ampersand 1\\
0\ampersand 0\ampersand 0\ampersand 0\ampersand 1 
\end{pmatrix}}"', loop, distance=2em, in=305, out=235]             &  & 
4 \arrow["{\begin{pmatrix}
1\ampersand1\ampersand0\ampersand0\\
0\ampersand1\ampersand1\ampersand0\\
0\ampersand0\ampersand1\ampersand0\\
0\ampersand0\ampersand0\ampersand1
\end{pmatrix}}"', loop, distance=2em, in=305, out=235]  &  &         \\
                                                                                              &  &                                                                       \\
                                                                                              &  &                                                                       \\
                                                                                              &  &                                                                       \\   
                                                                                              &  &                                                                       \\   
\end{tikzcd}
\end{center}
\begin{center}
\begin{tikzcd}
                                                                                              & {} &                                                                       \\
                                                                                              & {} &                                                                       \\
                                                                                              & {} \arrow["nine",u, Rightarrow] &                                                                       \\
\end{tikzcd}
\end{center}
\begin{center}
\begin{tikzcd}[boxedcd={inner sep=1pt}]
                                                                                              &  &                                                                       \\
&  1 \arrow["a"', loop, distance=2em, in=305, out=235] \arrow[rr, "b"] \arrow[rr] \arrow[rr]     &  & 
2 \arrow["c"', loop, distance=2em, in=305, out=235]  &                   \\
                                                                                              &  &                                                                       \\
\end{tikzcd}
\end{center}
\begin{center}
\begin{tikzcd}[boxedcd={inner sep=1pt}]
                                                                                              &  &                                                                       \\
&  {\{1,2,3\}} \arrow["{(2,1,3)}"', loop, distance=2em, in=305, out=235] 
\arrow[rr, "{(4,5,6)}"] &  & 
{\{4,5,6\}} \arrow["{(5,6,4)}"', loop, distance=2em, in=305, out=235]  & \\
                                                                                              &  &                                                                       \\
\end{tikzcd}
\end{center}

\[
F(a) \eta_{1} = \eta_{1} G(a)\\
F(b) \eta_{2} = \eta_{1} G(b)
\]
\end{example}

\section{$\mathbb{K}$-linear Category (Algebroid)}

Group: Category with one object.

Groupoid: A small category in which every morphism is an isomorphism.

Algebroid

EmbeddingOfSumOfImages

What is an Algebroid? Bialgebroid?

\section{Additive Category}



\section{Abelian Category}

\section{The Category of Categories}

\section{The Categories of Functors}

\section{The Representation of a Category}

\section{Representation}

Grundidee von FunctorCategory

Standard-Monoidale Struktur von der Zielkategorie z.B. TensorUnit(C)

\section{Algorithms}
\lstinputlisting[numbers=left,firstnumber=60,firstline=60,lastline=142]{\pkgpath/catreps/gap/CatRepsWithCAP.gi}

\begin{algorithm}\capstart
    \caption{\texttt{RightQuiverFromConcreteCategory}}\label{algo:RightQuiverFromConcreteCategory}
	\SetKwInput{Input}{Input~}
	\SetKwInput{Output}{Output~}
	\Input{~a finite concrete category $C$ with $n$ objects}
	\Output{~the right quiver $q(n)$}
	\BlankLine
	let $Obj$ be the set of objects of $C$\;
	let $n := Length(Obj)$\;
	let $gMor$ be the set of generating morphisms of $C$\;
	let $A$ be the empty set and let $i := 1$\;
	\ForEach{morphism $mor$ in $gMor$}{
	    let $A_{i,1}$ be the position of $Source( mor )$ in $Obj$\;
	    let $A_{i,2}$ be the position of $Range( mor )$ in $Obj$\;
	    let $i := i+1$\;
	}
	\BlankLine
	let $q$ be the right quiver with vertices $\{1,\dots,n\}$ and arrows $A$.
	\BlankLine
	\Return q\;
\end{algorithm}

We want the endomorphism relations so that the path algebra is finite-dimensional and we
get a finite Gröbner basis.

\begin{algorithm}\capstart
    \caption{\texttt{RelationsOfEndomorphisms}}\label{algo:RelationsOfEndomorphisms}
	\SetKwInput{Input}{~Input}
	\SetKwInput{Output}{~Output}
	\Input{~a commutative ring $k$ and a finite concrete category $C$}
	\Output{~the endomorphism relations of the category $C$}
	\BlankLine
	let $q := \texttt{RightQuiverFromConcreteCategory}(C)$\;
	let $kq$ be the path algebra generated by $k$ and $q$\;
	let $gMor$ be the set of generating morphisms of $C$\;
	let $A := Arrows(q)$\;
	let $relsEndo$ be the empty set\;
	\ForEach{$i = 1, \dots, Length(gMor)$}{
	    let $mor := gMor_i$
	    \If{$mor$ is not an endomorphism}{
		continue\;
	    }
	    let $m := 0$ and let $powers$ be the empty set\;
	    let $foundEqual$ be false\;
	    \While{$mor^{m}\nin powers$}{
		let $n := 1$\;
		\While{$\neg foundEqual$}{
		    \If{$mor^{(m+n)} = mor^{m}$}{
		    	Add the relation $kq.(A_{i})^{(m+n)}-kq.(A_{i})^{m}$ to relsEndo\;
		    	foundEqual := true\;
		    }
		    n := n+1\;
		}
		Add $mor^{m}$ to powers\;
		m := m+1\;
	    }
	}
	\Return{relsEndo}\;
\end{algorithm}

Proof that algorithm is correct
Proof that it terminates.

Wir haben BasisOfExternalHom benutzt um Decompose in CAP umzusetzen um EmbeddingOfSubRepresentation umzusetzen um
WeakDirectSumDecomposition umzusetzen.

\begingroup
     \parindent 0pt
     \parskip 2ex
     \def\enotesize{\normalsize}
     \theendnotes
\endgroup 

\input{bib/sources.bib}

\end{document}