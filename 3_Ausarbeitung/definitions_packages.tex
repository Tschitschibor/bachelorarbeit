%\usepackage[top=37mm,bottom=37mm,left=27mm,right=27mm]{geometry}

%% From Gliederung preambel
% left=2.5cm,right=2.5cm,top=2.5cm,bottom=2.5cm

%\usepackage[ % % top 37, bottom 37
  %          top=27mm,bottom=27mm,left=27mm,right=27mm,%
    %        footskip=.6cm]{geometry}
            
\usepackage[ 
  a4paper,
  footskip=0.7cm,
  margin=2.7cm,
  top=1.1cm,
  bottom=1.4cm
]{geometry}            

\def\changemargin#1#2{\list{}{\rightmargin#2\leftmargin#1}\item[]}
\let\endchangemargin=\endlist

\usepackage[utf8]{inputenc}
\usepackage[T1]{fontenc}%
\usepackage{fancyvrb}
\usepackage{enumitem}

\usepackage{fixltx2e}%
\usepackage{graphicx}%

% size of table of contents
\usepackage{tocloft}
\renewcommand{\cftsecfont}{\small\bfseries}
\renewcommand{\cftsubsecfont}{\small\mdseries}
\renewcommand{\cftsubsubsecfont}{\small\mdseries}

%%% For captions and references
\usepackage{enotez} %[backref]
\usepackage{hyperref}
\usepackage{hypcap}
\newcommand{\Algoref}[1]{%
	\hyperref[algo:#1]{Algorithm~\ref*{algo:#1}}%
}
\newcommand{\algoref}[1]{%
	\hyperref[algo:#1]{Algorithm~\ref*{algo:#1}}%
}
\newcommand{\Funcref}[1]{%
	\hyperref[func:#1]{Function~\ref*{func:#1}}%
}
\newcommand{\funcref}[1]{%
	\hyperref[func:#1]{\texttt{#1}}%
}

%%% For footnotes at end of text
\usepackage{endnotes}

%%% hyperendnotes.sty
\makeatletter
\newif\ifenotelinks
\newcounter{Hendnote}
% Redefining portions of endnotes-package:
\let\savedhref\href
\let\savedurl\url
\def\endnotemark{%
\@ifnextchar[\@xendnotemark{%
\stepcounter{endnote}%
\protected@xdef\@theenmark{\theendnote}%
\protected@xdef\@theenvalue{\number\c@endnote}%
\@endnotemark
}%
}%
\def\@xendnotemark[#1]{%
\begingroup\c@endnote#1\relax
\unrestored@protected@xdef\@theenmark{\theendnote}%
\unrestored@protected@xdef\@theenvalue{\number\c@endnote}%
\endgroup
\@endnotemark
}%
\def\endnotetext{%
\@ifnextchar[\@xendnotenext{%
\protected@xdef\@theenmark{\theendnote}%
\protected@xdef\@theenvalue{\number\c@endnote}%
\@endnotetext
}%
}%
\def\@xendnotenext[#1]{%
\begingroup
\c@endnote=#1\relax
\unrestored@protected@xdef\@theenmark{\theendnote}%
\unrestored@protected@xdef\@theenvalue{\number\c@endnote}%
\endgroup
\@endnotetext
}%
\def\endnote{%
\@ifnextchar[\@xendnote{%
\stepcounter{endnote}%
\protected@xdef\@theenmark{\theendnote}%
\protected@xdef\@theenvalue{\number\c@endnote}%
\@endnotemark\@endnotetext
}%
}%
\def\@xendnote[#1]{%
\begingroup
\c@endnote=#1\relax
\unrestored@protected@xdef\@theenmark{\theendnote}%
\unrestored@protected@xdef\@theenvalue{\number\c@endnote}%
\show\@theenvalue
\endgroup
\@endnotemark\@endnotetext
}%
\def\@endnotemark{%
\leavevmode
\ifhmode
\edef\@x@sf{\the\spacefactor}\nobreak
\fi
\ifenotelinks
\expandafter\@firstofone
\else
\expandafter\@gobble
\fi
{%
\Hy@raisedlink{%
\hyper@@anchor{Hendnotepage.\@theenvalue}{\empty}%
}%
}%
\hyper@linkstart{link}{Hendnote.\@theenvalue}%
\makeenmark
\hyper@linkend
\ifhmode
\spacefactor\@x@sf
\fi
\relax
}%
\long\def\@endnotetext#1{%
\if@enotesopen
\else
\@openenotes
\fi
\immediate\write\@enotes{%
\@doanenote{\@theenmark}{\@theenvalue}%
}%
\begingroup
\def\next{#1}%
\newlinechar='40
\immediate\write\@enotes{\meaning\next}%
\endgroup
\immediate\write\@enotes{%
\@endanenote
}%
}%
\def\theendnotes{%
\immediate\closeout\@enotes
\global\@enotesopenfalse
\begingroup
\makeatletter
\edef\@tempa{`\string>}%
\ifnum\catcode\@tempa=12
\let\@ResetGT\relax
\else
\edef\@ResetGT{\noexpand\catcode\@tempa=\the\catcode\@tempa}%
\@makeother\>%
\fi
\def\@doanenote##1##2##3>{%
\def\@theenmark{##1}%
\def\@theenvalue{##2}%
\par
\smallskip %<-small vertical gap between endnotes
\begingroup
\def\href{\expandafter\savedhref}%
\def\url{\expandafter\savedurl}%
\@ResetGT
\edef\@currentlabel{\csname p@endnote\endcsname\@theenmark}%
\enoteformat
}%
\def\@endanenote{%
\par\endgroup
}%
% Redefine, how numbers are formatted in the endnotes-section:
\renewcommand*\@makeenmark{%
\hbox{\normalfont\@theenmark~}%
}%
% header of endnotes-section
\enoteheading
% font-size of endnotes
\enotesize
\input{\jobname.ent}%
\endgroup
}%
\def\enoteformat{%
\rightskip\z@
\leftskip1.8em
\parindent\z@
\leavevmode\llap{%
\setcounter{Hendnote}{\@theenvalue}%
\addtocounter{Hendnote}{-1}%
\refstepcounter{Hendnote}%
\ifenotelinks
\expandafter\@secondoftwo
\else
\expandafter\@firstoftwo
\fi
{\@firstofone}%
{\hyperlink{Hendnotepage.\@theenvalue}}%
{\makeenmark}%
}%
}%
% stop redefining portions of endnotes-package:
\makeatother
% Toggle switch in order to turn on/off back-links in the
% endnote-section:
\enotelinkstrue
%\enotelinksfalse
%\let\footnote{\endnote}
%% Heading of endnotes section
\renewcommand*{\notesname}{Annotations}

\makeatletter
\renewcommand*{\enoteheading}{%
   \section*{\notesname%
   \@mkboth{\MakeUppercase{\notesname}}{\MakeUppercase{\notesname}}}%
\mbox{}\par\vskip-\baselineskip}
\makeatother



%%% For switching languages in quotes
\usepackage[english]{babel}
%\usepackage[english, german]{babel} %% makes troubles

%%% For quotation
%% english guillemets have to be custom defined in /tex/latex/csquotes/csquotes.cfg
%\usepackage[english = guillemets, autostyle = true,autopunct,csdisplay = true]{csquotes}
\usepackage[autostyle = true,autopunct,csdisplay = true]{csquotes}

%%% For proper underline
\usepackage{soul}
%\setuldepth{gjpqy}
%\setuldepth\strut
\setuldepth{-1}

%%% Color
\usepackage{xcolor}
\usepackage{color}
\definecolor{FireBrick}{rgb}{0.5812,0.0074,0.0083}
\definecolor{RoyalBlue}{rgb}{0.0236,0.0894,0.6179}
\definecolor{RoyalGreen}{rgb}{0.0236,0.6179,0.0894}
\definecolor{RoyalRed}{rgb}{0.6179,0.0236,0.0894}
\definecolor{LightBlue}{rgb}{0.8544,0.9511,1.0000}
\definecolor{Black}{rgb}{0.0,0.0,0.0}

\definecolor{linkColor}{rgb}{0.0,0.0,0.554}
\definecolor{citeColor}{rgb}{0.0,0.0,0.554}
\definecolor{fileColor}{rgb}{0.0,0.0,0.554}
\definecolor{urlColor}{rgb}{0.0,0.0,0.554}
\definecolor{promptColor}{rgb}{0.0,0.0,0.589}
\definecolor{brkpromptColor}{rgb}{0.589,0.0,0.0}
\definecolor{gapinputColor}{rgb}{0.589,0.0,0.0}
\definecolor{gapoutputColor}{rgb}{0.0,0.0,0.0}

%%  for a long time these were red and blue by default,
%%  now black, but keep variables to overwrite
\definecolor{FuncColor}{rgb}{0.0,0.0,0.0}
%% strange name because of pdflatex bug:
\definecolor{Chapter }{rgb}{0.0,0.0,0.0}
\definecolor{DarkOlive}{rgb}{0.1047,0.2412,0.0064}

%% command for ColorPrompt style examples
\newcommand{\gapprompt}[1]{\color{promptColor}{\bfseries #1}}
\newcommand{\gapbrkprompt}[1]{\color{brkpromptColor}{\bfseries #1}}
\newcommand{\gapinput}[1]{\color{gapinputColor}{#1}}

%%% For source code listings
\usepackage{listings}[2013/08/05]
\input{pfad.tex}
%\lstloadlanguages{GAP}

%%% For algorithm styles
\usepackage{xspace}
\usepackage[linesnumbered,ruled]{algorithm2e}
\usepackage{algpseudocode}
\SetKw{Continue}{continue}
\SetKw{Break}{break}
\SetKw{Not}{not\xspace}
\SetKw{AndAlg}{and\xspace}

%%% Math theorem styles
\usepackage{amsthm}

\newtheorem{theorem}{Theorem}[subsection]
\theoremstyle{definition}
\newtheorem{lemma}[theorem]{Lemma}
\newtheorem{corollary}[theorem]{Corollary}
\newtheorem{definition}[theorem]{Definition}
\newtheorem{remark}[theorem]{Remark}
\newtheorem{proposition}[theorem]{Proposition}
\newtheorem{example}[theorem]{Example}
\newtheorem{doctrine}[theorem]{Doctrine}
\newtheorem{computation}[theorem]{Computation}

%%% make equations count from subsection
\usepackage{chngcntr}
\counterwithin{equation}{subsection}

%%% for nested proofs
\newenvironment{subproof}[1][\proofname]{%
  \renewcommand{\qedsymbol}{$\mathbin{/\mkern-6mu/}$}%
  \begin{proof}[#1]%
}{%
  \end{proof}%
}

%%% for nicer Product sign
\newcommand{\invamalg}{\mathbin{\rotatebox[origin=c]{180}{$\amalg$}}}

%%% For Math
\usepackage{amsmath}
\usepackage{amsfonts}
\usepackage{amsbsy}
\usepackage{amssymb}
\usepackage{mathtools}
\usepackage{esvect}
\usepackage{commath}
\usepackage[sc,osf]{mathpazo}

%%% Macros for our recurring categories
\newcommand{\kmat}{\Bbbk\textnormal{-}\mathbf{mat}}
\newcommand{\kAlgebroid}{\Bbbk\textnormal{-}\mathrm{algebroid}}
\newcommand{\Rmat}{R\textnormal{-}\mathbf{mat}}
\newcommand{\HomAkmat}{\mathrm{Hom_{\Bbbk}}(\mathcal{A},\kmat)}
\newcommand{\HomARmat}{\mathrm{Hom_{R}}(\mathcal{A},\Rmat)}
\newcommand{\HomA}{\mathrm{Hom}_{\mathcal{A}}}
\newcommand{\FinSets}{\mathrm{FinSets}}
\newcommand{\Cat}{\mathrm{\textbf{Cat}}}
\newcommand{\Set}{\mathrm{\textbf{Set}}}
\newcommand{\Quiv}{\mathrm{\textbf{Quiv}}}
\newcommand{\kChat}{\widehat{\Bbbk\mathcal{C}}}

%%% Macros for the software packages
\newcommand{\Gap}{\textsc{Gap}}
\newcommand{\QPA}{\textsc{QPA$2$}}
\newcommand{\CatReps}{\texttt{CatReps}}
\newcommand{\catreps}{\texttt{catreps}}
\newcommand{\CAP}{\textsc{CAP}}
\newcommand{\homalgProject}{\texttt{homalg\_project}}
\newcommand{\FunctorCategories}{\texttt{FunctorCategories}}

%%% vel means or in latin, easier to remember
\newcommand{\vel}{\vee}

%%% For arrows and categories
%\usepackage[all]{xy} %%not used anymore
\usepackage{tikz-cd}

%%% For calculations and loops inside tikz and latex
\usepackage{calc}
\usepackage{pgffor}

\newcounter{modresult}
\newcommand*{\themodulo}[2]{%
\setcounter{modresult}{%
#1-(#1/#2)*#2%
}%
#1 mod #2 = \themodresult\par
}

%%% For matrices
\let\ampersand =&

%%% Math operators bold
%\newcommand{\Category}{Category}
% just use /textup{#1} inside math environment instead of redefining every math operator.

%%% tikz
\usetikzlibrary{positioning}
\usetikzlibrary{arrows}

%%% for captions of tikzpictures and other figures
\usepackage{capt-of}

%%% For function restrictions
\newcommand\restrict[1]{\raisebox{-.5ex}{$|$}_{#1}}

%%% For dotted box around diagrams
\tikzcdset{
    boxedcd/.style={
        every matrix/.append style={
            draw=black,
            dotted,
            rounded corners,
            #1
        },
    },
}

%%% For dotted arrows in math and in text
%% dottedrightarrow
\makeatletter
\newbox\dottedrightarrow@box
\setbox\dottedrightarrow@box\hbox
  {%
    \begin{tikzpicture}
      \draw[dotted,->] (0,0) -- (1.5em,0);
    \end{tikzpicture}%
  }
\newcommand*\dottedrightarrow
  {\relax\ifmmode\expandafter\dottedrightarrow@m\else\expandafter\dottedrightarrow@t\fi}
\newcommand*\dottedrightarrow@t[1][1.5em]
  {\resizebox{#1}{!}{\raisebox{.5ex}{\usebox\dottedrightarrow@box}}}
\newcommand*\dottedrightarrow@m[1][]
  {%
    \if\relax\detokenize{#1}\relax
      \mathchoice% values are trial and error based\ldots
        {\dottedrightarrow@t}
        {\dottedrightarrow@t}
        {\dottedrightarrow@t[1.1em]}
        {\dottedrightarrow@t[0.9em]}%
    \else
      \dottedrightarrow@t[#1]%
    \fi
  }
\makeatother
\let\olddottedrightarrow\dottedrightarrow
\renewcommand{\dottedrightarrow}{\raisebox{-.2em}{\,\,\olddottedrightarrow\,\,}}
%% dottedleftarrow
\makeatletter
\newbox\dottedleftarrow@box
\setbox\dottedleftarrow@box\hbox
  {%
    \begin{tikzpicture}
      \draw[dotted,<-] (0,0) -- (1.5em,0);
    \end{tikzpicture}%
  }
\newcommand*\dottedleftarrow
  {\relax\ifmmode\expandafter\dottedleftarrow@m\else\expandafter\dottedleftarrow@t\fi}
\newcommand*\dottedleftarrow@t[1][1.5em]
  {\resizebox{#1}{!}{\raisebox{.5ex}{\usebox\dottedleftarrow@box}}}
\newcommand*\dottedleftarrow@m[1][]
  {%
    \if\relax\detokenize{#1}\relax
      \mathchoice% values are trial and error based\ldots
        {\dottedleftarrow@t}
        {\dottedleftarrow@t}
        {\dottedleftarrow@t[1.1em]}
        {\dottedleftarrow@t[0.9em]}%
    \else
      \dottedleftarrow@t[#1]%
    \fi
  }
\makeatother
\let\olddottedleftarrow\dottedleftarrow
\renewcommand{\dottedleftarrow}{\raisebox{-.2em}{\,\,\olddottedleftarrow\,\,}}

%%% For some big dots (they still don't look very big)
\makeatletter
\newcommand*{\bigcdot}{}% Check if undefined
\DeclareRobustCommand*{\bigcdot}{%
  \mathbin{\mathpalette\bigcdot@{}}%
}
\newcommand*{\bigcdot@scalefactor}{.5}
\newcommand*{\bigcdot@widthfactor}{1.15}
\newcommand*{\bigcdot@}[2]{%
  % #1: math style
  % #2: unused
  \sbox0{$#1\vcenter{}$}% math axis
  \sbox2{$#1\cdot\m@th$}%
  \hbox to \bigcdot@widthfactor\wd2{%
    \hfil
    \raise\ht0\hbox{%
      \scalebox{\bigcdot@scalefactor}{%
        \lower\ht0\hbox{$#1\bullet\m@th$}%
      }%
    }%
    \hfil
  }%
}
\makeatother