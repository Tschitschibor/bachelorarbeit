% mainfile: ../main.tex

\section{Einleitung}

Wie viele Vokablen braucht man, um Mathematik zu betreiben?

\[ \mathbb{K}^{\{1,2,3\}} \leq \mathbb{K}^{\{1,2,3,4\}} \]
\[ \{1,2,3\} \subset \{1,2,3,4\} \]
\[ 3 < 4 \]

Diese drei Sachverhalte drücken eigentlich alle dasselbe aus.



Denken in Kategorien, aber kleinteilig auf Elementebene programmieren? Wenn wir die Kontrolle über unsere Datenstrukturen nicht abgeben wollen,
sondern versuchen, im Quellcode selbst zu optimieren geht das unweigerlich aufkosten der Lesbarkeit des Codes. Und wenn die kleinen Performance-Gewinne gegenüber dem Platzhirsch in ein paar Jahren dadurch aufgehoben werden, weil sich andere Experten allein um Optimierung gekümmert haben, stellt man fest,
man hat aufs falsche Pferd gesetzt.\\
Denken in Kategorien und programmieren in Kategorien. Man geht ein, zwei, beliebig viele Abstraktionsstufen höher, und überlässt die eigentliche Rechnung
dafür optimierten Programmen. Wenn ich einen Kern einer Matrix berechne, die Matrix selbst das Ergebnis komplizierter kategorientheoretischer Konstruktionen
ist, dann ist es wichtiger zu wissen, was die Matrix ist, wie sie entstanden ist, als das Ergebnis der Berechnungen elementweise zu kennen.

Mit dem $\mathtt{CAP}$-Project haben [Posur et al.] gezeigt, dass es möglich ist, kategoriell zu denken und zu programmieren, und dabei die eigentliche
Rechenarbeit an optimierte Computeralgebraprogramme wie Singular oder Magma auszugliedern. Man muss bei der Rechnung eben nicht von Grund auf
anfangen. Das bringt uns zurück zur Frage, wie viele Vokablen braucht man, um Mathematik zu betreiben?

Die Mengentheorie kommt aus mit dem Element-Symbol $\in$ und den Mengenklammern $\{$ und $\}$, sowie mit den
Prädikaten der Logik. Nimmt man dann noch ein paar Axiome hinzu, kommt man zu Axiomensystemen wie Zermelo-Fraenkel (ZF) oder ZFC.