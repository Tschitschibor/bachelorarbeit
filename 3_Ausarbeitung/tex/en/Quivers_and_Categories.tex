% mainfile: ../main.tex

This section serves two purposes: On the one hand, it is an introduction to quivers and category theory. On the other hand it introduces
concrete categories which we want to represent, and all the additional constructions that are needed to that goal.

\subsection{Quivers}
In this section, we first want to define the category \textbf{Quiv} and how it is the prototype for the category \textbf{Cats}.
In order to describe the category \textbf{Quiv} of quivers, we first have to define what a category is and for this we need
the definition of a quiver. Lateron we will revisit this definition as we can define quivers as the objects in the quiver category \textbf{Quiv}.

\begin{definition}{(Quiver)}\label{def:quiver}\\
A \ul{directed graph} or \ul{quiver} $q$ consists of a class of \ul{objects} (or \ul{vertices}) $q_{0} = \textup{Obj}\,q$ and
a class of \ul{morphisms} (or \ul{arrows}) $q_{1} = \textup{Mor}\,q$ together with two defining maps
\[
\begin{tikzcd}[column sep=small]
{s,t\colon q_{1}} \arrow[rr, shift left = 0.7ex] \arrow[rr, shift right = 0.7ex] & & q_{0}
\end{tikzcd}
\]
$s$ called \ul{source} and $t$ called \ul{target}.
\end{definition}

In the next definition we are giving a new characterization for $q_{1}$ by looking at all arrows between two fixed objects.

\begin{definition}{(Hom-set of a (locally) small quiver)}\label{def:hom-set}
\renewcommand{\labelenumi}{(\theenumi)}
\begin{enumerate}
\item Given two objects $M, N \in q_{0}$ we write $\textup{Hom}_{q}(M,N)$ or $q(M,N)$ for the fiber
$(s,t)^{-1} (\{(M,N)\})$ of the product map 
\begin{tikzcd}[column sep=small]
(s, t) : q_{1} \arrow[rr] &  & q_{0} \times q_{0} 
\end{tikzcd} over the pair $(M,N) \in q_{0} \times q_{0}$.
This is the class of all morphisms with source $= M$ and target $= N$.
We indicate this by writing
\begin{tikzcd}[column sep=small]
\varphi : M \arrow[rr] &  & N
\end{tikzcd} or 
\begin{tikzcd}[column sep=small]
M \arrow[rr,"\varphi"] &  & N.
\end{tikzcd} Hence $q_{1}$ is the disjoint union $\bigcup\limits^{\bigcdot}_{M,N \in q_{0}} \textup{Hom}_{q}(M,N) = q_{1}$.
As usual we define $\textup{End}_{q}(M):= \textup{Hom}_{q}(M,M)$.
\item If the class $\textup{Hom}_{q}(M,N)$ is a \ul{set} for all pairs $(M,N)$ then we call the quiver \ul{locally small}.
We therefore talk about \ul{Hom-sets}.
If additionally, $q_{0}$ is a set, then the quiver is called \ul{small}.
\item A quiver with a finite set of objects and a finite set of morphisms is called a \ul{finite} quiver.
\end{enumerate}
\end{definition}

When we don't assume the category to be locally small, but still talk about its hom-sets, we mean the class of morphisms,
if we don't explicitly use the fact that it's a set of morphisms.

\begin{example}\label{q(2)}{(Quiver with 2 objects and 3 morphisms)}\\
\[
\begin{tikzcd}
1 \arrow["a"', loop, distance=2em, in=305, out=235] \arrow[rr, "b"] &  & 2 \arrow["c"', loop, distance=2em, in=305, out=235]
\end{tikzcd}
\]
The objects of this quiver $q$ are $q_{0} = \{1, 2\}$, and the morphisms are $q_{1} = \{a, b, c\}$ with\\
$s (a) = 1 = t (a)$, $s (c) = 2 = t (c)$ and $s (b) = 1, t (b) = 2$.\\
\noindent Thus $\textup{End}_{q}(1) = \{a\}, \textup{End}_{q}(2) = \{c\}$ and $\textup{Hom}_{q}(1,2) = \{b\}$ whereas
$\textup{Hom}_{q}(2,1)=\emptyset$.\\

\noindent In \texttt{QPA} this quiver is encoded as \texttt{q(2)[a:1->1,b:1->2,c:2->2]} where the first \texttt{(2)} in parentheses stands for the total
number of objects.
\end{example}

\begin{definition}{(Composable arrows; path in a quiver)}\label{def:path}\\
Since we already have the source and target maps, we say two arrows $a, b \in q_{1}$ are \ul{composable} if $t(a) = s(b)$ or
$t(b) = s(a)$. In this case we can write a sequence of composable arrows $p = a_{1}a_{2}\cdots a_{n}$ where $t(a_{i}) = s(a_{i+1})$ for $i=1,\dots,n-1$.
We call this sequence a \ul{path} from $s(a_{1})$ to $t(a_{n})$ and the integer $n \in \mathbb{Z}_{\geq0}$ the \ul{length} $l(p)$ of the path $p$.
Although it's not an arrow, we can define the source and target of a path $p = a_{1}\cdots a_{n}$ as $s(p) := s(a_{1})$ and $t(p) := t(a_{n})$.
A path $p = a_{1}\cdots a_{n}$ with $s(a_{1}) = t(a_{n})$, i.e. $s(p) = t(p)$, is called \ul{cyclic}.\\
For an endomorphism $a \in \textup{End}_{q}(M)$ we write $a^{n}$ for $aa \cdots a$ ($n$ times). In the case of $n=0$ an \ul{empty path}
whose source and target are the vertex $i \in q_{0}$ is called the \ul{trivial path at $i$} and is denoted $e_{i}$. Note that the composition of paths
$e_{i}e_{i}$ has length zero starting at $i$ therefore $e_{i}^{2}=e_{i}$.
\end{definition}

\begin{lemma} Let Q be a quiver. If there is a path of length at least $\abs{Q_{0}}$, then there are cyclic paths,
and thus infinitely many paths.\cite{[leit4]}
\begin{proof}
Assume that there exists a path of length greater or equal to $\abs{Q_{0}}$. Then there exists a path of length $n = \abs{Q_{0}}$, say
$\alpha_{1}\cdots \alpha_{n}$. Consider the vertices $x_{i}=s(\alpha_{i})$ for $1 \leq i \leq n$ and $x_{n+1}=t(\alpha_{n})$. Then these
are $n+1$ vertices, thus there has to exist $i<j$ with $x_{i}=x_{j}$. Let $\omega=\alpha_{i}\cdots \alpha_{j-1}$, this is a path with source and target
$x_{i}=x_{j}$, thus a cyclic path. But then $\omega^{m}$ is a path for any natural number $m$. The path $\omega$ has length $j-i\geq1$, thus
$\omega^{m}$ has length $m(j-i)$. This shows that these paths are pairwise different.
\end{proof}
\end{lemma}

\begin{example}{(A quiver with no cycles)}\\
\[
\begin{tikzcd}
2 \arrow[rrrr, "\psi"] \arrow[rrrrddd, "\psi\rho", pos=0.3] &  &  &  &
3 \arrow[ddd, "\rho"] \\
 &  &  &  & \\
 &  &  &  & \\
1 \arrow[uuu, "\varphi"] \arrow[rrrruuu, "\varphi\psi", pos=0.3] \arrow[rrrr, "\varphi\psi\rho" '] &  &  &  & 4
\end{tikzcd}
\]
The longest path $1\rightarrow2\rightarrow3\rightarrow4$ has length 3. If after the object $4$ another arrow would go to either $1,2,3$ or $4$ itself,
we would have a cyclic path and thus infinitely many paths.
\end{example}

\subsection{Categories}

\begin{definition}{(Category)}\label{def:category}\\
\noindent A \ul{category} $\mathcal{C}$ is a quiver with two further maps:
\begin{enumerate}
\renewcommand{\labelenumi}{(id)}
\item The \ul{identity map} $1_{( )}$ mapping every object $X \in\mathcal{C}_{0}$ to its \ul{identity morphism} $1_{X}$:
\[
\begin{tikzcd}[column sep=small]
\mathcal{C}_{0} \arrow[rr,"1"] &  & \mathcal{C}_{1}
\end{tikzcd}
\]
\renewcommand{\labelenumi}{($\mu$)}
\item And for any two \ul{composable} morphisms $\varphi$ and $\psi \in \mathcal{C}_{1}$, i.e. with $t(\varphi) = s(\psi)$, the
\ul{composition map} $\mu$, which maps $\varphi, \psi \in \mathcal{C}_{1}\times\mathcal{C}_{1}$ to $\mu(\varphi,\psi) \in \mathcal{C}_{1}$ which
we also write as $\varphi\psi$. 
\[
\begin{tikzcd}[column sep=small]
\mathcal{C}_{1} \times \mathcal{C}_{1} \arrow[rr,"\mu"] &  & \mathcal{C}_{1}
\end{tikzcd}
\]
\end{enumerate}
\noindent The defining properties for $1$ and $\mu$ are:
\renewcommand{\labelenumi}{(\theenumi)}
\begin{enumerate}
\item $s(1_{M}) = M = t(1_{M})$, i.e.\\
$1_{M} \in \textup{End}_{\mathcal{C}} \forall M \in \mathcal{C}$.

\item $s(\varphi\psi) = s(\varphi)$ and\\
$t(\varphi\psi) = t(\psi)$\\
for all composable morphisms $\varphi, \psi \in \mathcal{C}$.
\[
\begin{tikzcd}[column sep=small]
\mu : \textup{Hom}_{\mathcal{C}}(M,L) \times \textup{Hom}_{\mathcal{C}}(L,N) \arrow[rr] &  & \textup{Hom}_{\mathcal{C}}(M,N)
\end{tikzcd}
\]
\item \label{associativity_of_composition} \begin{minipage}{.55\textwidth} $(\varphi\psi)\rho = \varphi(\psi\rho)$ \hfill{} [associativity of composition]\end{minipage}
\begin{minipage}{.45\textwidth}\phantom{}\end{minipage}
\item \label{unit_property} \begin{minipage}{.55\textwidth} $1_{s(\varphi)}\varphi = \varphi = \varphi1_{t(\varphi)}$ \hfill{} [unit property]\end{minipage}
\begin{minipage}{.45\textwidth}\phantom{}\end{minipage}\\
The identity is a left and right unit of the composition.
\end{enumerate}
\end{definition}

So with categories you always answer the four questions
\begin{itemize}
\item What are the objects? (which includes the question What are the identity morphisms?)
\item What are the morphisms?
\item How do you compose morphisms?
\item Why is the composition associative?
\end{itemize}

\subsection{Functors}

Categories are themselves objects in the category of categories, which leads to a question: What is a morphism between categories?

\begin{definition}{(Functor)}\label{def:functor}\\
\noindent A \ul{functor} $F : \mathcal{C} \rightarrow \mathcal{D}$, between categories $\mathcal{C}$ and $\mathcal{D}$, consists of the
following data:

\begin{itemize}
\item An object $Fc\in\mathcal{D}_{0}$, for each object $c \in \mathcal{C}_{0}$.
\item A morphism $Ff : Fc \rightarrow Fc' \in \mathcal{D}_{1}$, for each morphism $f : c \rightarrow c' \in \mathcal{C}_{1}$, so that the
source and target of $Ff$ are, respectively, equal to $F$ applied to the source or target of $f$, in other words,
$s(Ff) = Fs(f)$ and $t(Ff) = Ft(f)$.
\end{itemize}

\noindent The assignments are required to satisfy the following two \ul{functoriality axioms}:
\begin{itemize}\label{functoriality}
\item For any composable pair $f, g \in \mathcal{C}_{1}, Fg \cdot Ff = F(g \cdot f)$.
\item For each object $c \in \mathcal{C}_{0}, F(1_{c}) = 1_{Fc}$.
\end{itemize}

Put concisely, a functor consists of a mapping on objects and a mapping on morphisms that preserves all of the structure of a category,
namely domains and codomains, composition, and identities.
\end{definition}

So with functors you always answer the four questions
\begin{itemize}
\item How does it work on objects?
\item How does it work on morphisms?
\item Why does it respect composition?
\item Why does it respect identity morphisms?
\end{itemize}

We have already seen an example for a functor in definition \ref{def:hom-set} where we defined the hom-set $\textup{Hom}(M,N)$ between two
objects $M$ and $N$. There are two ways to leave blank one of the objects and thus define the 

\begin{example}{(partial Hom-functor)}\label{def:Hom_functor}
Let $\mathcal{C}$ be a category and $P \in \mathcal{C}_{0}$ any object. The \ul{Hom-functor}, also called \ul{partial Hom-functor},
\begin{enumerate}
\item $\textup{Hom}(P,-)$ is a functor from $\mathcal{C}$ to $\mathcal{C}_{1}$ where objects in $\mathcal{C}_{1}$ are the hom-sets 
$\textup{Hom}(P,N)$, and morphisms are maps from one hom-set to another.
$\textup{Hom}(P,-)$ works on objects by mapping the object $N \in \mathcal{C}_{0}$ to
the hom-set $\textup{Hom}(P,N) \in \mathcal{C}_{1}$.
$\textup{Hom}(P,-)$ works on morphisms by mapping the morphism $(f : M \rightarrow N ) \in \mathcal{C}_{1}$ to the transformation
$\textup{Hom}(P,f) : \textup{Hom}(P,M) \rightarrow \textup{Hom}(P,N); \varphi \mapsto \varphi f$, so for every morphism
$\varphi \in \textup{Hom}(P,M)$, you post-compose $f \in \textup{Hom}(M,N)$ to get a new morphism $\varphi f \in \textup{Hom}(P,N)$.

\item $\textup{Hom}(-,P)$ is a functor from $\mathcal{C}$ to $\mathcal{C}_{1}$ where objects in $\mathcal{C}_{1}$ are the hom-sets 
$\textup{Hom}(N,P)$, and morphisms are maps from one hom-set to another.
$\textup{Hom}(-,P)$ works on objects by mapping the object $N \in \mathcal{C}_{0}$ to
the hom-set $\textup{Hom}(N,P) \in \mathcal{C}_{1}$.
$\textup{Hom}(-,P)$ works on morphisms by mapping the morphism $(f : M \rightarrow N ) \in \mathcal{C}_{1}$ to the transformation
$\textup{Hom}(f,P) : \textup{Hom}(N,P) \rightarrow \textup{Hom}(M,P); \varphi \mapsto f\varphi$, so for every morphism
$\varphi \in \textup{Hom}(N,P)$, you pre-compose $f \in \textup{Hom}(M,N)$ to get a new morphism $f\varphi \in \textup{Hom}(M,P)$.
\end{enumerate}

The important difference between these two functors was how they worked on morphisms. If in both cases we take a morphism
$f : M \rightarrow N$ as given, then we have to arrange the source and target for $\textup{Hom}(P,f)$ and $\textup{Hom}(f,P)$
according to the post-composition and pre-composition. Thus if we wanted $\textup{Hom}(f,P)$ to be defined by pre-composition
$\varphi \mapsto f\varphi$, then we were forced to invert $M$ and $N$ as source and target to get 
$\textup{Hom}(f,P): \textup{Hom}(N,P) \rightarrow \textup{Hom}(M,P)$. 
This process of inverting source and target is caught in the following definition.
\end{example}

\begin{definition}{(covariant / contravariant functor)}\endnote{(Def 1.3.5. in \cite{[context]}, p. 17 (35/258))}\\
The way we defined a functor in definition \ref{def:functor} was in the \ul{covariant} way.\\
A \ul{contravariant} functor $F : \mathcal{C} \rightarrow \mathcal{D}$ works on objects the same way as a covariant one, i.e.
an object $Fc \in \mathcal{D}_{0}$ for each object $c \in \mathcal{C}_{0}$. For morphisms on the other hand, we have
a morphism $F f : Fc' \rightarrow Fc \in \mathcal{D}_{1}$ for each morphism $f : c \rightarrow c' \in \mathcal{C}_{1}$, so that
$s(F f) = F t(f)$ and $t(F f) = F s(f)$.
The \ul{functoriality axioms} are also inverted for a contravariant functor:
For any composable pair, $f, g \in \mathcal{C}_{1}$, $F f \cdot F g = F(g \cdot f)$.
For the identity morphisms, it is again the same as in the covariant case:
For each object $c \in \mathcal{C}_{0}$, $F(1_{c}) = 1_{Fc}$.
\end{definition}

In the following definition, we define different subclasses of functors. These adjectives often come in opposite pairs, so that you may be
tempted to think, duality lets you just swap all the adjectives for the opposite ones, but be careful there. E.g. when 
$\textup{Hom}(P,-)$ is a \ul{covariant}, \ul{left-exact} functor, the opposite $\textup{Hom}(-,P)$ is a \ul{contravariant}, but still \ul{left-exact} functor.
But their respective \ul{right-exactedness} is equivalent to opposite concepts concerning \ul{projective} and \ul{injective} objects.

\begin{definition}{(Exact functor)}\label{def:exact_functor}\endnote{(Def 4.5.9. in \cite{[context]}, p. 139 (157/258))}
A functor 
\end{definition}
left-/ right- exact

full

faithful functor

\subsection{Natural transformations}

With fixed categories $\mathcal{C}$ and $\mathcal{D}$ we can consider functors $F, G \in \textup{Hom}(\mathcal{C},\mathcal{D})$ themselves
as objects in the category $\textup{Hom}(\mathcal{C},\mathcal{D})$ of functors between $\mathcal{C}$ and $\mathcal{D}$. In this \ul{functor category},
the morphisms between two functors are called \ul{natural transformations}.

\begin{definition}{(Natural transformations)}\label{def:natural_transformation}\\
\noindent Given categories $\mathcal{C}$ and $\mathcal{D}$ and functors $F : \mathcal{C} \rightarrow \mathcal{D}$ and
$G : \mathcal{C} \rightarrow \mathcal{D}$, a \ul{natural transformation} $\alpha : F \Rightarrow G$ consists of:
\begin{itemize}
\item an arrow $\alpha_{c} : Fc \rightarrow Gc \in \mathcal{D}_{1}$ for each object $c \in \mathcal{C}_{0}$, the collection of which
define the \ul{components} of the natural transformation, so that, for any morphism $f : c \rightarrow c' \in \mathcal{C}_{1}$, the following
square of morphisms in $\mathcal{D}$
\[\begin{tikzcd}
Fc \arrow[rr, "\alpha_{c}"] \arrow[dd, "Ff"] &  & Gc \arrow[dd, "Gf"] \\
                                             &  &                     \\
Fc' \arrow[rr, "\alpha_{c'}"]                &  & Gc'                
\end{tikzcd}\]

\ul{commutes}, i.e., has a a common composite $Fc \rightarrow Gc' \in \mathcal{D}_{1}$.
\end{itemize}

\end{definition}






















