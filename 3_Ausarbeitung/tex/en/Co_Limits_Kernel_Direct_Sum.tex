% mainfile: ../main.tex

\subsection{Limit and colimit of a functor}

\begin{definition}{(Source of a functor)}
Let $D : \mathbf{I} \rightarrow \mathcal{C}$ be a functor. A \ul{source} of $D$ consists of the following data:
\begin{enumerate}
\renewcommand{\labelenumi}{(\theenumi)}
\item An object $S \in \mathcal{C}$.
\item A dependent function $s$ mapping an object $i \in \mathbf{I}$ to a morphism
$s(i) : S \rightarrow D(i)$ such that for all $i, j \in \mathbf{I}, \iota : i \rightarrow j$, we have $D(\iota) \cdot s(i) = s(j)$.
\end{enumerate}
\end{definition}

\begin{definition}{(Limit and colimit of a functor)}
Let $D : \mathbf{I} \rightarrow \mathcal{C}$ be a functor. A \ul{limit} of $D$ consists of the
following data:
\begin{enumerate}
\renewcommand{\labelenumi}{(\theenumi)}
\item A source of $D$ given by the data $(\mathrm{lim}\, D, (\lambda(i) : \mathrm{lim}\, D \rightarrow D(i))_{i\in\mathbf{I}})$.
\item A dependent function $u$ mapping every source $\tau = (T, (\tau(i) : T \rightarrow D(i))_{i \in \mathbf{I}})$ to a
morphism $u(\tau) : T \rightarrow \mathrm{lim}\, D$ such that $\lambda(i) \cdot u(\tau) = \tau(i)$ for all $i \in \mathbf{I}$.\label{itm:2}
\item For any other dependent function $v$ satisfying (\ref{itm:2}), we have $u = v$.
\end{enumerate}
A \ul{colimit} of $D$ is a limit of $D' : \mathbf{I} \rightarrow \mathcal{C}^{\mathrm{op}}$.
\end{definition}

\begin{definition}{(Limits of type \textbf{I})}
Let $\mathbf{I}$ be a category. We say a category $\mathcal{C}$ \ul{has limits of type} $\mathbf{I}$ if it is
equipped with a dependent function $\lambda$ mapping a functor $D : \mathbf{I} \rightarrow \mathcal{C}$ to a limit
$(\mathrm{lim}\, D, \lambda_{D}, u_{D})$ of $D$.
We say $\mathcal{C}$ \ul{has colimits of type} $\mathbf{I}$ if $\mathcal{C}^{\mathrm{op}}$ has limits of that type.
\end{definition}

\begin{definition}{(Initial object, terminal object, zero object)}\label{def:init_term_zero_object}
\renewcommand{\labelenumi}{(\theenumi)}
\begin{enumerate}
\item A \ul{terminal object} $T$ in a category $\mathcal{C}$ is an object such that $\textup{Hom}_{\mathcal{C}}(-,T)$ is a singleton.
\item An \ul{initial object} $I$ in a category $\mathcal{C}$ is an object such that $\textup{Hom}_{\mathcal{C}}(I,-)$ is a singleton.
\item An object is a \ul{zero object} if it is both initial and terminal.
\end{enumerate}
\end{definition}

\begin{example}
Depending on \textbf{I} some limits and colimits have special names:\\
\begin{center}
\begin{tabular}{c|c|c}
\textbf{I} & limit & colimit \\
\hline
$\emptyset$ & terminal object & initial object \\
a set & direct product & coproduct \\
$\cdot \rightarrow \cdot \leftarrow \cdot$ & pullback & - \\
$\cdot \leftarrow \cdot \rightarrow \cdot$  & - & pushout \\
$ \cdot \rightrightarrows \cdot$ & equalizer & coequalizer
\end{tabular}
\end{center}
\end{example}

\begin{definition}{(Zero morphism)}\label{def:zero_morphism}\\
A \ul{zero morphism} in a category with a zero object $Z$ is a morphism factoring over $Z$, i.e. $\varphi : M \rightarrow N$ is called a zero
morphism, if\\
\begin{minipage}{.35\textwidth}
\begin{tikzcd}
M \arrow[rr, "\varphi"] \arrow[rd, "\varphi_{1}"] &                              & N \\
                                                  & Z \arrow[ru, "\varphi_{2}"'] &  
\end{tikzcd}
\end{minipage}
\begin{minipage}{.65\textwidth}
$\exists \varphi_{1} : M \rightarrow Z, \varphi_{2} : Z \rightarrow N$\\
such that $\varphi = \varphi_{1}\varphi_{2}$.
\end{minipage}
\end{definition}


\subsection{Monomorphisms and epimorphisms}

\subsection{Kernel and cokernel; image and coimage}

\subsection{Direct sum and direct product}