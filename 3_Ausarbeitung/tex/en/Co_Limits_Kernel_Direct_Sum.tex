% mainfile: ../main.tex

\subsection{Limit and colimit of a functor}

\begin{definition}{(Source of a functor)}
Let $D : \mathbf{I} \rightarrow \mathcal{C}$ be a functor. A \ul{source} of $D$ consists of the following data:
\begin{enumerate}
\renewcommand{\labelenumi}{(\theenumi)}
\item An object $S \in \mathcal{C}$.
\item A dependent function $s$ mapping an object $i \in \mathbf{I}_{0}$ to a morphism
$s(i) : S \rightarrow D(i)$ such that for all $i, j \in \mathbf{I}, \iota : i \rightarrow j$, we have $D(\iota) \cdot s(i) = s(j)$.
\end{enumerate}
\end{definition}

\begin{definition}{(Limit and colimit of a functor)}
Let $D : \mathbf{I} \rightarrow \mathcal{C}$ be a functor. A \ul{limit} of $D$ consists of the
following data:
\begin{enumerate}
\renewcommand{\labelenumi}{(\theenumi)}
\item A source of $D$ given by the data $(\mathrm{lim}\, D, (\lambda(i) : \mathrm{lim}\, D \rightarrow D(i))_{i\in\mathbf{I}_{0}})$.
\item A dependent function $u$ mapping every source $\tau = (T, (\tau(i) : T \rightarrow D(i))_{i \in \mathbf{I}})$ to a
morphism $u(\tau) : T \rightarrow \mathrm{lim}\, D$ such that $\lambda(i) \cdot u(\tau) = \tau(i)$ for all $i \in \mathbf{I}$.\label{itm:2}
\item For any other dependent function $v$ satisfying (\ref{itm:2}), we have $u = v$.
\end{enumerate}
A \ul{colimit} of $D$ is a limit of $D' : \mathbf{I} \rightarrow \mathcal{C}^{\mathrm{op}}$.
\end{definition}

\begin{definition}{(Limits of type \textbf{I})}
Let $\mathbf{I}$ be a category. We say a category $\mathcal{C}$ \ul{has limits of type} $\mathbf{I}$ if it is
equipped with a dependent function $\lambda$ mapping a functor $D : \mathbf{I} \rightarrow \mathcal{C}$ to a limit
$(\mathrm{lim}\, D, \lambda_{D}, u_{D})$ of $D$.
We say $\mathcal{C}$ \ul{has colimits of type} $\mathbf{I}$ if $\mathcal{C}^{\mathrm{op}}$ has limits of that type.
\end{definition}

\begin{example}\label{ex:limits}
Depending on \textbf{I} some limits and colimits have special names:
\begin{center}
\begin{tabular}{c|c|c}
generating quiver of $\mathbf{I}$ & limit & colimit \\
\hline
$\emptyset$ & terminal object & initial object \\
$\cdot \text{\phantom{$\rightarrow$}} \cdot$ & binary product & binary coproduct \\
$\cdot \rightarrow \cdot \leftarrow \cdot$ & binary pullback & - \\
$\cdot \leftarrow \cdot \rightarrow \cdot$  & - & binary pushout \\
$ \cdot \rightrightarrows \cdot$ & binary equalizer & binary coequalizer
\end{tabular}
\end{center}
\end{example}

In the following section, we give explicit definitions for the limits in \ref{ex:limits} and examples for categories with such limits.

\subsection{Examples for limits and colimits}

\begin{definition}{(Product, coproduct)}\label{def:prod_coprod}
Let $I$ be an index set and $\{A_{i}\}_{i\in I}$ a family of objects in a category $\mathcal{C}$.
\begin{enumerate}
\renewcommand{\labelenumi}{(prod)}
\item The \ul{product} of the family $\{A_{i}\}_{i\in I}$ is an object $\invamalg A_{i}$ together with a family of morphisms
\[
\{ \pi_{i} : \invamalg A_{i} \rightarrow A_{i} \}
\]
called \ul{projections}, such that the following universal property is satisfied:\\
For any object $M \in \mathcal{C}_{0}$ and any family $\{ \varphi_{i} : M \rightarrow A_{i} \}_{i\in I}$ of morphisms, there exists
a unique morphism $\varphi : M \rightarrow \invamalg A_{i}$ called the \ul{product morphism} such that
\[
\varphi \pi_{i} = \varphi_{i} \, \forall i \in I.
\]
\begin{tikzcd}
                                                                                                                            &  &                                                          & A_{1} \\
M \arrow[rrru, "\varphi_{1}", bend left] \arrow[rrrd, "\varphi_{2}"', bend right] \arrow[rr, "\exists^{1} \varphi", dashed] &  & A_{1}\invamalg A_{2} \arrow[ru, "\pi_{1}"] \arrow[rd, "\pi_2"] &       \\
                                                                                                                            &  &                                                          & A_{2}
\end{tikzcd}
\renewcommand{\labelenumi}{(coprod)}
\item The dual notion to product is the \ul{coproduct} of the family $\{A_{i}\}_{i\in I}$, that is an object $\amalg A_{i}$ together with
a family of morphisms
\[
\{ \iota_{i} : A_{i} \rightarrow \amalg A_{i} \}
\]
called \ul{coprojections} or sometimes \ul{injections} or \ul{inclusions}, such that the following universal property is satisfied:\\
For any object $M \in \mathcal{C}$ and any family $\{ \psi_{i} : A_{i} \rightarrow M \}$ of morphisms, there exists a unique
morphism $\psi : \amalg A_{i} : M$ called the \ul{coproduct morphism} such that
\[
\iota_{i} \psi = \psi_{i} \, \forall i \in I.
\]
\begin{tikzcd}
  &  &                                                           & A_{1} \arrow[llld, "\psi_{1}"', bend right] \arrow[ld, "\iota_{1}"'] \\
M &  & A_{1}\amalg A_{2} \arrow[ll, "\exists^{1} \psi"', dashed] &                                                                      \\
  &  &                                                           & A_{2} \arrow[lllu, "\psi_{2}", bend left] \arrow[lu, "\iota_2"]     
\end{tikzcd}
\end{enumerate}
\end{definition}

\begin{remark}{(Terminal object, initial object, zero object)}\label{def:init_term_zero_object}
\renewcommand{\labelenumi}{(\theenumi)}
\begin{enumerate}
\item A \ul{terminal object} $T$ in a category $\mathcal{C}$ is an object such that $\textup{Hom}_{\mathcal{C}}(-,T)$ is a singleton.
\item An \ul{initial object} $I$ in a category $\mathcal{C}$ is an object such that $\textup{Hom}_{\mathcal{C}}(I,-)$ is a singleton.
\item An object $Z$ or $0$ is a \ul{zero object} if it is both initial and terminal.
\end{enumerate}
\end{remark}

\begin{definition}{(Zero morphism)}\label{def:zero_morphism}\\
A \ul{zero morphism} in a category with a zero object $0$ is a morphism factoring over $0$, i.e. $\varphi : M \rightarrow N$ is called a zero
morphism, if\\
\begin{minipage}{.35\textwidth}
\begin{tikzcd}
M \arrow[rr, "\varphi"] \arrow[rd, "\varphi_{1}"] &                              & N \\
                                                  & 0 \arrow[ru, "\varphi_{2}"'] &  
\end{tikzcd}
\end{minipage}
\begin{minipage}{.65\textwidth}
$\exists \varphi_{1} : M \rightarrow 0, \varphi_{2} : 0 \rightarrow N$\\
such that $\varphi = \varphi_{1}\varphi_{2}$.
\end{minipage}
Since the zero object $0$ is both initial and terminal, a zero morphism is uniquely defined by its source and target, thus we can
talk about \textit{the} zero morphism from $M$ to $N$, which we denote by $0_{M,N}$.
\end{definition}

\begin{definition}{(Ab-category)}
An \ul{Ab-category} (also called \ul{pre-additive category}) is a category in which all homomorphism sets are abelian groups,
and composition distributes over addition.\\
In other words, a category $\mathcal{C}$ is an \ul{Ab-category} if for every pair of objects $M,N \in \mathcal{C}_{0}$,
$( \textup{Hom}_{\mathcal{C}}(M,N), + )$ is an abelian group (with the zero morphism $0_{M,N}$ as the neutral element),
and for all morphisms $\gamma, \delta \in \textup{Hom}_{\mathcal{C}}(M,N),
\alpha, \beta \in \textup{Hom}_{\mathcal{C}}(N,L)$
\begin{align}
(\gamma + \delta)\alpha &=\label{eq:dist1} \gamma\alpha + \delta\alpha \textup{ and }\\
\gamma(\alpha+\beta) &=\label{eq:dist2} \gamma\alpha + \gamma\beta.
\end{align}
Note that every hom-set has its own unique zero morphism. E.g. in $\kmat$ the $2 \times 3$ zero-matrix
$\mathbf{0} \in \textup{Hom}_{\kmat}(2,3)$ is different from the $4 \times 4$ zero-matrix $\mathbf{0} \in \textup{Hom}_{\kmat}(4,4)$.
\end{definition}

\begin{example}{(The matrix category $\kmat$ over a commutative ring $\Bbbk$)}\label{ex:matrix_category}
\begin{itemize}
\item Objects are natural numbers $\kmat_{0} = \mathbb{N} = \mathbb{N}_{0} = \{0,1,2,\dots\}$
\item Morphisms $\Rmat_{1} \ni (m \rightarrow n)$ are $m \times n$ matrices over $\Bbbk$.
We write the set of morphisms between $m$ and $n$, as $\Bbbk^{m\times n} := \textup{Hom}_{\kmat}(m,n)$. Identity morphisms are the
identity matrices.
\item Composition is matrix multiplication (associative).
\item It is a skeletal category, i.e. $m$ is isomorphic to $n \Rightarrow m = n$. Only quadratic matrices ($m = n$) can be
isomorphisms.
\end{itemize}
In this category, the number $0$ is \ul{the} zero object.\\
A zero matrix (zero morphism) is a matrix factoring through the zero object $0$.\\
\begin{minipage}{.2\textwidth}\phantom{ }\end{minipage}
\begin{minipage}{.25\textwidth}
$R^{m\times n} \ni A = 0_{m,n}$
\end{minipage}
\begin{minipage}{.08\textwidth}
$\Longleftrightarrow$
\end{minipage}
\begin{minipage}{.32\textwidth}
\begin{tikzcd}
m \arrow[rr, "A"] \arrow[rd, "(m \times 0)"'] &                               & n \\
                                              & 0 \arrow[ru, "(0 \times n)"'] &  
\end{tikzcd}\\
$\Rightarrow A = (m \times 0) \cdot (0 \times n)$.
\end{minipage}
\begin{minipage}{.15\textwidth}\phantom{ }\end{minipage}\\

\noindent The matrices $(m \times 0)$ and $(0 \times n)$ have zero columns or zero rows respectively, but it is
important to note that for each $m \in \kmat_{0}$ there is exactly one such matrix $(m \times 0)$ and $(0 \times m)$
(that's what initial and terminal object means), and for different $m$, these morphisms are different.
\end{example}

\begin{example}{($\kmat$ is an Ab-category)}
For two natural numbers $m,n \in {\kmat}_{0} = \mathbb{N} = \mathbb{N}_{0}$, the set of morphisms with source $m$ and target $n$ is
$\Bbbk^{m\times n}$, the set of $m \times n$-matrices. This is an abelian group:
\begin{itemize}
\item The neutral element of the addition is the $m \times n$ zero matrix $\mathbf{0}$.
\item Addition of matrices is associative and commutative, so it is an abelian group.
\end{itemize}
Composition of matrices is defined as matrix multiplication, which is bilinear, i.e. satisfies the distributive laws \eqref{eq:dist1} and \eqref{eq:dist2}.
\end{example}

\begin{definition}{(Biproduct)}\label{def:biproduct}
Let $I$ be an index set and $\{S_{i}\}_{i\in I}$ a family of objects in a category $\mathcal{C}$.
A \ul{biproduct} is a product and a coproduct simultaneously, i.e. consists of the following data:
\begin{itemize}
\item an object $S \in \mathcal{C}_{0}$,
\item a family of morphisms $\pi = \{ \pi_{i} : S \rightarrow S_{i} \}_{i\in I}$,
\item a family of morphisms $\iota = \{ \iota_{i} : S_{i} \rightarrow S \}_{i\in I}$,
\item a dependent function $u_{\text{in}}$ mapping every family $\tau = \{ \tau_{i} : T \rightarrow S_{i} \}_{i\in I}$ to a morphism
$u_{\text{in}}(\tau) : T \rightarrow S$ such that $u_{\text{in}}(\tau) \pi_{i} \sim \tau_{i}$ for all $i \in I$,
\item a dependent function $u_{\text{out}}$ mapping every family $\rho = \{ \rho_{i} : S_{i} \rightarrow R \}_{i\in I}$ to a morphism
$u_{\text{out}}(\rho) : S \rightarrow R$ such that $\iota_{i} u_{\text{out}}(\rho) \sim \rho_{i}$ for all $i \in I$,
\end{itemize}
\end{definition}

\begin{definition}{(Direct sum)}\label{def:direct_sum}
A \ul{direct sum} is a biproduct of objects in an Ab-category such that
\begin{itemize}
\item $\sum_{i\in I}  \pi_{i} \iota_{i} \sim 1_{S}$,
\item $ \iota_{i} \pi_{j} \sim \delta_{i, j} =  \begin{cases}
            1_{S_{i}} & \text{ if } i = j  \\
            0_{ij} & \text{ if } i \neq j
        \end{cases}$,
\end{itemize}
where $\delta_{i, j} \in \mathrm{Hom}(S_{i}, S_{j})$ is the identity if $i = j$, and the zero morphism $0_{ij} := 0_{S_{i}, S_{j}}$ otherwise.
\end{definition}

\begin{example}{($\kmat$ is an additive category)}\label{ex:kmat_has_direct_sums}
The matrix category $\kmat$ is an Ab-category. The direct sum of two ($I = \{1,2\}$) natural numbers $S_{1} = m, S_{2} = n \in \kmat_{0}$ is
\begin{itemize}
\item the object $S = m+n$,
\item the two morphisms $\pi_{1} : m+n \rightarrow m$, $\pi_{2} : m+n \rightarrow n$ which are $(m+n) \times m$- and $(m+n) \times n$-matrices.
\item the two morphisms $\iota_{1} : m \rightarrow m+n$, $\iota_{2} : n \rightarrow m+n$ which are $m \times (m+n)$- and $n \times (m+n)$-matrices.
\item a family of two morphisms $\tau = \{ \tau_{1} : t \rightarrow m, \tau_{2} : t \rightarrow n\}$ gets mapped to a $t \times (m+n)$-matrix
$u_{\text{in}}(\tau) : t \rightarrow m+n$ with $u_{\text{in}}(\tau) \pi_{1} = \tau_{1}$ and $u_{\text{in}}(\tau) \pi_{2} = \tau_{2}$.
\item a family of two morphisms $\rho = \{ \rho_{1} : m \rightarrow r, \rho_{2} : n \rightarrow r \}$ gets mapped to a $(m+n) \times r$-matrix
$u_{\text{out}}(\rho) : m+n \rightarrow r$ with $\iota_{1} u_{\text{out}} = \rho_{1}$ and $\iota_{2} u_{\text{out}} = \rho_{2}$.
\end{itemize}
\end{example}

\begin{example}{(number example)}
The direct sum of $m = 3, n = 5 \in \kmat_{0}$ is the object $m+n = 8 \in \kmat_{0}$ together with the following morphisms:
\begin{align*}
\pi_{1} = \begin{pmatrix}
1 \ampersand \cdot \ampersand \cdot \\
\cdot \ampersand 1 \ampersand \cdot \\
\cdot \ampersand \cdot \ampersand 1 \\
\cdot \ampersand \cdot \ampersand \cdot \\
\cdot \ampersand \cdot \ampersand \cdot \\
\cdot \ampersand \cdot \ampersand \cdot \\
\cdot \ampersand \cdot \ampersand \cdot \\
\cdot \ampersand \cdot \ampersand \cdot
\end{pmatrix},
\pi_{2} = \begin{pmatrix}
\cdot \ampersand \cdot \ampersand \cdot \ampersand \cdot \ampersand \cdot \\
\cdot \ampersand \cdot \ampersand \cdot \ampersand \cdot \ampersand \cdot \\
\cdot \ampersand \cdot \ampersand \cdot \ampersand \cdot \ampersand \cdot \\
1 \ampersand \cdot \ampersand \cdot \ampersand \cdot \ampersand \cdot \\
\cdot \ampersand 1 \ampersand \cdot \ampersand \cdot \ampersand \cdot \\
\cdot \ampersand \cdot \ampersand 1 \ampersand \cdot \ampersand \cdot \\
\cdot \ampersand \cdot \ampersand \cdot \ampersand 1 \ampersand \cdot \\
\cdot \ampersand \cdot \ampersand \cdot \ampersand \cdot \ampersand 1
\end{pmatrix}, 
\begin{array}{rr}
\iota_{1} &= \begin{pmatrix}
1 \ampersand \cdot \ampersand \cdot \ampersand \cdot \ampersand \cdot \ampersand \cdot \ampersand \cdot \ampersand \cdot \\
\cdot \ampersand 1 \ampersand \cdot \ampersand \cdot \ampersand \cdot \ampersand \cdot \ampersand \cdot \ampersand \cdot \\
\cdot \ampersand \cdot \ampersand 1 \ampersand \cdot \ampersand \cdot \ampersand \cdot \ampersand \cdot \ampersand \cdot
\end{pmatrix} \\
\\
\iota_{2} &= \begin{pmatrix}
\cdot \ampersand \cdot \ampersand \cdot \ampersand 1 \ampersand \cdot \ampersand \cdot \ampersand \cdot \ampersand \cdot \\
\cdot \ampersand \cdot \ampersand \cdot \ampersand \cdot \ampersand 1 \ampersand \cdot \ampersand \cdot \ampersand \cdot \\
\cdot \ampersand \cdot \ampersand \cdot \ampersand \cdot \ampersand \cdot \ampersand 1 \ampersand \cdot \ampersand \cdot \\
\cdot \ampersand \cdot \ampersand \cdot \ampersand \cdot \ampersand \cdot \ampersand \cdot \ampersand 1 \ampersand \cdot \\
\cdot \ampersand \cdot \ampersand \cdot \ampersand \cdot \ampersand \cdot \ampersand \cdot \ampersand \cdot \ampersand 1
\end{pmatrix}
\end{array}
\end{align*}\\
\noindent One can easily verify that $\pi_{1} \iota_{1} + \pi_{2} \iota_{2} = 1_{(3+5)} = 1_{8}$ and e.g. $\iota_{1} \pi_{2} = 0_{3,5}$.\\

\noindent \begin{minipage}[t]{.5\textwidth}
For $t = 4$, $\tau = (\tau_{1}, \tau_{2})$ defined as
\begin{align*}
\tau_{1} = \begin{pmatrix}
1 \ampersand 2 \ampersand 2 \\
4 \ampersand 3 \ampersand 1 \\
\cdot \ampersand 1 \ampersand \cdot \\
1 \ampersand 2 \ampersand 1
\end{pmatrix},
\tau_{2} = \begin{pmatrix}
1 \ampersand 1 \ampersand 2 \ampersand 2 \ampersand 3 \\
3 \ampersand 4 \ampersand 4 \ampersand 5 \ampersand 5 \\
6 \ampersand 6 \ampersand 7 \ampersand 7 \ampersand 8 \\
8 \ampersand 9 \ampersand 9 \ampersand 4 \ampersand 4
\end{pmatrix}
\end{align*}
we get the matrix
\begin{align*}
u_{\text{in}}(\tau) = \begin{pmatrix}
1 \ampersand 2 \ampersand 2 \ampersand 1 \ampersand 1 \ampersand 2 \ampersand 2 \ampersand 3 \\
4 \ampersand 3 \ampersand 1 \ampersand 3 \ampersand 4 \ampersand 4 \ampersand 5 \ampersand 5 \\
\cdot \ampersand 1 \ampersand \cdot \ampersand 6 \ampersand 6 \ampersand 7 \ampersand 7 \ampersand 8 \\
1 \ampersand 2 \ampersand 1 \ampersand 8 \ampersand 9 \ampersand 9 \ampersand 4 \ampersand 4
\end{pmatrix}
\end{align*}
\end{minipage}
\begin{minipage}[t]{.5\textwidth}
and for $r = 2$, $\rho = (\rho_{1}, \rho_{2})$ we get the matrix
\begin{align*}
\begin{array}{rr}
\rho_{1} &= \begin{pmatrix}
\cdot \ampersand 1 \\
1 \ampersand 1 \\
2 \ampersand 2
\end{pmatrix} \\
\\
\rho_{2} &= \begin{pmatrix}
4 \ampersand 5 \\
7 \ampersand \cdot \\
\cdot \ampersand 5 \\
\cdot \ampersand \cdot \\
1 \ampersand 1
\end{pmatrix}
\end{array}
u_{\text{out}}(\rho) = \begin{pmatrix}
\cdot \ampersand 1 \\
1 \ampersand 1 \\
2 \ampersand 2 \\
4 \ampersand 5 \\
7 \ampersand \cdot \\
\cdot \ampersand 5 \\
\cdot \ampersand \cdot \\
1 \ampersand 1
\end{pmatrix}
\end{align*}
\end{minipage}\\

\noindent One can easily verify that e.g. $\tau_{1} = u_{\mathrm{in}}(\tau) \pi_{1}$ and $\rho_{2} = \iota_{2} u_{\mathrm{out}}(\rho)$.
\end{example}

\begin{remark}{(Addition of matrices)}
The abelian group structure on $\mathrm{Hom}_{\kmat}(m,n)$ can be derived from the direct sum:
For $\rho_{1}, \rho_{2} \in \mathrm{Hom}_{\kmat}(m,n)$
\[
\rho_{1} + \rho_{2} = u_{\mathrm{in}}(1_{m},1_{m}) u_{\mathrm{out}}(\rho_{1},\rho_{2}) 
= u_{\mathrm{in}}(\rho_{1},\rho_{2})u_{\mathrm{out}}(1_{n},1_{n})
\]
This makes $\kmat$ an example of an additive category.
\end{remark}

\begin{definition}\label{def:additive_category}\endnote{ ref. \cite{[Posur]}, Def. 2.8}
An \ul{additive category} is an Ab-category $\mathcal{C}$ together with a dependent function $\oplus^{\mathcal{C}}$ mapping
a finite set $I$ and a family $(A_{i})_{i\in I}$ of objects in $\mathcal{C}$ to a corresponding direct sum $(\oplus_{i\in I}^{\mathcal{C}} A_{i},
(\pi_{i})_{i\in I}, (\iota_{i})_{i\in I})$.
\end{definition}

%%% additive functor

\subsection{Kernel and cokernel; image and coimage}

\begin{definition}{(Kernel)}
In an additive category $\mathcal{C}$, the \ul{kernel} of a morphism $f : A \rightarrow B \in \mathcal{C}_{1}$ is the equalizer of $f$ and $0_{A,B}$,
i.e. consists of the data
\begin{enumerate}
\renewcommand{\labelenumi}{(\theenumi)}
\item An object $K = \mathrm{ker}(f)$
\item A morphism $\mathrm{KernelEmbedding}(f) := \kappa : K \rightarrow A$ such that
\[
\kappa \cdot f = 0_{K,B}
\]
\item A dependent function $\mathrm{KernelLift}(f,-) := ( - /\kappa)$ mapping a morphism $\tau : T \rightarrow A$ with $\tau f = 0_{T,B}$ to a
morphism $\mathrm{KernelLift}(f,\tau) (\tau / \kappa) : T \rightarrow K$ such that
\[
\tau =\label{eq:kernel_lift} (\tau / \kappa) \cdot \kappa.
\]
\item For any other dependent function $v$ satisfying \eqref{eq:kernel_lift}, $v = ( - / \kappa)$.
\end{enumerate}
\end{definition}

\begin{definition}{(Cokernel)}
In an additive category $\mathcal{C}$, the \ul{cokernel} of a morphism $f : A \rightarrow B \in \mathcal{C}_{1}$ is the coequalizer of
$f$ and $0_{A,B}$, i.e. consists of the data
\begin{enumerate}
\renewcommand{\labelenumi}{(\theenumi)}
\item An object $C = \mathrm{coker}(f)$
\item A morphism $\mathrm{CokernelProjection}(f) := \varepsilon : B \rightarrow C$ such that
\[
f \cdot \varepsilon = 0_{A,C}
\]
\item A dependent function $\mathrm{CokernelCoLift}(f,-) := ( \varepsilon \backslash -)$ mapping a morphism $\tau : B \rightarrow T$ with
$f \tau  = 0_{A,T}$ to a morphism $\mathrm{CokernelCoLift}(f,\tau) := ( \varepsilon \backslash \tau) : C \rightarrow T$ such that
\[
\tau =\label{eq:cokernel_colift} \varepsilon \cdot (\varepsilon \backslash \tau).
\]
\item For any other dependent function $v$ satisfying \eqref{eq:cokernel_colift}, $v = ( - \backslash \kappa)$.
\end{enumerate}
\end{definition}

\begin{definition}{(Pre-Abelian category)}
A \ul{pre-abelian category} consists of the following data:
\begin{enumerate}
\item An additive category $\mathcal{C}$.
\item A dependent function mapping every morphism $f : A \rightarrow B$ for $A, B \in \mathcal{C}_{0}$ to a
kernel of $f$.
\item A dependent function mapping every morphism $f : A \rightarrow B$ for $A, B \in \mathcal{C}_{0}$ to a
cokernel of $f$.
\end{enumerate}
\end{definition}

%%% exact functor

\begin{definition}{(Abelian category)}\endnote{(From \cite{[context]}, appendix E.5, Def. E.5.1)}
A category $\mathcal{C}$ is \ul{abelian} if
\begin{itemize}
\item it has a zero object $0$,
\item it has all \ul{binary products} and \ul{binary coproducts},
\item it has all \ul{kernels} and \ul{cokernels}, defined repsectively to be the \ul{equalizer} and
\ul{coequalizer} of a map $f : A \rightarrow B$ with the zero map $A \rightarrow 0 \rightarrow B$, and
\item all monomorphisms and epimorphisms arise as kernels or cokernels, respectively.
\end{itemize}
\end{definition}

%%% functors between abelian categories.


\begin{theorem}
The functor category has all limits, colimits and bilimits which exist in the target category.
\end{theorem}

Instead of proving this in general, we prove this as part of \ref{thm:functor_category_abelian} for the direct sum, which is a very special bilimit.

