Goal is to show that

The functor category with values in $\kmat$ is an Abelian category with enough projectives (constructively) and direct sum decomposition (constructively).

\subsection{Abelian category}

\begin{lemma}
Let $\mathcal{A}$ be a finite-dimensional algebroid over some field $\Bbbk$. The functor category
\[
\HomAkmat
\]
is an Ab-category.
\begin{proof}
\begin{itemize}
\renewcommand{\labelenumi}{(\theenumi)}
\item Let $F, G \in \mathrm{Hom}(\mathcal{A},\kmat)$. We want to show that
$\mathrm{Hom}_{\mathrm{Hom}(\mathcal{A},\kmat)}(F,G)$ is an abelian group.
We define the morphisms in $\mathrm{Hom}(\mathcal{A},\kmat)_{1}$ as natural transformations
between the functors, i.e. a morphism $\eta F \Rightarrow G$ is defined by its components $\eta_{o}$ for each object $o \in \mathcal{A}$,
such that $G(o) = \eta_{o} F(o)$. Each $\eta_{o}$ is a morphism in $\kmat$, i.e. a matrix. And for these we already
showed that it's an abelian category, therefore $\mathrm{Hom}_{\mathrm{Hom}(\mathcal{A},\kmat)}(F,G)$ is an abelian group.
\item The composition of two morphisms $\eta, \gamma \in \mathrm{Hom}(\mathcal{A},\kmat)_{1}$ 
\end{itemize}
\end{proof}
\end{lemma}

\begin{lemma}
Let $\mathcal{A}$ be a finite-dimensional algebroid over some field $\Bbbk$. The functor category
\[
\mathrm{Hom}(\mathcal{A},\kmat)
\]
is an additive category.
\begin{proof}
\begin{itemize}
\renewcommand{\labelenumi}{(\theenumi)}
\item It is an Ab-category, which we showed in \ref{}.
\item
\end{itemize}
\end{proof}
\end{lemma}

\begin{lemma}
Let $\mathcal{A}$ be a finite-dimensional algebroid over some field $\Bbbk$. The functor category
\[
\mathrm{Hom}(\mathcal{A},\kmat)
\]
is a pre-abelian category.
\begin{proof}
\begin{itemize}
\renewcommand{\labelenumi}{(\theenumi)}
\item It is an additive category, which we showed in \ref{}.
\item
\end{itemize}
\end{proof}
\end{lemma}

\begin{theorem}
Let $\mathcal{A}$ be a finite-dimensional algebroid over some field $\Bbbk$. The functor category $\HomAkmat$
is an abelian category.
\end{theorem}
\begin{proof}
We prove that $\HomAkmat$ is an Ab-category, then that it is also an additive category and a pre-abelian category and finally that
it is an abelian category.
\begin{enumerate}
\renewcommand{\labelenumi}{(\theenumi)}
\item In order to show that $\HomAkmat$ is an Ab-category, we must show that 
\begin{enumerate}
\renewcommand{\labelenumii}{(\roman{enumii})}
\item for any two objects $F,G \in \HomAkmat_{0}$ the hom-set $\mathrm{Hom}_{\HomAkmat}(F,G)$ between them is an Abelian group, and
\item that the composition of two morphisms\\
$\mu : \mathrm{Hom}_{\HomAkmat}(F,G) \times \mathrm{Hom}_{\HomAkmat}(G,H) \rightarrow \mathrm{Hom}_{\HomAkmat}(F,H)$ is a
bilinear map, i.e. for $\eta, \varepsilon \in \mathrm{Hom}_{\HomAkmat}(F,G)$ and $\varphi, \psi \in \mathrm{Hom}_{\HomAkmat}(G,H)$ and
for $x \in \Bbbk$
\begin{align}
(\eta + \varepsilon)\varphi &= \eta\varphi + \varepsilon\varphi \\
\eta(\varphi + \psi) &= \eta\varphi + \eta\psi \\
(x\eta)\varphi &= \eta(x\varphi) = x(\eta\varphi).
\end{align}
\end{enumerate}

\begin{subproof}[Proof of (i)]
For any object $c \in \mathcal{A}$, the set of components $\mathrm{Hom}_{\kmat}(Fc,Gc)$ is a $\Bbbk$-vector space and therefore an
Abelian group.

We define the addition and scalar multiplication
\begin{align*}
+ :&& \mathrm{Hom}_{\HomAkmat}(F,G) &\times \mathrm{Hom}_{\HomAkmat}(F,G) \rightarrow \mathrm{Hom}_{\HomAkmat}(F,G)\\
\cdot :&& \Bbbk &\times \mathrm{Hom}_{\HomAkmat}(F,G) \rightarrow \mathrm{Hom}_{\HomAkmat}(F,G)
\end{align*}
component-wise: $\forall \eta, \varepsilon \in \mathrm{Hom}_{\HomAkmat}(F,G), \forall x \in \Bbbk, \forall c \in \mathcal{A}$
\begin{align}
(\eta+\varepsilon)_{c} &:= \eta_{c} + \varepsilon_{c}\\
(x \eta)_{c} &:= x\eta_{c}
\end{align}
where the right-hand side is the usual addition and scalar multiplication of matrices.

We identify as the neutral element $0_{F,G} \in \mathrm{Hom}_{\HomAkmat}(F,G)$ (or simply $0$ when the context is clear)
the natural transformation $0$ with each component $0_{c}$ being the $Fc\times Gc$ zero matrix.
For each natural transformation $\eta$ the additive inverse $-\eta$ is defined component-wise as
\begin{align}
(-\eta)_{c} &:= -\eta_{c}.
\end{align}
We also confirm that the addition is commutative:
\begin{align}
(\eta+\varepsilon)_{c} &= \eta_{c} + \varepsilon_{c}\\
    &= \varepsilon_{c} + \eta_{c}\\
    &= (\varepsilon + \eta)_{c}
\end{align}
This concludes that for each $F, G \in \HomAkmat,\, \mathrm{Hom}_{\HomAkmat}(F,G)$ is an Abelian group.
\end{subproof}
\begin{subproof}[Proof of (ii)]
Let $F, G, H \in \HomAkmat$ and let $\eta, \varepsilon \in \mathrm{Hom}_{\HomAkmat}(F,G)$,
$\varphi, \psi \in \mathrm{Hom}_{\HomAkmat}(G,H)$ and $x \in \Bbbk$.\\
The composition $\eta\varphi \in \mathrm{Hom}_{\HomAkmat}(F,H)$ is defined by component-wise matrix-multiplication,
i.e. $\forall c \in \mathcal{A}$
\begin{align*}
(\eta\varphi)_{c} := \eta_{c}\varphi_{c}
\end{align*}
and from this follows the bilinearity of the composition, since the matrix multiplication is bilinear.\\
This concludes the first part of the proof, i.e. $\HomAkmat$ is an Ab-category.
\end{subproof}

\item Next we show that $\HomAkmat$ is an additive category, i.e. it is
\begin{enumerate}
\renewcommand{\labelenumii}{(\roman{enumii})}
\item An Ab-category with
\item A dependent function $\oplus$ mapping a finite set $I$ and a collection $(F_{i})_{i\in I}$ of objects in $\HomAkmat$
to a corresponding direct sum $( \oplus_{i\in I} F_{i}, (\pi_{i})_{i\in I}, (\iota_{i})_{i\in I} )$.
\end{enumerate}
\begin{subproof}[Proof of (ii)]
Let $I = \{1,\dots,n\}$ be a finite set, and $F_{i} \in \HomAkmat, 1\leq i\leq n$. The object $\oplus_{i=1}^{n} F_{i}$ is a functor, mapping
each object $c \in \mathcal{A}$ to the natural number $\oplus_{i=1}^{n} F_{i}c := \sum_{i=1}^{n} F_{i}c \in \kmat_{0}$.\\
A morphism $a : c \rightarrow c' \in \mathcal{A}_{1}$ with $F_{i} a : F_{i} c \rightarrow F_{i} c', \, 1\leq i\leq n$ gets mapped to the matrix 
$\oplus_{i=1}^{n} F_{i} a : \oplus_{i=1}^{n} F_{i}c \rightarrow \oplus_{i=1}^{n} F_{i}c'$ that is the block-diagonal matrix of the $F_{i}a$'s.
In other words, we can identify each $F_{j}a$ as a $F_{j} c \times F_{j} c'$ submatrix:
\[
\left(F_{j}a\right)_{k,l} = \left(\oplus_{i=1}^{n} F_{i} a\right)_{\sum_{i=1}^{j-1} F_{i}c + k,\sum_{i=1}^{j-1} F_{i}c' + l},
1 \leq k \leq F_{j}c, 1 \leq l \leq F_{j}c'
\]
\end{subproof}

\item
\end{enumerate}
\end{proof}

\subsection{Enough Projective objects (constructively)}

\begin{theorem}{($\HomAkmat$ has enough projectives)}
Let $\mathcal{A}$ be a finite-dimensional algebroid over some field $\Bbbk$. The functor category $\HomAkmat$ has sufficiently many
projectives (ref. \ref{def:enough_projectives}).
\end{theorem}
\begin{proof}
Let $F \in \HomAkmat_{0}$. We want to find a projective object $P \in \HomAkmat_{0}$ and an epimorphism $\eta: P \twoheadrightarrow F$.






\end{proof}

\subsection{Direct sum decomposition (constructively)}

\begin{tikzcd}
F1 \arrow["Fa"', loop, distance=2em, in=125, out=55] \arrow[rr, "Fb"] \arrow[dd, "\eta_{1}"]
&  & F2 \arrow["Fc"', loop, distance=2em, in=125, out=55] \arrow[dd, "\eta_{2}"] \\
&  &                                                                                 \\
G1 \arrow["Ga"', loop, distance=2em, in=305, out=235] \arrow[rr, "Gb"]
&  & G2 \arrow["Gc"', loop, distance=2em, in=305, out=235]
\end{tikzcd}

\begin{align}
Fa\,\eta_{1} &= \eta_{1} Ga \\
Fb\,\eta_{2} &= \eta_{1} Gb \\
Fc\,\eta_{2} &= \eta_{2} Gc
\end{align}

For the first equation we have

\begin{align}
Fa\,\eta_{1} - \eta_{1} Ga = 0
\end{align}

thus

\begin{align}
\left( Fa\,\eta_{1} - \eta_{1} Ga \right)_{i,j} = 0,\, 1\leq i \leq 5, 1\leq j \leq 3
\end{align}

