In this thesis we study the functor category of a finite concrete category with values in a matrix category over a field.

We give an introduction to quivers, categories, functors and natural transformations in \cref{sec:quivers_categories} allowing a student with
no background in category theory to understand this thesis. We are also introducing finite concrete categories which we
model as finite subcategories of $\mathrm{FinSets}$, the category of finite sets.
The main source for this section is \cite[Chapter 1, Sections 1.1 - 1.4]{[context]}.

Then we define (co-)limits in \cref{sec:limits_colimits} and use them to build up a hierarchy of categorical doctrines leading up
to three equivalent definitions for Abelian categories.
As an example for an Abelian category we define the matrix category $\kmat$
and follow the steps from Ab-category, category with zero object, additive category, pre-Abelian category and
finally Abelian category, each time giving constructive algorithms for that doctrine in $\kmat$. This section is in part a 
citation from \cite[I.1.38-I.1.47 and I.2]{[Posur]} and from the lecture notes \cite{[AlgAlg]}.

In \cref{sec:algebroids} we revisit the relation between quivers and categories from chapter 1
and give an example for an adjunction, the free-forgetful adjunction. The rest of chapter 3 is concerned with
bringing our finite concrete category into a form that is easily representable while maintaining its finite structure.
The endomorphism monoid at each object plays an important role. We formulate and prove the $\sigma$-lemma
which holds in general finite monoids. For a finite concrete category with explicitly
cyclic endomorphism monoids we can compute the relations of endomorphisms. The algorithm is an application
of the orbit algorithm and terminates due to the $\sigma$ lemma.
This allows us to write the finite category $\mathcal{C}$ as a quotient modulo finitely many relations of a free category $Fq$
generated by a finite quiver $q$.
Since we consider functors from $\mathcal{C}$ into the $\Bbbk$-linear category $\kmat$, we can equally consider $\Bbbk$-linear functors
from the $\Bbbk$-linear closure $\Bbbk\mathcal{C}$ of $\mathcal{C}$. This closure is a finite-dimensional $\Bbbk$-algebroid. 
The $\Bbbk$-linear closure $\Bbbk\mathcal{C}$ can be modeled as the $\Bbbk$-algebra of paths of the
quiver $q$ modulo finitely many relations. This is a finite-dimensional algebra with a complete set of orthogonal idempotents.

\cref{sec:functor_cat_abelian} is the synthesis of \cref{sec:limits_colimits,sec:algebroids}, in that we define the category of representations $\CatReps$
of our finite concrete category, which is the category $\HomAkmat$ of $\Bbbk$-linear functors from the $\Bbbk$-algebroid
$\mathcal{A}$ to $\kmat$. Using only categorical constructions provided by the axioms of the abelian category $\kmat$, we prove that
the functor category with values in an abelian category is an abelian category.

A thesis on category theory would be incomplete without mentioning Yoneda's lemma. In \cref{sec:yoneda_projective} we
formulate Yoneda's lemma, citing a proof in \cite[2.2]{[context]}, and define the Yoneda embedding of a category $\mathcal{C}$ into the
functor category from $\mathcal{C}^{\text{op}}$ to $\mathbf{Set}$. For our purposes we
define our $\Bbbk$-linear version of Yoneda's embedding of the opposite $\Bbbk$-algebroid $\mathcal{A}^{\text{op}}$
into the functor category $\HomAkmat$. In the rest of chapter 5 we define projective objects and that
Yoneda projectives are in fact projective objects, closing with a statement that $\HomAkmat$ has enough projectives.

The last two sections deal with the hom-structure of $\HomAkmat$. Since the morphisms between two
functors are natural transformations, they must fulfill certain naturality constraints leading to Sylvester equations. 
In \cref{sec:direct_sum_decompositon} we also give a probabilistic algorithm for the direct sum decomposition of functors. The proof of the decomposition is a linear
analogue of the $\sigma$ lemma.

In \cref{sec:hom_based_invariants} we give hom-based invariants that distinguish between non-isomorphic objects in our functor category, which provide 
categorical proofs that distinguish e.g. a Yoneda projective from other objects.

The result of this thesis is the \Gap package \CatReps which was originally meant to wrap Peter Webb's
pre-package \catreps into the categorical framework offered by \CAP. Since the category of representations is just an application of the package
\FunctorCategories, most of the functionality used by \CatReps is provided through \FunctorCategories.
Currently \CatReps consists of the methods
\begin{itemize}
\item \funcref{ConvertToMapOfFinSets},
\item \funcref{ConcreteCategoryForCAP},
\item \funcref{RightQuiverFromConcreteCategory},
\item \funcref{RelationsOfEndomorphisms},
\item \funcref{AsFpCategory},
\item \funcref{Algebroid}.
\end{itemize}
The content of this package might moveto \FunctorCategories in the future. All the algorithms developed and implemented while working on this
thesis are included as code listings in the appendix.
