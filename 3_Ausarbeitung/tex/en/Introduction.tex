In this thesis 

We give an introduction to quivers, categories, functors and natural transformations in chapter 1 allowing a student with
no background in category theory to understand this thesis. We are also introducing finite concrete categories 
which we model as finite subcategories of $\mathrm{FinSets}$.

Then we define (co-)limits in chapter 2 and use them to build up a hierarchy of categorical doctrines leading up
to three equivalent definitions for Abelian categories.
As an example for an Abelian category we define the matrix category $\kmat$
and follow the steps from Ab-category, category with zero object, additive category, pre-Abelian category and
finally Abelian category, each time giving constructive algorithms for that doctrine in $\kmat$.

In chapter 3 we revisit the relation between quivers and categories from chapter 1
and give an example for an adjunction, the free-forgetful adjunction. The rest of chapter 3 is concerned with
bringing our finite concrete category into a form that is easily representable while maintaining its finite structure.
The endomorphism monoid at each object plays an important role. We formulate and prove the $\sigma$-lemma
which holds in general finite monoids. For a finite concrete category with explicitly
cyclic endomorphism monoids we can - using the $\sigma$ lemma -  compute the relations of endomorphisms as generators of an ideal
of the path algebra. This allows us to compute the $\Bbbk$-linear closure of a finite concrete category
as a finite-dimensional quotient algebra with orthogonal idempotents, i.e. a finite-dimensional $\Bbbk$-algebroid.

Chapter 4 now is the synthesis of chapters 2 and 3, in that we define the category of representations $\CatReps$
of our finite concrete category, which is the category $\HomAkmat$ of $\Bbbk$-linear functors from a $\Bbbk$-algebroid
to $\kmat$. Using only categorical constructions provided by the axioms of the abelian category $\kmat$, we prove that
the functor category with values in an abelian category is an abelian category.

A thesis on category theory would be incomplete without mentioning Yoneda's lemma. In chapter 5 we
formulate Yoneda's lemma and define the Yoneda embedding of a category $\mathcal{C}$ into the
functor category from $\mathcal{C}^{\text{op}}$ to $\mathbf{Set}$. For our purposes we
define our $\Bbbk$-linear version of Yoneda's embedding of the opposite $\Bbbk$-algebroid $\mathcal{A}^{\text{op}}$
into the functor category $\HomAkmat$. In the rest of chapter 5 we define projective objects and that
Yoneda projectives are in fact projective objects, closing with a statement that $\HomAkmat$ has enough projectives.

The last two chapters deal with the hom-structure of our functor category $\HomAkmat$. In chapter 6 we give
algorithms for a probabilistic direct sum decomposition of functors. The proof of the decomposition is a linear
analogue of the $\sigma$ lemma.

In chapter 7 we give hom-based invariants that distinguish between different objects in our functor category, which provide 
categorical proofs that distinguish e.g. a Yoneda projective from other objects.

Chapter 1 from \cite{[context]}, chapter 2 from \cite{[Posur]}. Algorithms in chapter 6 based on Peter Webb's catreps.
