% mainfile: ../main.tex

\subsection{Additional structure on the Hom-set of a category}

....

\begin{example}{(Group as a category)}\\
\noindent A group $\mathbf{G}$ defines a category $\mathcal{B}\mathbf{G}$ with a single object $\ast$. The group elements are its morphisms, which are
all automorphisms (i.e. bijective endomorphisms) of the single object. Composition of morphisms is defined by the binary group operation.
The identity element $e \in G$ acts as the identity morphism for the unique object in this category. The hom-set of that category is itself
a group.
\end{example}

This example can be generalized to categories where the hom-set is a ring or an R-algebra. But for this we need a commutative ring R.

Our goal is to represent finite concrete categories, for this we need the source and target categories of our functors, which the
representations are.
As subcategories of $\textup{FinSets}$, our finite concrete categories only have definitions for their objects and their
morphisms, methods to check when two morphisms are congruent or equivalent, but not much else.
A competing theory to category theory is that of quivers and path algebras. We already used their terminology in
\ref{def:path}, \ref{la:cyclic_paths} and \ref{def:path_algebra}, for instance when talking about the trivial path,
which in the language of category theory is nothing but the identity morphism, composition of arrows to a path is nothing but
composition of morphisms (if you make the path explicit by writing a new arrow for every path).

So what we called a path algebra in \ref{def:path_algebra} is a different data structure for a category. 
For one, the path algebra is an algebra, i.e. a vector space with additional structure, and thus a single set, comparable to the
class of morphisms $\mathcal{C}_{1}$ of a category $\mathcal{C}$.
But as it is an algebra, it not only contains the generating morphisms of the category, but also $\Bbbk$-linear combinations of
morphisms and paths. This is what our concrete categories lack, and what additional structure we have to give them in order
to represent them by matrices.

In practise, there is already developed software for \ul{q}uivers and \ul{p}ath \ul{a}lgebras, namely the \textsc{Gap} package
\textsc{QPA$2$}\endnote{(see \cite{[QPA2]})}.
What we are actually doing to represent finite concrete categories, is going from $\mathcal{C} \in \mathbf{Cats}$ to $q \in \mathbf{Quiv}$,
in theory by \ul{forgetting} (see \ref{ex:forgetful_functor}) the category concepts of identity morphism and composition, in practise by calculating the
underlying quiver $q$, and then for a commutative ring $\Bbbk$, constructing the path algebra $\Bbbk q$. In this step the path algebra
is infinite-dimensional, since there are infinitely many paths according to lemma \ref{la:cyclic_paths}, and \textsc{QPA$2$}'s function
\texttt{BasisPathsBetweenVertices} only works for finite-dimensional path algebras. Thus in a next step we have to provide
additional data in the form of generators of ideals of the path algebra, by which we can divide and build the quotient path algebra,
which is then finite-dimensional. This is the purpose of \texttt{RelationsOfEndomorphisms}.

Once we have a finite-dimensional path algebra $\Bbbk q$, we let \textsc{QPA$2$} calculate generators of the non-endomorphism relations,
and when we have a complete set of relations, that will be our definitive quotient quiver algebra $\Bbbk q$, which we then take it back into the category
theoretical context by constructing the $\Bbbk$-\textbf{Algebroid} $\mathcal{A}$ from the path algebra $\Bbbk q$.

The source category for our representation is then the $\Bbbk$-\textbf{Algebroid} $\mathcal{A}$ and not anymore our finite concrete
category $\mathcal{C}$, but it behaves in the same way regarding composition of morphisms and which morphisms are congruent.

The target category of our category representations will be $\Bbbk$-\textbf{Mat} which we will describe in the next section,
especially all the nice properties $\Bbbk$-\textbf{Mat} has, and how they get carried over to our functor category with $\Bbbk$-\textbf{Mat} as
target.\endnote{
In \cite{[Ab-Cat]}, Posur used the equivalence between categories $\textup{mat}_{\Bbbk} \cong \textup{vec}^{\text{fd}}_{\Bbbk}$,
as described in \cite{[context]}, \textsc{Example} 1.5.6 on page 30 (48/258), to justify that $\Bbbk$-\textbf{Mat} is a good
\textbf{computational model} to
%\setquotestyle[guillemets]{english} don't do that!
\blockquote{transform otherwise inaccessible mathematical objects into computationally easily graspable entities}
\setquotestyle{default}, which is what we are doing with \textbf{CatReps}.
}

With source and target categories defined, the category where our category representations lie in is \textbf{CatReps} for which we
show that it's a subcategory of the \textbf{Functor Category}. And even more in the next section.
\[
\mathbf{CatReps_{\mathcal{C}}} = \textup{Hom}(\Bbbk\mathbf{-Algebroid_{\mathcal{C}}}, \kmat)
\]

\subsection{Generating morphisms of a category and the underlying quiver}

$\textup{gmorphisms} := \{g_{1},\dots,g_{r}\} \rightarrow$ concrete category with set of generating morphisms $\textup{gmorphisms}$.

This is the $\textup{Free}$ functor from $\mathbf{Quiv}$ to $\mathbf{Cat}$, taking a quiver and adding the missing structure of
identity morphisms and composition of arrows to that category. The result is a category.

The $\textup{forgetful}$ functor from $\mathbf{Cat}$ to $\mathbf{Quiv}$ is going the other way around and leaves all
morphisms that we now have in the category, but forgets their relations, what was identity, what was composition.

Given a field $\Bbbk$, we have the path algebra $\Bbbk q$ with all the arrows as a basis.

Given relations on endomorphisms and on the other morphisms, we make the quotient path algebra.

This is already a category, and now it has more structure.

\subsection{Ab-categories}

\begin{definition}{(Ab-category)}
An \ul{Ab-category} is a category in which all homomorphism sets are abelian groups, and composition distributes over addition.\\
In other words, a category $\mathcal{C}$ is an \ul{Ab-category} if for every pair of objects $M,N \in \mathcal{C}_{0}$,
$( \textup{Hom}_{\mathcal{C}}(M,N), + )$ is an abelian group (with the neutral element called \ul{zero morphism}),
and for all morphisms $\gamma, \delta \in \textup{Hom}_{\mathcal{C}}(M,N),
\alpha, \beta \in \textup{Hom}_{\mathcal{C}}(N,L)$
\begin{align}\label{eq:dist}
(\gamma + \delta)\alpha &= \gamma\alpha + \delta\alpha \textup{ and }\\
\gamma(\alpha+\beta) &= \gamma\alpha + \gamma\beta.
\end{align}
Note that every hom-set has its own unique zero morphism. E.g. in $\textup{Mat}_{\mathbb{Q}}$ the $2 \times 3$ zero-matrix
$\mathbf{0} \in \textup{Hom}(2,3)$ is different from the $4 \times 4$ zero-matrix $\mathbf{0} \in \textup{Hom}(4,4)$.
\end{definition}

\begin{definition}{(semisimple)}
A ring R is semisimple if ...
\end{definition}

\begin{example}{(The matrix category $\Rmat$ over a commutative ring $R$)}\label{ex:matrix_category}
\begin{itemize}
\item Objects are natural numbers $\textup{Obj}(\textup{Mat}_{R}) = \mathbb{N} = \mathbb{N}_{0} = \{0,1,2,\dots\}$
\item Morphisms $\textup{Mor}(\textup{Mat}_{R}) \ni (m \rightarrow n)$ are $m \times n$ matrices over $R$.
We write the set of morphisms between $m$ and $n$, as $R^{m\times n} := \textup{Hom}(m,n)$. Identity morphisms are the
identity matrices.
\item Composition is matrix multiplication (associative).
\item It is a skeletal category, i.e. $m$ is isomorphic to $n \Rightarrow m = n$. Only quadratic matrices ($m = n$) can be
isomorphisms.
\end{itemize}
In this category, the number $0$ is \ul{the} zero object.\\
A zero matrix (zero morphism) is a matrix factoring through the zero object $0$.\\
\begin{minipage}{.2\textwidth}\phantom{ }\end{minipage}
\begin{minipage}{.25\textwidth}
Matrix $R^{m\times n} \ni A = 0$
\end{minipage}
\begin{minipage}{.08\textwidth}
$\Longleftrightarrow$
\end{minipage}
\begin{minipage}{.32\textwidth}
\begin{tikzcd}
m \arrow[rr, "A"] \arrow[rd, "(m \times 0)"'] &                               & n \\
                                              & 0 \arrow[ru, "(0 \times n)"'] &  
\end{tikzcd}\\
$\Rightarrow A = (m \times 0) \cdot (0 \times n)$.
\end{minipage}
\begin{minipage}{.15\textwidth}\phantom{ }\end{minipage}\\
\noindent The ``matrices'' $(m \times 0)$ and $(0 \times n)$ have zero columns or zero rows respectively, but it is
important to note that for each $m \in \textup{Obj}(\textup{Mat}_{R})$ there is exactly one such matrix $(m \times 0)$ and $(0 \times m)$
(that's what initial and terminal object means), and for different $m$, these morphisms are different.
\end{example}


\begin{example}{($\kmat$ is an Ab-category)}
For two natural numbers $m,n \in {\kmat}_{0} = \mathbb{N} = \mathbb{N}_{0}$, the set of morphisms with source $m$ and target $n$ is
$\Bbbk^{m\times n}$, the set of $m \times n$-matrices. This is an abelian group:
\begin{itemize}
\item The neutral element of the addition is the $m \times n$ zero matrix $\mathbf{0}$.
\item Addition of matrices is associative and commutative, so it's an abelian group.
\end{itemize}
The distributive laws \eqref{eq:dist} for composable morphisms hold.
\end{example}









\begin{definition}{(Abelian category)}\endnote{(From \cite{[context]}, appendix E.5, Def. E.5.1)}
A category $\mathcal{C}$ is \ul{abelian} if
\begin{itemize}
\item it has a \ul{zero object} $0$, that is both initial and terminal,
\item it has all \ul{binary products} and \ul{binary coproducts},
\item it has all \ul{kernels} and \ul{cokernels}, defined repsectively to be the \ul{equalizer} and
\ul{coequalizer} of a map $f : A \rightarrow B$ with the zero map $A \rightarrow 0 \rightarrow B$, and
\item all monomorphisms and epimorphisms arise as kernels or cokernels, respectively.
\end{itemize}
\end{definition}

\begin{definition}{($R$-linear category)}
Let $R$ be a commutative ring.
\end{definition}

For $R = \mathbb{Z}$ an $R$-linear category is nothing but an Ab-category.


\begin{definition}
Once source and target categories $\mathcal{C}, \mathcal{D}$ are both $R$-linear categories we define the functor category
$\mathrm{Hom_{R}}(\mathcal{C},\mathcal{D})$ the subcategory of $R$-linear functors.
\end{definition}



