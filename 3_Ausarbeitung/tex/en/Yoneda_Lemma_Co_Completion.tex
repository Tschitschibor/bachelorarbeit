% mainfile: ../main.tex

\section{Yoneda's Lemma: Completion and cocompletion of a category}

\subsection{Embedding categories}

\begin{lemma}{(Yoneda's Lemma)}

\begin{proof}

\end{proof}
\end{lemma}

\begin{lemma}\label{la:Hom_exact_proj_Lift_along_epis}
Let $\mathcal{C}$ be a category. For an object $P \in \mathcal{C}_{0}$ the following are equivalent:
\begin{itemize}
\item The covariant functor $\textup{Hom}(P,-)$ is exact.
\item For all epimorphisms $\varphi : M \twoheadrightarrow N$ and morphisms $\theta : P \rightarrow N$, there exists a
projective lift $\psi : P\,\dottedrightarrow\,M$ such that $\theta = \psi\varphi$.\\
\begin{tikzcd}
M \arrow[r, "\varphi", two heads] & N \\
	& P \arrow[u, "\theta", "=\,\psi\varphi"'] \arrow[lu, "\psi", dotted]
\end{tikzcd}
\end{itemize}
\begin{proof}
For an object $L\in \mathcal{C}_{0}$, $\textup{Hom}(L,-)$ is always a covariant left-exact functor, i.e. respects monos.\\
\ul{''$\Leftarrow$'' :} Prove that $\textup{Hom}(P,-)$  is right exact, i.e. respects epis.\\
For this, let $M, N \in \mathcal{C}_{0}$ and $\varphi : M \twoheadrightarrow N$ be an epi. The Hom-functor works on morphisms
by mapping the Hom-sets of the source and target objects of the morphism, i.e.
$\textup{Hom}(P,\varphi) : \textup{Hom}(P,M) \rightarrow \textup{Hom}(P,N)$, given by $\rho \mapsto \rho\varphi\, \forall \rho \in \textup{Hom}(P,M)$.
Now given that $\varphi$ is an epi, we want to show that $\textup{Hom}(P,\varphi)$ is also an epi.\\
Let $O \in \mathcal{C}_{0}$,  $\gamma : N \rightarrow O$ and $\varepsilon : N \rightarrow O$ such that
$\textup{Hom}(P,\gamma) : \textup{Hom}(P,N) \rightarrow \textup{Hom}(P,O);\, \theta \mapsto \theta\gamma$ and
$\textup{Hom}(P,\varepsilon) : \textup{Hom}(P,N) \rightarrow \textup{Hom}(P,O);\, \theta \mapsto \theta\varepsilon$ and
$\textup{Hom}(P,\varphi)\textup{Hom}(P,\gamma) = \textup{Hom}(P,\varphi)\textup{Hom}(P,\varepsilon)$. 
From the functoriality axioms (ref. definition \ref{def:functor} of a functor) it follows that $\textup{Hom}(P,\varphi\gamma) = \textup{Hom}(P,\varphi\varepsilon)$. This implies
\begin{equation}\label{eqn:Hom_functoriality}\rho(\varphi\gamma) = \rho(\varphi\varepsilon)\, \forall \rho \in \textup{Hom}(P,M)\end{equation}. 

\begin{tikzcd}
M \arrow[r, "\varphi", shift right, two heads] & N \arrow[r, "\gamma"'] \arrow[r, shift left=2, "\varepsilon"] & O \\
	& P \arrow[lu, "\rho"] \arrow[u, "\theta"] \arrow[ru, outer sep=2, pos=.55, "\rho(\varphi\gamma) = \rho(\varphi\varepsilon)"'] \arrow[ru, shift right=2]
\end{tikzcd}

We want to show that the parallel morphisms $\textup{Hom}(P,\gamma)$ and $\textup{Hom}(P,\varepsilon)$ are the same, i.e. for all
$\theta \in \textup{Hom}(P,N), \theta\gamma = \theta\varepsilon$. Our assumtion that there exists a projective lift helps us in this situation:
$\forall \theta \in \textup{Hom}(P,N)\, \exists\, \rho \in \textup{Hom}(P,M)$ such that $\theta = \rho\varphi$ and therefore with the above 
equation \eqref{eqn:Hom_functoriality},
$\theta\gamma = (\rho\varphi)\gamma = \rho(\varphi\gamma) = \rho(\varphi\varepsilon) = (\rho\varphi)\varepsilon = \theta\varepsilon$
and therefore $\textup{Hom}(P,\gamma) = \textup{Hom}(P,\varepsilon)$, i.e. $\textup{Hom}(P,\varphi)$ is epi.\\

\noindent\ul{''$\Rightarrow$'':} Let $\textup{Hom}(P,-)$ be right exact. Let $M, N \in \mathcal{C}_{0}$, the morphism
$\varphi : M \twoheadrightarrow N$ be an epi and $\theta : P \rightarrow N$ any morphism.
We want to show the existence of a morphism $\psi : P \dottedrightarrow\, M$ such that $\theta = \psi\varphi$.
With $\textup{Hom}(P,-)$ being exact, we have that $\textup{Hom}(P,\varphi) : \textup{Hom}(P,M) \twoheadrightarrow \textup{Hom}(P,N)$ is
an epi, and is given by $\textup{Hom}(P,M) \ni \rho \mapsto \rho\varphi \in \textup{Hom}(P,N)$.
\begin{itemize}
\item \ul{Case 1}: $\mathcal{C}$ is locally small, i.e. for the two objects $P, N \in \mathcal{C}_{0},$ there is a \ul{set} $\textup{Hom}(P,N)$
of morphisms between them. The $\textup{Hom}$-functor moves the morphisms from a general categorical context 
in $\mathcal{C}$ into the category of sets, i.e. $\textup{Hom}(P,\varphi)$ is a function in the category of sets.
And for those it's true that every epimorphism is surjective. Thus $\forall \theta \in \textup{Hom}(P,N)\, \exists \rho \in \textup{Hom}(P,M)$ such
that $\theta = (\textup{Hom}(P,\varphi))(\rho) = \rho\varphi$. This $\rho$ is the projective lift $\psi := \rho$ we were looking for.
\item \ul{Case 2}: If $\mathcal{C}$ is not locally small, we have to argument from the epimorphism itself and can't use surjectivity.
In $\mathcal{C}^{\text{op}}$ the morphism $\textup{Hom}(\varphi,P) : \textup{Hom}(N,P) \hookrightarrow \textup{Hom}(M,P)$
becomes a monomorphism, i.e. for two parallel morphisms
$h : \textup{Hom}(O,P) \rightarrow \textup{Hom}(N,P)$, $k : \textup{Hom}(O,P) \rightarrow \textup{Hom}(N,P)$, 
$\textup{Hom}(O,P) \ni \rho \mapsto h\rho = k\rho \textup{Hom}(N,P)$ it follows $h = k$.
Let $\theta \in \textup{Hom}(P,N)$
\end{itemize}
\end{proof}
\end{lemma}

\begin{definition}{(Projective object)}\label{def:proj_object}\\
An object $P$ in a category $\mathcal{C}$ that satisfies one (and thus both) of the equivalent properties in Lemma
 \ref{la:Hom_exact_proj_Lift_along_epis} is called a \ul{projective object}.
\end{definition}

\begin{lemma}{(dual to Lemma \ref{la:Hom_exact_proj_Lift_along_epis})}\label{la:dual_Hom_exact_proj_colift}
Let $\mathcal{C}$ be a category. For an object $P \in \mathcal{C}_{0}$ the following are equivalent:
\begin{itemize}
\item The contravariant functor $\textup{Hom}(-,P)$ is exact.
\item For all monomorphisms $\varphi : M \hookleftarrow N$ and morphisms $\theta : P \leftarrow N$, there exists a
projective colift $\psi : P \dottedleftarrow M$ such that $\theta = \varphi\psi$.\\
\begin{tikzcd}
M \arrow[rd, "\psi"', dotted] & N \arrow[l, "\varphi", hook] \arrow[d, "=\,\varphi\psi", "\theta"'] \\
	& P 
\end{tikzcd}
\end{itemize}
\begin{proof}
For an object $L \in \mathcal{C}_{0}, \textup{Hom}(-,L)$ is always a contravariant left-exact functor, i.e. respects monos.\\
\ul{''$\Leftarrow$'' :} Prove that $\textup{Hom}(-,P)$  is right exact, i.e. respects epis.

\end{proof}
\end{lemma}

\begin{definition}{(Yoneda projective, CatReps version)}\label{la:yoneda_projective}\endnote{(ref. Chapter 4 in \cite{[CategoryAlgebras_Webb]}; I first
want to understand Yoneda projectives in the context of CatReps, and then follow up with the general case)}\\
Consider the category of functors $\textup{Hom}( \mathcal{A}, \Bbbk\textup{-Mat} )$ from a $\Bbbk$-Algebroid $\mathcal{A}$ to the matrix category
over the same field $\Bbbk$, where the objects are representations of the algebroid by matrices, i.e. functors between the two categories, and
the morphisms are natural transformations between these functors.\\
\noindent Given an object $o \in \mathcal{A}$, the \ul{Yoneda projective} is the submodule of the category algebra consisting of all arrows
starting at $o$.
\end{definition}

Given that Yoneda projectives are objects, we can ask questions about their properties as objects in a category. On the other hand, they are
functors in the functor category, so addressing them as functors we can ask our four questions \ref{four_functor_questions} 
How does it work on objects? How does it work on morphisms? Why does it respect composition? Why does it respect identity morphisms?

If we want to check the property of a projective object for a yoneda projective $Y \in \textup{CatReps}_{0}$, we have to consider the
hom-sets $\textup{Hom}(Y,-)$, i.e. how exactly do the morphisms $\eta \in \textup{CatReps}_{1}$ work?


\begin{lemma}
Yoneda projectives are projective objects.
\begin{proof}
\end{proof}
\end{lemma}

\[
\begin{pmatrix}1 \ampersand 2 \ampersand 3 \ampersand 4\end{pmatrix}
\begin{pmatrix} 1\to5, \ampersand 2\to6 \\ 3\to7, \ampersand 4\to8 \end{pmatrix}
\begin{pmatrix}6 \ampersand 8\end{pmatrix}
\begin{pmatrix}5\to9\\6\to10\\7\to11\\8\to12\end{pmatrix}
\begin{pmatrix}9\ampersand10\end{pmatrix}\begin{pmatrix}11\ampersand12\end{pmatrix}
\begin{pmatrix}9\to13, \ampersand 10\to14\\11\to15 \ampersand 12\to16\end{pmatrix}
\textup{id}
\]

\subsection{Yoneda Projective}

Consider the concrete category 
\[
\begin{tikzcd}
{\{1,2,3,4\}} \arrow["{(1,2,3,4)}"', loop, distance=2em, in=125, out=55] \arrow[rr, "{\begin{pmatrix} 1\to5, 2\to6, \\ 3\to7, 4\to8 \end{pmatrix}}"] \arrow[dd] \arrow[rrdd] &  & {\{5,6,7,8\}} \arrow["{(6,8)}"', loop, distance=2em, in=125, out=55] \arrow[dd, " \begin{pmatrix}5\to9\\6\to10\\7\to11\\8\to12\end{pmatrix}"', bend left] \arrow[lldd] \\
                                                                                                                                                                             &  &                                                                                                                                                                        \\
{\{13,14,15,16\}} \arrow["\textup{id}"', loop, distance=2em, in=305, out=235]                                                                                                &  & {\{9,10,11,12\}} \arrow["{(9,10)(11,12)}"', loop, distance=2em, in=305, out=235] \arrow[ll, "{\begin{pmatrix}9\to13, 10\to14,\\ 11\to 15, 12\to16\end{pmatrix}}"]     
\end{tikzcd}
\]

and its $\mathbb{K}$-Algebroid $\textup{kq}$

\[
\begin{tikzcd}
1 \arrow["a"', loop, distance=2em, in=125, out=55] \arrow[rrr, "b"] \arrow[ddd, "d"', bend right] \arrow[rrrddd, "c", pos=0.25] &  &  & 2 \arrow["e"', loop, distance=2em, in=125, out=55] \arrow[ddd, "f", bend left] \arrow[lllddd, "g", pos=0.25] \\
                                                                                                                      &  &  &                                                                                                     \\
                                                                                                                      &  &  &                                                                                                     \\
4 \arrow["j"', loop, distance=2em, in=215, out=145]                                                                   &  &  & 3 \arrow["h"', loop, distance=2em, in=35, out=325] \arrow[lll, "i"]                                
\end{tikzcd}
\]

together with the relations

\[
[a^{4} - (1), e^{2} - (2), h^{2} - (3), j^{1} - (4), bf - c, bef-ach, bg-d, ci-d, achi-beg, a^{3}beg-chi, fi-g]
\]

The resulting category algebra has dimension 43.

We can look at the submodule of the category algebra consisting of all arrows starting at \texttt{kq.1}.
This is what the function \texttt{YonedaProjective( CatReps, kq.1 )} gives us:

\texttt{
proj1 := YonedaProjective( CatReps, kq.1 );
<(1)->4, (2)->8, (3)->8, (4)->8; (a)->4x4, (b)->4x8, (c)->4x8,
(d)->4x8, (e)->8x8, (f)->8x8, (g)->8x8, (h)->8x8, (i)->8x8, (4)->8x8>
}

The number 4 associated with object (1) tells us that the submodule of all arrows starting and ending at (1) has dimension 4.
Its basis is the set of paths $\{a, a^{2}, a^{3}, a^{4} = (1) \}$.

Likewise in

\texttt{
proj4 := YonedaProjective( CatReps, kq.4 );
<(1)->0, (2)->0, (3)->0, (4)->1; (a)->0x0, (b)->0x0, (c)->0x0,
(d)->0x1, (e)->0x0, (f)->0x0, (g)->0x1, (h)->0x0, (i)->0x1, (4)->1x1>
}

The submodule of all arrows starting at (4) is only of dimension 1, since it's already the identity arrow $\{j = (4)\}$.

Dimension of the (quotient of the) path algebra is 43.
Sum of all dimensions of the yoneda projectives on each objects is 43.

\begin{definition}{(Projective object)}

\end{definition}

\begin{definition}{(Yoneda projective)}
Yoneda's projective representation given by the object $o$ is the submodule of the category algebra
consisting of all arrows starting at $o$.
\end{definition}

\begin{thm}
Yoneda projectives are projective objects.
\begin{proof}
TODO
\end{proof}
\end{thm}

Conjecture: 

Dimension of the path algebra = Sum of dimensions of the yoneda projectives on each object.

What does the yoneda projective mean???

Function that creates examples for concrete categories so that I can check my conjecture.

Proof: Follows from Yoneda

Bilder der Yoneda-Einbettung sind projektive Objekte in der Funktor-Kategorie. Das sind die YonedaProjectives.