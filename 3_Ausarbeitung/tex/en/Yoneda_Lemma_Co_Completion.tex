
\subsection{The category of presheaves}

\begin{definition}{(The category of presheaves)}\endnote{(cited from ncatlab \cite{[ncatlab_presheaves]})}\\
For $\mathcal{C}$ a small category, its \ul{category of presheaves} is the functor category
\[ \mathrm{PSh}(\mathcal{C}) := \mathrm{Hom}(\mathcal{C}^{\text{op}}, \Set) \]
from the opposite category of $\mathcal{C}$ to $\Set$.
An object in this category is a \ul{presheaf}.\\
\noindent Taking $\mathcal{C}^{\text{op}}$ instead of $\mathcal{C}$ (and with $(\mathcal{C}^{\text{op}})^{\text{op}} = \mathcal{C}$)
we get the functor category as in \ref{def:functor_category}
\[
\mathrm{Hom}(\mathcal{C}, \Set) = \mathrm{PSh}(\mathcal{C}^{\text{op}})
\]
\end{definition}

\begin{remark}{(General properties of presheaves)}\\
The category of presheaves $\mathrm{PSh}(\mathcal{C})$ is called the \ul{free cocompletion} of $\mathcal{C}$.
\end{remark}

\begin{definition}{(Representable functor)}\label{def:repres_functor}\endnote{(from ncatlab \ref{[ncatlab_repres_functor]})}
\begin{enumerate}
\renewcommand{\labelenumi}{(\theenumi)}
\item A functor from a locally small category $\mathcal{C}$ to $\Set$ is \ul{representable} if there is an object $c \in \mathcal{C}$ and a
natural isomorphism between $F$ and $\mathrm{Hom}(c,-)$ (for a covariant $F$, otherwise $\mathrm{Hom}(-,c)$ for a contravariant $F$),
in which case one says thet the functor $F$ is \ul{represented by} the object $c$.
\item A \ul{representation} for a functor $F$ is a choice of object $c \in \mathcal{C}$ together with a specified natural isomorphism
$\mathrm{Hom}(c,-) \cong F$ (for a covariant $F$, or $\mathrm{Hom}(-,c) \cong F$ for a contravariant $F$).
\end{enumerate}
\end{definition}

\begin{lemma}{(Yoneda's Lemma)}
For any functor $F : \mathcal{C} \rightarrow \mathrm{Set}$, whose source $\mathcal{C}$ is locally small and any
object $c \in \mathcal{C}_{0}$, there is a bijection
\[
\mathrm{Hom}(\mathrm{Hom}_{\mathcal{C}}(c,-), F) \cong Fc
\]
that associates a natural transformation $\alpha : \mathcal{C}(c,-) \Rightarrow F$ to the element $\alpha_{c}(1_{c}) \in Fc$.
Moreover, this correspondence is natural in both $c$ and $F$.
\end{lemma}
As $\mathcal{C}$ is locally small but not necessarily small, a priori the collection of natural transformations
$\mathrm{Hom}(\mathrm{Hom}_{\mathcal{C}}(c,-),F)$ might be large. However, the bijection in the Yoneda lemma proves that this particular
collection of natural transformations indeed forms a set.
\begin{proof}{of the bijection}

\end{proof}
\begin{proof}{of naturality}

\end{proof}

\begin{definition}{(Yoneda embedding)}\label{def:yoneda_embedding}\endnote{(cited from ncatlab \cite{[ncatlab_yoneda_emb]})}\\
The \ul{Yoneda embedding} for a locally small category $\mathcal{C}$ is the functor
\[
Y : \mathcal{C} \hookrightarrow \mathrm{Hom}(\mathcal{C}^{\text{op}}, \mathrm{Set})
\]
from $\mathcal{C}$ to the category of presheaves over $\mathcal{C}$ which is the image of the hom-functor
\[
\mathrm{Hom} : \mathcal{C}^{\text{op}}\times\mathcal{C} \rightarrow \mathrm{Set}
\]
under the $\mathrm{Hom}$ adjunction
\[
\mathrm{Hom}(\mathcal{C}^{\text{op}}\times\mathcal{C}, \mathrm{Set}) \simeq
\mathrm{Hom}(\mathcal{C},\mathrm{Hom}(\mathcal{C}^{\text{op}}, \mathrm{Set}))
\]
in the closed symmetric monoidal category $\mathrm{Cat}$.
If instead we have the opposite category $\mathcal{C}^{\text{op}}$, then we get the embedding into the functor category:
\[
Y^{\text{op}} : \mathcal{C}^{\text{op}} \hookrightarrow \mathrm{Hom}(\mathcal{C},\mathrm{Set})
\]
\end{definition}

\begin{remark}[Our $\Bbbk$-linear version of Yoneda's embedding]
Let $\mathcal{A}$ be a $\Bbbk$-linear category with finite-dimensional $\Bbbk$-vector spaces as hom-sets. Then we get
$\Bbbk$-linear versions of Yoneda's lemma and Yondeda's embedding:

For any $\Bbbk$-linear functor $F : \mathcal{A} \rightarrow \kmat$ and any object $i \in \mathcal{A}_{0}$, there is a bijection
\[
\mathrm{Hom}_{\HomAkmat}(\mathrm{Hom}_{\mathcal{A}}(-,i), F) \cong F(i)
\]

\[
Y : \mathcal{A} \hookrightarrow \mathrm{Hom_{\Bbbk}}(\mathcal{A^{\text{op}}},\kmat)
\]

\[
Y^{\text{op}} : \mathcal{A}^{\text{op}} \hookrightarrow \HomAkmat
\]

\end{remark}

$\HomAkmat$ is Abelian and therefore finitely complete and finitely cocomplete.

\subsection{Projective objects and the Yoneda projective}

\begin{lemma}\label{la:Hom_exact_proj_Lift_along_epis}
Let $\mathcal{C}$ be a locally small category. For an object $P \in \mathcal{C}_{0}$ the following are equivalent:
\begin{itemize}
\item The covariant functor $\mathrm{Hom}(P,-)$ is exact.
\item For all epimorphisms $\varphi : M \twoheadrightarrow N$ and morphisms $\theta : P \rightarrow N$, there exists a
projective lift $\psi : P\dottedrightarrow M$ such that $\theta = \psi\varphi$.\\
\begin{tikzcd}
M \arrow[r, "\varphi", two heads] & N \\
	& P \arrow[u, "\theta", "=\,\psi\varphi"'] \arrow[lu, "\psi", dotted]
\end{tikzcd}
\end{itemize}
\begin{proof}
For an object $L\in \mathcal{C}_{0}$, $\mathrm{Hom}(L,-)$ is always a covariant left-exact functor, i.e. respects monos.\\
\setlist[description]{font=\normalfont}
\begin{description}
\item[``$\Leftarrow$:''] Prove that $\mathrm{Hom}(P,-)$  is right exact, i.e. respects epis.\\
For this, let $M, N \in \mathcal{C}_{0}$ and $\varphi : M \twoheadrightarrow N$ be an epi. The Hom-functor works on morphisms
by mapping the Hom-sets of the source and target objects of the morphism, i.e.
$\mathrm{Hom}(P,\varphi) : \mathrm{Hom}(P,M) \rightarrow \mathrm{Hom}(P,N)$, given by $\rho \mapsto \rho\varphi\, \forall \rho \in \mathrm{Hom}(P,M)$.
Now given that $\varphi$ is an epi, we want to show that $\mathrm{Hom}(P,\varphi)$ is also an epi.\\
Let $O \in \mathcal{C}_{0}$,  $\gamma : N \rightarrow O$ and $\varepsilon : N \rightarrow O$ such that
$\mathrm{Hom}(P,\gamma) : \mathrm{Hom}(P,N) \rightarrow \mathrm{Hom}(P,O);\, \theta \mapsto \theta\gamma$ and
$\mathrm{Hom}(P,\varepsilon) : \mathrm{Hom}(P,N) \rightarrow \mathrm{Hom}(P,O);\, \theta \mapsto \theta\varepsilon$ and
$\mathrm{Hom}(P,\varphi)\mathrm{Hom}(P,\gamma) = \mathrm{Hom}(P,\varphi)\mathrm{Hom}(P,\varepsilon)$. 
From the functoriality axioms (ref. definition \ref{def:functor} of a functor) it follows that $\mathrm{Hom}(P,\varphi\gamma) = \mathrm{Hom}(P,\varphi\varepsilon)$. This implies
\begin{equation}\label{eqn:Hom_functoriality}\rho(\varphi\gamma) = \rho(\varphi\varepsilon)\, \forall \rho \in \mathrm{Hom}(P,M)\end{equation}. 

\begin{tikzcd}
M \arrow[r, "\varphi", shift right, two heads] & N \arrow[r, "\gamma"'] \arrow[r, shift left=2, "\varepsilon"] & O \\
	& P \arrow[lu, "\rho"] \arrow[u, "\theta"] \arrow[ru, outer sep=2, pos=.55, "\rho(\varphi\gamma) = \rho(\varphi\varepsilon)"'] \arrow[ru, shift right=2]
\end{tikzcd}

We want to show that the parallel morphisms $\mathrm{Hom}(P,\gamma)$ and $\mathrm{Hom}(P,\varepsilon)$ are the same, i.e. for all
$\theta \in \mathrm{Hom}(P,N), \theta\gamma = \theta\varepsilon$. Our assumtion that there exists a projective lift helps us in this situation:
$\forall \theta \in \mathrm{Hom}(P,N)\, \exists\, \rho \in \mathrm{Hom}(P,M)$ such that $\theta = \rho\varphi$ and therefore with the above 
equation \eqref{eqn:Hom_functoriality},
$\theta\gamma = (\rho\varphi)\gamma = \rho(\varphi\gamma) = \rho(\varphi\varepsilon) = (\rho\varphi)\varepsilon = \theta\varepsilon$
and therefore $\mathrm{Hom}(P,\gamma) = \mathrm{Hom}(P,\varepsilon)$, i.e. $\mathrm{Hom}(P,\varphi)$ is epi.\\

\item[``$\Rightarrow$:''] Let $\mathrm{Hom}(P,-)$ be right exact. Let $M, N \in \mathcal{C}_{0}$, the morphism
$\varphi : M \twoheadrightarrow N$ be an epi and $\theta : P \rightarrow N$ any morphism.
We want to show the existence of a morphism $\psi : P \dottedrightarrow\, M$ such that $\theta = \psi\varphi$.
With $\mathrm{Hom}(P,-)$ being exact, we have that $\mathrm{Hom}(P,\varphi) : \mathrm{Hom}(P,M) \twoheadrightarrow \mathrm{Hom}(P,N)$ is
an epi, and is given by $\mathrm{Hom}(P,M) \ni \rho \mapsto \rho\varphi \in \mathrm{Hom}(P,N)$.\\
$\mathcal{C}$ is locally small, i.e. for the two objects $P, N \in \mathcal{C}_{0},$ there is a \ul{set} $\mathrm{Hom}(P,N)$
of morphisms between them. The $\mathrm{Hom}$-functor moves the morphisms from a general categorical context 
in $\mathcal{C}$ into the category of sets, i.e. $\mathrm{Hom}(P,\varphi)$ is a function in the category of sets.
And for those it's true that every epimorphism is surjective. Thus $\forall \theta \in \mathrm{Hom}(P,N)\, \exists \rho \in \mathrm{Hom}(P,M)$ such
that $\theta = (\mathrm{Hom}(P,\varphi))(\rho) = \rho\varphi$. This $\rho$ is the projective lift $\psi := \rho$ we were looking for.
\end{description}
\end{proof}
\end{lemma}

\begin{definition}{(Projective object)}\label{def:proj_object}\\
An object $P$ in a category $\mathcal{C}$ that satisfies one (and thus both) of the equivalent properties in Lemma
 \ref{la:Hom_exact_proj_Lift_along_epis} is called a \ul{projective object}.
\end{definition}

The dual statement to Lemma \ref{la:Hom_exact_proj_Lift_along_epis} is
\begin{lemma}\label{la:dual_Hom_exact_proj_colift}
Let $\mathcal{C}$ be a category. For an object $P \in \mathcal{C}_{0}$ the following are equivalent:
\begin{itemize}
\item The contravariant functor $\mathrm{Hom}(-,P)$ is exact.
\item For all monomorphisms $\varphi : M \hookleftarrow N$ and morphisms $\theta : P \leftarrow N$, there exists a
projective colift $\psi : P \dottedleftarrow M$ such that $\theta = \varphi\psi$.\\
\begin{tikzcd}
M \arrow[rd, "\psi"', dotted] & N \arrow[l, "\varphi", hook] \arrow[d, "=\,\varphi\psi", "\theta"'] \\
	& P 
\end{tikzcd}
\end{itemize}
\end{lemma}

\begin{definition}{(Injective object)}\label{def:inj_object}\\
An object $P$ in a category $\mathcal{C}$ that satisfies one (and thus both) of the equivalent properties in Lemma
 \ref{la:dual_Hom_exact_proj_colift} is called an \ul{injective object}.
\end{definition}

\begin{definition}{(Yoneda projective)}
Let $\mathcal{A}$ be a $\Bbbk$-algebroid with finitely many objects and finite-dimensional hom-sets over $\Bbbk$.
With the Yoneda embedding 
\[
Y^{\text{op}} : \begin{cases}\mathcal{A}^{\text{op}} \hookrightarrow \HomAkmat \\
i \mapsto \mathrm{Hom}_{\mathcal{A}}(-,i)
\end{cases}
\]
we see that the image
$Y^{\text{op}}(i)$ of the object $i \in \mathcal{A}^{\text{op}}$ is the representable functor $\mathrm{Hom}_{\mathcal{A}}(-,i)$ in $\HomAkmat$. 
It is called the \ul{$i$-th Yoneda projective}.
\end{definition}

Analogous to Prop. \ref{prop:Alg-Alg-Correspondence} one can easily prove that a $\Bbbk$-representation $F$ of $\mathcal{A}$ corresponds to a 
module $\bigoplus_{i \in \mathcal{A}_{0}} F(i)$ over the algebra $\mathbf{A} = \mathrm{Algebra}(\mathcal{A})$.
In particular the Yoneda projective $Y^{\text{op}}(i) = \mathrm{Hom}_{\mathcal{A}}(-,i)$ corresponds to the $\mathbf{A}$-module
$M_{i} = \bigoplus_{j \in \mathcal{A}_{0}} \mathrm{Hom}_{\mathcal{A}}(j,i)$.

\begin{lemma}
Yoneda projectives are projective objects in $\HomAkmat$.
\end{lemma}
\begin{proof}\phantom{}\\
We want to show that $\mathrm{Hom}_{\HomAkmat}(Y^{\text{op}}(i),-)$ is right exact, i.e. finitely cocontinuous.

\noindent For this, let
\[
\begin{tikzcd}
0 \arrow[r] & F_{1} \arrow[r, "\eta"] & F_{2} \arrow[r,"\rho"] & F_{3} \arrow[r] & 0
\end{tikzcd}
\]
be a short exact sequence in $\HomAkmat$, i.e.
\begin{enumerate}
\item $\eta$ is a monomorphism,
\item $\rho$ is an epimorphism and
\item the image of $\eta$ equals the kernel of $\rho$.
\end{enumerate}

\noindent Then applying the functor $\mathrm{Hom}_{\HomAkmat}(Y^{\text{op}}(i),-)$ on the sequence and simplifying with Yoneda's lemma

\begin{align*}
\mathrm{Hom}_{\HomAkmat}(\mathrm{Hom}_{\mathcal{A}}(-,i),F) &\simeq F(i) \\
\mathrm{Hom}_{\HomAkmat}(\mathrm{Hom}_{\mathcal{A}}(-,i),\rho) &\simeq \rho_{i}
\end{align*}

We only have to measure exactness on the components
\[
\begin{tikzcd}
0 \arrow[r] & F_{1}(i) \arrow[r, "\eta_{i}"] &
F_{2}(i) \arrow[r,"\rho_{i}"] & F_{3}(i) \arrow[r] & 0\mbox{,}
\end{tikzcd}
\]
but this coincides with the definition of the exactness of functors. So the proof is a tautology by Yoneda's lemma.
\end{proof}


\subsection{Enough Projective objects (constructively)}

\begin{definition}{(Sufficiently many projectives)}\label{def:enough_projectives}
A category $\mathcal{C}$ is said to have \ul{sufficiently many} (or \ul{enough}) \ul{projectives} if for any object $A \in \mathcal{C}_{0}$
there exists an epimorphism $P \twoheadrightarrow A$, where $P$ is projective.
\end{definition}

We state without a proof the following fact about our functor category:

\begin{theorem}{($\HomAkmat$ has enough projectives)}\\
Let $\mathcal{A}$ be a finite-dimensional algebroid over some field $\Bbbk$. The functor category $\HomAkmat$ has sufficiently many
projectives.
\end{theorem}
\begin{proof}
(no proof)
\end{proof}