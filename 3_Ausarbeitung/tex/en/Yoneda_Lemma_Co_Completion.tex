% mainfile: ../main.tex

%% What does completion and cocompletion mean?
%% What does embedding mean?
\subsection{The category of presheaves}

\begin{definition}{(The category of presheaves)}\endnote{(cited from ncatlab \cite{[ncatlab_presheaves]})}\\
For $\mathcal{C}$ a small category, its \ul{category of presheaves} is the functor category
\[ \textup{PSh}(\mathcal{C}) := \textup{Hom}(\mathcal{C}^{\text{op}}, \textup{Set}) \]
from the opposite category of $\mathcal{C}$ to $\textup{Set}$.
An object in this category is a \ul{presheaf}.\\
\noindent Taking $\mathcal{C}^{\text{op}}$ instead of $\mathcal{C}$ (and with $(\mathcal{C}^{\text{op}})^{\text{op}} = \mathcal{C}$) we get the
functor category as in \ref{def:functor_category}
\[
\textup{Hom}(\mathcal{C}, \textup{Set}) = \textup{PSh}(\mathcal{C}^{\text{op}}) \]
\end{definition}

\begin{remark}{(General properties of presheaves)}\\
The category of presheaves $\textup{PSh}(\mathcal{C})$ is the \ul{free cocompletion} of $\mathcal{C}$.
\end{remark}

\begin{definition}{(Representable functor)}\label{def:repres_functor}\endnote{(from ncatlab \ref{[ncatlab_repres_functor]})}
\begin{enumerate}
\renewcommand{\labelenumi}{(\theenumi)}
\item A functor from a locally small category $\mathcal{C}$ to $\textup{Hom}$ is \ul{representable} if there is an object $c \in \mathcal{C}$ and a
natural isomorphism between $F$ and $\textup{Hom}(c,-)$ (for a covariant $F$, otherwise $\textup{Hom}(-,c)$ for a contravariant $F$), in which case
one says thet the functor $F$ is \ul{represented by} the object $c$.
\item A \ul{representation} for a functor $F$ is a choice of object $c \in \mathcal{C}$ together with a specified natural isomorphism
$\textup{Hom}(c,-) \cong F$ (for a covariant $F$, or $\textup{Hom}(-,c) \cong F$ for a contravariant $F$).
\end{enumerate}
\end{definition}

\begin{lemma}{(Yoneda's Lemma)}
Let $\mathcal{C}$ be a locally small category. For $X \in [\mathcal{C}^{\text{op}}, \textup{Set}]$ any presheaf, there is a canonical isomorphism
\[
\textup{Hom}_{[\mathcal{C}^{\text{op}}, \textup{Set}]}(y(c), X) \cong X(c)
\]
between the hom-set of presheaf homomorphisms from the representable presheaf $y(c)$ to $X$, and the value of $X$ at $c$.
\end{lemma}

\begin{lemma}{(Yoneda's Lemma)}
For any functor $F : \mathcal{C} \rightarrow \textup{Set}$, whose source $\mathcal{C}$ is locally small and any
object $c \in \mathcal{C}$, there is a bijection
\[
\textup{Hom}(\mathcal{C}(c,-), F) \cong Fc
\]
that associates a natural transformation $\alpha : \mathcal{C}(c,-) \Rightarrow F$ to the element $\alpha_{c}(1_{c}) \in Fc$.
Moreover, this correspondence is natural in both $c$ and $F$.
\end{lemma}
As $\mathcal{C}$ is locally small but not necessarily small, a priori the collection of natural transformations
$\textup{Hom}(\mathcal{C}(c,-),F)$ might be large. However, the bijection in the Yoneda lemma proves that this particular
collection of natural transformations indeed forms a set.
\begin{proof}{of the bijection}

\end{proof}
\begin{proof}{of naturality}

\end{proof}

What is the forgetful functor $U : \mathbf{Cat} \rightarrow \mathbf{Quiv}$ represented by? Is it even representable?
A representation for $U$ is an object $c \in \mathcal{C}_{0}$, together with an isomorphism $\mathcal{C}(c,-) \cong U$.
Target needs to be $\textup{Set}$.


\begin{definition}{(Yoneda Embedding)}\label{def_func:yoneda_embedding}\endnote{(quoted directly from the \texttt{FunctorCategories} \cite{[FunctorCat]} documentation}\\
$\mathtt{YonedaEmbedding(B)}$\\
Returns: A \textsc{CAP} functor.

\noindent The input is an algebroid $\mathcal{B}$ defined by some quiver $\Bbbk$-algebra $\mathcal{A}$. The output is the Yoneda embedding functor
from $\mathcal{B}$ into the functors category $\textup{Hom}( \mathtt{AlgebroidOverOppositeAlgebra}(\mathcal{B}), \mathcal{C} )$, where
$\mathcal{C}$ is a matrix category over $\Bbbk$.
\end{definition}

\begin{definition}{(Yoneda embedding)}\label{def:yoneda_embedding}\endnote{(cited from ncatlab \cite{[ncatlab_yoneda_emb]})}\\
The \ul{Yoneda embedding} for a locally small category $\mathcal{C}$ is the functor
\[
Y : \mathcal{C} \hookrightarrow \textup{Hom}(\mathcal{C}^{\text{op}}, \textup{Set})
\]
from $\mathcal{C}$ to the category of presheaves over $\mathcal{C}$ which is the image of the hom-functor
\[
\textup{Hom} : \mathcal{C}^{\text{op}}\times\mathcal{C} \rightarrow \textup{Set}
\]
under the $\textup{Hom}$ adjunction
\[
\textup{Hom}(\mathcal{C}^{\text{op}}\times\mathcal{C}, \textup{Set}) \simeq
\textup{Hom}(\mathcal{C},\textup{Hom}(\mathcal{C}^{\text{op}}, \textup{Set}))
\]
in the closed symmetric monoidal category $\textup{Cat}$.
If instead we have the opposite category $\mathcal{C}^{\text{op}}$, then we get the embedding into the functor category:
\[
Y : \mathcal{C}^{\text{op}} \hookrightarrow \textup{Hom}(\mathcal{C},\textup{Set})
\]
\end{definition}

\begin{remark}
Let $\mathcal{A}$ be a $\Bbbk$-linear category with finite $\Bbbk$-dimensional hom-sets. Then we get
a $\Bbbk$-linear version of Yoneda's lemma:

\[
Y : \mathcal{A} \hookrightarrow \mathrm{Hom_{\Bbbk}}(\mathcal{A^{\text{op}}},\kmat)
\]

\[
Y : \mathcal{A}^{\text{op}} \hookrightarrow \HomAkmat
\]

\end{remark}

\subsection{Projective objects and the Yoneda projective}

\begin{lemma}\label{la:Hom_exact_proj_Lift_along_epis}
Let $\mathcal{C}$ be a locally small category. For an object $P \in \mathcal{C}_{0}$ the following are equivalent:
\begin{itemize}
\item The covariant functor $\textup{Hom}(P,-)$ is exact.
\item For all epimorphisms $\varphi : M \twoheadrightarrow N$ and morphisms $\theta : P \rightarrow N$, there exists a
projective lift $\psi : P\,\dottedrightarrow\,M$ such that $\theta = \psi\varphi$.\\
\begin{tikzcd}
M \arrow[r, "\varphi", two heads] & N \\
	& P \arrow[u, "\theta", "=\,\psi\varphi"'] \arrow[lu, "\psi", dotted]
\end{tikzcd}
\end{itemize}
\begin{proof}
For an object $L\in \mathcal{C}_{0}$, $\textup{Hom}(L,-)$ is always a covariant left-exact functor, i.e. respects monos.\\
\ul{''$\Leftarrow$'' :} Prove that $\textup{Hom}(P,-)$  is right exact, i.e. respects epis.\\
For this, let $M, N \in \mathcal{C}_{0}$ and $\varphi : M \twoheadrightarrow N$ be an epi. The Hom-functor works on morphisms
by mapping the Hom-sets of the source and target objects of the morphism, i.e.
$\textup{Hom}(P,\varphi) : \textup{Hom}(P,M) \rightarrow \textup{Hom}(P,N)$, given by $\rho \mapsto \rho\varphi\, \forall \rho \in \textup{Hom}(P,M)$.
Now given that $\varphi$ is an epi, we want to show that $\textup{Hom}(P,\varphi)$ is also an epi.\\
Let $O \in \mathcal{C}_{0}$,  $\gamma : N \rightarrow O$ and $\varepsilon : N \rightarrow O$ such that
$\textup{Hom}(P,\gamma) : \textup{Hom}(P,N) \rightarrow \textup{Hom}(P,O);\, \theta \mapsto \theta\gamma$ and
$\textup{Hom}(P,\varepsilon) : \textup{Hom}(P,N) \rightarrow \textup{Hom}(P,O);\, \theta \mapsto \theta\varepsilon$ and
$\textup{Hom}(P,\varphi)\textup{Hom}(P,\gamma) = \textup{Hom}(P,\varphi)\textup{Hom}(P,\varepsilon)$. 
From the functoriality axioms (ref. definition \ref{def:functor} of a functor) it follows that $\textup{Hom}(P,\varphi\gamma) = \textup{Hom}(P,\varphi\varepsilon)$. This implies
\begin{equation}\label{eqn:Hom_functoriality}\rho(\varphi\gamma) = \rho(\varphi\varepsilon)\, \forall \rho \in \textup{Hom}(P,M)\end{equation}. 

\begin{tikzcd}
M \arrow[r, "\varphi", shift right, two heads] & N \arrow[r, "\gamma"'] \arrow[r, shift left=2, "\varepsilon"] & O \\
	& P \arrow[lu, "\rho"] \arrow[u, "\theta"] \arrow[ru, outer sep=2, pos=.55, "\rho(\varphi\gamma) = \rho(\varphi\varepsilon)"'] \arrow[ru, shift right=2]
\end{tikzcd}

We want to show that the parallel morphisms $\textup{Hom}(P,\gamma)$ and $\textup{Hom}(P,\varepsilon)$ are the same, i.e. for all
$\theta \in \textup{Hom}(P,N), \theta\gamma = \theta\varepsilon$. Our assumtion that there exists a projective lift helps us in this situation:
$\forall \theta \in \textup{Hom}(P,N)\, \exists\, \rho \in \textup{Hom}(P,M)$ such that $\theta = \rho\varphi$ and therefore with the above 
equation \eqref{eqn:Hom_functoriality},
$\theta\gamma = (\rho\varphi)\gamma = \rho(\varphi\gamma) = \rho(\varphi\varepsilon) = (\rho\varphi)\varepsilon = \theta\varepsilon$
and therefore $\textup{Hom}(P,\gamma) = \textup{Hom}(P,\varepsilon)$, i.e. $\textup{Hom}(P,\varphi)$ is epi.\\

\noindent\ul{''$\Rightarrow$'':} Let $\textup{Hom}(P,-)$ be right exact. Let $M, N \in \mathcal{C}_{0}$, the morphism
$\varphi : M \twoheadrightarrow N$ be an epi and $\theta : P \rightarrow N$ any morphism.
We want to show the existence of a morphism $\psi : P \dottedrightarrow\, M$ such that $\theta = \psi\varphi$.
With $\textup{Hom}(P,-)$ being exact, we have that $\textup{Hom}(P,\varphi) : \textup{Hom}(P,M) \twoheadrightarrow \textup{Hom}(P,N)$ is
an epi, and is given by $\textup{Hom}(P,M) \ni \rho \mapsto \rho\varphi \in \textup{Hom}(P,N)$.\\
$\mathcal{C}$ is locally small, i.e. for the two objects $P, N \in \mathcal{C}_{0},$ there is a \ul{set} $\textup{Hom}(P,N)$
of morphisms between them. The $\textup{Hom}$-functor moves the morphisms from a general categorical context 
in $\mathcal{C}$ into the category of sets, i.e. $\textup{Hom}(P,\varphi)$ is a function in the category of sets.
And for those it's true that every epimorphism is surjective. Thus $\forall \theta \in \textup{Hom}(P,N)\, \exists \rho \in \textup{Hom}(P,M)$ such
that $\theta = (\textup{Hom}(P,\varphi))(\rho) = \rho\varphi$. This $\rho$ is the projective lift $\psi := \rho$ we were looking for.
\end{proof}
\end{lemma}

\begin{definition}{(Projective object)}\label{def:proj_object}\\
An object $P$ in a category $\mathcal{C}$ that satisfies one (and thus both) of the equivalent properties in Lemma
 \ref{la:Hom_exact_proj_Lift_along_epis} is called a \ul{projective object}.
\end{definition}

\begin{lemma}{(dual to Lemma \ref{la:Hom_exact_proj_Lift_along_epis})}\label{la:dual_Hom_exact_proj_colift}
Let $\mathcal{C}$ be a category. For an object $P \in \mathcal{C}_{0}$ the following are equivalent:
\begin{itemize}
\item The contravariant functor $\textup{Hom}(-,P)$ is exact.
\item For all monomorphisms $\varphi : M \hookleftarrow N$ and morphisms $\theta : P \leftarrow N$, there exists a
projective colift $\psi : P \dottedleftarrow M$ such that $\theta = \varphi\psi$.\\
\begin{tikzcd}
M \arrow[rd, "\psi"', dotted] & N \arrow[l, "\varphi", hook] \arrow[d, "=\,\varphi\psi", "\theta"'] \\
	& P 
\end{tikzcd}
\end{itemize}
\begin{proof}
For an object $L \in \mathcal{C}_{0}, \textup{Hom}(-,L)$ is always a contravariant left-exact functor, i.e. respects monos.\\
\ul{''$\Leftarrow$'' :} Prove that $\textup{Hom}(-,P)$  is right exact, i.e. respects epis.

\end{proof}
\end{lemma}

\begin{definition}{(Injective object)}\label{def:inj_object}\\
An object $P$ in a category $\mathcal{C}$ that satisfies one (and thus both) of the equivalent properties in Lemma
 \ref{la:dual_Hom_exact_proj_colift} is called an \ul{injective object}.
\end{definition}

\begin{definition}{(Yoneda representation)}
$F$
\end{definition}

\begin{definition}{(Yoneda projective, CatReps version)}\label{la:yoneda_projective}\endnote{(ref. Chapter 4 in \cite{[CategoryAlgebras_Webb]}; I first
want to understand Yoneda projectives in the context of CatReps, and then follow up with the general case)}\\
Consider the category of functors $\textup{Hom}( \mathcal{A}, \Bbbk\textup{-Mat} )$ from a $\Bbbk$-Algebroid $\mathcal{A}$ to the matrix category
over the same field $\Bbbk$, where the objects are representations of the algebroid by matrices, i.e. functors between the two categories, and
the morphisms are natural transformations between these functors.\\
\noindent Given an object $o \in \mathcal{A}$, the \ul{Yoneda projective} is the submodule of the category algebra consisting of all arrows
starting at $o$.
\end{definition}

Given that Yoneda projectives are objects, we can ask questions about their properties as objects in a category. On the other hand, they are
functors in the functor category, so addressing them as functors we can ask our four questions \ref{four_functor_questions} 
How does it work on objects? How does it work on morphisms? Why does it respect composition? Why does it respect identity morphisms?

If we want to check the property of a projective object for a yoneda projective $Y \in \textup{CatReps}_{0}$, we have to consider the
hom-sets $\textup{Hom}(Y,-)$, i.e. how exactly do the morphisms $\eta \in \textup{CatReps}_{1}$ work?



\begin{lemma}
Yoneda projectives are projective objects.
\begin{proof}
The category $\textup{CatReps} := \textup{Hom}(\Bbbk \mathbf{q}, \Bbbk\mathbf{Mat})$ is locally small.
Let $o \in \Bbbk \mathbf{q}_{0}$ be any object, and $P \in \textup{CatReps}_{0}$ be the yoneda representation of $o$ in $\textup{CatReps}$.
Let $M, N \in \textup{CatReps}_{0}$, and $\textup{CatReps}_{1} \ni \eta : M \twoheadrightarrow N$ be an epimorphism.
A morphism $\theta : P \rightarrow N$ is a natural transformation between the yoneda representation $P$ and another functor $N$.
\end{proof}
\end{lemma}


Direct summands and direct sums of projectives are projective.

Yoneda projectives, projective indecomposables -> every projective object is a direct sum of projective indecomposables

projective indecomposables for a given concrete category

\begin{definition}{(Sufficiently many projectives)}\label{def:enough_projectives}\endnote{(ref. \cite{[sciencedirect_proj_objects]})}
A category $\mathcal{C}$ is said to have \ul{sufficiently many} (or \ul{enough}) \ul{projectives} if for any object $A \in \mathcal{C}_{0}$
there exists an epimorphism $P \twoheadrightarrow A$, where $P$ is projective.
\end{definition}





\begin{definition}{(Projective object)}

\end{definition}

\begin{theorem}
Yoneda projectives are projective objects.
\begin{proof}
TODO
\end{proof}
\end{theorem}


\subsection{Enough Projective objects (constructively)}

\begin{theorem}{($\HomAkmat$ has enough projectives)}
Let $\mathcal{A}$ be a finite-dimensional algebroid over some field $\Bbbk$. The functor category $\HomAkmat$ has sufficiently many
projectives (ref. \ref{def:enough_projectives}).
\end{theorem}
\begin{proof}
(no proof)
Let $F \in \HomAkmat_{0}$. We want to find a projective object $P \in \HomAkmat_{0}$ and an epimorphism $\eta: P \twoheadrightarrow F$.
\end{proof}