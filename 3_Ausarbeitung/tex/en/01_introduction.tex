% mainfile: ../main.tex

\section{Introduction}

\subsection{Purpose of this thesis}

Die Aufgabe der vorliegenden Arbeit besteht darin, das GAP-Paket "catreps" von Peter Webb mit CAP zu re-organisieren. Dabei werden die Algorithmen, welche in catreps direkt implementiert sind, soweit wie möglich durch vorhandene Methoden von CAP ersetzt.
Insbesondere wird das CAP-Paket FunctorCategories Anwendung finden, weil ich zeigen werde, dass catreps, also die Kategorie der Darstellungen einer konkreten endlichen Kategorie, nichts anderes ist als eine Unterkategorie von FunctorCategories, also der Kategorie aller Funktoren zwischen Kategorien. 
Da catreps selbst bereits eine Verallgemeinerung der Darstellung endlicher Gruppen ist (eine Gruppe ist nichts anderes als eine Kategorie mit einem Objekt, in dem jeder Morphismus ein Isomorphismus ist), stellt FunctorCategories wohl den allgemeinsten Rahmen dar, den man sich vorstellen kann.

In this thesis I will define what a category is, then I go further in the doctrine of enriched categories, especially monoidal categories.
The morphisms in the category of categories are the functors between categories. I will treat the functor category where the functors themselves are
objects and natural transformations the morphisms between them. I will show that any representation of a category (and thus any representation
of a group) is a functor, so the category of representations (of a category) is a subcategory of the functor category.
I will show how the monoidal structure of the category of representations arises from the counit and the comultiplication on the Bialgebroid structure
on the category.

\noindent Throughout the thesis I will give proofs of existence by providing an algorithm that computes the object that exists. I will be using CAP, the gap package
developed by Sebastian Gutsche et al. Another purpose of this thesis is the translation of the work of Peter Webb, who used gap directly, into our CAP
framework. This includes his decomposition algorithm for a representation. As the category of representations is just a subcategory of the functor category,
most of the work will be done inside the package FunctorCategories by Prof. Mohamed Barakat. In this thesis I will also write the documentation for the
package FunctorCategories.
