
We want to distinguish between different objects $F$, $G$ in our functor category $\HomAkmat$. A good method for this is by comparing the output of a
binary morphism\\
$\varphi( -, - ) : \HomAkmat \times \HomAkmat \rightarrow \HomAkmat$ with the first input being the same constant functor,
e.g. the tensor unit $\mathtt{const} : \mathcal{A}_{0} \rightarrow \{1\}; \mathcal{A}_{1} \rightarrow \{1\}^{1\times 1}$.
Then if $\varphi( \mathtt{const}, F )$ is different from $\varphi( \mathtt{const}, G )$, we can be sure that $F$ and $G$ are categorically different objects.

We are defining such a morphism $\varphi$ in three steps.

\subsection{The algorithm $\mathtt{MorphismOntoSumOfImagesOfAllMorphisms}$}

In an Abelian category the hom-set $\mathtt{hom} := \mathtt{BasisOfExternalHom}( M, N )$ of morphisms from $M$ to $N$ is a family of
morphisms with same source $M$ and same target $N$.
\[
\begin{tikzcd}[
  row sep=0.5ex,
]
M \arrow[r, "b_{1}"] & N \\
M \arrow[r, "b_{2}"] & N \\
M \arrow[r, "\dots"] & N
\end{tikzcd}
\]
If we take this basis $\mathtt{hom}$ as an index set $I := \{1,\dots,\abs{\mathtt{hom}}\}$, we can make a direct sum out of that many copies of $M$,
\[
S := \bigoplus_{i\in I} M.
\]
Then for these morphisms $\mathtt{hom}$ we can calculate their universal morphism out of the direct sum as in \ref{def:biproduct}, i.e.
\begin{align*}
\mathtt{UniversalMorphismFromDirectSum}( \mathtt{hom} ) := u_{\text{out}}(\mathtt{hom}) : S \rightarrow N.
\end{align*}

In the case where $\mathtt{hom} = \emptyset$, i.e. there are no (non-zero) morphisms from $M$ to $N$, we still get a
universal morphism with target $N$, namely the unique morphism from the zero object
\[
\mathtt{UniversalMorphismFromZeroObject}( N ) := 0_{0,N} : 0 \rightarrow N
\]
as in \ref{ex:sum_of_empty}.

In both cases we get one morphism that encodes all the information of the whole family of morphisms, especially their images. 
\[
\begin{tikzcd}
\bigoplus_{i\in I} M \arrow[r, "u_{\text{out}}(\mathtt{hom})"'] & N
\end{tikzcd}
\]

\begin{algorithm}[H]\capstart
    \caption{\texttt{MorphismOntoSumOfImagesOfAllMorphisms}}\label{algo:MorphismOntoSumOfImagesOfAllMorphisms}
	\SetKwInput{Input}{Input~}
	\SetKwInput{Output}{Output~}
	\Input{~two objects $M$ and $N$ in an abelian category}
	\Output{~a morphism with source $\oplus_{i \in I} M$ and target $N$, where $\abs{I} = \abs{\mathtt{hom}}$}
	\BlankLine
	$\mathtt{hom} := \mathtt{BasisOfExternalHom}( M, N )$\;
	\If(\tcp*[f]{the direct sum of the empty set is the zero object}){$\mathtt{hom} = \emptyset$}{
	    \Return $\mathtt{UniversalMorphismFromZeroObject}( N )$\;
	}
	\BlankLine
	\Return $\mathtt{UniversalMorphismFromDirectSum}( \mathtt{hom} )$\;
\end{algorithm}

\subsection{The algorithm $\mathtt{EmbeddingOfSumOfImagesOfAllMorphisms}$}

Since we are in the setting of an Abelian category, the morphism $u_{\text{out}}(\mathtt{hom})$ from above can
be factored with the Corollary \ref{cor:epi_mono_factorization} into an epimorphism $\pi$ and a monomorphism $\iota$.
\[
\begin{tikzcd}
\bigoplus_{i\in I} M \arrow[rr, "u_{\text{out}}(\mathtt{hom})"] \arrow[rd, "\pi"', two heads] &                              & N \\
                                                                                              & I \arrow[ru, "\iota"', hook] &  
\end{tikzcd}
\]

The monomorphism $\iota$ is the image embedding of $u_{\text{out}}(\mathtt{hom})$.

\begin{algorithm}[H]\capstart
    \caption{\texttt{EmbeddingOfSumOfImagesOfAllMorphisms}}\label{algo:EmbeddingOfSumOfImagesOfAllMorphisms}
	\SetKwInput{Input}{Input~}
	\SetKwInput{Output}{Output~}
	\Input{~two objects $M$ and $N$ in an abelian category}
	\Output{~an embedding morphism with source $I$ and target $N$}
	\BlankLine
	\Return $\mathtt{ImageEmbedding}( \mathtt{MorphismOntoSumOfImagesOfAllMorphisms}( M, N ) )$\;
\end{algorithm}

\subsection{The algorithm $\mathtt{SumOfImagesOfAllMorphisms}$}

As a last step, we are now looking at the source $I$ of the above embedding $\iota : I \hookrightarrow N$. The image
embedding has all the information from $u_{\text{out}}(\mathtt{hom})$, but its source $I$ is considerably ``smaller'' than the
direct sum $\bigoplus_{i\in I} M$ that we started with. 

\begin{algorithm}[H]\capstart
    \caption{\texttt{SumOfImagesOfAllMorphisms}}\label{algo:SumOfImagesOfAllMorphisms}
	\SetKwInput{Input}{Input~}
	\SetKwInput{Output}{Output~}
	\Input{~two objects $M$ and $N$ in an abelian category}
	\Output{~the image $I$ over which the direct sum factors}
	\BlankLine
	\Return $\mathtt{Source}( \mathtt{EmbeddingOfSumOfImagesOfAllMorphisms}( M, N ) )$\;
\end{algorithm}

Since the objects and morphisms we calculated,
\begin{itemize}
\item $\mathtt{hom} := \mathtt{BasisOfExternalHom}(\mathtt{const},F)$,
\item $u_{\text{out}}(\mathtt{hom}) := \mathtt{UniversalMorphismFromDirectSum}(\mathtt{hom})$,
\item $\iota := \mathtt{ImageEmbedding}(u_{\text{out}}(\mathtt{hom}))$ and
\item the image object $I := \mathtt{Source}(\iota)$ are all unique up to isomorphism,
\end{itemize}
we get as output an object that is unique up to unique isomorphism, and thus a good candidate to distinguish
between two categorically different objects $F$ and $G$ as we wanted.


