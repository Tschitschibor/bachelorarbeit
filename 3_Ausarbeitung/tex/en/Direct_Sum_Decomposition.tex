


We have shown in section \ref{sect:abelian_cat} that for a family $\{F_{i}\}_{i\in I}$ of functors in $\HomAkmat$, there is the
direct sum $F := \bigoplus_{i\in I} F_{i}$ with embeddings $\iota_{i} : F_{i} \rightarrow F$ (and projections $\pi_{i} : F \rightarrow F_{i}$).
In this section we show constructively the other direction, i.e. that for each $F$ there is a direct sum decomposition
$\{F_{i}\}_{i\in I}$ and $\iota_{i} : F_{i} \rightarrow F$ such that
\[
F = \bigoplus_{i\in I} F_{i}
\]


\subsection{$\mathrm{Hom}$-structure of $\HomAkmat$}

The following algorithms assume that we already have an algorithm to compute a $\Bbbk$-basis of the external hom
$\mathrm{Hom}_{\HomAkmat}(F,F)$.
That algorithm $\mathtt{BasisOfExternalHom}( F, G )$ for two functors $F$ and $G$ works by solving systems of matrix equations, also known as
Sylvester equations, which can be solved using the Kronecker product.

As already noticed by Sylvester, one can bring a two-sided homogeneous system of linear equations (of morphisms in $\kmat$) into a classical
one-sided homogeneous system of linear equations (of morphisms in $\kmat$).

\begin{example}[Sylvester equations for a natural transformation $\eta : F \Rightarrow G$ between two representations of $C_{3}C_{3}$]
The naturality constraints, that the diagram on the right

\begin{minipage}{0.05\textwidth} % left margin
\phantom{}
\end{minipage}
\begin{minipage}{0.25\textwidth} % c3c3
\[
\begin{tikzcd}[boxedcd={inner xsep=2.2em, inner ysep=1.1em}]
1 \arrow[r, "b"] \arrow["a"', loop, distance=2em, in=215, out=145] & 2 \arrow["c"', loop, distance=2em, in=35, out=325]
\end{tikzcd}
\]
\end{minipage}
\begin{minipage}{0.02\textwidth} % gap between
\phantom{}
\end{minipage}
\begin{minipage}{0.15\textwidth} % -F-> =eta=> -G->
\[
\begin{tikzcd}
{} \arrow[rr, "F"]  & {} \arrow[dd, "\eta" description, Rightarrow] & {} \\
                    &                                               &    \\
{} \arrow[rr, "G"'] & {}                                            & {}
\end{tikzcd}
\]
\end{minipage}
\begin{minipage}{0.05\textwidth} % gap between
\phantom{}
\end{minipage}
\begin{minipage}{0.30\textwidth} % naturality CD
\[
\begin{tikzcd}
F1 \arrow["Fa"', loop, distance=2em, in=215, out=145] \arrow[rr, "Fb"] \arrow[dd, "\eta_{1}"] &  & F2 \arrow["Fc"', loop, distance=2em, in=35, out=325] \arrow[dd, "\eta_{2}"] \\
                                                                                              &  &                                                                             \\
G1 \arrow["Ga"', loop, distance=2em, in=215, out=145] \arrow[rr, "Gb"]                        &  & G2 \arrow["Gc"', loop, distance=2em, in=35, out=325]                       
\end{tikzcd}
\]
\end{minipage}
\begin{minipage}{0.17\textwidth} % right margin
\phantom{}
\end{minipage}\\
commutes for all three morphisms $a, b, c \in \mathcal{A}_{1}$, with $\mathcal{A}$ being (the $\Bbbk$-linear closure of) our
finite concrete category $C_{3}C_{3}$, are the following:

\begin{subequations}\label{eq:naturality_constraints_c3c3}
\begin{align}
Fa\,\eta_{1} &= \eta_{1} Ga \\
Fb\,\eta_{2} &= \eta_{1} Gb \\
Fc\,\eta_{2} &= \eta_{2} Gc
\end{align}
\end{subequations}

We bring the terms with the matrix variables $\eta_{j}$ all on one side:

\begin{subequations}
\begin{align}
Fa\,\eta_{1} - \eta_{1} Ga &= 0 \\
Fb\,\eta_{2} - \eta_{1} Gb &= 0 \\
Fc\,\eta_{2} - \eta_{2} Gc &= 0
\end{align}
\end{subequations}

This system of equations can be obtained from the generalized system of Sylvester equations\endnote{(This is equation (A.17) in
\cite{[FengYagoubi2017]}.)}

\begin{align}\label{eq:general_sylvester_equations}
\sum_{1\leq j \leq k} \Phi_{i\,j}\, \Theta_{j}\, \Psi_{i\,j} = P_{i}, \quad 1\leq i \leq f
\end{align}

by setting $f = 3$, $k = 2$, and defining the constant matrices $\Phi$, $\Psi$ and the right-hand side $P$ as follows:

For the left matrices $\Phi$ we have
\begin{align*}
\Phi_{1,1} = \mathrm{Diag}(Fa, 0_{F2,F2}), &\quad \Phi_{1,2} = \mathrm{Diag}(-1_{F1}, 0_{F2,F2}), \\
\Phi_{2,1} = \mathrm{Diag}(0_{F1,F1}, Fb), &\quad \Phi_{2,2} = \mathrm{Diag}(-1_{F1}, 0_{F2,F2}), \\
\Phi_{3,1} = \mathrm{Diag}(0_{F1,F1}, Fc), &\quad \Phi_{3,2} = \mathrm{Diag}(0_{F1,F1}, -1_{F2}).
\end{align*}

For the right matrices $\Psi$ we have
\begin{align*}
\Psi_{1,1} = \mathrm{Diag}(1_{G1}, 0_{G2,G2}), &\quad \Psi_{1,2} = \mathrm{Diag}(Ga, 0_{F2,F2}), \\
\Psi_{2,1} = \mathrm{Diag}(0_{G1,G1}, 1_{G2}), &\quad \Psi_{2,2} = \mathrm{Diag}(Gb, 0_{F2,F2}), \\
\Psi_{3,1} = \mathrm{Diag}(0_{G1,G1}, 1_{G2}), &\quad \Psi_{3,2} = \mathrm{Diag}(0_{G1,G1}, Gc).
\end{align*}

For the right-hand side we have $P_{1} = P_{2} = P_{3} = 0_{F1+F2,\,G1+G2}$.

Then the matrix variable $\Theta$ is
\begin{align*}
\Theta_{1} = \Theta_{2} = \mathrm{Diag}(\eta_{1}, \eta_{2}).
\end{align*}

With these definitions, our system of matrix equations \eqref{eq:naturality_constraints_c3c3} is equivalent to the following system of
block matrix equations:

\begin{subequations}
\begin{align}
\begin{pmatrix} Fa & 0 \\ 0 & 0 \end{pmatrix} \begin{pmatrix} \eta_{1} & 0 \\ 0 & \eta_{2} \end{pmatrix}
\begin{pmatrix} 1_{G1} & 0 \\ 0 & 0 \end{pmatrix}
+ \begin{pmatrix} -1_{F1} & 0 \\ 0 & 0 \end{pmatrix} \begin{pmatrix} \eta_{1} & 0 \\ 0 & \eta_{2} \end{pmatrix}
\begin{pmatrix} Ga & 0 \\ 0 & 0 \end{pmatrix} = \begin{pmatrix} 0 & 0 \\ 0 & 0 \end{pmatrix} \\
\begin{pmatrix} 0 & 0 \\ 0 & Fb \end{pmatrix} \begin{pmatrix} \eta_{1} & 0 \\ 0 & \eta_{2} \end{pmatrix}
\begin{pmatrix} 0 & 0 \\ 0 & 1_{G2} \end{pmatrix}
+ \begin{pmatrix} -1_{F1} & 0 \\ 0 & 0 \end{pmatrix} \begin{pmatrix} \eta_{1} & 0 \\ 0 & \eta_{2} \end{pmatrix}
\begin{pmatrix} Gb & 0 \\ 0 & 0 \end{pmatrix} = \begin{pmatrix} 0 & 0 \\ 0 & 0 \end{pmatrix} \\
\begin{pmatrix} 0 & 0 \\ 0 & Fc \end{pmatrix} \begin{pmatrix} \eta_{1} & 0 \\ 0 & \eta_{2} \end{pmatrix}
\begin{pmatrix}0 & 0 \\ 0 & 1_{G2} \end{pmatrix}
+ \begin{pmatrix} 0 & 0 \\ 0 & -1_{F2} \end{pmatrix} \begin{pmatrix} \eta_{1} & 0 \\ 0 & \eta_{2} \end{pmatrix}
\begin{pmatrix} 0 & 0 \\ 0 & Gc \end{pmatrix} = \begin{pmatrix} 0 & 0 \\ 0 & 0 \end{pmatrix} 
\end{align}
\end{subequations}

And can be solved with the method suitable for the general system of Sylvester equations \eqref{eq:general_sylvester_equations}.
With the constants $F1, F2, G1, G2, Fa, Fb, Fc, Ga, Gb, Gc$ given as coefficients for the equations, we can thus solve the system and
get a finite basis of the external hom, i.e. a basis for $\mathrm{Hom}_{\HomAkmat}(F,G)$.
\end{example}

\subsection{The algorithm $\mathtt{DecomposeOnceByRandomEndomorphism}$}

\begin{algorithm}[H]\capstart
    \caption{\texttt{DecomposeOnceByRandomEndomorphism}}\label{algo:DecomposeOnceByRandomEndomorphism}
	\SetKwInput{Input}{Input~}
	\SetKwInput{Output}{Output~}
	\Input{~a functor $F$ in a functor category}
	\Output{~a pair $[\iota : I \rightarrow F, \kappa : K \rightarrow F]$ of morphisms such that $I \oplus K = F$ with $I \neq 0$ and $K \neq 0$ or
	$\mathtt{fail}$ if it was unable to further decompose $F$; }
	\BlankLine
	$d := \max \{ \mathrm{dim}_{\Bbbk}Fc \}_{c \in \mathcal{A}_{0}}$\;
	\If{$d = 0$}{
	    \Return $\mathtt{fail}$\tcp*{the zero representation is indecomposable}
	}
	$\mathcal{B} = [\beta_{1},\dots,\beta_{h}]$ is a $\Bbbk$-basis of $\mathrm{Hom}_{\HomAkmat}(F,F)$\;
	add $0_{F,F}$ to $\mathcal{B}$\;
	$n := \lfloor\log_{2}(d)\rfloor + 1$\;
	\BlankLine
	\For{$b \in [h+1, h, \dots,2]$}{
	    $\alpha := \beta_{b} + \mathrm{random}(\Bbbk) \cdot \beta_{b-1}$\tcp*{a heuristic ansatz for a random endomorphism}
	    \For{$i \in [ 1, \dots, n ]$}{
	        $\alpha_{2} := \alpha^{2}$\;
	        \nl\tcc{We do not expect the exponentiation to produce an idempotent, still this is a very cheap test:}
	        \If{$\alpha = \alpha_{2}$}{
	            \Break\;
	        }
	        $\alpha := \alpha_{2}$\;
	    }
	    \BlankLine
	    
	    \If{$\alpha = 0$}{
	        \Continue\tcp*{try another endomorphism}
	    }
	    
	    $\kappa := \mathrm{KernelEmbedding}(\alpha)$\;
	    
	    \If{$\kappa = 0$}{
	        \Continue\tcp*{try another endomorphism}
	    }
	    \BlankLine
	    $\iota := \mathrm{ImageEmbedding}(\alpha)$\;
	    \Return $[ \iota, \kappa ]$\;
	}
	\BlankLine
	\Return $\mathtt{fail}$\tcp*{The input functor $F$ is indecomposable with a high probability.}
\end{algorithm}
\phantom{}\\

To justify proposition \ref{prop:Decompose_terminates_correct} below we need the following lemma which is a linear analogue of the
$\sigma$-lemma \ref{la:sigma-lemma}.

\begin{lemma}
If $\alpha$ is an endomorphism of a $d$-dimensional vector space, then $\alpha^{e}\restrict{\mathrm{Im}(\alpha^{d})}$ is an automorphism for
all $e \geq d$.
\end{lemma}
\begin{proof}
This follows from the inclusion $\mathrm{Im}(\alpha^{2}) \subseteq \mathrm{Im}(\alpha)$ and that the dimension of the vector space is equal to
$d$.
\end{proof}

\begin{proposition}\label{prop:Decompose_terminates_correct}
\algoref{DecomposeOnceByRandomEndomorphism} terminates with the correct output.
\end{proposition}
\begin{proof}
For the input $F = 0$ we have $Fc = 0\,\forall c \in \mathcal{A}$ and thus $d = 0$ which returns $\mathtt{fail}$, i.e. there is no
decomposition $0 = I \oplus K$ with $I \neq 0$ and $K \neq 0$.\\

\noindent For any other input $F \neq 0$ there is a $c \in \mathcal{A}_{0}$ with $Fc > 0$, thus in line 1 we have $d > 0$.\\
Since $F \neq 0$ the vector space $\mathrm{End}_{\HomAkmat}(F)$ is at least $1$-dimensional, so our basis $\mathcal{B}$ has
length $h \geq 1$ and doesn't contain $0_{F,F}$. Thus after line 6 we can assume $\mathrm{Length}(\mathcal{B}) = h+1 \geq 2$ and
the list $[h+1,h,\dots,2]$ in line 8 to be nonempty, thus we will enter the for loop at least once.\\

\noindent In the for-loop in lines 10-17, we are squaring $\alpha$ at most $n$ times with
\[
2^{n} = 2^{\lfloor\log_{2}(d)\rfloor+1} \geq d.
\]
Hence $\alpha_{c}^{2^{n}}$ is an automorphism on its image $\mathrm{Im}(\alpha_{c}^{2^{n}}) = \mathrm{Im}(\alpha_{c}^{d})$
by the above lemma and with the definition of $d$. Let $I = \mathrm{Im}(\alpha^{2^{n}})$ and $K = \mathrm{Ker}(\alpha^{2^{n}})$
with image embedding $\iota : I \rightarrow F$ and kernel embedding $\kappa : K \rightarrow F$. Then 
\begin{align}
F = I \oplus K.
\end{align}
\end{proof}

\begin{remark}
The set of endomorphisms which yield a nontrivial decomposition of a decomposable functor is Zariski-open and hence Zariski-dense in
the vector space $\mathrm{End}_{\HomAkmat}(F)$.
\end{remark}

%gap example DecomposeOnce (fortyone)
\begin{computation} \label{comp:decompose_once}
In this \Gap{} session we will apply \algoref{DecomposeOnceByRandomEndomorphism}
first on an indecomposable representation, which will $\mathtt{fail}$ as expected, and then on a decomposable one,
resulting in the two embeddings $\mathtt{iota}$ and $\mathtt{kappa}$. The former with indecomposable source,
and the latter $\mathtt{kappa}$ with a source that can be further decomposed.\\

\begin{Verbatim}[commandchars=!@|,fontsize=\small,frame=single,label=Example]
  !gapprompt@gap>| !gapinput@LoadPackage( "CatReps" );|
  true
  !gapprompt@gap>| !gapinput@c3c3 := ConcreteCategoryForCAP( [ [2,3,1], [4,5,6], [,,,5,6,4] ] );
|
  A finite concrete category
  !gapprompt@gap>| !gapinput@GF3 := HomalgRingOfIntegers( 3 );
|
  GF(3)
  !gapprompt@gap>| !gapinput@kq := Algebroid( GF3, c3c3 );
|
  Algebroid generated by the right quiver q(2)[a:1->1,b:1->2,c:2->2]
  !gapprompt@gap>| !gapinput@SetIsLinearClosureOfACategory( kq, true );
|
  !gapprompt@gap>| !gapinput@CatReps := Hom( kq, GF3 );
|
  The category of functors: Algebroid generated by the right quiver
  q(2)[a:1->1,b:1->2,c:2->2] -> Category of matrices over GF(3)
  !gapprompt@gap>| !gapinput@d := [[1,1,0,0,0],[0,1,1,0,0],[0,0,1,0,0],[0,0,0,1,1],[0,0,0,0,1]];;
|
  !gapprompt@gap>| !gapinput@e := [[0,1,0,0],[0,0,1,0],[0,0,0,0],[0,1,0,1],[0,0,1,0]];;
|
  !gapprompt@gap>| !gapinput@f := [[1,1,0,0],[0,1,1,0],[0,0,1,0],[0,0,0,1]];;
|
  !gapprompt@gap>| !gapinput@nine := AsObjectInHomCategory( kq, [ 5, 4 ], [ d, e, f ] );
|
  <(1)->5, (2)->4; (a)->5x5, (b)->5x4, (c)->4x4>
  !gapprompt@gap>| !gapinput@DecomposeOnceByRandomEndomorphism( nine );
|
  fail
\end{Verbatim}
 The above shows that our representation \texttt{nine} is indecomposable (with a high probability). We use the tensor product to generate another
representation \texttt{fortyone}, that is hopefully decomposable, and inspect the two resulting embeddings \texttt{iota} and \texttt{kappa}. 
\begin{Verbatim}[commandchars=!@|,fontsize=\small,frame=single,label=Example]
  !gapprompt@gap>| !gapinput@fortyone := TensorProductOnObjects( nine, nine );
|
  <(1)->25, (2)->16; (a)->25x25, (b)->25x16, (c)->16x16>
  !gapprompt@gap>| !gapinput@result := DecomposeOnceByRandomEndomorphism( fortyone );
|
  [ <(1)->3x25, (2)->1x16>, <(1)->22x25, (2)->15x16> ]
  !gapprompt@gap>| !gapinput@iota := result[1];
|
  <(1)->3x25, (2)->1x16>
  !gapprompt@gap>| !gapinput@kappa := result[2];
|
  <(1)->22x25, (2)->15x16>
  !gapprompt@gap>| !gapinput@Display( fortyone );
|
  An object in The category of functors: Algebroid generated by the
  right quiver q(2)[a:1->1,b:1->2,c:2->2] -> Category of matrices
  over GF(3) defined by the following data:
  
  
  Image of <(1)>:
  A vector space object over GF(3) of dimension 25
  
  Image of <(2)>:
  A vector space object over GF(3) of dimension 16
  
  Image of (1)-[{ Z(3)^0*(a) }]->(1):
   1 1 . . . 1 1 . . . . . . . . . . . . . . . . . .
   . 1 1 . . . 1 1 . . . . . . . . . . . . . . . . .
   . . 1 . . . . 1 . . . . . . . . . . . . . . . . .
   . . . 1 1 . . . 1 1 . . . . . . . . . . . . . . .
   . . . . 1 . . . . 1 . . . . . . . . . . . . . . .
   . . . . . 1 1 . . . 1 1 . . . . . . . . . . . . .
   . . . . . . 1 1 . . . 1 1 . . . . . . . . . . . .
   . . . . . . . 1 . . . . 1 . . . . . . . . . . . .
   . . . . . . . . 1 1 . . . 1 1 . . . . . . . . . .
   . . . . . . . . . 1 . . . . 1 . . . . . . . . . .
   . . . . . . . . . . 1 1 . . . . . . . . . . . . .
   . . . . . . . . . . . 1 1 . . . . . . . . . . . .
   . . . . . . . . . . . . 1 . . . . . . . . . . . .
   . . . . . . . . . . . . . 1 1 . . . . . . . . . .
   . . . . . . . . . . . . . . 1 . . . . . . . . . .
   . . . . . . . . . . . . . . . 1 1 . . . 1 1 . . .
   . . . . . . . . . . . . . . . . 1 1 . . . 1 1 . .
   . . . . . . . . . . . . . . . . . 1 . . . . 1 . .
   . . . . . . . . . . . . . . . . . . 1 1 . . . 1 1
   . . . . . . . . . . . . . . . . . . . 1 . . . . 1
   . . . . . . . . . . . . . . . . . . . . 1 1 . . .
   . . . . . . . . . . . . . . . . . . . . . 1 1 . .
   . . . . . . . . . . . . . . . . . . . . . . 1 . .
   . . . . . . . . . . . . . . . . . . . . . . . 1 1
   . . . . . . . . . . . . . . . . . . . . . . . . 1
  
  A morphism in Category of matrices over GF(3)
  
  
  Image of (1)-[{ Z(3)^0*(b) }]->(2):
   . . . . . 1 . . . . . . . . . .
   . . . . . . 1 . . . . . . . . .
   . . . . . . . . . . . . . . . .
   . . . . . 1 . 1 . . . . . . . .
   . . . . . . 1 . . . . . . . . .
   . . . . . . . . . 1 . . . . . .
   . . . . . . . . . . 1 . . . . .
   . . . . . . . . . . . . . . . .
   . . . . . . . . . 1 . 1 . . . .
   . . . . . . . . . . 1 . . . . .
   . . . . . . . . . . . . . . . .
   . . . . . . . . . . . . . . . .
   . . . . . . . . . . . . . . . .
   . . . . . . . . . . . . . . . .
   . . . . . . . . . . . . . . . .
   . . . . . 1 . . . . . . . 1 . .
   . . . . . . 1 . . . . . . . 1 .
   . . . . . . . . . . . . . . . .
   . . . . . 1 . 1 . . . . . 1 . 1
   . . . . . . 1 . . . . . . . 1 .
   . . . . . . . . . 1 . . . . . .
   . . . . . . . . . . 1 . . . . .
   . . . . . . . . . . . . . . . .
   . . . . . . . . . 1 . 1 . . . .
   . . . . . . . . . . 1 . . . . .
  
  A morphism in Category of matrices over GF(3)
  
  
  Image of (2)-[{ Z(3)^0*(c) }]->(2):
   1 1 . . 1 1 . . . . . . . . . .
   . 1 1 . . 1 1 . . . . . . . . .
   . . 1 . . . 1 . . . . . . . . .
   . . . 1 . . . 1 . . . . . . . .
   . . . . 1 1 . . 1 1 . . . . . .
   . . . . . 1 1 . . 1 1 . . . . .
   . . . . . . 1 . . . 1 . . . . .
   . . . . . . . 1 . . . 1 . . . .
   . . . . . . . . 1 1 . . . . . .
   . . . . . . . . . 1 1 . . . . .
   . . . . . . . . . . 1 . . . . .
   . . . . . . . . . . . 1 . . . .
   . . . . . . . . . . . . 1 1 . .
   . . . . . . . . . . . . . 1 1 .
   . . . . . . . . . . . . . . 1 .
   . . . . . . . . . . . . . . . 1
  
  A morphism in Category of matrices over GF(3)
  !gapprompt@gap>| !gapinput@S := DirectSum( [ Source( iota ), Source( kappa ) ] );
|
  <(1)->25, (2)->16; (a)->25x25, (b)->25x16, (c)->16x16>
  !gapprompt@gap>| !gapinput@Display( S );
|
  An object in The category of functors: Algebroid generated by the
  right quiver q(2)[a:1->1,b:1->2,c:2->2] -> Category of matrices
  over GF(3) defined by the following data:
  
  
  Image of <(1)>:
  A vector space object over GF(3) of dimension 25
  
  Image of <(2)>:
  A vector space object over GF(3) of dimension 16
  
  Image of (1)-[{ Z(3)^0*(a) }]->(1):
   . 2 . . . . . . . . . . . . . . . . . . . . . . .
   1 2 2 . . . . . . . . . . . . . . . . . . . . . .
   . . 1 . . . . . . . . . . . . . . . . . . . . . .
   . . . 1 1 . . . 1 1 . . . . . . . . . . . . . . .
   . . . . 1 1 . . . 1 1 . . . . . . . . . . . . . .
   . . . . . 1 . . . . 1 . . . . . . . . . . . . . .
   . . . . . . 1 1 . . . 1 1 . . . . . . . . . . . .
   . . . . . . . 1 . . . . 1 . . . . . . . . . . . .
   . . . . . . . . 1 1 . . . 1 1 . . . . . . . . . .
  . . . . . . . . . 1 1 . . . 1 1 . . . . . . . . .
  . . . . . . . . . . 1 . . . . 1 . . . . . . . . .
  . . . . . . . . . . . 1 1 . . . 1 1 . . . . . . .
  . . . . . . . . . . . . 1 . . . . 1 . . . . . . .
  . . . . . . . . . . . . . 1 1 . . . . . . . . . .
  . . . . . . . . . . . . . . 1 1 . . . . . . . . .
  . . . . . . . . . . . . . . . 1 . . . . . . . . .
  . . . . . . . . . . . . . . . . 1 1 . . . . . . .
  . . . . . . . . . . . . . . . . . 1 . . . . . . .
  . . . . . . . . . . . . . . . . . . 1 1 . 1 1 . .
  . . . . . . . . . . . . . . . . . . . 1 1 . 1 1 .
  . . . . . . . . . . . . . . . . . . . . 1 . . 1 .
  . . . . . . . . . . . . . . . . . . . . . 1 1 . .
  . . . . . . . . . . . . . . . . . . . . . . 1 1 .
  . . . . . . . . . . . . . . . . . . . . . . . 1 .
  . . . . . . . . . . . . . . . . . . . . . . . . 1
  
  A morphism in Category of matrices over GF(3)
  
  
  Image of (1)-[{ Z(3)^0*(b) }]->(2):
   2 . . . . . . . . . . . . . . .
   1 . . . . . . . . . . . . . . .
   . . . . . . . . . . . . . . . .
   . . . . . . 1 . . . . . . . . .
   . . . . . . . 1 . . . . . . . .
   . . . . . . . . . . . . . . . .
   . . . . . . 1 . 1 . . . . . . .
   . . . . . . . 1 . . . . . . . .
   . . . . . . . . . . 1 . . . . .
   . . . . . . . . . . . 1 . . . .
   . . . . . . . . . . . . . . . .
   . . . . . . . . . . 1 . 1 . . .
   . . . . . . . . . . . 1 . . . .
   . . . . . . . . . . . . . . . .
   . . . . . . . . . . . . . . . .
   . . . . . . . . . . . . . . . .
   . . . . . . . . . . . . . . . .
   . . . . . . . . . . . . . . . .
   . . . . . . 1 . . . . . . . 1 .
   . . . . . . . 1 . . . . . . . 1
   . . . . . . . . . . . . . . . .
   . . . . . . . . . . 1 . . . . .
   . . . . . . . . . . . 1 . . . .
   . . . . . . . . . . . . . . . .
   . . . . . . . 2 . . 1 . 1 . . 2
  
  A morphism in Category of matrices over GF(3)
  
  
  Image of (2)-[{ Z(3)^0*(c) }]->(2):
   1 . . . . . . . . . . . . . . .
   . 1 1 . . 1 1 . . . . . . . . .
   . . 1 1 . . 1 1 . . . . . . . .
   . . . 1 . . . 1 . . . . . . . .
   . . . . 1 . . . 1 . . . . . . .
   . . . . . 1 1 . . 1 1 . . . . .
   . . . . . . 1 1 . . 1 1 . . . .
   . . . . . . . 1 . . . 1 . . . .
   . . . . . . . . 1 . . . 1 . . .
   . . . . . . . . . 1 1 . . . . .
   . . . . . . . . . . 1 1 . . . .
   . . . . . . . . . . . 1 . . . .
   . . . . . . . . . . . . 1 . . .
   . . . . . . . . . . . . . 1 1 .
   . . . . . . . . . . . . . . 1 1
   . . . . . . . . . . . . . . . 1
  
  A morphism in Category of matrices over GF(3)
\end{Verbatim}
 Comparing the matrices of \texttt{fortyone} with those of \texttt{S}, we see that after decomposing once, we have separated one small matrix on
the diagonal: A $3\times 3$-matrix from \texttt{S(kq.a)}, a $3 \times 1$-matrix from \texttt{S(kq.b)} and a $1\times 1$-matrix from \texttt{S(kq.c)}. This matches with the source of the embedding \texttt{iota}. 
\begin{Verbatim}[commandchars=!@|,fontsize=\small,frame=single,label=Example]
  !gapprompt@gap>| !gapinput@Display( iota );
|
  A morphism in The category of functors: Algebroid generated by the
  right quiver q(2)[a:1->1,b:1->2,c:2->2] -> Category of matrices
  over GF(3) defined by the following data:
  
  
  Image of <(1)>:
   2 2 . 1 1 . . . . . . . . . . 1 2 1 2 1 . . . . .
   1 2 1 2 1 1 2 1 2 1 . . . . . 2 . . 1 . 2 . . 1 .
   . . 2 . . . 1 2 . 2 2 . . 1 . . . 1 . . . 2 . . 1
  
  A split monomorphism in Category of matrices over GF(3)
  
  
  Image of <(2)>:
   . . . . . . . . . . . . . . . 1
  
  A split monomorphism in Category of matrices over GF(3)
  !gapprompt@gap>| !gapinput@Source( iota );|
  <(1)->3, (2)->1; (a)->3x3, (b)->3x1, (c)->1x1>
  !gapprompt@gap>| !gapinput@Display( Source( iota ) );|
  An object in The category of functors: Algebroid generated by the
  right quiver q(2)[a:1->1,b:1->2,c:2->2] -> Category of matrices
  over GF(3) defined by the following data:
  
  Image of <(1)>:
  A vector space object over GF(3) of dimension 3
  
  Image of <(2)>:
  A vector space object over GF(3) of dimension 1
  
  Image of (1)-[{ Z(3)^0*(a) }]->(1):
   . 2 .
   1 2 2
   . . 1
  
  A morphism in Category of matrices over GF(3)
  
  
  Image of (1)-[{ Z(3)^0*(b) }]->(2):
   2
   1
   .
  
  A morphism in Category of matrices over GF(3)
  
  
  Image of (2)-[{ Z(3)^0*(c) }]->(2):
   1
  
  A morphism in Category of matrices over GF(3)
\end{Verbatim}
 We can then look at the other embedding of the direct sum decomposition, \texttt{kappa}. The iteration of \texttt{WeakDirectSumDecomposition} will continue then with \texttt{Source( kappa )}. Each time the random endomorphism might decompose the representation resulting in matrices of
 smaller and smaller dimensions.
\begin{Verbatim}[commandchars=!@|,fontsize=\small,frame=single,label=Example]
  !gapprompt@gap>| !gapinput@Source( kappa );|
  <(1)->22, (2)->15; (a)->22x22, (b)->22x15, (c)->15x15>
  !gapprompt@gap>| !gapinput@Display( Source( kappa ) );|
  An object in The category of functors: Algebroid generated by the
  right quiver q(2)[a:1->1,b:1->2,c:2->2] -> Category of matrices
  over GF(3) defined by the following data:
  
  Image of <(1)>:
  A vector space object over GF(3) of dimension 22
  
  Image of <(2)>:
  A vector space object over GF(3) of dimension 15
  
  Image of (1)-[{ Z(3)^0*(a) }]->(1):
   1 1 . . . 1 1 . . . . . . . . . . . . . . .
   . 1 1 . . . 1 1 . . . . . . . . . . . . . .
   . . 1 . . . . 1 . . . . . . . . . . . . . .
   . . . 1 1 . . . 1 1 . . . . . . . . . . . .
   . . . . 1 . . . . 1 . . . . . . . . . . . .
   . . . . . 1 1 . . . 1 1 . . . . . . . . . .
   . . . . . . 1 1 . . . 1 1 . . . . . . . . .
   . . . . . . . 1 . . . . 1 . . . . . . . . .
   . . . . . . . . 1 1 . . . 1 1 . . . . . . .
   . . . . . . . . . 1 . . . . 1 . . . . . . .
   . . . . . . . . . . 1 1 . . . . . . . . . .
   . . . . . . . . . . . 1 1 . . . . . . . . .
   . . . . . . . . . . . . 1 . . . . . . . . .
   . . . . . . . . . . . . . 1 1 . . . . . . .
   . . . . . . . . . . . . . . 1 . . . . . . .
   . . . . . . . . . . . . . . . 1 1 . 1 1 . .
   . . . . . . . . . . . . . . . . 1 1 . 1 1 .
   . . . . . . . . . . . . . . . . . 1 . . 1 .
   . . . . . . . . . . . . . . . . . . 1 1 . .
   . . . . . . . . . . . . . . . . . . . 1 1 .
   . . . . . . . . . . . . . . . . . . . . 1 .
   . . . . . . . . . . . . . . . . . . . . . 1
   
  A morphism in Category of matrices over GF(3)
   
   
  Image of (1)-[{ Z(3)^0*(b) }]->(2):
   . . . . . 1 . . . . . . . . .
   . . . . . . 1 . . . . . . . .
   . . . . . . . . . . . . . . .
   . . . . . 1 . 1 . . . . . . .
   . . . . . . 1 . . . . . . . .
   . . . . . . . . . 1 . . . . .
   . . . . . . . . . . 1 . . . .
   . . . . . . . . . . . . . . .
   . . . . . . . . . 1 . 1 . . .
   . . . . . . . . . . 1 . . . .
   . . . . . . . . . . . . . . .
   . . . . . . . . . . . . . . .
   . . . . . . . . . . . . . . .
   . . . . . . . . . . . . . . .
   . . . . . . . . . . . . . . .
   . . . . . 1 . . . . . . . 1 .
   . . . . . . 1 . . . . . . . 1
   . . . . . . . . . . . . . . .
   . . . . . . . . . 1 . . . . .
   . . . . . . . . . . 1 . . . .
   . . . . . . . . . . . . . . .
   . . . . . . 2 . . 1 . 1 . . 2
  
  A morphism in Category of matrices over GF(3)
 
 
  Image of (2)-[{ Z(3)^0*(c) }]->(2):
   1 1 . . 1 1 . . . . . . . . .
   . 1 1 . . 1 1 . . . . . . . .
   . . 1 . . . 1 . . . . . . . .
   . . . 1 . . . 1 . . . . . . .
   . . . . 1 1 . . 1 1 . . . . .
   . . . . . 1 1 . . 1 1 . . . .
   . . . . . . 1 . . . 1 . . . .
   . . . . . . . 1 . . . 1 . . .
   . . . . . . . . 1 1 . . . . .
   . . . . . . . . . 1 1 . . . .
   . . . . . . . . . . 1 . . . .
   . . . . . . . . . . . 1 . . .
   . . . . . . . . . . . . 1 1 .
   . . . . . . . . . . . . . 1 1
   . . . . . . . . . . . . . . 1
  
  A morphism in Category of matrices over GF(3)
  !gapprompt@gap>| !gapinput@result2 := DecomposeOnceByRandomEndomorphism( Source( kappa ) );|
  [ <(1)->3x22, (2)->3x15>, <(1)->19x22, (2)->12x15> ]
\end{Verbatim}

\end{computation}

% julia worksheet

\subsection{The algorithm $\mathtt{WeakDirectSumDecomposition}$}

\begin{algorithm}[H]\capstart
    \caption{\texttt{WeakDirectSumDecomposition}}\label{algo:WeakDirectSumDecomposition}
	\SetKwInput{Input}{Input~}
	\SetKwInput{Output}{Output~}
	\Input{~a functor $F$ in a functor category}
	\Output{~a list $[\eta_i : F_{i} \rightarrow F]$ of embeddings such that $\oplus_{i} F_{i} = F$ and each $F_{i}$ is indecomposable
	(with a high probability).}
	\BlankLine
	$\mathtt{queue} := [ 1_{F} ]$\;
	$\mathtt{summands} := \emptyset$\;
	
	\While{ $\mathtt{queue} \neq \emptyset$ }{
	    let $\eta$ be the last element in $\mathtt{queue}$ and delete $\eta$ from $\mathtt{queue}$\;
	    $\mathtt{result} := \mathtt{DecomposeOnceByRandomEndomorphism}(s(\eta))$\;
	    \eIf(\tcp*[f]{$s(\eta)$ was indecomposable (with a high probability)}){$\mathtt{result} = \mathtt{fail}$}{
	        add $\eta$ to $\mathtt{summands}$\;
	    }{
	        $[\iota,\kappa] = \mathtt{result}$\;
	        append $[\iota\eta, \kappa\eta]$ to $\mathtt{queue}$\;
	    }
	}
	\BlankLine
	\Return $\mathtt{summands}$\;
\end{algorithm}

\begin{proposition}
\algoref{WeakDirectSumDecomposition} terminates with the correct output.
\end{proposition}
\begin{proof} We assert certain truths about the algorithm by formulating invariants, and how they stay constant in each line of the algorithm.\\

\begin{subproof}[Proof that the output is correct]\phantom{}\\
\noindent In line 1, the morphism $1_{F} : F \rightarrow F$, which is initially the only morphism in $\mathtt{queue}$, satisfies $t(1_{F}) = F$.\\
In line 10, since $\iota : I \rightarrow s(\eta)$ and $\kappa : K \rightarrow s(\eta)$ are each composable with $\eta$, then we are
appending the list $[\iota\eta, \kappa\eta]$ of morphisms with target $t(\iota\eta) = t(\kappa\eta) = F$ to the $\mathtt{queue}$.\\
Thus in each step of the algorithm the $\mathtt{queue}$ only contains morphisms $\eta$ with target $t(\eta) = F$.\\

\noindent In line 2, the list $\mathtt{summands}$ is initially empty.\\
In line 7, we add a morphism $\eta$ from the $\mathtt{queue}$ to the list $\mathtt{summands}$ only if
in line 6 we checked that $s(\eta)$ is indecomposable.\\
Thus in each step of the algorithm the list $\mathtt{summands}$ only contains indecomposable morphisms with target $F$.\\

\noindent Initially with $\mathtt{queue} = [1_{F}]$ and $\mathtt{summands} = \emptyset$ we have
\begin{align}
F = \label{eq:direct_sum_decomposition}
\bigoplus_{\eta \in \mathtt{queue}} s(\eta) \oplus \bigoplus_{\eta \in \mathtt{summands}} s(\eta)
\end{align}
For the first run of the while loop we take $\eta_{1} := 1_{F}$ from the $\mathtt{queue}$. Then there are two possibilities:\\
If $F$ was indecomposable, we now add $\eta_{1}$ to $\mathtt{summands}$, and have $\mathtt{queue} = \emptyset$ and
$\mathtt{summands} = [1_{F}]$ which also satisfies \eqref{eq:direct_sum_decomposition}.\\
Otherwise we get a decomposition of $F$ with $\iota_{1} : I_{1} \rightarrow F$ and $\kappa_{1} : K_{1} \rightarrow F$. In this case
$\mathtt{summands}$ stays empty, and instead we have $\mathtt{queue} = [\iota_{1}\eta_{1}, \kappa_{1}\eta_{1}]$. For
\eqref{eq:direct_sum_decomposition} to hold, we need to prove that
\[
I_{1} \oplus K_{1} = F
\]
which is exactly what we proved above for the output of $\mathtt{DecomposeOnceByRandomEndomorphism}$.\\
Thus after the first while loop, equation \eqref{eq:direct_sum_decomposition} holds.\\

In each run of the while loop, we are replacing $I_{j}$ with $I_{j,1}$ and $K_{j,1}$ with $I_{j} = I_{j,1} \oplus K_{j,1}$ and
$K_{j}$ with $I_{j,2}$ and $K_{j,2}$ with $K_{j} = I_{j,2} \oplus K_{j,2}$, if possible. That is we have 
\begin{alignat*}{5}
F &=  &&I_{1} &&\oplus &&K_{1} \\
&= (I_{11} &&\oplus K_{11}) &&\oplus (I_{12} &&\oplus K_{12}) \\
&= \dots
\end{alignat*}

This decomposition tree can end earlier for some $K_{j}$ and $I_{j'}$, which are already decomposed into indecomposables after
two steps, while for other $K_{i}$ and $I_{i'}$ the decomposition can go on for three or more steps, as illustrated in the following figure:

\[
\begin{tikzcd}
                                                   &                                                           &                                                         &                                                             & F                                                          &                                                           &                                                          &                                                          \\
                                                   &                                                           & F \arrow[rru, "\eta_{1}", bend left]                    &                                                             &                                                            &                                                           & F \arrow[llu, "\eta_{1}"', bend right]                   &                                                          \\
                                                   &                                                           & I_{1} \arrow[u, "\iota_{1}"] \arrow[rr, dash]         &                                                             & \bigoplus \arrow[uu, "I_{1} \oplus K_{1}" description]     &                                                           & K_{1} \arrow[u, "\kappa_{1}"'] \arrow[ll, dash]        &                                                          \\
                                                   & I_{1} \arrow[ru, "\eta_{11}"', bend left=49, shift right] &                                                         &                                                             & I_{1} \arrow[llu, "\eta_{11}", pos=0.35, bend right]                 & K_{1} \arrow[ru, "\eta_{12}"', pos=0.40, bend left=49, shift left] &                                                          & K_{1} \arrow[lu, "\eta_{12}", bend right=49, shift left] \\
                                                   & I_{11} \arrow[u, "\iota_{11}"] \arrow[r, dash]          & \bigoplus \arrow[uu, "I_{11}\oplus K_{11}" description] &                                                             & K_{11} \arrow[ll, dash] \arrow[u, "\kappa_{11}"]         & I_{12} \arrow[u, "\iota_{12}"] \arrow[r, dash]          & \bigoplus \arrow[uu, "I_{12} \oplus K_{12}" description] & K_{12} \arrow[u, "\kappa_{12}"'] \arrow[l, dash]       \\
I_{11} \arrow[ru, "\eta_{111}"', bend left=49]     &                                                           & I_{11} \arrow[lu, "\eta_{111}", bend right=49]          & K_{11} \arrow[ru, "\eta_{112}"', bend left=49]              &                                                            & K_{11} \arrow[lu, "\eta_{112}", bend right=49]            &                                                          &                                                          \\
I_{111} \arrow[r, dash] \arrow[u, "\iota_{111}"] & \bigoplus \arrow[uu, "I_{111}\oplus K_{111}" description] & K_{111} \arrow[l, dash] \arrow[u, "\kappa_{111}"]     & I_{112} \arrow[u, "\iota_{112}"] \arrow[r, dash]          & \bigoplus \arrow[uu, "I_{112} \oplus K_{112}" description] & K_{112} \arrow[u, "\kappa_{112}"] \arrow[l, dash]       &                                                          &                                                          \\
                                                   &                                                           & I_{112} \arrow[ru, "\eta_{1121}"', bend left=49]        &                                                             & I_{112} \arrow[lu, "\eta_{1121}", bend right=49]           &                                                           &                                                          &                                                          \\
                                                   &                                                           & I_{1121} \arrow[u, "\iota_{1121}"] \arrow[r, dash]    & \bigoplus \arrow[uu, "I_{1121}\oplus K_{1121}" description] & K_{1121} \arrow[u, "\kappa_{1121}"] \arrow[l, dash]      &                                                           &                                                          &                                                         
\end{tikzcd}
\]

Assuming that the bottom-most nodes in the tree represent indecomposable functors (for which the algorithm stopped in line 6), in this example
we have a decomposition of $F$ as the direct sum

\begin{align*}
F &= I_{1} \oplus K_{1} \\
&= I_{11} \oplus K_{11} \oplus I_{12} \oplus K_{12} \\
&= I_{111} \oplus K_{111} \oplus I_{112} \oplus K_{112} \oplus I_{12} \oplus K_{12} \\
&= I_{111} \oplus K_{111} \oplus I_{1121} \oplus K_{1121} \oplus K_{112} \oplus I_{12} \oplus K_{12} \\
\end{align*}
and with $I_{111}, K_{111}, I_{1121}, K_{1121}, K_{112}, I_{12}, K_{12} $ indecomposable, we are finished. Tracing the morphisms from
the bottom to the top gives us the list of embeddings, i.e. morphisms with source an indecomposable representation and target $F$,
which are thus added to $\mathtt{summands}$.

\begin{subproof}[Proof that the algorithm terminates, i.e. that $\mathtt{queue}$ will be empty]\phantom{}\\
\noindent In each run of the while loop, a morphism $\eta$ from the $\mathtt{queue}$ gets either deleted and not replaced,
since $s(\eta)$ was indecomposable, or it gets deleted and replaced by two morphisms $[\iota\eta, \kappa\eta]$.\\
If at some point, $s(\eta)$ is indecomposable for every $\eta \in \mathtt{queue}$, then the $\eta$ will be deleted from the
$\mathtt{queue}$ (and added to $\mathtt{summands}$) until the $\mathtt{queue}$ is empty. Then the while loop will end.\\

\noindent The case that each $\eta$ gets replaced by $[\iota\eta, \kappa\eta]$ where again the $s(\iota\eta)$ and $s(\kappa\eta)$ are
decomposable must come to an end after a finite number of steps:\\

\noindent Whenever $G := s(\eta)$ is decomposable with $I \oplus K = G$ and $I, K \neq 0$, the variable
$d_{G} :=\max_{c \in \mathcal{A}_{0}} Gc$ also gets decomposed into $d_{I} := \max_{c \in \mathcal{A}_{0}} Ic$ and
$d_{K} := \max_{c \in \mathcal{A}_{0}} Kc$ with
\begin{align*}
0 < d_{I} < d_{G} \\
0 < d_{K} < d_{G}
\end{align*}
and both have the lower bound of $0$. Thus after a finite number of steps there are only $\eta$ with indecomposable $s(\eta)$ in the $\mathtt{queue}$
so that by the above result, $\mathtt{queue}$ will eventually become empty and the loop will end.
\end{subproof}

\noindent With the $\mathtt{queue} = \emptyset$ and $\mathtt{summands}$ containing only $\eta$ with indecomposable $s(\eta)$,
equation \eqref{eq:direct_sum_decomposition} becomes
\begin{align}
F = \bigoplus_{\eta \in \mathtt{summands}} s(\eta)
\end{align}
\end{subproof}
This proves that the algorithm terminates with the correct output.
\end{proof}

\begin{computation}\label{comp:direct_sum_decomposition}
Continuing with the computation \ref{comp:decompose_once}, we want to see the result of our direct sum decomposition
of $\mathtt{fortyone}$:

\begin{Verbatim}[commandchars=!@B,fontsize=\small,frame=single,label=Example]
  !gapprompt@gap>B !gapinput@etas := WeakDirectSumDecomposition( fortyone : random := false );;B
  !gapprompt@gap>B !gapinput@iso := UniversalMorphismFromDirectSum( etas );B
  <(1)->25x25, (2)->16x16>
  !gapprompt@gap>B !gapinput@Display( Source( iso ) );B
  An object in The category of functors: Algebroid generated by the
  right quiver q(2)[a:1->1,b:1->2,c:2->2] -> Category of matrices
  over GF(3) defined by the following data:
  
  
  Image of <(1)>:
  A vector space object over GF(3) of dimension 25
  
  Image of <(2)>:
  A vector space object over GF(3) of dimension 16
  
  Image of (1)-[{ Z(3)^0*(a) }]->(1):
   1 1 . . . . . . . . . . . . . . . . . . . . . . .
   . 1 . . . . . . . . . . . . . . . . . . . . . . .
   1 1 1 . . . . . . . . . . . . . . . . . . . . . .
   . . . 1 2 . . . . . . . . . . . . . . . . . . . .
   . . . . 1 . . . . . . . . . . . . . . . . . . . .
   . . . 1 1 1 . . . . . . . . . . . . . . . . . . .
   . . . . . . 1 2 . . . . . . . . . . . . . . . . .
   . . . . . . . 1 . . . . . . . . . . . . . . . . .
   . . . . . . 1 1 1 . . . . . . . . . . . . . . . .
   . . . . . . . . . 1 2 . . . . . . . . . . . . . .
   . . . . . . . . . . 1 . . . . . . . . . . . . . .
   . . . . . . . . . 1 1 1 . . . . . . . . . . . . .
   . . . . . . . . . . . . 1 . . . . . . . . . . . .
   . . . . . . . . . . . . . 1 2 . . . . . . . . . .
   . . . . . . . . . . . . . . 1 . . . . . . . . . .
   . . . . . . . . . . . . . 2 . 1 . . . . . . . . .
   . . . . . . . . . . . . . . . . 2 2 . . . . . . .
   . . . . . . . . . . . . . . . . 1 . . . . . . . .
   . . . . . . . . . . . . . . . . 2 . 1 . . . . . .
   . . . . . . . . . . . . . . . . . . . 1 2 . . . .
   . . . . . . . . . . . . . . . . . . . . 1 . . . .
   . . . . . . . . . . . . . . . . . . . 2 . 1 . . .
   . . . . . . . . . . . . . . . . . . . . . . . 2 .
   . . . . . . . . . . . . . . . . . . . . . . 1 2 2
   . . . . . . . . . . . . . . . . . . . . . . . . 1
  
  A morphism in Category of matrices over GF(3)
  
  
  Image of (1)-[{ Z(3)^0*(b) }]->(2):
   . . . . . . . . . . . . . . . .
   . . . . . . . . . . . . . . . .
   . . . . . . . . . . . . . . . .
   . . . . . . . . . . . . . . . .
   . . . . . . . . . . . . . . . .
   . . . . . . . . . . . . . . . .
   . . . . . . . . . . . . . . . .
   . . . . . . . . . . . . . . . .
   . . . . . . . . . . . . . . . .
   . . . . . . . . . . . . . . . .
   . . . . . . . . . . . . . . . .
   . . . . . . . . . . . . . . . .
   . . . . . 2 . . . . . . . . . .
   . . . . . . . . 1 . . . . . . .
   . . . . . . 1 . . . . . . . . .
   . . . . . . 1 2 . . . . . . . .
   . . . . . . . . . . . 2 . . . .
   . . . . . . . . . . . 2 . . . .
   . . . . . . . . . . 1 . . . . .
   . . . . . . . . . . . . . . 1 .
   . . . . . . . . . . . . . . . .
   . . . . . . . . . . . . . 2 . .
   . . . . . . . . . . . . . . . 2
   . . . . . . . . . . . . . . . 1
   . . . . . . . . . . . . . . . .
  
  A morphism in Category of matrices over GF(3)
  
  
  Image of (2)-[{ Z(3)^0*(c) }]->(2):
   1 . . . . . . . . . . . . . . .
   . 1 1 . . . . . . . . . . . . .
   1 . 1 . . . . . . . . . . . . .
   . . . 1 1 . . . . . . . . . . .
   . . . . 1 1 . . . . . . . . . .
   . . . . . 1 . . . . . . . . . .
   . . . . . . 1 . . . . . . . . .
   . . . . . . . 1 1 . . . . . . .
   . . . . . . 2 . 1 . . . . . . .
   . . . . . . . . . 1 1 . . . . .
   . . . . . . . . . . 1 1 . . . .
   . . . . . . . . . . . 1 . . . .
   . . . . . . . . . . . . 1 1 . .
   . . . . . . . . . . . . . 1 1 .
   . . . . . . . . . . . . . . 1 .
   . . . . . . . . . . . . . . . 1
  
  A morphism in Category of matrices over GF(3)
\end{Verbatim}
You can clearly see the diagonal block structure of the matrices in comparison with $\mathtt{fortyone}$.
\end{computation}










