\label{sect:abelian_cat}
Goal is to show that

The functor category with values in $\kmat$ is an Abelian category with enough projectives (constructively) and direct sum decomposition (constructively).

\subsection{Abelian category}

\begin{theorem}\label{thm:functor_category_abelian}
Let $\mathcal{A}$ be a finite-dimensional algebroid over some field $\Bbbk$. The functor category $\HomAkmat$
is an abelian category.
\end{theorem}
\begin{proof}
We prove that $\HomAkmat$ is an Ab-category, then that it is also an additive category and a pre-abelian category and finally that
it is an abelian category.
\begin{enumerate}
\renewcommand{\labelenumi}{(\theenumi)}
\item In order to show that $\HomAkmat$ is an Ab-category, we must show that 
\begin{enumerate}
\renewcommand{\labelenumii}{(\roman{enumii})}
\item for any two objects $F,G \in \HomAkmat_{0}$ the hom-set $\mathrm{Hom}_{\HomAkmat}(F,G)$ between them is an Abelian group, and
\item that the composition of two morphisms\\
$\mu : \mathrm{Hom}_{\HomAkmat}(F,G) \times \mathrm{Hom}_{\HomAkmat}(G,H) \rightarrow \mathrm{Hom}_{\HomAkmat}(F,H)$ is a
bilinear map, i.e. for $\eta, \varepsilon \in \mathrm{Hom}_{\HomAkmat}(F,G)$ and $\varphi, \psi \in \mathrm{Hom}_{\HomAkmat}(G,H)$ and
for $x \in \Bbbk$
\begin{align}
(\eta + \varepsilon)\varphi &= \eta\varphi + \varepsilon\varphi \\
\eta(\varphi + \psi) &= \eta\varphi + \eta\psi \\
(x\eta)\varphi &= \eta(x\varphi) = x(\eta\varphi).
\end{align}
\end{enumerate}

\begin{subproof}[Proof of (i)]
For any object $c \in \mathcal{A}$, the set of components $\mathrm{Hom}_{\kmat}(Fc,Gc)$ is a $\Bbbk$-vector space and therefore an
Abelian group.

We define the addition and scalar multiplication
\begin{align*}
+ :&& \mathrm{Hom}_{\HomAkmat}(F,G) &\times \mathrm{Hom}_{\HomAkmat}(F,G) \rightarrow \mathrm{Hom}_{\HomAkmat}(F,G)\\
\cdot :&& \Bbbk &\times \mathrm{Hom}_{\HomAkmat}(F,G) \rightarrow \mathrm{Hom}_{\HomAkmat}(F,G)
\end{align*}
component-wise: $\forall \eta, \varepsilon \in \mathrm{Hom}_{\HomAkmat}(F,G), \forall x \in \Bbbk, \forall c \in \mathcal{A}$
\begin{align}
(\eta+\varepsilon)_{c} &:= \eta_{c} + \varepsilon_{c}\\
(x \eta)_{c} &:= x\eta_{c}
\end{align}
where the right-hand side is the usual addition and scalar multiplication of matrices.

We identify as the neutral element $0_{F,G} \in \mathrm{Hom}_{\HomAkmat}(F,G)$ (or simply $0$ when the context is clear)
the natural transformation $0$ with each component $0_{c}$ being the $Fc\times Gc$ zero matrix.
For each natural transformation $\eta$ the additive inverse $-\eta$ is defined component-wise as
\begin{align}
(-\eta)_{c} &:= -\eta_{c}.
\end{align}
We also confirm that the addition is commutative:
\begin{align}
(\eta+\varepsilon)_{c} &= \eta_{c} + \varepsilon_{c}\\
    &= \varepsilon_{c} + \eta_{c}\\
    &= (\varepsilon + \eta)_{c}
\end{align}
This concludes that for each $F, G \in \HomAkmat,\, \mathrm{Hom}_{\HomAkmat}(F,G)$ is an Abelian group.
\end{subproof}
\begin{subproof}[Proof of (ii)]
Let $F, G, H \in \HomAkmat$ and let $\eta, \varepsilon \in \mathrm{Hom}_{\HomAkmat}(F,G)$,
$\varphi, \psi \in \mathrm{Hom}_{\HomAkmat}(G,H)$ and $x \in \Bbbk$.\\
The composition $\eta\varphi \in \mathrm{Hom}_{\HomAkmat}(F,H)$ is defined by component-wise matrix-multiplication,
i.e. $\forall c \in \mathcal{A}$
\begin{align*}
(\eta\varphi)_{c} := \eta_{c}\varphi_{c}
\end{align*}
and from this follows the bilinearity of the composition, since the matrix multiplication is bilinear.\\
This concludes the first part of the proof, i.e. $\HomAkmat$ is an Ab-category.
\end{subproof}

\item Next we show that $\HomAkmat$ is an additive category, i.e. it is
\begin{enumerate}
\renewcommand{\labelenumii}{(\roman{enumii})}
\item An Ab-category with
\item A dependent function $\oplus$ mapping a finite set $I$ and a collection $(F_{i})_{i\in I}$ of objects in $\HomAkmat$
to a corresponding direct sum $( \oplus_{i\in I} F_{i}, (\pi_{i})_{i\in I}, (\iota_{i})_{i\in I}, u_{\mathrm{in}}, u_{\mathrm{out}} )$.
\end{enumerate}
\begin{subproof}[Proof of (ii)]
Let $I = \{1,\dots,n\}$ be a finite set, and $F_{i} \in \HomAkmat, 1\leq i\leq n$. The object $F := \oplus_{i=1}^{n} F_{i}$ is a functor, mapping
each object $c \in \mathcal{A}_{0}$ to the natural number $Fc = \oplus_{i=1}^{n} F_{i}c := \sum_{i=1}^{n} F_{i}c \in \kmat_{0}$.\\
A morphism $a : c \rightarrow c' \in \mathcal{A}_{1}$ with $F_{i} a : F_{i} c \rightarrow F_{i} c', \, 1\leq i\leq n$ gets mapped to the matrix 
$F a : Fc \rightarrow Fc'$ that is the block-diagonal matrix of the $F_{i}a$'s.
In other words, we can identify each $F_{j}a$ as a $F_{j} c \times F_{j} c'$ submatrix:
\[
\left(F_{j}a\right)_{k,l} = \left(F a\right)_{\sum_{i=1}^{j-1} F_{i}c + k,\,\sum_{i=1}^{j-1} F_{i}c' + l},
1 \leq k \leq F_{j}c, 1 \leq l \leq F_{j}c'
\]

define $\pi_{i}, \iota_{i}$, for $\tau, \rho, u_{\mathrm{in}}(\tau), u_{\mathrm{out}}(\rho)$,

prove:
properties are fulfilled.

The sets of projections $\pi = \{ \pi_{i} : F \rightarrow F_{i} \}_{i = 1,\dots,n}$ and coprojections
$\iota = \{ \iota_{i} : F_{i} \rightarrow F \}_{i = 1,\dots,n}$ are sets of natural transformations,
where each $\pi_{i}$ and $\iota_{i}$ is defined by its components:\\
For $c \in \mathcal{A}_{0}$ we have $(\pi_{i})_{c} : Fc \rightarrow F_{i}c$ and $(\iota_{i})_{c} : F_{i}c \rightarrow Fc$ with\\
\begin{minipage}{.35\textwidth}
\[
\sum_{i=1}^{n} (\pi_{i})_{c} (\iota_{i})_{c} = 1_{Fc}
\]
\end{minipage}
\begin{minipage}{.1\textwidth}
and
\end{minipage}
\begin{minipage}{.55\textwidth}
\[
(\iota_{i})_{c}(\pi_{j})_{c} = (\delta_{i,j})_{c} = \begin{cases}
1_{F_{i}c} & \text{ if } i = j \\
0_{F_{i}c, F_{j}c} & \text{ if } i \neq j
\end{cases}
\]
\end{minipage}

For a family of natural transformations $\tau = \{ \tau_{i} : G \rightarrow F_{i} \}_{i = 1,\dots,n}$

\begin{align*}
\forall c \in \mathcal{A}_{0},&& &&  (\oplus_{i \in I} F_{i}) c &= \oplus_{i \in I} (F_{i} c) \\
\forall c \in \mathcal{A}_{0},&& \forall i \in I,&& (\pi_{i} : F \rightarrow F_{i})_{c} &= (\pi_{i})_{c} : Fc \rightarrow F_{i} c \\
\forall c \in \mathcal{A}_{0},&& \forall i \in I,&& (\iota_{i} : F_{i} \rightarrow F)_{c} &= (\iota_{i})_{c} : F_{i} c \rightarrow Fc \\
\forall c \in \mathcal{A}_{0},&& \forall \tau = \{ \tau_{i} : G \rightarrow F_{i} \}_{i = 1,\dots,n},&&
(u_{\mathrm{in}}(\tau))_{c} &= u_{\mathrm{in}}(\tau_{c}) \\
\forall c \in \mathcal{A}_{0},&& \forall \rho = \{ \rho_{i} : F_{i} \rightarrow H \}_{i = 1,\dots,n},&&
(u_{\mathrm{out}}(\rho))_{c} &= u_{\mathrm{out}}(\rho_{c})
\end{align*}

where $\tau_{c} = \{ (\tau_{i})_{c} : Gc \rightarrow F_{i} c \}_{i \in I}$ and $\rho_{c} = \{ (\rho_{i})_{c} : F_{i} c \rightarrow Hc \}_{i \in I}$.

\begin{align*}
u_{\mathrm{in}}(\tau_{c}) p_{i} &= (\tau_{i})_{c} \\
j_{i} u_{\mathrm{out}}(\rho_{c}) &= (\rho_{i})_{c}
\end{align*}

with $\{ p_{i} : Fc \rightarrow F_{i}c \}_{i\in I}$ and $\{ j_{i} : F_{i}c \rightarrow Fc \}_{i \in I}$ the projection and injection morphisms
of the direct sum of matrices.

Prove that $p_{i} = (\pi_{i})_{c}$ and $j_{i} = (\iota_{i})_{c}$.

The direct sum of a family of functors $\{ F_{i}\}_{i \in I}$ is a functor $F := \oplus_{i \in I} F_{i}$, mapping each object $c \in \mathcal{A}_{0}$
to an object $Fc \in \kmat_{0}$, where $Fc$ is the direct sum of the family $\{ F_{i}c \}_{i \in I}$.
The family of projections $\{ \pi_{i} : F \rightarrow F_{i} \}$ of the direct sum $F$ of functors is a family of natural transformations.
The components of the $\pi_{i}$ under the same object $c \in \mathcal{A}_{0}$ form a family of projections $\{ (\pi_{i})c : Fc \rightarrow F_{i} c\}$
that are the projections in factors of the direct sum $Fc$. Likewise for the family of injections.

We conclude that the 
\end{subproof}

\begin{subproof}[Proof of (ii)]
Let $I = \{1,\dots,n\}$ be a finite set. For each $c \in \mathcal{A}_{0}$ the family $(F_{i})_{i\in I}$ of functors in $\HomAkmat_{0}$ maps $c$ to a
family $(S_{i})_{i\in I} := (F_{i}c)_{i\in I}$ of objects in $\kmat_{0}$. Let $S := \oplus_{i\in I} S_{i}$ be the direct sum of the $S_{i}$'s with projections
$p = \{ p_{i} : S \rightarrow S_{i}\}_{i\in I}$ and coprojections $q = \{ q_{i} : S_{i} \rightarrow S\}_{i\in I}$ as well as dependent functions
$v_{\text{in}}$ and $v_{\text{out}}$ such that each family of morphisms $t = \{ t_{i} : T \rightarrow S_{i}\}_{i\in I}$ gets mapped to a
morphism $v_{\text{in}}(t)$ and each family of morphisms $r = \{ r_{i} : S_{i} \rightarrow R\}_{i\in I}$ gets mapped to a
morphism $v_{\text{out}}(r)$ such that
\begin{align*}
v_{\text{in}}(t) p_{i} &= t_{i} \forall i\in I, \\
q_{i} v_{\text{out}}(r) &= r_{i} \forall i\in I.
\end{align*}
The $p$ and $q$ also satisfy
\begin{align*}
\sum_{i\in I} p_{i} q_{i} &= 1_{S} \text{  and  } \\
q_{i} p_{j} &= \begin{cases}
            1_{S_{i}} & \text{ if } i = j \\
            0_{ij} & \text{ if } i \neq j
        \end{cases}
\end{align*}
We now want to define a direct sum in the functor category, i.e. for the family $(F_{i})_{i\in I}$ the tuple
$(F = \oplus_{i\in I} F_{i}, \pi = (\pi_{i})_{i\in I}, \iota = (\iota_{i})_{i\in I}, u_{\text{in}}, u_{\text{out}})$ with the desired properties of a direct sum.
Let $F$ be the functor mapping each $c\in \mathcal{A}_{0}$ to the object $Fc := S = \oplus_{i\in I} F_{i}c$. This already fixes the source and target of
$Fa$ for a morphism $a : c \rightarrow c' \in \mathcal{A}_{1}$, namely
\begin{align*}
Fa &: Fc \rightarrow Fc' \\
Fa &: \oplus_{i\in I} F_{i}c \rightarrow \oplus_{i\in I} F_{i}c'
\end{align*}
$\pi$ such that $(\pi_{i})_{c} : Fc \rightarrow F_{i}c = p_{i} : S \rightarrow S_{i}$.
\end{subproof}

\item Next we show that $\HomAkmat$ is a pre-abelian category, i.e. an
\begin{enumerate}
\renewcommand{\labelenumii}{(\roman{enumii})}
\item additive category with
\item a dependent function mapping a morphism $\eta : F \rightarrow G \in \HomAkmat_{1}$ to a kernel of $\eta$ and
\item a dependent function mapping a morphism $\eta : F \rightarrow G \in \HomAkmat_{1}$ to a cokernel of $\eta$.
\end{enumerate}
The proof of the existence of kernels and cokernels goes the same way as for the direct sum. In general we proved that the functor category
has all finite limits and colimits of the target category.


\end{enumerate}
\end{proof}

We have seen that there is a $\Bbbk$-vector space of natural transformations between two functors $F$ and $G$ in $\HomAkmat$.
Numbering the finitely many objects in $\mathcal{A}_{0}$ by $1,2,\dots,o$ with $o \in \mathbb{N}$
With $Fc$ and $Gc$ being natural numbers, the components $\alpha_{c}$ of a natural transformation $\alpha$ are matrices.
The $\Bbbk$-vector space $\Bbbk^{Fc\times Gc}$ is $Fc\times Gc$-dimensional, so for all of $\mathcal{A}_{0}$ we have an
upper bound for the vector space dimension:
\begin{align}
\mathrm{dim}_{\Bbbk} \mathrm{Hom}_{\HomAkmat}(F,G) &\leq \sum_{c\in \mathcal{A}_{0}} \mathrm{dim}_{\Bbbk} \Bbbk^{Fc\times Gc} \\
    &= \sum_{c\in \mathcal{A}_{0}} Fc\times Gc
\end{align}
But in this calculation, we didn't include the naturality constraints, i.e. for each morphism $a : c -> c' \in \mathcal{A}_{1}$
\begin{align}
Fa \alpha_{c'} &= \alpha{c} Ga
\end{align}

The naive approach to build a basis for each object, and then take the cartesian product over all objects, and only then
checking the naturality constraints, is terribly slow.

Use Yoneda's Lemma!



