

\subsection{Finite concrete categories and the free/forgetful adjunction}

Our model for a finite concrete category $\mathcal{C}$ is that of a finite subcategory of $\FinSets$. In particular we restrict ourselves
to finite concrete categories that are generated by a finite set of morphisms $\mathtt{SetOfGeneratingMorphisms}$.

The $\mathtt{SetOfGeneratingMorphisms} = \{ a_{1},a_{2},\dots,a_{n} \}$ already defines a finite quiver:

\begin{definition}{(Finite quiver generated by a finite set of morphisms)}\label{def:quiver_generated}
Let $M = \{ a_{1}, a_{2}, \dots, a_{n} \}$ be a finite set of morphisms. We say a quiver $q$ is \ul{generated by $M$}, if
\begin{align}
q_{1} &= M, \text{ and } \\
q_{0} &= \{ o : \exists a \in M, s(a) = o \vel t(a) = o \}
\end{align}
In this case the quiver $q$ is finite.
\end{definition}

The fact that every category is also a quiver can be expressed in the following as the existence of a certain forgetful functor:

\begin{example}{(Forgetful functor $U : \mathcal{C} \rightarrow \mathcal{D}$)}
\begin{enumerate}
\renewcommand{\labelenumi}{(\theenumi)}
\item We denote by the letter $U$ (for \textit{underlying}) a \ul{forgetful functor} between two categories $U : \mathcal{C} \rightarrow \mathcal{D}$ if
we can identify every object $c \in \mathcal{C}$ as an object $Uc \in \mathcal{D}$ by \textit{forgetting} some additional structure that $c$ had
in $\mathcal{C}$ but that is not defined for objects in $\mathcal{D}$. The object $Uc$ is called the \ul{underlying object} of $c$.
\item For a morphism $a : c \rightarrow c' \in \mathcal{C}$ that was some structure-preserving map between $c$ and $c'$, if that structure doesn't
exist in the category $\mathcal{D}$ then $Ua : Uc \rightarrow Uc'$ \textit{forgets} the structure-preserving property of $a$.
\item There are other conceivable functors that even map morphisms between two objects
(e.g. functors between two categories are morphisms in $\mathrm{\textbf{Cat}}$) to objects (e.g. functors in the functor category). If you now
want to get back the morphism from the object, you again are using a forgetful functor (e.g. to get the \ul{underlying functor} of the functor object).
\end{enumerate}
Some authors define a forgetful functor in the strict sense that its target category $\mathcal{D} = \mathrm{\textbf{Set}}$, i.e. it forgets all structure;
and functors that only forget some but not all of the algebraic structure are called \ul{intermediate forgetful functors}.
\end{example}

\noindent We are interested in the forgetful functor with $\mathcal{C} = \Cat$ and $\mathcal{D} = \Quiv$:

\begin{example}{(Forgetful functor $U  : \Cat \hookrightarrow \Quiv$)}
Let $\mathcal{C} \in \Cat$ be a category. The quiver $q = U\mathcal{C}$ is defined by
\begin{align}
q_{0} &= \mathcal{C}_{0} \\
q_{1} &= \mathcal{C}_{1}
\end{align}
In particular every identity morphism $1_{c} \in \mathcal{C}_{1}$ for an object $c \in \mathcal{C}$ now is just any other endomorphism
on that object (but it is still true that $s(1_{c}) = t(1_{c}) = c$).
And every morphism $\varphi\psi \in \mathcal{C}_{1}$ that was the composition of $\varphi$ with $\psi$ is now just
any morphism without much deeper connection to $\varphi$ and $\psi$ apart from
\begin{align}
s(\varphi\psi) &= s(\varphi) \text{ and } \\
t(\varphi\psi) &= t(\psi),
\end{align}
which is still true in $\Quiv$. Of course, associativity and unital property of the composition $\mu$ doesn't exist in $\Quiv$ since there is no composition
of arrows.
\end{example}

\begin{example}{(Underlying quiver)}\label{ex:underlying_quiver}\\

\noindent\begin{minipage}{.08\textwidth}
\phantom{}
\end{minipage}
\begin{minipage}{.37\textwidth}
\begin{tikzcd}[boxedcd={inner xsep=1.5em, inner ysep=3em}]
2 \arrow[rrrr, "b"] \arrow[rrrrddd, "e", pos=0.3] \arrow["h"', loop, distance=2em, in=125, out=55] &  &  &  &
3 \arrow[ddd, "c"] \arrow["i"', loop, distance=2em, in=125, out=55]\\
 &  &  &  & \\
 &  &  &  & \\
1 \arrow[uuu, "a"] \arrow[rrrruuu, "d", pos=0.3] \arrow[rrrr, bend left, "f" ', shift right=2]
\arrow[rrrr, "f", bend right] \arrow["g"', loop, distance=2em, in=305, out=235] &  &  &  &
4 \arrow["j"', loop, distance=2em, in=305, out=235]
\end{tikzcd}
\end{minipage}
%
\begin{minipage}{.10\textwidth}
$\xhookleftarrow{\text{   U   }}$
\end{minipage}
%
\begin{minipage}{.37\textwidth}
\begin{tikzcd}[boxedcd={inner xsep=1.5em, inner ysep=3em}]
B \arrow[rrrr, "\psi"] \arrow[rrrrddd, "\psi\rho", pos=0.3] \arrow["1_{B}"', loop, distance=2em, in=125, out=55] &  &  &  &
C \arrow[ddd, "\rho"] \arrow["1_{C}"', loop, distance=2em, in=125, out=55]\\
 &  &  &  & \\
 &  &  &  & \\
A \arrow[uuu, "\varphi"] \arrow[rrrruuu, "\varphi\psi", pos=0.3] \arrow[rrrr, bend left, "(\varphi\psi)\rho" ', shift right=2]
\arrow[rrrr, "\varphi(\psi\rho)", bend right] \arrow["1_{A}"', loop, distance=2em, in=305, out=235] &  &  &  &
D \arrow["1_{D}"', loop, distance=2em, in=305, out=235]
\end{tikzcd}
\end{minipage}
\begin{minipage}{.08\textwidth}
\phantom{}
\end{minipage}\\

\noindent In the category on the right, associativity of composition guaranteed that $(\varphi\psi)\rho = \varphi(\psi\rho)$, so those two arrows
were already the same, so they are mapped to the same arrow $f = U((\varphi\psi)\rho) = U(\varphi(\psi\rho))$ in the quiver on the left.
We didn't have to draw both arrows for $f$, but since they are equal, there is still only one arrow in the hom-set $\textup{Hom}_{q}(1,4)=\{f,f\} = \{f\}$.\\
All the other identities are not preserved under the forgetful functor, e.g. $d$ doesn't know what it has to do with $a$ and $b$ apart from
$s(d) = s(a)$ and $t(d) = t(b)$. Especially the former identity arrows are now just endomorphisms with no defining property.\\
The paths $g^{2}f, gf$ and $fj^{3}$ are all different, while in the category, they all simplify to
$1_{A}1_{A}(\varphi\psi)\rho = 1_{A}(\varphi\psi)\rho = (\varphi\psi)\rho1_{D}1_{D}1_{D} =  (\varphi\psi)\rho$ due to the unit property and associativity.
\end{example}

The category $\mathcal{C}$ in the last example has the set of morphisms $\mathcal{C}_{1} =
\{ 1_{A}, 1_{B}, 1_{C}, 1_{D}, \varphi, \psi, \rho, \varphi\psi, \psi\rho, \varphi\psi\rho \}$, i.e. 10 morphisms. But once the three morphisms
$\varphi, \psi$ and $\rho$ were defined, the other seven morphisms were forced from the unit and composition axioms of a category.

\begin{example}{(Category generated by one endomorphism)}\label{ex:category_generated_by_one_endomorphism}
As another example, take a category $\mathcal{M}$ with one object $\ast$ and apart from $1_{\ast}$ one other endomorphism
$\alpha : \ast \rightarrow \ast$. It already has a priori countably infinitely many morphisms
$\mathcal{M}_{1} = \{ 1_{\ast}, \alpha, \alpha^{2}, \alpha^{3}, \dots \}$. But the information to generate that category is all encoded in the
one morphism $\alpha$.
\end{example}

What we are looking for is a construction of a finite concrete category from a finite set of generating morphisms. For this we can take
the generated quiver from \ref{def:quiver_generated} and from it the free category.

\begin{definition}{(The free category $F : \Quiv \rightarrow \Cat$)}\label{ex:free_category}\endnote{(ref. \cite{[context]} Example 4.1.13)}
The \ul{free category} $Fq$ on a quiver $q$ has $q_{0}$ as its set of objects. The set $(Fq)_{1}$ of morphisms consists of identities $1_{c}$ for each
object $c\in q_{0}$ together with all finite paths of arrows $a \in q_{1}$. Composition is defined by concatenation of paths.
\end{definition}

\begin{definition}{(Free $\dashv$ forgetful adjunction)}
The functor pair $F : \Quiv \leftrightarrows \Cat : U$ is an example for an adjunction, i.e. for the functors $F : \Quiv \rightarrow \Cat$ and
$U : \Cat \rightarrow \Quiv$ there is an isomorphism
\begin{align}
\mathrm{Hom}_{\Cat}(Fq, \mathcal{C}) \cong \mathrm{Hom}_{\Quiv}(q,U\mathcal{C})
\end{align}
for each $q \in \Quiv$ and $\mathcal{C} \in \Cat$, that is natural in both $q$ and $\mathcal{C}$. Here $U$ is \ul{right adjoint} to $F$ and
the forgetful functor $U$ admits a \ul{left adjoint}, free construction $F$.
\end{definition}

We can view the free category functor in action in two different ways: Where does the category $\mathcal{C}$ in example \ref{ex:underlying_quiver}
come from, i.e. what is the quiver $q$ such that $\mathcal{C} = Fq$? And where does it go after we forget the category structure, i.e. what is
the category $F(U(\mathcal{C}))$? We will illustrate the answers to both questions in the next example:

\begin{example}{(Generating quiver $\xrightarrow{F}$ category $\xrightarrow{U}$ underlying quiver $\xrightarrow{F}$ category)}
\begin{enumerate}
\renewcommand{\labelenumi}{(\theenumi)}
\item The free category generated by the quiver:
\[
\noindent\begin{minipage}{.08\textwidth}
\phantom{}
\end{minipage}
%
\begin{minipage}{.37\textwidth}
\begin{tikzcd}[boxedcd={inner xsep=1.5em, inner ysep=3em}]
B \arrow[rrrr, "\psi"] &  &  &  & C \arrow[ddd, "\rho"] \\
 &  &  &  & \\
 &  &  &  & \\
A \arrow[uuu, "\varphi"] &  &  &  & D
\end{tikzcd}
\end{minipage}
%
\begin{minipage}{.10\textwidth}
$\xrightarrow{\text{     }F\text{     }}$
\end{minipage}
%
\begin{minipage}{.37\textwidth}
\begin{tikzcd}[boxedcd={inner xsep=1.5em, inner ysep=3em}]
B \arrow[rrrr, "\psi"] \arrow[rrrrddd, "\psi\rho", pos=0.3] \arrow["1_{B}"', loop, distance=2em, in=125, out=55] &  &  &  &
C \arrow[ddd, "\rho"] \arrow["1_{C}"', loop, distance=2em, in=125, out=55]\\
 &  &  &  & \\
 &  &  &  & \\
A \arrow[uuu, "\varphi"] \arrow[rrrruuu, "\varphi\psi", pos=0.3] \arrow[rrrr, bend left, "(\varphi\psi)\rho" ', shift right=2]
\arrow[rrrr, "\varphi(\psi\rho)", bend right] \arrow["1_{A}"', loop, distance=2em, in=305, out=235] &  &  &  &
D \arrow["1_{D}"', loop, distance=2em, in=305, out=235]
\end{tikzcd}
\end{minipage}
\begin{minipage}{.08\textwidth}
\phantom{}
\end{minipage}
\]
\item The free category generated by the underlying quiver:
\[
\noindent\begin{minipage}{.08\textwidth}
\phantom{}
\end{minipage}
\begin{minipage}{.37\textwidth}
\begin{tikzcd}[boxedcd={inner xsep=1.5em, inner ysep=3em}]
2 \arrow[rrrr, "b"] \arrow[rrrrddd, "e", pos=0.3] \arrow["h"', loop, distance=2em, in=125, out=55] &  &  &  &
3 \arrow[ddd, "c"] \arrow["i"', loop, distance=2em, in=125, out=55]\\
 &  &  &  & \\
 &  &  &  & \\
1 \arrow[uuu, "a"] \arrow[rrrruuu, "d", pos=0.3] \arrow[rrrr, bend left, "f" ', shift right=2]
\arrow["g"', loop, distance=2em, in=305, out=235] &  &  &  &
4 \arrow["j"', loop, distance=2em, in=305, out=235]
\end{tikzcd}
\end{minipage}
%
\begin{minipage}{.10\textwidth}
$\xrightarrow{\text{     }F\text{     }}$
\end{minipage}
%
\begin{minipage}{.37\textwidth}
\begin{tikzcd}[boxedcd={inner xsep=3em, inner ysep=3em}]
2 \arrow[rrrr, "b"] \arrow[rrrrddd, "e", pos=0.75] \arrow["h"', loop, distance=2em, in=125, out=55] \arrow["1_{2}"', loop, distance=2em, in=215, out=145] \arrow[rrrrddd, "bc", pos=0.75, shift left=5]                                                                                                                                                                                                                                                                                       &  &  &  & 3 \arrow[ddd, "c"] \arrow["i"', loop, distance=2em, in=125, out=55] \arrow["1_{3}"', loop, distance=2em, in=35, out=325] \\
                                                                                                                                                                                                                                                                                                                                                                                                                                                                          &  &  &  &                                                                                                                          \\
                                                                                                                                                                                                                                                                                                                                                                                                                                                                          &  &  &  &                                                                                                                          \\
1 \arrow[uuu, "gah"', pos=0.65, shift right=3] \arrow[rrrruuu, "d", pos=0.75] \arrow[rrrr, "f", bend left, shift right=2] \arrow[rrrr, "abc", bend right] \arrow["g"', loop, distance=2em, in=305, out=235] \arrow["1_{1}"', loop, distance=2em, in=215, out=145] \arrow[uuu, "ah" description, pos=0.6] \arrow[uuu, "ga" description, pos=0.45, shift left=3] \arrow[uuu, "a" description, pos=0.3, shift left=6] \arrow[rrrruuu, "ab", pos=0.67, shift left=6] \arrow[rrrr, "dc", shift right=3] \arrow[rrrr, "ae", shift left=2] &  &  &  & 4 \arrow["j"', loop, distance=2em, in=305, out=235] \arrow["1_{4}"', loop, distance=2em, in=35, out=325]                
\end{tikzcd}
\end{minipage}
%
\begin{minipage}{.08\textwidth}
\phantom{}
\end{minipage}
\]
As you can see, this picture gets cluttered very fast (I didn't draw all morphisms). The reason for this is the existence of non-identity endomorphisms.
As we have shown in lemma \ref{la:cyclic_paths}, one non-identity endomorphism is enough for a quiver to have infinitely many paths.
Here it is even worse than in the example \ref{ex:category_generated_by_one_endomorphism} where we had countably many endomorphisms
$\alpha^{n}, n \in \mathbb{N}$. Through the arrow $a : 1 \rightarrow 2$ we can concatenate countably many morphisms
$g^{m}ah^{n}, (m,n) \in \mathbb{N}\times\mathbb{N}$ and even $g^{n_{1}}ah^{n_{2}}bi^{n_{3}}cj^{n_{4}}, (n_{1}, n_{2}, n_{3}, n_{4}) \in \mathbb{N}^{4}$.
If we were to construct the path algebra (see \ref{def:path_algebra}) on the quiver, it already had an infinite basis.
\end{enumerate}
\end{example}

\begin{example}{(Continued example \ref{ex:category_generated_by_one_endomorphism})}\\

As a last example to see how bad it can get from seemingly innocent quivers, take the category with 1 object and its identity morphism:
\[
\noindent\begin{minipage}{.005\textwidth}
\phantom{}
\end{minipage}
\begin{minipage}{.08\textwidth}
\begin{tikzcd}
\ast \arrow["1_{\ast}"', loop, distance=2em, in=305, out=235]
\end{tikzcd}
\end{minipage}
%
\begin{minipage}{.05\textwidth}
$\xrightarrow{\text{     }U\text{     }}$
\end{minipage}
%
\begin{minipage}{.08\textwidth}
\begin{tikzcd}
\ast \arrow["a"', loop, distance=2em, in=305, out=235]
\end{tikzcd}
\end{minipage}
%
\begin{minipage}{.05\textwidth}
$\xrightarrow{\text{     }F\text{     }}$
\end{minipage}
%
\begin{minipage}{.15\textwidth}
\begin{tikzcd}
\ast \arrow["a"', loop, distance=2em, in=305, out=235] \arrow["1_{\ast}"', loop, distance=2em, in=125, out=55] \arrow["{a^{2}, a^{3},\dots}"', loop, distance=2em, in=35, out=325]
\end{tikzcd}
\end{minipage}
%
\begin{minipage}{.05\textwidth}
$\xrightarrow{\text{     }U\text{     }}$
\end{minipage}
%
\begin{minipage}{.08\textwidth}
\begin{tikzcd}
\ast \arrow["{a, b, c,\dots}"', loop, distance=2em, in=305, out=235]
\end{tikzcd}
\end{minipage}
%
\begin{minipage}{.05\textwidth}
$\xrightarrow{\text{     }F\text{     }}$
\end{minipage}
%
\begin{minipage}{.25\textwidth}
\begin{tikzcd}
\ast \arrow["{a, b, c,\dots}"', loop, distance=2em, in=305, out=235] \arrow["1_{\ast}"', loop, distance=2em, in=125, out=55] \arrow["{a^{2},\dots,ab,\dots,ababbaba,\dots,b^{2},\dots,c^{2},\dots}"', loop, distance=2em, in=35, out=325]
\end{tikzcd}
\end{minipage}
\begin{minipage}{.10\textwidth}
\phantom{}
\end{minipage}
\]
After the first forgetful functor, we are in the situation of \ref{ex:category_generated_by_one_endomorphism} where then the first free functor
gives us countably many morphisms.
The following forgetful functor only renames them to countably many distinct morphisms.
In the last step, we are constructing the free monoid on countably many generators. Among other things it contains the free monoid on two
generators.\endnote{As my father rightly remarked when I showed this to him, \enquote{You can hide the whole world inside there!}}
\end{example}

We can learn two lessons from these examples:
\begin{enumerate}
\item An adjunction is more general than an equivalence of categories. The free functor $F$ doesn't just \textit{undo} the forgetful functor $U$.
You will end up with much more than you started with.
\item If we want to still work with finite categories, we really need to control the size of our hom-sets, especially regarding the endomorphisms.
This is the topic in the next section.
\end{enumerate}

\subsection{Relations of endomorphisms}


\begin{lemma}{($\sigma$-Lemma)}\\

\begin{minipage}{.01\textwidth}\phantom{}
\end{minipage}
%
\begin{minipage}{.55\textwidth}
Let $\mathcal{C}$ be a finite concrete category. Then for each object $M \in \mathcal{C}_{0}$ the set
$\textup{End}_{\mathcal{C}}(M)$ is a monoid and for each endomorphism $f \in \textup{End}_{\mathcal{C}}(M)$
there exist $m,n \in \mathbb{N}$ such that $f^{(m+n)}=f^{m}$. If $m = 0$ and $n \geq 1$ then $f$ is bijective with $f^{-1} = f^{n-1}$.
\begin{proof}
The properties of a monoid are precisely the associativity of composition and the unit property from \ref{associativity_of_composition} and
\ref{unit_property}.
Since $\abs{\textup{End}_{\mathcal{C}}(M)}<\infty$ there are only finitely many endomorphisms
$f_{1},\dots, f_{N} \in \textup{End}_{\mathcal{C}}(M)$.
Let $\{f^{k} | k \in \mathbb{N} \} \subset \textup{End}_{\mathcal{C}}(M)$, i.e. there is an inclusion function 
$\{f^{k} | k \in \mathbb{N}\} \rightarrow \{f_{j} | j \in \{1,\dots,N\}\}; f^{k} \mapsto f_{j}$ not necessarily surjective and 
by the pigeonhole principle highly non injective, since $\abs{\mathbb{N}}>\abs{\textup{End}_{\mathcal{C}}(M)}$.
Let $m := \mathrm{min} \{ k \in \mathbb{N}| \exists j \in \{ 1,\dots,N\} : f^{k} =  f_{j} \}$
\end{proof}
\end{minipage}
%
\begin{minipage}{.45\textwidth}
%[$\sigma$ lemma illustrated for different sizes of $m$ and $n$]
\begin{tikzpicture}[->,>=stealth',auto,node distance=1cm,
  thick,main node/.style={circle,draw,font=\sffamily\Large\bfseries}]

  % nodes on the leg of sigma
  \node[main node] (1) [circle, fill, inner sep=2pt] {};
  \node[main node] (2) [left of=1, circle, fill, inner sep=2pt] {};
  \node[main node] (3) [left of=2, circle, fill, inner sep=2pt] {};
  \node[main node] (4) [left of=3, circle, fill, inner sep=2pt] {};
  
  % nodes on the circle of sigma
  \node[main node] (5) [above right=-0.84098cm and -1.10106cm of 4, circle, fill, inner sep=2pt] {};
  \node[main node] (6) [above right=-1.95902cm and -0.73779cm of 4, circle, fill, inner sep=2pt] {};
  \node[main node] (7) [above right=-1.95902cm and 0.43779cm of 4, circle, fill, inner sep=2pt] {};
  \node[main node] (8) [above right=-0.84098cm and 0.80106cm of 4, circle, fill, inner sep=2pt] {};
  \node[main node] (9) [above right=-0.150cm and -0.150cm of 4, circle, fill, inner sep=2pt] {};
  
  
  \path[every node/.style={font=\sffamily\small}]
    (1) edge[-] node [left] {} (2)
    (2) edge[-] node [left] {} (3)
    (3) edge[-] node [left] {} (4)
    (1) edge["$f^{0} = 1_{M}$"', pos=0.70, out=300, in=40, min distance=7mm, looseness=8] (1)
    (1) edge["$f^{1}$"', bend right] node [left] {} (2)
    (2) edge["$f^{2}$"', bend right] node [left] {} (3)
    (3) edge["$f^{3}$"', bend right] node [left] {} (4)
    (
\end{tikzpicture}
\phantom{}
\end{minipage}
\end{lemma}

\begin{example}

\end{example}


\subsection{$\Bbbk$-linear categories, Algebroids and Algebras with idempotents}




