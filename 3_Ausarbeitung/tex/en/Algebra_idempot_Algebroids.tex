

\subsection{Finite concrete categories}

Our model for a finite concrete category $\mathcal{C}$ is that of a finite subcategory of $\FinSets$. In particular we restrict ourselves
to finite concrete categories that are generated by a finite set of morphisms $\mathtt{SetOfGeneratingMorphisms}$.

The $\mathtt{SetOfGeneratingMorphisms} = \{ a_{1},a_{2},\dots,a_{n} \}$ already defines a finite quiver:

\begin{definition}{(Finite quiver generated by a finite set of morphisms)}\label{def:quiver_generated}
Let $M = \{ a_{1}, a_{2}, \dots, a_{n} \}$ be a finite set of morphisms. We say a quiver $q$ is \ul{generated by $M$}, if
\begin{align}
q_{1} &= M, \text{ and } \\
q_{0} &= \{ o : \exists a \in M, s(a) = o \vel t(a) = o \}
\end{align}
In this case the quiver $q$ is finite.
\end{definition}

The fact that every category is also a quiver can be expressed in the following as the existence of a certain forgetful functor:

\begin{example}{(Forgetful functor $U : \mathcal{C} \rightarrow \mathcal{D}$)}
\begin{enumerate}
\renewcommand{\labelenumi}{(\theenumi)}
\item We denote by the letter $U$ (for \textit{underlying}) a \ul{forgetful functor} between two categories $U : \mathcal{C} \rightarrow \mathcal{D}$ if
we can identify every object $c \in \mathcal{C}$ as an object $Uc \in \mathcal{D}$ by \textit{forgetting} some additional structure that $c$ had
in $\mathcal{C}$ but that is not defined for objects in $\mathcal{D}$. The object $Uc$ is called the \ul{underlying object} of $c$.
\item For a morphism $a : c \rightarrow c' \in \mathcal{C}$ that was some structure-preserving map between $c$ and $c'$, if that structure doesn't
exist in the category $\mathcal{D}$ then $Ua : Uc \rightarrow Uc'$ \textit{forgets} the structure-preserving property of $a$.
\item There are other conceivable functors that even map morphisms between two objects
(e.g. functors between two categories are morphisms in $\mathrm{\textbf{Cat}}$) to objects (e.g. functors in the functor category). If you now
want to get back the morphism from the object, you again are using a forgetful functor (e.g. to get the \ul{underlying functor} of the functor object).
\end{enumerate}
Some authors define a forgetful functor in the strict sense that its target category $\mathcal{D} = \mathrm{\textbf{Set}}$, i.e. it forgets all structure,
and functors that only forget some but not all of the algebraic structure are called \ul{intermediate forgetful functors}.
\end{example}

\noindent We are interested in the forgetful functor with $\mathcal{C} = \Cat$ and $\mathcal{D} = \Quiv$:

\begin{example}{(Forgetful functor $U  : \Cat \hookrightarrow \Quiv$)}
Let $\mathcal{C} \in \Cat$ be a category. The quiver $q = U\mathcal{C}$ is defined by
\begin{align}
q_{0} &= \mathcal{C}_{0} \\
q_{1} &= \mathcal{C}_{1}
\end{align}
In particular every identity morphism $1_{c} \in \mathcal{C}_{1}$ for an object $c \in \mathcal{C}$ now is just any other endomorphism
on that object (but it is still true that $s(1_{c}) = t(1_{c}) = c$).
And every morphism $\varphi\psi \in \mathcal{C}_{1}$ that was the composition of $\varphi$ with $\psi$ is now just
any morphism without much deeper connection to $\varphi$ and $\psi$ apart from
\begin{align}
s(\varphi\psi) &= s(\varphi) \text{ and } \\
t(\varphi\psi) &= t(\psi),
\end{align}
which is still true in $\Quiv$. Of course, associativity and unital property of the composition $\mu$ doesn't exist in $\Quiv$ since there is no composition
of arrows.
\end{example}

\begin{example}{(Underlying quiver)}\\

\noindent\begin{minipage}{.08\textwidth}
\phantom{}
\end{minipage}
\begin{minipage}{.37\textwidth}
\begin{tikzcd}[boxedcd={inner xsep=1.5em, inner ysep=3em}]
2 \arrow[rrrr, "b"] \arrow[rrrrddd, "e", pos=0.3] \arrow["h"', loop, distance=2em, in=125, out=55] &  &  &  &
3 \arrow[ddd, "c"] \arrow["i"', loop, distance=2em, in=125, out=55]\\
 &  &  &  & \\
 &  &  &  & \\
1 \arrow[uuu, "a"] \arrow[rrrruuu, "d", pos=0.3] \arrow[rrrr, bend left, "f" ', shift right=2]
\arrow[rrrr, "f", bend right] \arrow["g"', loop, distance=2em, in=305, out=235] &  &  &  &
4 \arrow["j"', loop, distance=2em, in=305, out=235]
\end{tikzcd}
\end{minipage}
%
\begin{minipage}{.10\textwidth}
$\xhookleftarrow{\text{   U   }}$
\end{minipage}
%
\begin{minipage}{.37\textwidth}
\begin{tikzcd}[boxedcd={inner xsep=1.5em, inner ysep=3em}]
B \arrow[rrrr, "\psi"] \arrow[rrrrddd, "\psi\rho", pos=0.3] \arrow["1_{B}"', loop, distance=2em, in=125, out=55] &  &  &  &
C \arrow[ddd, "\rho"] \arrow["1_{C}"', loop, distance=2em, in=125, out=55]\\
 &  &  &  & \\
 &  &  &  & \\
A \arrow[uuu, "\varphi"] \arrow[rrrruuu, "\varphi\psi", pos=0.3] \arrow[rrrr, bend left, "(\varphi\psi)\rho" ', shift right=2]
\arrow[rrrr, "\varphi(\psi\rho)", bend right] \arrow["1_{A}"', loop, distance=2em, in=305, out=235] &  &  &  &
D \arrow["1_{D}"', loop, distance=2em, in=305, out=235]
\end{tikzcd}
\end{minipage}
\begin{minipage}{.08\textwidth}
\phantom{}
\end{minipage}\\

\noindent In the category on the right, associativity of composition guaranteed that $(\varphi\psi)\rho = \varphi(\psi\rho)$, so those two arrows
were already the same, so they are mapped to the same arrow $f = U((\varphi\psi)\rho) = U(\varphi(\psi\rho))$ in the quiver on the right.
We didn't have to draw both arrows for $f$, but since they are equal, there is still only one arrow in the hom-set $\textup{Hom}_{q}(1,4)=\{f,f\} = \{f\}$.\\
All the other identities are not preserved under the forgetful functor, e.g. $d$ doesn't know what it has to do with $a$ and $b$ apart from
$s(d) = s(a)$ and $t(d) = t(b)$. Especially the former identity arrows are now just endomorphisms with no defining property.\\
The paths $g^{2}f, gf$ and $fj^{3}$ are all different, while in the category, they all simplify to
$1_{A}1_{A}(\varphi\psi)\rho = 1_{A}(\varphi\psi)\rho = (\varphi\psi)\rho1_{D}1_{D}1_{D} =  (\varphi\psi)\rho$ due to the unit property and associativity.
\end{example}

The category $\mathcal{C}$ in the last example has the set of morphisms $\mathcal{C}_{1} =
\{ 1_{A}, 1_{B}, 1_{C}, 1_{D}, \varphi, \psi, \rho, \varphi\psi, \psi\rho, \varphi\psi\rho \}$, i.e. 10 morphisms. But once the three morphisms
$\varphi, \psi$ and $\rho$ were defined, the other seven morphisms were forced from the unit and composition axioms of a category.\\
Or as another example a category $\mathcal{M}$ with one object $\ast$ and apart from $1_{\ast}$ one other endomorphism
$\alpha : \ast \rightarrow \ast$, already has a priori infinitely many morphisms
$\mathcal{M}_{1} = \{ 1_{\ast}, \alpha, \alpha^{2}, \alpha^{3}, \dots \}$. But the information to generate that category is all encoded in the
one morphism $\alpha$.

What we are looking for is a construction of a finite concrete category from a finite set of generating morphisms. For this we can take
the generated quiver from ref{def:quiver_generated} and complete it to a category.

\begin{definition}{The free category completion}

\end{definition}




