% mainfile: ../main.tex

\subsection{Additional structure on the Hom-set of a category}

....

\begin{example}{(Group as a category)}\\
\noindent A group $\mathbf{G}$ defines a category $\mathcal{B}\mathbf{G}$ with a single object $\ast$ and the morphisms being the group elements. The group elements are its morphisms, which are
all automorphisms (i.e. bijective endomorphisms) of the single object. Composition of morphisms is defined by the binary group operation.
The identity element $e \in G$ acts as the identity morphism for the unique object in this category. The hom-set of that category is itself
a group.
$\Bbbk G$ group algebra, $\Bbbk \mathcal{C}$ category algebra
\end{example}

Our goal is to represent finite concrete categories, for this we need the source and target categories of our functors, which the
representations are.
As subcategories of $\textup{FinSets}$, our finite concrete categories only have definitions for their objects and their
morphisms, methods to check when two morphisms are congruent or equivalent, but not much else.


\begin{definition}{(Idempotent)}

\end{definition}

\begin{theorem}
Let $\mathcal{C}$ be a finite additive category.
\end{theorem}











A competing theory to category theory is that of quivers and path algebras. We already used their terminology in
\ref{def:path}, \ref{la:cyclic_paths} and \ref{def:path_algebra}, for instance when talking about the trivial path,
which in the language of category theory is nothing but the identity morphism, composition of arrows to a path is nothing but
composition of morphisms (if you make the path explicit by writing a new arrow for every path).

So what we called a path algebra in \ref{def:path_algebra} is a different data structure for a category. 
For one, the path algebra is an algebra, i.e. a vector space with additional structure, and thus a single set, comparable to the
class of morphisms $\mathcal{C}_{1}$ of a category $\mathcal{C}$.
But as it is an algebra, it not only contains the generating morphisms of the category, but also $\Bbbk$-linear combinations of
morphisms and paths. This is what our concrete categories lack, and what additional structure we have to give them in order
to represent them by matrices.

In practise, there is already developed software for \ul{q}uivers and \ul{p}ath \ul{a}lgebras, namely the \textsc{Gap} package
\textsc{QPA$2$}\endnote{(see \cite{[QPA2]})}.
What we are actually doing to represent finite concrete categories, is going from $\mathcal{C} \in \mathbf{Cats}$ to $q \in \mathbf{Quiv}$,
in theory by \ul{forgetting} (see \ref{ex:forgetful_functor}) the category concepts of identity morphism and composition, in practise by calculating the
underlying quiver $q$; and then for a commutative ring $\Bbbk$, constructing the path algebra $\Bbbk q$. In this step the path algebra
is infinite-dimensional, since there are infinitely many paths according to lemma \ref{la:cyclic_paths}, and \textsc{QPA$2$}'s function
\texttt{BasisPathsBetweenVertices} only works for finite-dimensional path algebras. Thus in a next step we have to provide
additional data in the form of generators of \ul{ideals of the path algebra}, by which we can divide and build the quotient path algebra,
which is then finite-dimensional. This is the purpose of \texttt{RelationsOfEndomorphisms}.

\subsection{Ideals of the path algebra}

\begin{definition}{(Ideal of a path algebra)}

\end{definition}

\begin{lemma}{($\Bbbk Q$ / rel is finite-dimensional)}

Let $\Bbbk Q$ be the path algebra from a small quiver $Q$ with $Q_{0}$ finite and for every $x \in Q_{0}$, $\mathrm{End}_{Q}(x) = \mathrm{Aut}_{Q}(x)$
is a finite group generated by one morphism $a_{x}$, with $n = n(x) \in \mathbb{N}$ such that $a_{x}^{n} = e_{x}$.



... defines an ideal $I$ of $\Bbbk Q$

... then the quotient algebra $\Bbbk Q / I$ is finite-dimensional.
\end{lemma}

Once we have a finite-dimensional path algebra $\Bbbk q$, we let \textsc{QPA$2$} calculate generators of the non-endomorphism relations,
and when we have a complete set of relations, that will be our definitive quotient quiver algebra $\Bbbk q$, which we then take it back into the category
theoretical context by constructing the $\Bbbk$-\textbf{Algebroid} $\mathcal{A}$ from the path algebra $\Bbbk q$.

An algebra with idempotents defines an algebroid.

The source category for our representation is then the $\Bbbk$-\textbf{Algebroid} $\mathcal{A}$ and not anymore our finite concrete
category $\mathcal{C}$, but it behaves in the same way regarding composition of morphisms and which morphisms are congruent.

The target category of our category representations will be $\Bbbk$-\textbf{Mat} which we will describe in the next section,
especially all the nice properties $\Bbbk$-\textbf{Mat} has, and how they get carried over to our functor category with $\Bbbk$-\textbf{Mat} as
target.\endnote{
In \cite{[Ab-Cat]}, Posur used the equivalence between categories $\textup{mat}_{\Bbbk} \cong \textup{vec}^{\text{fd}}_{\Bbbk}$,
as described in \cite{[context]}, \textsc{Example} 1.5.6 on page 30 (48/258), to justify that $\Bbbk$-\textbf{Mat} is a good
\textbf{computational model} to
\blockquote{transform otherwise inaccessible mathematical objects into computationally easily graspable entities},
which is what we are doing with \textbf{CatReps}.
}

With source and target categories defined, the category where our category representations lie in is \textbf{CatReps} for which we
show that it's a subcategory of the \textbf{Functor Category}. And even more in the next section.
\[
\textup{Hom}(\Bbbk\mathcal{C}, \kmat)
\]

\subsection{Generating morphisms of a category and the underlying quiver}

$\textup{gmorphisms} := \{g_{1},\dots,g_{r}\} \rightarrow$ concrete category with set of generating morphisms $\textup{gmorphisms}$.

This is the $\textup{Free}$ functor from $\mathbf{Quiv}$ to $\mathbf{Cat}$, taking a quiver and adding the missing structure of
identity morphisms and composition of arrows to that category. The result is a category.

The $\textup{forgetful}$ functor from $\mathbf{Cat}$ to $\mathbf{Quiv}$ is going the other way around and leaves all
morphisms that we now have in the category, but forgets their relations, what was identity, what was composition.

Given a field $\Bbbk$, we have the path algebra $\Bbbk q$ with all the arrows as a basis.

Given relations on endomorphisms and on the other morphisms, we make the quotient path algebra.

This is already a category, and now it has more structure.

\subsection{Ab-categories}


\begin{definition}{(semisimple)}
semisimple category
\end{definition}

\begin{theorem}{(Wedderburn)}
$\kmat$ of a semisimple ring k is a semisimple category.
\end{theorem}
TODO








\begin{definition}{($R$-linear category)}
Let $R$ be a commutative ring. An \ul{$R$-linear category} $\mathcal{A}$ is a category where every hom-set is an
$R$-module, and where for $x,y,z \in \mathcal{A}_{0}$ composition of morphisms
\[
\mu : \mathrm{Hom}_{\mathcal{A}}(x,y) \times \mathrm{Hom}_{\mathcal{A}}(y,z) \rightarrow \mathrm{Hom}_{\mathcal{A}}(x,z)
\]
is $R$-linear.

Note that this does imply that $\mathcal{A}$ is a pre-additive category, but it need not be additive.
\end{definition}

\begin{definition}{($R$-linear functor)}
Let $R$ be a commutative ring. A \ul{functor of $R$-linear categories} or an \ul{$R$-linear functor} is a functor
$F : \mathcal{A} \rightarrow \mathcal{B} $ where for all objects $x, y \in \mathcal{A}_{0}$, the map
$F : \mathrm{Hom}_{\mathcal{A}}(x,y) \rightarrow \mathrm{Hom}_{\mathcal{B}}(F(x), F(y))$ is a homomorphism
of $R$-modules.
\end{definition}

\begin{example}
Consider the following category $\mathcal{A}$ with four objects $\mathcal{A}_{0} = \{1,2,3,4\}$ and
identity morphisms $e_{1}, e_{2}, e_{3}, e_{4}$:
\[
\begin{tikzcd}
1 \arrow["e_{1}"', loop, distance=2em, in=215, out=145] \arrow[dd, "v"'] &  & 3 \arrow["x"', loop, distance=2em, in=35, out=325] \arrow["e_{3}"', loop, distance=2em, in=215, out=145] \arrow[dd, "w"', bend right=49] \arrow[dd, "xw" description, bend right=23] \arrow[dd, "wy" description, bend left=23] \arrow[dd, "xwy", bend left=49] \\
                                                                         &  &                                                                                                                                                                                                                                                           \\
2 \arrow["e_{2}"', loop, distance=2em, in=215, out=145]                  &  & 4 \arrow["y"', loop, distance=2em, in=5, out=295] \arrow["e_4"', loop, distance=2em, in=245, out=175]                                                                                                                                                   
\end{tikzcd}
\]
Note that the existence of $xw$ and $wy$ follows from the composition axioms of our category, but with the same argument,
we should also define $x^{2} = xx, x^{3} = xxx, ... $ and $y^{2} = yy, ... $ which are a priori all distinct morphisms.
Defining a set of relations of endomorphisms fixes this problem, so we define $x^{2} = e_{3}$ and $y^{2} = e_{4}$.
The identity axiom already implies the relations $e_{i}^{2} = e_{i}$ for $i = 1,2,3,4$.

If we take a field $\Bbbk$ for the commutative ring, the identity morphisms $e_{1}, e_{2}, e_{3}, e_{4}$ together with the other morphisms
\begin{align*}
v &: 1 \rightarrow 2 \\
x &: 3 \rightarrow 3 \\
xw, w, wy, xwy &: 3 \rightarrow 4 \\
y &: 4 \rightarrow 4
\end{align*}
form bases for the vector spaces
\begin{align*}
\HomA(1,1) &= \left< e_{1} \right> \\
\HomA(1,2) &= \left< v \right> \\
\HomA(2,2) &= \left< e_{2} \right> \\
\HomA(3,3) &= \left< e_{3}, x \right> \\
\HomA(3,4) &= \left< w, xw, wy, xwy \right> \\
\HomA(4,4) &= \left< e_{4}, y \right>.
\end{align*}
Linearity of composition necessitates that adding two arrows and then composing the result with another arrow
is the same as first composing each arrow and then adding them, e.g.
\begin{align*}
(x+x)w &= xw + xw \\
(e_{3} + e_{3})w &= (2\,e_{3})w = e_{3}(2w) = 2w
\end{align*}
The $\Bbbk$-dimensions of the vector spaces are $1, 1, 1, 2, 4, 2$ with a total of 11. Does that mean that
somewhere there exists an 11-dimensional $\Bbbk$-vector space that has these six vectorspaces as subspaces? Indeed it does:

The $\Bbbk$-vector space $\Bbbk\mathcal{A}$ with $\mathcal{A}_{1} = \{ e_{1},e_{2},e_{3},e_{4}, v,x,w,xw, wy, xwy, y \}$ as a basis.
The composition of arrows is already partially defined in the cases where two arrows are composable. To extend composition for all
arrows we first define for all arrows $\alpha, \beta \in \mathcal{A}_{1}$:
\begin{equation}
\alpha\beta :=\label{eq:cat_alg_mult} \begin{cases}
\alpha\cdot\beta & \text{ if $\alpha$ and $\beta$ can be composed} \\
0 & otherwise
\end{cases}
\end{equation}
and then extend this product to the whole of $\Bbbk\mathcal{A}$ using bilinearity of multiplication.
This construction turns $\Bbbk\mathcal{A}$ into not only a $\Bbbk$-vector space, but also an associative algebra.
\end{example}

\begin{definition}{(Category algebra)}
Let $\Bbbk$ be a commutative unital ring. For a category $\mathcal{C}$, we define the \ul{category algebra} $\Bbbk \mathcal{C}$
to be the free $\Bbbk$-module with $\mathcal{C}_{1}$ as a basis and the product of morphisms defined as in \eqref{eq:cat_alg_mult}.
\end{definition}

\begin{lemma}
If the set of objects $\mathcal{C}_{0}$ is finite, then the category algebra $\Bbbk\mathcal{C}$ has a unit element, namely
$\sum_{i\in \mathcal{C}_{0}} e_{i}$.
\end{lemma}
\begin{proof}
Let $e = \sum_{i\in \mathcal{C}_{0}} e_{i}$. We want to show for a general $w \in \Bbbk\mathcal{C}$ that $ew = w = we$.
\begin{enumerate}
\item Let $w = \sum_{l=1}^{N} \lambda_{l} b_{l}$ with $\{b_{l}\}_{1\leq l\leq N}$ a basis for $\mathrm{Hom}_{\mathcal{C}}(j,k)$ and
$\lambda_{l} \in \Bbbk$. Then $ew = e_{j}w + \sum_{i \neq j} e_{i} w = \sum_{l=1}^{N} \lambda_{l} (e_{j} b_{l})
+ \sum_{i \neq j} \sum_{l=1}^{N} \lambda_{l} (e_{i} b_{l})$ with each $e_{j} b_{l} = b_{l}$ in the first sum and
$e_{i} b_{l} = 0$ in the second sum, thus $ew = \sum_{l=1}^{N} \lambda_{l} b_{l} = w$ and similarly $we_{k} = w$ and $we_{i} = 0$ for
$i \neq k$.
\item 
\end{enumerate}
\end{proof}

\begin{definition}{(Algebra with enough idempotents)}
An \ul{algebra with enough idempotents} is an algebra $A$ which admits a family of nonzero orthogonal idempotents
$(e_{i})_{i\in I}$ such that $\oplus_{i\in I} e_{i} A = A = \oplus_{i\in I} Ae_{i}$.
This family $(e_{i})_{i\in I}$ is called \ul{complete} or \ul{distinguished family of orthogonal idempotents}.
\end{definition}



\begin{definition}
Once source and target categories $\mathcal{C}, \mathcal{D}$ are both $R$-linear categories we define the functor category
$\mathrm{Hom_{R}}(\mathcal{C},\mathcal{D})$ as the subcategory of $R$-linear functors.
\end{definition}



