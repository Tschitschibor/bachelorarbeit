

\begin{proof}
\[
\begin{tikzcd}
Y(i) \arrow[d, "\rho"'] \arrow[rd, "\eta"] \arrow[rrd, shift left=2] &                                                                   &   \\
F \arrow[r, "\varphi", two heads]                                  & G \arrow[r, "\theta"', shift right] \arrow[r, "\psi", shift left] & H
\end{tikzcd}
\]

We want to show that $Y(i)$ is a projective object, i.e. that for each epimorphism $\varphi : F \twoheadrightarrow G$ and each
morphism $\eta : Y(i) \rightarrow G$, there exists a projective lift $\rho : Y(i) \dottedrightarrow F$ such that $\eta = \rho\varphi$.

$\varphi_{j}$ epimorphism in $\kmat$ we have a $\rho = \mathrm{LeftDivide}( \eta, \varphi )$. Or in category theory terms,
CokernelCoLift or KernelLift... see Diss. Posur. p. 38.

Let $\mathcal{A}$ be a $\Bbbk$-algebroid with finitely many objects and finite-dimensional hom-sets over $\Bbbk$ and let
$i \in \mathcal{A}^{\text{op}}$.
We want to show that $\mathrm{Hom}_{\HomAkmat}(Y(i),-)$ is exact.\\
Let $F,G \in \HomAkmat_{0}$ and $\varphi : F \twoheadrightarrow G$ an epimorphism in $\HomAkmat_{1}$.
We want to show that
\begin{alignat}{2}
&\mathrm{Hom}_{\HomAkmat}(Y(i),\varphi) &&: \begin{cases}\mathrm{Hom}_{\HomAkmat}(Y(i), F) 
\rightarrow \mathrm{Hom}_{\HomAkmat}(Y(i), G); \\
\rho \mapsto \rho\varphi
\end{cases}
\end{alignat}
is an epimorphism.
The composition $\rho\varphi$ of natural transformations is defined component-wise:
\[
(\rho\varphi)_{j} := \rho_{j}\varphi_{j} : \begin{tikzcd}
Y(i)(j) \arrow[rd, "\rho_{j}"'] \arrow[rr, "\rho_{j}\varphi_{j}"] &                                            & G(j) \\
                                                                  & F(j) \arrow[ru, "\varphi_{j}"', two heads] &     
\end{tikzcd}
\]
where $Y(i)(j) = \mathrm{Hom}_{\mathcal{A}}(j,i)$ is the representation of the object $j$ by the functor
$Y(i) = \mathrm{Hom}_{\mathcal{A}}(-,i)$, and since we assumed our hom-sets in $\mathcal{A}$ to be finite-dimensional over $\Bbbk$,
$Y(i)(j) \in \kmat_{0}$ is also the $\Bbbk$-dimension of $\mathrm{Hom}_{\mathcal{A}}(j,i)$.\\
Let two parallel morphisms 
\begin{alignat}{2}
&\mathrm{Hom}_{\HomAkmat}(Y(i),\psi),
&&\,\mathrm{Hom}_{\HomAkmat}(Y(i),\theta) : \\
&\mathrm{Hom}_{\HomAkmat}(Y(i),G)
\rightrightarrows &&\,\mathrm{Hom}_{\HomAkmat}(Y(i),H)
\end{alignat}
such that
\begin{alignat}{3}
& &\mathrm{Hom}_{\HomAkmat}(Y(i),\varphi) &\cdot \mathrm{Hom}_{\HomAkmat}(Y(i),\psi) \\
&= \,&\mathrm{Hom}_{\HomAkmat}(Y(i),\varphi) &\cdot \mathrm{Hom}_{\HomAkmat}(Y(i),\theta); \\
& &\rho\varphi\psi &= \rho\varphi\theta, \\
& &\forall \rho &\in \mathrm{Hom}_{\HomAkmat}(Y(i), F)
\end{alignat}
i.e. component-wise
\[
\begin{tikzcd}
Y(i)(j) \arrow[rr, "\rho_{j}\varphi_{j}"] \arrow[rrrr, "\rho_{j}\varphi_{j}\theta_{j}"', bend right] \arrow[rrrr, "\rho_{j}\varphi_{j}\psi_{j}", bend left] &  & G(j) \arrow[rr, "\psi_{j}", shift left] \arrow[rr, "\theta_{j}"', shift right] &  & H(j)
\end{tikzcd}
\]
$\rho_{j}\varphi_{j}\psi_{j} = \rho_{j}\varphi_{j}\theta_{j}$ for all objects $j \in \mathcal{A}_{0}$. We are finished when we have shown that
$\psi_{j} = \theta_{j}$ for all $j \in \mathcal{A}_{0}$.

By Yoneda's lemma we have 
\begin{alignat}{2}
&\mathrm{Hom}_{\HomAkmat}(\mathrm{Hom}_{\mathcal{A}}(-,i),F) &&\cong F(i), \\
&\mathrm{Hom}_{\HomAkmat}(\mathrm{Hom}_{\mathcal{A}}(-,i),G) &&\cong G(i)\, \text{ and } \\
&\mathrm{Hom}_{\HomAkmat}(\mathrm{Hom}_{\mathcal{A}}(-,i),\varphi) &&\cong \varphi_{i}.
\end{alignat}
This means that the morphism
\begin{alignat*}{2}
&\mathrm{Hom}_{\HomAkmat}(\mathrm{Hom}_{\mathcal{A}}(-,i),\varphi) &&:\\
&\mathrm{Hom}_{\HomAkmat}(\mathrm{Hom}_{\mathcal{A}}(-,i),F)
	&&\rightarrow \mathrm{Hom}_{\HomAkmat}(\mathrm{Hom}_{\mathcal{A}}(-,i),G)
\end{alignat*}
defined by
\begin{alignat*}{3}
&\rho &&\mapsto \rho\varphi &&\forall \rho \in \mathrm{Hom}_{\HomAkmat}(\mathrm{Hom}_{\mathcal{A}}(-,i),F),\,\text{ i.e. }\\
&\rho &&: Y(i) \Rightarrow F\, &&\text{ is a natural transformation with components }\\
&\rho_{j} &&: Y(i)(j) \rightarrow F(j)&&
\end{alignat*}
is equivalent to the matrix
\begin{align}
\varphi_{i} : F(i) &\rightarrow G(i)
\end{align}
For two parallel morphisms 
\begin{alignat}{2}
&\mathrm{Hom}_{\HomAkmat}(\mathrm{Hom}_{\mathcal{A}}(-,i),\psi),
&&\,\mathrm{Hom}_{\HomAkmat}(\mathrm{Hom}_{\mathcal{A}}(-,i),\theta) : \\
&\mathrm{Hom}_{\HomAkmat}(\mathrm{Hom}_{\mathcal{A}}(-,i),G)
\rightrightarrows &&\,\mathrm{Hom}_{\HomAkmat}(\mathrm{Hom}_{\mathcal{A}}(-,i),H)
\end{alignat}
with
\begin{align}
(\rho\varphi)\psi &= (\rho\varphi)\theta
\end{align}
we have the equivalent matrices
\begin{align}
\psi_{i}, \theta_{i} : G(i) \rightrightarrows H(i)
\end{align}
with
\begin{align}
\varphi_{i}\psi_{i} &= \varphi_{i}\theta_{i}
\end{align}
we already have $\psi = \theta$?
\end{proof}
