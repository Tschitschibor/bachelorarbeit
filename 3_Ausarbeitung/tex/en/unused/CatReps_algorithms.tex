% mainfile: ../main.tex

\section{Algorithms}

\begin{algorithm}\capstart
   \caption{\texttt{ConvertToMapOfFinSets}}\label{algo:ConvertToMapOfFinSets}
      \SetKwInput{Input}{Input~}
      \SetKwInput{Output}{Output~}
      \Input{~a list $objects$ of objects in FinSets and a morphism $gen$ given as a list of images in the convention of catreps}
      \Output{~the corresponding map of finite sets from source $S$ to target $T$}
      \BlankLine
      let $T$ be the first object $O \in objects$ such that $gen \cap O \not= \emptyset$\;
      \If{$gen \cap O = \emptyset \, \forall O \in objects$}{
         Error "unable to find target set"
      }
      let $fl$ be the flattening of $objects$ as a list\;
      let $S$ be the sublist of $fl$ according to positions $i$ such that $gen[i]$ is bound\;
      set $S$ to be the first object $O \in objects$ such that $O = S$\;
      \If{$S \not= O \, \forall O \in objects$}{
          Error "unable to find source set"
      }
      \BlankLine
      let $G$ be the list of pairs $[ i, gen[i] ], i \in S$;
      \BlankLine
      \Return MapOfFinSets( S, G, T );
\end{algorithm}

We can now create finite concrete categories with objects not starting from 1, to demonstrate that
\texttt{ConcreteCategoryForCAP( [ [,,,5,6,4], [,,,7,8,9], [,,,,,,8,9,7] ] )} and\\
\texttt{ConcreteCategoryForCAP( [ [2,3,1], [4,5,6], [,,,5,6,4] ] )} yield
equivalent categories, i.e. their underlying quivers are the same and they give the same category of representations.

\begin{algorithm}\capstart
    \caption{\texttt{RightQuiverFromConcreteCategory}}\label{algo:RightQuiverFromConcreteCategory}
	\SetKwInput{Input}{Input~}
	\SetKwInput{Output}{Output~}
	\Input{~a finite concrete category $C$ with $n$ objects}
	\Output{~the right quiver $q(n)$}
	\BlankLine
	let $Obj$ be the set of objects of $C$\;
	let $n := Length(Obj)$\;
	let $gMor$ be the set of generating morphisms of $C$\;
	let $A$ be the empty set and let $i := 1$\;
	\ForEach{morphism $mor$ in $gMor$}{
	    let $A_{i,1}$ be the position of $Source( mor )$ in $Obj$\;
	    let $A_{i,2}$ be the position of $Range( mor )$ in $Obj$\;
	    let $i := i+1$\;
	}
	\BlankLine
	let $q$ be the right quiver with vertices $\{1,\dots,n\}$ and arrows $A$.
	\BlankLine
	\Return q\;
\end{algorithm}

\begin{algorithm}[H]\capstart
    \caption{\texttt{RelationsOfEndomorphisms}}\label{algo:RelationsOfEndomorphisms}
	\SetKwInput{Input}{~Input}
	\SetKwInput{Output}{~Output}
	\Input{~a commutative ring $k$ and a finite concrete category $C$}
	\Output{~the endomorphism relations of the category $C$ given as generators of an ideal of the path algebra}
	\BlankLine
	$q := \mathtt{RightQuiverFromConcreteCategory}(C)$\;
	$kq := \mathtt{PathAlgebra}(k, q)$\;
	$gMor := \mathtt{SetOfGeneratingMorphisms}(C)$\;
	$A := \mathtt{Arrows}(q)$\;
	$relsEndo := \emptyset$\;
	\For{$i = 1, \dots, \mathtt{Length}(gMor)$}{
	    let $mor := gMor_i$\\
	    \If{\Not{$\mathtt{IsEndomorphism}(mor)$}}{
		continue\;
	    }
	    $m := 0$\;
	    $mpowers := \emptyset$\;
	    $foundEqual := \mathtt{false}$\;
	    \While{$mor^{m}\notin mpowers$}{
		let $n := 1$\;
	    	$npowers := \emptyset$\;
		\While{$\Not{foundEqual}$}{
		    \If{$mor^{(m+n)} = mor^{m}$}{
		    	Add the relation $kq.(A_{i})^{(m+n)}-kq.(A_{i})^{m}$ to relsEndo\;
		    	foundEqual := true\;
		    }
		    Add $mor^{(m+n)}$ to mpowers\;
		    n := n+1\;
		}
		Add $mor^{m}$ to mpowers\;
		m := m+1\;
	    }
	}
	\Return{relsEndo}\;
\end{algorithm}

\begin{algorithm}\capstart
   \caption{\texttt{Algebroid}}\label{algo:Algebroid}
      \SetKwInput{Input}{~Input}
      \SetKwInput{Output}{~Output}
      \Input{~a commutative ring $k$ and a finite concrete category $C$}
      \Output{~the $k$-linear closure of the category $C$ over the commutative ring $k$}
      \BlankLine
      
      \Return{}\;
\end{algorithm}