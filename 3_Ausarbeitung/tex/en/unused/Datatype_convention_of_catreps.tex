Since the goal of this thesis is a translation of the package \texttt{catreps} by Peter Webb et al. into CAP, this section is
a short overview of the package catreps.

\blockquote[\cite{[Webb2020]}]{In this package a category is stored as a concrete category (i.e. a category where the objects are sets and
morphisms are maps of sets).
A category is stored as a record (cat, say) with fields cat.objects, cat.generators, cat.domain, cat.codomain.
Each object in the list cat.object is a set, and each morphism in the list of generator morphisms cat.generators
is stored as a mapping of sets, which we notate as the list of its values.}

\begin{Verbatim}[commandchars=!@|,fontsize=\small,frame=single,label=Example]
  !gapprompt@gap>| !gapinput@c3c3 := ConcreteCategory( [ [2,3,1], [4,5,6], [,,,5,6,4] ] );|
  rec( codomain := [ 1, 2, 2 ], domain := [ 1, 1, 2 ],
       generators := [ [ 2, 3, 1 ], [ 4, 5, 6 ], [ ,,, 5, 6, 4 ] ],
       objects := [ [ 1, 2, 3 ], [ 4, 5, 6 ] ], operations := rec(  ) )
\end{Verbatim}

\noindent The list of values as seen in the example above may be easy to type in, but does have its disadvantages: If for example you want to store the
morphism that maps the set $\{9\}$ to itself, i.e. the identity morphism $1_{\{9\}}$, you first have to write the eight commas that are not part of that
morphism definition \texttt{ [ ,,,,,,,,9 ] } and you might make a mistake by forgetting one comma.
Another issue is that the source object of a morphism \texttt{gen} is only implicitly given by those list entries \texttt{i} for which
\texttt{ IsBound( gen[i] ) = true }.

Using instead \texttt{MapOfFinSets} in \textsc{Cap} solves both of these issues, and it lets us use a different model for concrete categories in \textsc{Cap},
i.e. that of a subcategory of \texttt{FinSets}, for which we already have an implementation in \textsc{Cap}. 
Another advantages of this method is that a \texttt{MapOfFinSets} can cache known properties about itself:

\begin{Verbatim}[commandchars=!@|,fontsize=\small,frame=single,label=Example]
  !gapprompt@gap>| !gapinput@S := FinSet( [1,2,3] );|
  <An object in FinSets>
  !gapprompt@gap>| !gapinput@T := FinSet( [4,5,6] );|
  <An object in FinSets>
  !gapprompt@gap>| !gapinput@map1 := MapOfFinSets( S, [ [1,1], [2,2], [3,3] ], S );|
  <A morphism in FinSets>
  !gapprompt@gap>| !gapinput@IsAutomorphism( map1 );|
  true
  !gapprompt@gap>| !gapinput@map1;|
  <An automorphism in FinSets>
\end{Verbatim}

Going further in the cited example,\\
\blockquote[\cite{[Webb2020]}]{The following constructs a representation:}
\begin{Verbatim}[commandchars=!@|,fontsize=\small,frame=single,label=Example]
  !gapprompt@gap>| !gapinput@one:=One(GF(3));;|
  !gapprompt@gap>| !gapinput@d:=[[1,1,0,0,0],[0,1,1,0,0],[0,0,1,0,0],[0,0,0,1,1],[0,0,0,0,1]]*one;;|
  !gapprompt@gap>| !gapinput@e:=[[0,1,0,0],[0,0,1,0],[0,0,0,0],[0,1,0,1],[0,0,1,0]]*one;;|
  !gapprompt@gap>| !gapinput@f:=[[1,1,0,0],[0,1,1,0],[0,0,1,0],[0,0,0,1]]*one;;|
  !gapprompt@gap>| !gapinput@nine:=CatRep(c3c3,[d,e,f],GF(3));|
  rec(
category := rec( generators := [ [ 2, 3, 1 ], [ 4, 5, 6 ], [ ,,, 5, 6, 4 ] ]
, operations := rec( ), objects := [ [ 1, 2, 3 ], [ 4, 5, 6 ] ],
domain := [ 1, 1, 2 ], codomain := [ 1, 2, 2 ] ),
genimages := [ [ [ Z(3)^0, Z(3)^0, 0*Z(3), 0*Z(3), 0*Z(3) ],
[ 0*Z(3), Z(3)^0, Z(3)^0, 0*Z(3), 0*Z(3) ],
[ 0*Z(3), 0*Z(3), Z(3)^0, 0*Z(3), 0*Z(3) ],
[ 0*Z(3), 0*Z(3), 0*Z(3), Z(3)^0, Z(3)^0 ],
[ 0*Z(3), 0*Z(3), 0*Z(3), 0*Z(3), Z(3)^0 ] ],
[ [ 0*Z(3), Z(3)^0, 0*Z(3), 0*Z(3) ], [ 0*Z(3), 0*Z(3), Z(3)^0, 0*Z(3) ]
, [ 0*Z(3), 0*Z(3), 0*Z(3), 0*Z(3) ],
[ 0*Z(3), Z(3)^0, 0*Z(3), Z(3)^0 ],
[ 0*Z(3), 0*Z(3), Z(3)^0, 0*Z(3) ] ],
[ [ Z(3)^0, Z(3)^0, 0*Z(3), 0*Z(3) ], [ 0*Z(3), Z(3)^0, Z(3)^0, 0*Z(3) ]
, [ 0*Z(3), 0*Z(3), Z(3)^0, 0*Z(3) ],
[ 0*Z(3), 0*Z(3), 0*Z(3), Z(3)^0 ] ] ], field := GF(3),
dimension := [ 5, 4 ] )
\end{Verbatim}

we see that \texttt{catreps} works with \textsc{Gap} matrices directly whereas with \textsc{Cap} we use \texttt{HomalgMatrix} and
\texttt{RingsForHomalg} which lets us delegate computation to faster computer algebra systems like \texttt{Singular} or \texttt{Magma}.
What is also noticable is the big chunk of output we get as a result of \texttt{CatRep(c3c3,[d,e,f],GF(3))}. In \textsc{Cap} we hide the output
and give a short description of the resulting object or morphism, and use the \texttt{Display} function to display the whole result.

All in all, there are plenty of reasons to change to \textsc{Cap}. In the meantime, in order to still support inputs in the convention of
\texttt{catreps}, I wrote a converter function \funcref{ConvertToMapOfFinSets}.