
\begin{example}[Naturality equations in $C_{3}C_{3}$]

As an example, let $\mathcal{A}$ be our concrete category $C_{3}C_{3}$ from \ref{ex:quiver_q(2)[a:1->1,b:1->2,c:2->2]}
with objects $1,2$ and morphisms
$a : 1\rightarrow 1,\, b : 1 \rightarrow 2,\, c : 2 \rightarrow 2$. Then the following rectangle describes how the functors $F$, $G$, $K$ and $C$
act on $\mathcal{A}$:

\[
\begin{tikzcd}
K1 \arrow["Ka"', loop, distance=2em, in=125, out=55] \arrow[d, "Kb"] \arrow[r, "\kappa_{1}"] & F1 \arrow["Fa"', loop, distance=2em, in=125, out=55] \arrow[d, "Fb"] \arrow[r, "\eta_{1}"] & G1 \arrow["Ga"', loop, distance=2em, in=125, out=55] \arrow[d, "Gb"] \arrow[r, "\varepsilon_{1}"] & C1 \arrow[d, "Cb"] \arrow["Ca"', loop, distance=2em, in=125, out=55] \\
K2 \arrow["Kc"', loop, distance=2em, in=305, out=235] \arrow[r, "\kappa_{2}"]                & F2 \arrow["Fc"', loop, distance=2em, in=305, out=235] \arrow[r, "\eta_{2}"]                & G2 \arrow["Gc"', loop, distance=2em, in=305, out=235] \arrow[r, "\varepsilon_{2}"]                & C2 \arrow["Cc"', loop, distance=2em, in=305, out=235]               
\end{tikzcd}
\]

\noindent Naturality of $\kappa$, $\eta$ and $\varepsilon$ gives the following equations:\\
\begin{minipage}{.04\textwidth}
\phantom{}
\end{minipage}
\begin{minipage}{.28\textwidth}
\begin{subequations}
\begin{align}
Ka \kappa_{1} &=\label{eq:naturality_equations_first}
 \kappa_{1} Fa \\
Kb \kappa_{2} &= \kappa_{1} Fb \\
Kc \kappa_{2} &= \kappa_{2} Fc
\end{align}
\end{subequations}
\end{minipage}
\begin{minipage}{.04\textwidth}
\phantom{}
\end{minipage}
\begin{minipage}{.28\textwidth}
\begin{subequations}
\begin{align}
Fa \eta_{1} &= \eta_{1} Ga \\
Fb \eta_{2} &= \eta_{1} Gb \\
Fc \eta_{2} &= \eta_{2} Gc
\end{align}
\end{subequations}
\end{minipage}
\begin{minipage}{.04\textwidth}
\phantom{}
\end{minipage}
\begin{minipage}{.28\textwidth}
\begin{subequations}
\begin{align}
Ga \varepsilon_{1} &= \varepsilon_{1} Ca \\
Gb \varepsilon_{2} &= \varepsilon_{1} Cb \\
Gc \varepsilon_{2} &=\label{eq:naturality_equations_last}
 \varepsilon_{2} Cc
\end{align}
\end{subequations}
\end{minipage}
\begin{minipage}{.04\textwidth}
\phantom{}
\end{minipage}

\noindent which together with $\kappa \, \eta = 0_{K,G}$ and $\eta\, \varepsilon = 0_{F,C}$ gives the following equations:
\begin{subequations}
\begin{alignat}{4}
Ka\kappa_{1}\eta_{1} &= \kappa_{1}Fa\eta_{1} &= \kappa_{1}\eta_{1}Ga &= 0 : K1 \rightarrow G1 \\
Kb\kappa_{2}\eta_{2} &= \kappa_{1}Fb\eta_{2} &= \kappa_{1}\eta_{1}Gb &= 0 : K1 \rightarrow G2 \\
Kc\kappa_{2}\eta_{2} &= \kappa_{2}Fc\eta_{2} &= \kappa_{2}\eta_{2}Gc &= 0 : K2 \rightarrow G2
\end{alignat}
\end{subequations}
\begin{subequations}
\begin{alignat}{4}
Ka\kappa_{1}\eta_{1} &= \kappa_{1}Fa\eta_{1} &= \kappa_{1}\eta_{1}Ga &= 0 : K1 \rightarrow G1 \\
Kb\kappa_{2}\eta_{2} &= \kappa_{1}Fb\eta_{2} &= \kappa_{1}\eta_{1}Gb &= 0 : K1 \rightarrow G2 \\
Kc\kappa_{2}\eta_{2} &= \kappa_{2}Fc\eta_{2} &= \kappa_{2}\eta_{2}Gc &= 0 : K2 \rightarrow G2
\end{alignat}
\end{subequations}

To determine the morphism $Ka$, we are writing $\kappa_{1} : K1 \rightarrow F1$ and $Ka \kappa_{1} : K1 \rightarrow F1$ in one diagram:
\[
\begin{tikzcd}
K1 \arrow[rr, "\kappa_{1}", hook]                                 &                                   & F1 \arrow[rr, "\eta_{1}"] &  & G1 \\
                                                                  & K1 \arrow[ru, "\kappa_{1}", hook] &                           &  &    \\
K1 \arrow[ru, "Ka"] \arrow[uu, "(Ka\kappa_{1}/\eta_{1})", dashed] &                                   &                           &  &   
\end{tikzcd}
\]
Then we have two competing morphisms, $\kappa_{1}$ and $Ka \kappa_{1}$ both from $K1 \rightarrow F1$, and both with
\begin{align*}
\kappa_{1}\eta_{1} &= 0_{K1,G1} \\
Ka \kappa_{1}\eta_{1} &= 0_{K1,G1}
\end{align*}
Then we get a unique morphism $(Ka\kappa_{1}/\eta_{1}) := \mathrm{KernelLift}(\eta_{1},Ka\kappa_{1})$ such that
\begin{align*}
Ka\kappa_{1} &= (Ka\kappa_{1}/\eta_{1})\kappa_{1}
\end{align*}
But since $\kappa_{1}$ is a monomorphism, we get that
\begin{align*}
Ka = (Ka\kappa_{1}/\eta_{1}) = \mathrm{KernelLift}(\eta_{1},Ka\kappa_{1})
\end{align*}
which a priori doesn't help calculating $Ka$, since it stands on both sides of the equation, but with equation
\eqref{eq:naturality_equations_first} we get
\begin{align*}
Ka\kappa_{1} &= \kappa_{1} Fa \\
\end{align*}
and therefore
\begin{align*}
Ka = \mathrm{KernelLift}(\eta_{1},\kappa_{1} Fa)
\end{align*}
\end{example}