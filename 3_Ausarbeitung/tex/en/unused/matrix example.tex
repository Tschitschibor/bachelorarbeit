
\begin{example}[A number example for the direct sum in $\kmat$]
The direct sum of $m = 3, n = 5 \in \kmat_{0}$ is the object $m+n = 8 \in \kmat_{0}$ together with the following morphisms:
\begin{align*}
\pi_{1} = \begin{pmatrix}
1 \ampersand \cdot \ampersand \cdot \\
\cdot \ampersand 1 \ampersand \cdot \\
\cdot \ampersand \cdot \ampersand 1 \\
\cdot \ampersand \cdot \ampersand \cdot \\
\cdot \ampersand \cdot \ampersand \cdot \\
\cdot \ampersand \cdot \ampersand \cdot \\
\cdot \ampersand \cdot \ampersand \cdot \\
\cdot \ampersand \cdot \ampersand \cdot
\end{pmatrix},
\pi_{2} = \begin{pmatrix}
\cdot \ampersand \cdot \ampersand \cdot \ampersand \cdot \ampersand \cdot \\
\cdot \ampersand \cdot \ampersand \cdot \ampersand \cdot \ampersand \cdot \\
\cdot \ampersand \cdot \ampersand \cdot \ampersand \cdot \ampersand \cdot \\
1 \ampersand \cdot \ampersand \cdot \ampersand \cdot \ampersand \cdot \\
\cdot \ampersand 1 \ampersand \cdot \ampersand \cdot \ampersand \cdot \\
\cdot \ampersand \cdot \ampersand 1 \ampersand \cdot \ampersand \cdot \\
\cdot \ampersand \cdot \ampersand \cdot \ampersand 1 \ampersand \cdot \\
\cdot \ampersand \cdot \ampersand \cdot \ampersand \cdot \ampersand 1
\end{pmatrix}, 
\begin{array}{rr}
\iota_{1} &= \begin{pmatrix}
1 \ampersand \cdot \ampersand \cdot \ampersand \cdot \ampersand \cdot \ampersand \cdot \ampersand \cdot \ampersand \cdot \\
\cdot \ampersand 1 \ampersand \cdot \ampersand \cdot \ampersand \cdot \ampersand \cdot \ampersand \cdot \ampersand \cdot \\
\cdot \ampersand \cdot \ampersand 1 \ampersand \cdot \ampersand \cdot \ampersand \cdot \ampersand \cdot \ampersand \cdot
\end{pmatrix} \\
\\
\iota_{2} &= \begin{pmatrix}
\cdot \ampersand \cdot \ampersand \cdot \ampersand 1 \ampersand \cdot \ampersand \cdot \ampersand \cdot \ampersand \cdot \\
\cdot \ampersand \cdot \ampersand \cdot \ampersand \cdot \ampersand 1 \ampersand \cdot \ampersand \cdot \ampersand \cdot \\
\cdot \ampersand \cdot \ampersand \cdot \ampersand \cdot \ampersand \cdot \ampersand 1 \ampersand \cdot \ampersand \cdot \\
\cdot \ampersand \cdot \ampersand \cdot \ampersand \cdot \ampersand \cdot \ampersand \cdot \ampersand 1 \ampersand \cdot \\
\cdot \ampersand \cdot \ampersand \cdot \ampersand \cdot \ampersand \cdot \ampersand \cdot \ampersand \cdot \ampersand 1
\end{pmatrix}
\end{array}
\end{align*}\\
\noindent One can easily verify that $\pi_{1} \iota_{1} + \pi_{2} \iota_{2} = 1_{(3+5)} = 1_{8}$ and e.g. $\iota_{1} \pi_{2} = 0_{3,5}$.\\

\noindent \begin{minipage}[t]{.5\textwidth}
For $t = 4$, $\tau = (\tau_{1}, \tau_{2})$ defined as
\begin{align*}
\tau_{1} = \begin{pmatrix}
1 \ampersand 2 \ampersand 2 \\
4 \ampersand 3 \ampersand 1 \\
\cdot \ampersand 1 \ampersand \cdot \\
1 \ampersand 2 \ampersand 1
\end{pmatrix},
\tau_{2} = \begin{pmatrix}
1 \ampersand 1 \ampersand 2 \ampersand 2 \ampersand 3 \\
3 \ampersand 4 \ampersand 4 \ampersand 5 \ampersand 5 \\
6 \ampersand 6 \ampersand 7 \ampersand 7 \ampersand 8 \\
8 \ampersand 9 \ampersand 9 \ampersand 4 \ampersand 4
\end{pmatrix}
\end{align*}
we get the matrix
\begin{align*}
u_{\text{in}}(\tau) = \begin{pmatrix}
1 \ampersand 2 \ampersand 2 \ampersand 1 \ampersand 1 \ampersand 2 \ampersand 2 \ampersand 3 \\
4 \ampersand 3 \ampersand 1 \ampersand 3 \ampersand 4 \ampersand 4 \ampersand 5 \ampersand 5 \\
\cdot \ampersand 1 \ampersand \cdot \ampersand 6 \ampersand 6 \ampersand 7 \ampersand 7 \ampersand 8 \\
1 \ampersand 2 \ampersand 1 \ampersand 8 \ampersand 9 \ampersand 9 \ampersand 4 \ampersand 4
\end{pmatrix}
\end{align*}
\end{minipage}
\begin{minipage}[t]{.5\textwidth}
and for $r = 2$, $\rho = (\rho_{1}, \rho_{2})$ we get the matrix
\begin{align*}
\begin{array}{rr}
\rho_{1} &= \begin{pmatrix}
\cdot \ampersand 1 \\
1 \ampersand 1 \\
2 \ampersand 2
\end{pmatrix} \\
\\
\rho_{2} &= \begin{pmatrix}
4 \ampersand 5 \\
7 \ampersand \cdot \\
\cdot \ampersand 5 \\
\cdot \ampersand \cdot \\
1 \ampersand 1
\end{pmatrix}
\end{array}
u_{\text{out}}(\rho) = \begin{pmatrix}
\cdot \ampersand 1 \\
1 \ampersand 1 \\
2 \ampersand 2 \\
4 \ampersand 5 \\
7 \ampersand \cdot \\
\cdot \ampersand 5 \\
\cdot \ampersand \cdot \\
1 \ampersand 1
\end{pmatrix}
\end{align*}
\end{minipage}\\

\noindent One can easily verify that e.g. $\tau_{1} = u_{\mathrm{in}}(\tau) \pi_{1}$ and $\rho_{2} = \iota_{2} u_{\mathrm{out}}(\rho)$.
\end{example}
