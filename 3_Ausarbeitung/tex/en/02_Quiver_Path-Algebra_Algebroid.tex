% mainfile: ../main.tex

\section{Directed Quiver, Path Algebra and the Algebroid}

\begin{definition}{(Quiver)}\\
A \ul{directed quiver} $q$ consists of a class of \ul{objects} (or \ul{vertices}) $q_{0} = \textup{Obj}\,q$ and
a class of \ul{morphisms} (or \ul{arrows}) $q_{1} = \textup{Mor}\,q$ together with two defining maps
\[
\begin{tikzcd}[column sep=small]
{s,t\colon q_{1}} \arrow[rr, shift left = 0.7ex] \arrow[rr, shift right = 0.7ex] & & q_{0}
\end{tikzcd}
\]
$s$ called \ul{source} and $t$ called \ul{target}.
\end{definition}

\noindent In \texttt{QPA}, the objects are coded by natural numbers, so the first object is $1$, the second $2$ and so on. The arrows are denoted by
small letters $a, b, c$ and so on. There is a difference between \texttt{RightQuiver} and \texttt{LeftQuiver} in that the right quiver is \ul{right-oriented}
(that is, the convention for order in multiplication of paths is the opposite of that used for left-oriented quivers).

\begin{example}\label{q(2)}{(Quiver with 2 objects and 3 morphisms)}\\
\[
\begin{tikzcd}
1 \arrow["a"', loop, distance=2em, in=305, out=235] \arrow[rr, "b"] &  & 2 \arrow["c"', loop, distance=2em, in=305, out=235]
\end{tikzcd}
\]
The objects of this quiver $q$ are $q_{0} = \{1, 2\}$, and the morphisms are $q_{1} = \{a, b, c\}$ with\\
$s (a) = 1 = t (a)$, $s (c) = 2 = t (c)$ and $s (b) = 1, t (b) = 2$.\\

\noindent In \texttt{QPA} this quiver is encoded as \texttt{q(2)[a:1->1,b:1->2,c:2->2]} where the first \texttt{(2)} in parentheses stands for the total
number of objects.

Category closure of quiver

\end{example}

Quiver -> CAT: U: forget 1, forget composition

search U^(-1)

Beispiel für Adjunktion

\begin{lemma} Let Q be a quiver. If there is a path of length at least $\abs{Q_{0}}$, then there are cyclic paths,
and thus infinitely many paths.
\end{lemma}
\begin{proof}
Assume that there exists a path of length greater or equal to $\abs{Q_{0}}$. Then there exists a path of length |$Q_{0}$|, say
$alpha_{n}\cdots alpha_{1}$. Consider the vertices $x_{i}=s(alpha_{i})$ for $1 \leq i \leq n$ and $x_{n+1}=t(alpha_{n})$. Then these
are $n+1$ vertices, thus there has to exist $i<j$ with $x_{i}=x_{j}$. Let $\omega=alpha_{j-1}\cdots alpha_{i}$, this is a path with target and source
$x_{i}=x_{j}$, thus a cyclic path. But then $\omega^{m}$ is a path for any natural number $m$. The path $\omega$ has length $j-i\geq1$, thus
$\omega^{m}$ has length $m(j-i)$. This shows that these paths are pairwise different.
\end{proof}
Path Algebra: