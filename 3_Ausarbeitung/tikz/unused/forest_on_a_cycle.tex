% A forest on a cycle
\begin{example}{(A forest on a cycle)}\label{ex:forest_on_a_cycle}\endnote{(A more general picture to the $\sigma$-Lemma below.
Own recreation in \texttt{tikz}, originally from \cite{[facchini_2019]})}
\begin{tikzpicture}[x=0.5cm,y=0.5cm]
\tikzstyle{cblack}=[circle, fill=black, scale=0.5]

%Nodes
\foreach \place/\x in {{(0,0)/0}, {(-4.5,0)/1}, {(-7,-3)/2}, {(-4.5,-6)/3},
  {(0,-6)/4}, {(2.5,-3)/5},
  {(-4.5,3)/6}, {(-7.5,6)/7}, {(-4.5,6)/8},
  {(0,3)/9}, {(0,6)/10}, {(0,9)/11},
  {(3,3)/12}, {(3,6)/13}, {(3,9)/14},
  {(7.5,6)/15}, {(7.5,9)/16}, {(7.5,12)/17}, {(10.5,9)/18}}
\node[cblack] (a\x) at \place {};

%Arrows
\foreach \i in {0,1,2,3,4,5}
{
  \pgfmathtruncatemacro\result{Mod(\i+1,6)}%
  \draw[->] (a\i) -> (a\result);
}
\path[->] (a7) edge (a6); 
\path[->] (a8) edge (a6) edge (a1);
\path[->] (a11) edge (a10) edge (a9) edge (a0);
\path[->] (a14) edge (a13) edge (a12); 
\path[->] (a12) edge (a0);
\path[->] (a18) edge (a15) (a15) edge (a12);
\path[->] (a17) edge (a16) edge (a15);
%\path[->] (a\x) edge (a\y);
\end{tikzpicture}
\end{example}