\documentclass{article}
\usepackage[utf8]{inputenc}
\usepackage[top=37mm,bottom=37mm,left=27mm,right=27mm]{geometry}

\usepackage{hyperref}
\usepackage{hypcap}

\usepackage{xcolor}

%%% For source code listings
\usepackage{listings}[2013/08/05]
\def \pkgpath {C:/Users/Tibor/AppData/Local/Packages/CanonicalGroupLimited.UbuntuonWindows_79rhkp1fndgsc/LocalState/rootfs/home/user/.gap/pkg}
%\lstloadlanguages{GAP}

%%% For algorithm styles
\usepackage[linesnumbered,ruled]{algorithm2e}

\begin{document}
\tableofcontents\label{toc}
\section{Preface}
\section{Introduction to quivers and category theory}
\section{The category FinSets}

There are algorithms whose sole purpose is to convert data structures, so they are not of much interest to the mathematical theory,
and then there are algorithms that implement our category theoretical calculations, so they are important to our theory.

The algorithms \hyperref[func:ConvertToMapOfFinSets]{\texttt{ConvertToMapOfFinSets}}, 
\hyperref[func:ConcreteCategoryForCAP]{\texttt{ConcreteCategoryForCAP}} and 
\hyperref[func:RightQuiverFromConcreteCategory]{\texttt{RightQuiverFromConcreteCategory}} are
more of the data structure conversion type, while 
\hyperref[func:RelationsOfEndomorphisms]{\texttt{RelationsOfEndomorphisms}}, 
\hyperref[func:Algebroid]{\texttt{Algebroid}},
\hyperref[func:EmbeddingOfSubRepresentation]{\texttt{EmbeddingOfSubRepresentation}} and 
\hyperref[func:WeakDirectSumDecomposition]{\texttt{WeakDirectSumDecomposition}} are also important to our theory.

\subsection{MapOfFinSets}

\begin{algorithm}\capstart
   \caption{\texttt{ConvertToMapOfFinSets}}\label{algo:ConvertToMapOfFinSets}
      \SetKwInput{Input}{Input~}
      \SetKwInput{Output}{Output~}
      \Input{~a list $objects$ of objects in FinSets and a morphism $gen$ given as a list of images in the convention of catreps}
      \Output{~the corresponding map of finite sets from source $S$ to target $T$}
      \BlankLine
      let $T$ be the first object $O \in objects$ such that $gen \cap O \not= \emptyset$\;
      \If{$gen \cap O = \emptyset \, \forall O \in objects$}{
         Error "unable to find target set"
      }
      let $fl$ be the flattening of $objects$ as a list\;
      let $S$ be the sublist of $fl$ according to positions $i$ such that $gen[i]$ is bound\;
      set $S$ to be the first object $O \in objects$ such that $O = S$\;
      \If{$S \not= O \, \forall O \in objects$}{
          Error "unable to find source set"
      }
      \BlankLine
      let $G$ be the list of pairs $[ i, gen[i] ], i \in S$;
      \BlankLine
      \Return MapOfFinSets( S, G, T );
\end{algorithm}

\section{The categories Functor-Categories and Cat-Reps}
\section{Conclusion}
\addcontentsline{toc}{section}{References}
\section*{References}
\appendix
\renewcommand{\thesection}{\Alph{section}}
\section{Implementation in \textsc{Cap}}

\lstlistoflistings\label{lol}

\renewcommand{\lstlistingname}{Procedure}
\lstset{
		basicstyle=\ttfamily\small,
		keywordstyle=\color{red},
		identifierstyle=,
		commentstyle=\color{green!70!black},
		stringstyle=\color{blue},
		showstringspaces=false,
		gobble=2,
		columns=fullflexible,
		tabsize=4,
		numbers=none, 
		frame=single}

%%% Check firstline and lastline for each function after change in the codebase ! ! !
%%% 1st line = InstallMethod
%%% last line = end );

%%% ConvertToMapOfFinSets
\lstinputlisting[
		firstline=8,
		lastline=33, 
		label=func:ConvertToMapOfFinSets,
		caption={$\mathtt{ConvertToMapOfFinSets}$},
		language=GAP
		]{\pkgpath/catreps/gap/CatRepsWithCAP.gi}
From the package \texttt{CatReps} (d$\And$i by Tibor Grün and improved by Mohamed Barakat on 12 May 2020) \\
Back to \hyperref[lol]{Index}

%%% ConcreteCategoryForCAP
\lstinputlisting[
		firstline=36,
		lastline=81, 
		label=func:ConcreteCategoryForCAP,
		caption={$\mathtt{ConcreteCategoryForCAP}$},
		language=GAP
		]{\pkgpath/catreps/gap/CatRepsWithCAP.gi}
From the package \texttt{CatReps} (d$\And$i by Mohamed Barakat on 20 Feb 2020) \\
Back to \hyperref[lol]{Index}
		
%%% RightQuiverFromConcreteCategory
\lstinputlisting[
		firstline=228,
		lastline=255, 
		label=func:RightQuiverFromConcreteCategory,
		caption={$\mathtt{RightQuiverFromConcreteCategory}$},
		language=GAP
		]{\pkgpath/catreps/gap/CatRepsWithCAP.gi}
From the package \texttt{CatReps} (d$\And$i by Tibor Grün on 7 May 2020) \\
Back to \hyperref[lol]{Index}
		
%%% RelationsOfEndomorphisms
\lstinputlisting[
		firstline=156,
		lastline=225, 
		label=func:RelationsOfEndomorphisms,
		caption={$\mathtt{RelationsOfEndomorphisms}$},
		language=GAP
		]{\pkgpath/catreps/gap/CatRepsWithCAP.gi}
From the package \texttt{CatReps} (d$\And$i by Tibor Grün on 7 May 2020) \\
Back to \hyperref[lol]{Index}
		
%%% Algebroid
\lstinputlisting[
		firstline=84,
		lastline=153, 
		label=func:Algebroid,
		caption={$\mathtt{Algebroid}$},
		language=GAP
		]{\pkgpath/catreps/gap/CatRepsWithCAP.gi}
From the package \texttt{CatReps} (template added by Mohamed Barakat on 16 April 2020, finalized by Tibor Grün on 7 May 2020) \\
Back to \hyperref[lol]{Index}
		
%%% YonedaEmbedding 
\lstinputlisting[
		firstline=199,
		lastline=251, 
		label=func:YonedaEmbedding,
		caption={$\mathtt{YonedaEmbedding}$},
		language=GAP
		]{\pkgpath/FunctorCategories/gap/Functors.gi}
From the package \texttt{FunctorCategories} (d$\And$i by Kamal Saleh on 18 May 2020) \\
Back to \hyperref[lol]{Index}

%%% DecomposeOnceByRandomEndomorphism
\lstinputlisting[
		firstline=8,
		lastline=66, 
		label=func:DecomposeOnceByRandomEndomorphism,
		caption={$\mathtt{DecomposeOnceByRandomEndomorphism}$},
		language=GAP
		]{\pkgpath/FunctorCategories/gap/DirectSumDecomposition.gi}
From the package \texttt{FunctorCategories}\\
Back to \hyperref[lol]{Index}
		
%%% WeakDirectSumDecomposition
\lstinputlisting[
		firstline=69,
		lastline=96, 
		label=func:WeakDirectSumDecomposition,
		caption={$\mathtt{WeakDirectSumDecomposition}$},
		language=GAP
		]{\pkgpath/FunctorCategories/gap/DirectSumDecomposition.gi}
From the package \texttt{FunctorCategories}\\
Back to \hyperref[lol]{Index}
		
%%% MorphismOntoSumOfImagesOfAllMorphisms
\lstinputlisting[
		firstline=245,
		lastline=263,
		breaklines=true,
		label=func:MorphismOntoSumOfImagesOfAllMorphisms,
		caption={$\mathtt{MorphismOntoSumOfImagesOfAllMorphisms}$},
		language=GAP
		]{\pkgpath/CategoryConstructor/gap/Tools.gi}
From the package \texttt{CategoryConstructor} (d$\And$i by Mohamed Barakat on 10 April 2020) \\
Back to \hyperref[lol]{Index}
		
%%% EmbeddingOfSumOfImagesOfAllMorphisms
\lstinputlisting[
		firstline=266,
		lastline=276,
		breaklines=true,
		label=func:EmbeddingOfSumOfImagesOfAllMorphisms,
		caption={$\mathtt{EmbeddingOfSumOfImagesOfAllMorphisms}$},
		language=GAP
		]{\pkgpath/CategoryConstructor/gap/Tools.gi}
From the package \texttt{CategoryConstructor} (d$\And$i by Mohamed Barakat on 10 April 2020)\\
Back to \hyperref[lol]{Index}
		
%%% SumOfImagesOfAllMorphisms
\lstinputlisting[
		firstline=279,
		lastline=288,
		breaklines=true,
		label=func:SumOfImagesOfAllMorphisms,
		caption={$\mathtt{SumOfImagesOfAllMorphisms}$},
		language=GAP
		]{\pkgpath/CategoryConstructor/gap/Tools.gi}
From the package \texttt{CategoryConstructor} (d$\And$i by Mohamed Barakat on 10 April 2020)\\
Back to \hyperref[lol]{Index}

		


\end{document}