\documentclass{article}
\usepackage[top=37mm,bottom=37mm,left=27mm,right=27mm]{geometry}

\usepackage[utf8]{inputenc}
\usepackage{fancyvrb}

%%% For captions and references
\usepackage{hyperref}
\usepackage{hypcap}
\newcommand{\Algoref}[1]{%
	\hyperref[algo:#1]{Algorithm~\ref*{algo:#1}}%
}
\newcommand{\algoref}[1]{%
	\hyperref[algo:#1]{Algorithm~\ref*{algo:#1}}%
}
\newcommand{\Funcref}[1]{%
	\hyperref[func:#1]{Function~\ref*{func:#1}}%
}
\newcommand{\funcref}[1]{%
	\hyperref[func:#1]{\texttt{#1}}%
}

%%% For footnotes at end of text
\usepackage{endnotes}
%\let\footnote{\endnote}
%% Heading of endnotes section
\renewcommand*{\notesname}{Annotations}

\makeatletter
\renewcommand*{\enoteheading}{%
   \section*{\notesname%
   \@mkboth{\MakeUppercase{\notesname}}{\MakeUppercase{\notesname}}}%
\mbox{}\par\vskip-\baselineskip}
\makeatother

%%% For quotation
\usepackage{csquotes}

%%% For proper underline
\usepackage{soul}
%\setuldepth{gjpqy}
%\setuldepth\strut
\setuldepth{-1}

%%% Color
\usepackage{xcolor}
\usepackage{color}
\definecolor{FireBrick}{rgb}{0.5812,0.0074,0.0083}
\definecolor{RoyalBlue}{rgb}{0.0236,0.0894,0.6179}
\definecolor{RoyalGreen}{rgb}{0.0236,0.6179,0.0894}
\definecolor{RoyalRed}{rgb}{0.6179,0.0236,0.0894}
\definecolor{LightBlue}{rgb}{0.8544,0.9511,1.0000}
\definecolor{Black}{rgb}{0.0,0.0,0.0}

\definecolor{linkColor}{rgb}{0.0,0.0,0.554}
\definecolor{citeColor}{rgb}{0.0,0.0,0.554}
\definecolor{fileColor}{rgb}{0.0,0.0,0.554}
\definecolor{urlColor}{rgb}{0.0,0.0,0.554}
\definecolor{promptColor}{rgb}{0.0,0.0,0.589}
\definecolor{brkpromptColor}{rgb}{0.589,0.0,0.0}
\definecolor{gapinputColor}{rgb}{0.589,0.0,0.0}
\definecolor{gapoutputColor}{rgb}{0.0,0.0,0.0}

%%  for a long time these were red and blue by default,
%%  now black, but keep variables to overwrite
\definecolor{FuncColor}{rgb}{0.0,0.0,0.0}
%% strange name because of pdflatex bug:
\definecolor{Chapter }{rgb}{0.0,0.0,0.0}
\definecolor{DarkOlive}{rgb}{0.1047,0.2412,0.0064}

%% command for ColorPrompt style examples
\newcommand{\gapprompt}[1]{\color{promptColor}{\bfseries #1}}
\newcommand{\gapbrkprompt}[1]{\color{brkpromptColor}{\bfseries #1}}
\newcommand{\gapinput}[1]{\color{gapinputColor}{#1}}

%%% For source code listings
\usepackage{listings}[2013/08/05]
\input{pfad.tex}
%\lstloadlanguages{GAP}

%%% For algorithm styles
\usepackage[linesnumbered,ruled]{algorithm2e}

%%% Math theorem styles
\usepackage{amsthm}

\newtheorem{thm}{Theorem}[subsection]
\newtheorem{lemma}[thm]{Lemma}
\theoremstyle{definition}
\newtheorem{definition}[thm]{Definition}
\newtheorem{remark}[thm]{Remark}
\newtheorem{example}[thm]{Example}


%%% For Math
\usepackage{amsmath}
\usepackage{amsfonts}
\usepackage{amsbsy}
\usepackage{amsthm}
\usepackage{amssymb}

\usepackage{mathtools}
\usepackage{commath}
\usepackage[sc,osf]{mathpazo}

%%% For arrows and categories
\usepackage[all]{xy}
\usepackage{tikz-cd}

%%% tikz
\usetikzlibrary{positioning}

%%% For dotted box around diagrams
\tikzcdset{
    boxedcd/.style={
        every matrix/.append style={
            draw=black,
            dotted,
            rounded corners,
            #1
        },
    },
}

%%% For dotted arrows in math and in text
%% dottedrightarrow
\makeatletter
\newbox\dottedrightarrow@box
\setbox\dottedrightarrow@box\hbox
  {%
    \begin{tikzpicture}
      \draw[dotted,->] (0,0) -- (1.5em,0);
    \end{tikzpicture}%
  }
\newcommand*\dottedrightarrow
  {\relax\ifmmode\expandafter\dottedrightarrow@m\else\expandafter\dottedrightarrow@t\fi}
\newcommand*\dottedrightarrow@t[1][1.5em]
  {\resizebox{#1}{!}{\raisebox{.5ex}{\usebox\dottedrightarrow@box}}}
\newcommand*\dottedrightarrow@m[1][]
  {%
    \if\relax\detokenize{#1}\relax
      \mathchoice% values are trial and error based\ldots
        {\dottedrightarrow@t}
        {\dottedrightarrow@t}
        {\dottedrightarrow@t[1.1em]}
        {\dottedrightarrow@t[0.9em]}%
    \else
      \dottedrightarrow@t[#1]%
    \fi
  }
\makeatother
\let\olddottedrightarrow\dottedrightarrow
\renewcommand{\dottedrightarrow}{\raisebox{-.2em}{\olddottedrightarrow}}
%% dottedleftarrow
\makeatletter
\newbox\dottedleftarrow@box
\setbox\dottedleftarrow@box\hbox
  {%
    \begin{tikzpicture}
      \draw[dotted,<-] (0,0) -- (1.5em,0);
    \end{tikzpicture}%
  }
\newcommand*\dottedleftarrow
  {\relax\ifmmode\expandafter\dottedleftarrow@m\else\expandafter\dottedleftarrow@t\fi}
\newcommand*\dottedleftarrow@t[1][1.5em]
  {\resizebox{#1}{!}{\raisebox{.5ex}{\usebox\dottedleftarrow@box}}}
\newcommand*\dottedleftarrow@m[1][]
  {%
    \if\relax\detokenize{#1}\relax
      \mathchoice% values are trial and error based\ldots
        {\dottedleftarrow@t}
        {\dottedleftarrow@t}
        {\dottedleftarrow@t[1.1em]}
        {\dottedleftarrow@t[0.9em]}%
    \else
      \dottedleftarrow@t[#1]%
    \fi
  }
\makeatother
\let\olddottedleftarrow\dottedleftarrow
\renewcommand{\dottedleftarrow}{\raisebox{-.2em}{\olddottedleftarrow}}


%%% For some big dots
\makeatletter
\newcommand*{\bigcdot}{}% Check if undefined
\DeclareRobustCommand*{\bigcdot}{%
  \mathbin{\mathpalette\bigcdot@{}}%
}
\newcommand*{\bigcdot@scalefactor}{.5}
\newcommand*{\bigcdot@widthfactor}{1.15}
\newcommand*{\bigcdot@}[2]{%
  % #1: math style
  % #2: unused
  \sbox0{$#1\vcenter{}$}% math axis
  \sbox2{$#1\cdot\m@th$}%
  \hbox to \bigcdot@widthfactor\wd2{%
    \hfil
    \raise\ht0\hbox{%
      \scalebox{\bigcdot@scalefactor}{%
        \lower\ht0\hbox{$#1\bullet\m@th$}%
      }%
    }%
    \hfil
  }%
}
\makeatother



\begin{document}
\tableofcontents\label{toc}
\section{Preface}

\section{Introduction to quivers and category theory}
% mainfile: ../main.tex

This section serves two purposes: On the one hand, it is an introduction to quivers and category theory. On the other hand it introduces
concrete categories which we want to represent, and all the additional constructions that are needed to that goal.

\subsection{Quivers}
In this section, we first want to define the category \textbf{Quiv} and how it is the prototype for the category \textbf{Cats}.
In order to describe the category \textbf{Quiv} of quivers, we first have to define what a category is and for this we need
the definition of a quiver. Lateron we will revisit this definition as we can define quivers as the objects in the quiver category \textbf{Quiv}.

\begin{definition}{(Quiver)}\label{def:quiver}\\
A \ul{directed graph} or \ul{quiver} $q$ consists of a class of \ul{objects} (or \ul{vertices}) $q_{0} = \textup{Obj}\,q$ and
a class of \ul{morphisms} (or \ul{arrows}) $q_{1} = \textup{Mor}\,q$ together with two defining maps
\[
\begin{tikzcd}[column sep=small]
{s,t\colon q_{1}} \arrow[rr, shift left = 0.7ex] \arrow[rr, shift right = 0.7ex] & & q_{0}
\end{tikzcd}
\]
$s$ called \ul{source} and $t$ called \ul{target}.
\end{definition}

In the next definition we are giving a new characterization for $q_{1}$ by looking at all arrows between two fixed objects.

\begin{definition}{(Hom-set of a (locally) small quiver)}\label{def:hom_set}
\renewcommand{\labelenumi}{(\theenumi)}
\begin{enumerate}
\item Given two objects $M, N \in q_{0}$ we write $\textup{Hom}_{q}(M,N)$ or $q(M,N)$ for the fiber
$(s,t)^{-1} (\{(M,N)\})$ of the product map 
\begin{tikzcd}[column sep=small]
(s, t) : q_{1} \arrow[rr] &  & q_{0} \times q_{0} 
\end{tikzcd} over the pair $(M,N) \in q_{0} \times q_{0}$.
This is the class of all morphisms with source $= M$ and target $= N$.
We indicate this by writing
\begin{tikzcd}[column sep=small]
\varphi : M \arrow[rr] &  & N
\end{tikzcd} or 
\begin{tikzcd}[column sep=small]
M \arrow[rr,"\varphi"] &  & N.
\end{tikzcd} Hence $q_{1}$ is the disjoint union $\bigcup\limits^{\bigcdot}_{M,N \in q_{0}} \textup{Hom}_{q}(M,N) = q_{1}$.
As usual we define $\textup{End}_{q}(M):= \textup{Hom}_{q}(M,M)$.
\item If the class $\textup{Hom}_{q}(M,N)$ is a \ul{set} for all pairs $(M,N)$ then we call the quiver \ul{locally small}.
We therefore talk about \ul{Hom-sets}.
If additionally, $q_{0}$ is a set, then the quiver is called \ul{small}.
\item A quiver with a finite set of objects and a finite set of morphisms is called a \ul{finite} quiver.
\end{enumerate}
\end{definition}

When we don't assume the category to be locally small, but still talk about its hom-sets, we mean the class of morphisms,
if we don't explicitly use the fact that it's a set of morphisms.

\begin{example}\label{q(2)}{(Quiver with 2 objects and 3 morphisms)}\\
\[
\begin{tikzcd}
1 \arrow["a"', loop, distance=2em, in=305, out=235] \arrow[rr, "b"] &  & 2 \arrow["c"', loop, distance=2em, in=305, out=235]
\end{tikzcd}
\]
The objects of this quiver $q$ are $q_{0} = \{1, 2\}$, and the morphisms are $q_{1} = \{a, b, c\}$ with\\
$s (a) = 1 = t (a)$, $s (c) = 2 = t (c)$ and $s (b) = 1, t (b) = 2$.\\
\noindent Thus $\textup{End}_{q}(1) = \{a\}, \textup{End}_{q}(2) = \{c\}$ and $\textup{Hom}_{q}(1,2) = \{b\}$ whereas
$\textup{Hom}_{q}(2,1)=\emptyset$.\\

\noindent In \texttt{QPA} this quiver is encoded as \texttt{q(2)[a:1->1,b:1->2,c:2->2]} where the first \texttt{(2)} in parentheses stands for the total
number of objects.
\end{example}

\begin{definition}{(Composable arrows; path in a quiver)}\label{def:path}\endnote{(ref. \ref{[leit4]} 4.1)}
Let $q$ be a quiver.
\begin{enumerate}
\renewcommand{\labelenumi}{(\theenumi)}
\item We say two arrows $a, b \in q_{1}$ are \ul{composable} if $t(a) = s(b)$ or $t(b) = s(a)$. In this case we can write a
sequence of composable arrows $p = a_{1}a_{2}\cdots a_{n}$ where $t(a_{i}) = s(a_{i+1})$ for $i=1,\dots,n-1$.
We call this sequence a \ul{path} from $s(a_{1})$ to $t(a_{n})$ and the integer $n \in \mathbb{Z}_{\geq0}$ the \ul{length} $l(p)$ of the path $p$.
Although it may not be an arrow, we can define the source and target of a path $p = a_{1}\cdots a_{n}$ as $s(p) := s(a_{1})$ and $t(p) := t(a_{n})$.
Then again we define two paths $p$ and $q$ as composable, if $t(p) = s(q)$ (or $t(q) = s(p)$) and we call $pq$ (or $qp$) the \ul{concatenation} or
\ul{composition} of the two paths. We can identify each arrow again as a path of length 1.
A path $p = a_{1}\cdots a_{n}$ with $s(a_{1}) = t(a_{n})$, i.e. $s(p) = t(p)$, is called \ul{cyclic}.
\item For an endomorphism $a \in \textup{End}_{q}(M)$ we write $a^{n}$ for $aa \cdots a$ ($n$ times).
\item In the case of $n=0$ an \ul{empty path} whose source and target are the vertex $i \in q_{0}$ is called the \ul{trivial path at $i$} and
is denoted $e_{i}$. Note that the composition of paths $e_{i}e_{i}$ has length zero starting at $i$ therefore $e_{i}^{2}=e_{i}$,
in other words, each $e_{i}$ is an \ul{idempotent}.
\end{enumerate}
\end{definition}

\begin{lemma}\label{la:cyclic_paths}
Let Q be a quiver. If there is a path of length at least $\abs{Q_{0}}$, then there are cyclic paths,
and thus infinitely many paths.\cite{[leit4]}
\end{lemma}
\begin{proof}
Assume that there exists a path of length greater or equal to $\abs{Q_{0}}$. Then there exists a path of length $n = \abs{Q_{0}}$, say
$\alpha_{1}\cdots \alpha_{n}$. Consider the vertices $x_{i}=s(\alpha_{i})$ for $1 \leq i \leq n$ and $x_{n+1}=t(\alpha_{n})$. Then these
are $n+1$ vertices, thus there has to exist $i<j$ with $x_{i}=x_{j}$. Let $\omega=\alpha_{i}\cdots \alpha_{j-1}$, this is a path with source and target
$x_{i}=x_{j}$, thus a cyclic path. But then $\omega^{m}$ is a path for any natural number $m$. The path $\omega$ has length $j-i\geq1$, thus
$\omega^{m}$ has length $m(j-i)$. This shows that these paths are pairwise different.
\end{proof}

\begin{example}{(A quiver with no cycles)}\\
\[
\begin{tikzcd}
2 \arrow[rrrr, "\psi"] \arrow[rrrrddd, "\psi\rho", pos=0.3] &  &  &  &
3 \arrow[ddd, "\rho"] \\
 &  &  &  & \\
 &  &  &  & \\
1 \arrow[uuu, "\varphi"] \arrow[rrrruuu, "\varphi\psi", pos=0.3] \arrow[rrrr, "\varphi\psi\rho" '] &  &  &  & 4
\end{tikzcd}
\]
The longest path $1\rightarrow2\rightarrow3\rightarrow4$ has length 3. If after the object $4$ another arrow would go to either $1,2,3$ or $4$ itself,
we would have a cyclic path and thus infinitely many paths.
\end{example}

\begin{definition}{(Path algebra of a quiver)}\label{def:path_algebra}\endnote{(from \ref{[leit4]} 4.1 )}
Let $\Bbbk$ be a field. For a quiver $Q$ let $\Bbbk Q$ be the vector space with basis the set of all paths in $Q$, together with the
following multiplication: if $w, w'$ are paths, let $ww'$ be the concatenation of $w$ and $w'$ if they are composable, and the zero vector
otherwise, and extend this multiplication bilinearly to $\Bbbk Q$. We call $\Bbbk Q$ the \ul{path algebra} of the quiver $Q$.
\end{definition}

Note that the addition of two paths $w + w'$ doesn't necessarily yield a path as result, but instead an abstract element of the
path algebra, that you can't easily see in the quiver.

\begin{lemma}\label{la:path_algebra_is_ass_algebra}\endnote{(from \ref{[leit4]} 4.1 )}
For a quiver $Q$ and a field $\Bbbk$, the path algebra $\Bbbk Q$ is an associative $\Bbbk$-algebra.
\end{lemma}
\begin{proof}
Let $w, w', w''$ be paths. Then both $(ww')w''$ and $w(w'w'')$ are the concatenation of $w$ on the left,
$w'$ in the middle and $w''$ on the right, in case both conditions $t(w) = s(w')$ and $t(w') = s(w'')$ are satisfied, and
otherwise the zero element (since $(ww')0 = 0, 0(w'w'') = 0$, according to bilinearity).\\
Since the multiplication was defined on a basis and extended bilinearly, the axioms of an algebra are clearly satisfied.
\end{proof}

\begin{lemma}\label{la:unit_in_path_algebra}
If the set of vertices of a quiver $Q_{0}$ is finite, then $\Bbbk Q$ has a unit element $\sum_{x\in Q_{0}} e_{x}$. In this case, $\Bbbk Q$ is a unital ring.
\end{lemma}
\begin{proof}
Let $e := \sum_{x\in Q_{0}} e_{x}$. Let $w$ be a path with $s(w) = x$ and $t(w) = y$, then $e_{x}w = w$ and $e_{z}w = 0$ for all $z \neq x$,
thus $ew = e_{x}w + \sum_{z\neq x} e_{z}w = w + 0 = w$. Similarly, $we_{y} = w$ and $we_{z} = 0$ for $z \neq x$.
\end{proof}

\subsection{Categories}

\begin{definition}{(Category)}\label{def:category}\\
\noindent A \ul{category} $\mathcal{C}$ is a quiver with two further maps:
\begin{enumerate}
\renewcommand{\labelenumi}{(id)}
\item The \ul{identity map} $1_{( )}$ mapping every object $X \in\mathcal{C}_{0}$ to its \ul{identity morphism} $1_{X}$:
\[
\begin{tikzcd}[column sep=small]
\mathcal{C}_{0} \arrow[rr,"1"] &  & \mathcal{C}_{1}
\end{tikzcd}
\]
\renewcommand{\labelenumi}{($\mu$)}
\item And for any two \ul{composable} morphisms $\varphi$ and $\psi \in \mathcal{C}_{1}$, i.e. with $t(\varphi) = s(\psi)$, the
\ul{composition map} $\mu$, which maps $\varphi, \psi \in \mathcal{C}_{1}\times\mathcal{C}_{1}$ to $\mu(\varphi,\psi) \in \mathcal{C}_{1}$ which
we also write as $\varphi\psi$. 
\[
\begin{tikzcd}[column sep=small]
\mathcal{C}_{1} \times \mathcal{C}_{1} \arrow[rr,"\mu"] &  & \mathcal{C}_{1}
\end{tikzcd}
\]
\end{enumerate}
\noindent The defining properties for $1$ and $\mu$ are:
\renewcommand{\labelenumi}{(\theenumi)}
\begin{enumerate}
\item $s(1_{M}) = M = t(1_{M})$, i.e.\\
$1_{M} \in \textup{End}_{\mathcal{C}} \forall M \in \mathcal{C}$.

\item $s(\varphi\psi) = s(\varphi)$ and\\
$t(\varphi\psi) = t(\psi)$\\
for all composable morphisms $\varphi, \psi \in \mathcal{C}$.
\[
\begin{tikzcd}[column sep=small]
\mu : \textup{Hom}_{\mathcal{C}}(M,L) \times \textup{Hom}_{\mathcal{C}}(L,N) \arrow[rr] &  & \textup{Hom}_{\mathcal{C}}(M,N)
\end{tikzcd}
\]
\item \label{associativity_of_composition} \begin{minipage}{.55\textwidth} $(\varphi\psi)\rho = \varphi(\psi\rho)$ \hfill{} [associativity of composition]\end{minipage}
\begin{minipage}{.45\textwidth}\phantom{}\end{minipage}
\item \label{unit_property} \begin{minipage}{.55\textwidth} $1_{s(\varphi)}\varphi = \varphi = \varphi1_{t(\varphi)}$ \hfill{} [unit property]\end{minipage}
\begin{minipage}{.45\textwidth}\phantom{}\end{minipage}\\
The identity is a left and right unit of the composition.
\end{enumerate}
\end{definition}

\noindent So with categories you always answer the four questions
\begin{itemize}\label{category_questions}
\item What are the objects? (which includes the question What are the identity morphisms?)
\item What are the morphisms?
\item How do you compose morphisms?
\item Why is the composition associative?
\end{itemize}

\subsection{Functors}

Categories are themselves objects in the category of categories, which leads to a question: What is a morphism between categories?

\begin{definition}{(Functor)}\label{def:functor}\\
\noindent A \ul{functor} $F : \mathcal{C} \rightarrow \mathcal{D}$, between categories $\mathcal{C}$ and $\mathcal{D}$, consists of the
following data:

\begin{itemize}
\item An object $Fc\in\mathcal{D}_{0}$, for each object $c \in \mathcal{C}_{0}$.
\item A function $Ff : Fc \rightarrow Fc' \in \mathcal{D}_{1}$, for each morphism $f : c \rightarrow c' \in \mathcal{C}_{1}$, so that the
source and target of $Ff$ are, respectively, equal to $F$ applied to the source or target of $f$, in other words,
$s(Ff) = Fs(f)$ and $t(Ff) = Ft(f)$.
\end{itemize}

\noindent The assignments are required to satisfy the following two \ul{functoriality axioms}:
\begin{itemize}\label{functoriality}
\item For any composable pair $f, g \in \mathcal{C}_{1}, Fg \cdot Ff = F(g \cdot f)$.
\item For each object $c \in \mathcal{C}_{0}, F(1_{c}) = 1_{Fc}$.
\end{itemize}

Put concisely, a functor consists of a mapping on objects and a mapping on morphisms that preserves all of the structure of a category,
namely domains and codomains, composition, and identities.
\end{definition}

\noindent So with functors you always answer the four questions
\begin{itemize}\label{four_functor_questions}
\item How does it work on objects?
\item How does it work on morphisms?
\item Why does it respect composition?
\item Why does it respect identity morphisms?
\end{itemize}

We have already seen an example for a functor in definition \ref{def:hom_set} where we defined the hom-set $\textup{Hom}(M,N)$ between two
objects $M$ and $N$. There are two ways to leave blank one of the objects and thus define the 

\begin{example}{(partial Hom-functor)}\label{ex:hom_functor}
Let $\mathcal{C}$ be a category and $P \in \mathcal{C}_{0}$ any object. The \ul{Hom-functor}, also called \ul{partial Hom-functor},
\begin{enumerate}
\item $\textup{Hom}(P,-)$ is a functor from $\mathcal{C}$ to $\mathcal{C}_{1}$ where objects in $\mathcal{C}_{1}$ are the hom-sets 
$\textup{Hom}(P,N)$, and morphisms are maps from one hom-set to another.
$\textup{Hom}(P,-)$ works on objects by mapping the object $N \in \mathcal{C}_{0}$ to
the hom-set $\textup{Hom}(P,N) \in \mathcal{C}_{1}$.
$\textup{Hom}(P,-)$ works on morphisms by mapping the morphism $(f : M \rightarrow N ) \in \mathcal{C}_{1}$ to the transformation
$\textup{Hom}(P,f) : \textup{Hom}(P,M) \rightarrow \textup{Hom}(P,N); \varphi \mapsto \varphi f$, so for every morphism
$\varphi \in \textup{Hom}(P,M)$, you post-compose $f \in \textup{Hom}(M,N)$ to get a new morphism $\varphi f \in \textup{Hom}(P,N)$.

\item $\textup{Hom}(-,P)$ is a functor from $\mathcal{C}$ to $\mathcal{C}_{1}$ where objects in $\mathcal{C}_{1}$ are the hom-sets 
$\textup{Hom}(N,P)$, and morphisms are maps from one hom-set to another.
$\textup{Hom}(-,P)$ works on objects by mapping the object $N \in \mathcal{C}_{0}$ to
the hom-set $\textup{Hom}(N,P) \in \mathcal{C}_{1}$.
$\textup{Hom}(-,P)$ works on morphisms by mapping the morphism $(f : M \rightarrow N ) \in \mathcal{C}_{1}$ to the transformation
$\textup{Hom}(f,P) : \textup{Hom}(N,P) \rightarrow \textup{Hom}(M,P); \varphi \mapsto f\varphi$, so for every morphism
$\varphi \in \textup{Hom}(N,P)$, you pre-compose $f \in \textup{Hom}(M,N)$ to get a new morphism $f\varphi \in \textup{Hom}(M,P)$.
\end{enumerate}

The important difference between these two functors was how they worked on morphisms. If in both cases we take a morphism
$f : M \rightarrow N$ as given, then we have to arrange the source and target for $\textup{Hom}(P,f)$ and $\textup{Hom}(f,P)$
according to the post-composition and pre-composition. Thus if we wanted $\textup{Hom}(f,P)$ to be defined by pre-composition
$\varphi \mapsto f\varphi$, then we were forced to invert $M$ and $N$ as source and target to get 
$\textup{Hom}(f,P): \textup{Hom}(N,P) \rightarrow \textup{Hom}(M,P)$. 
This process of inverting source and target is caught in the following definition.
\end{example}

\begin{definition}{(covariant / contravariant functor)}\endnote{(Def 1.3.5. in \cite{[context]}, p. 17 (35/258))}\\
The way we defined a functor in definition \ref{def:functor} was in the \ul{covariant} way.\\
A \ul{contravariant} functor $F : \mathcal{C} \rightarrow \mathcal{D}$ works on objects the same way as a covariant one, i.e.
an object $Fc \in \mathcal{D}_{0}$ for each object $c \in \mathcal{C}_{0}$. For morphisms on the other hand, we have
a morphism $F f : Fc' \rightarrow Fc \in \mathcal{D}_{1}$ for each morphism $f : c \rightarrow c' \in \mathcal{C}_{1}$, so that
$s(F f) = F t(f)$ and $t(F f) = F s(f)$.
The \ul{functoriality axioms} are also inverted for a contravariant functor:
For any composable pair, $f, g \in \mathcal{C}_{1}$, $F f \cdot F g = F(g \cdot f)$.
For the identity morphisms, it is again the same as in the covariant case:
For each object $c \in \mathcal{C}_{0}$, $F(1_{c}) = 1_{Fc}$.
\end{definition}

In the following definitions, we define different subclasses of functors. These adjectives often come in opposite pairs, so that you may be
tempted to think, duality lets you just swap all the adjectives for the opposite ones, but be careful there. E.g. when 
$\textup{Hom}(P,-)$ is a \ul{covariant}, \ul{left-exact} functor, the opposite $\textup{Hom}(-,P)$ is a \ul{contravariant}, but still \ul{left-exact} functor.
But their respective \ul{right-exactedness} is equivalent to dual concepts concerning \ul{projective} and \ul{injective} objects.

Limiten 

(mit Beispielen / dual)
Kernel

Pullback

Terminal object

Equalizer

\begin{definition}{(Exact functor)}\label{def:exact_functor}\endnote{(Def 4.5.9. in \cite{[context]}, p. 139 (157/258))}\\
A functor is \ul{right exact} or \ul{finitely cocontinuous} if it preserves finite colimits, and \ul{left exact} or \ul{finitely continuous} if it preserves finite limits.
\end{definition}

\begin{remark}
Without going into the details of defining what a limit and a colimit is, and with \ul{pullbacks} and \ul{pushouts} as specific kinds of
finite limits or colimits, and with the following proposition characterizing monomorphisms and epimorphisms,
we can give a definition for exact functor that is useful enough for our purposes.\endnote{(For a more on exact functors see above footnote,
on limits and colimits the same \cite{[context]}, chapter 3, pages 73 (91/258) onward, on pullback and pushout Def 3.1.15 p. 78 / Ex. 3.1.22, p. 80 f)}
\end{remark}

\begin{lemma}\label{prop:mono_pullback}
A morphism $f : a \rightarrow b$ is a monomorphism if and only if
the pullback of $f$ and $f$ exists and is $a$, together with the identity maps $1_{a} : a \rightarrow a$.
In other words, $f : a \rightarrow b$ is a monomorphism if and only if the commutative square
\[
\begin{tikzcd}
a \arrow[r, "1_{a}"] \arrow[d, "1_{a}"'] & a \arrow[d, "f"] \\
a \arrow[r, "f"]                         & b               
\end{tikzcd}
\]
is a pullback square.\endnote{(Cited from \cite{[Annoying Precision]})}

A dual statement exists for epimorphisms and pushouts, which are finite colimits.
\end{lemma}

\begin{corollary}{(from \ref{prop:mono_pullback})}\label{cor:preserve_mono_epi}

\begin{enumerate}
\item Being a monomorphism is a “limit property”: more precisely, any functor which preserves pullbacks
(in particular any functor which preserves finite limits, in particular any functor which preserves all limits)
preserves monomorphisms.
\item Being an epimorphism is a “colimit property”: more precisely, any functor which preserves pushouts
(in particular any functor which preserves finite colimits, in particular any functor which preserves all colimits)
preserves epimorphisms.\endnote{(Cited from \cite{[Annoying Precision]},
after pullback square and pushout square respectively)}
\end{enumerate}
\end{corollary}

\begin{lemma}
For functors between Abelian categories, left/right exactness is equivalent to preserving monos/epis.
\end{lemma}

\begin{lemma}\label{la:hom_functor_left_exact}
The hom functors $\textup{Hom}(P,-)$ and $\textup{Hom}(-,P)$ from \ref{ex:hom_functor} are left exact, i.e. respect monos.
\begin{proof}
Let $f : M \rightarrow N \in \mathcal{C}_{1}$ be a monomorphism, and let $O \in \mathcal{C}_{0}$ be any object.
Let $\mathfrak{g} : \textup{Hom}(P,N) \rightarrow \textup{Hom}(P,O); \varphi \mapsto \mathfrak{g}(\varphi)$
and $\mathfrak{h} : \textup{Hom}(P,N) \rightarrow \textup{Hom}(P,O); \varphi \mapsto \mathfrak{h}(\varphi)$
such that $\textup{Hom}(P,f) \cdot \mathfrak{g} : \textup{Hom}(P,M) \rightarrow \textup{Hom}(P,O); \psi \mapsto \mathfrak{g}(\psi f)$
and  $\textup{Hom}(P,f) \cdot \mathfrak{h} : \textup{Hom}(P,M) \rightarrow \textup{Hom}(P,O); \psi \mapsto \mathfrak{h}(\psi f)$
yield the same morphism, i.e. $\forall \psi \in \textup{Hom}(P,M), \mathfrak{g}(\psi f) = \mathfrak{h}(\psi f)$.
We want to show that - under the assumption that $f : M \rightarrow N$ was a monomorphism, already $\mathfrak{g} = \mathfrak{h}$.
TODO
\end{proof}
\end{lemma}

\begin{definition}{(Full functor)}\label{def:full_functor}\endnote{(Def 1.5.7. in \cite{[context]}, p. 30 (48/258))}\\
A functor $F : \mathcal{C} \rightarrow \mathcal{D}$ is \ul{full} if
$\forall x, y \in \mathcal{C}_{0}$, the map $\mathcal{C}(x, y) \rightarrow \mathcal{D}(Fx, Fy)$ is surjective.
\end{definition}

\begin{definition}{(Faithful functor)}\label{def:faithful_functor}\endnote{(ebd.)}\\
A functor $F$ as in \ref{def:full_functor} is \ul{faithful} if
$\forall x, y \in \mathcal{C}_{0}$, the map $\mathcal{C}(x, y) \rightarrow \mathcal{D}(Fx, Fy)$ is injective.
\end{definition}

\begin{definition}{(Essentially surjective on objects)}\label{def:ess_surj_o_o}\endnote{(ebd.)}\\
A functor $F$ as in \ref{def:full_functor} is \ul{essentially surjective on objects} if for every object $d \in \mathcal{D}_{0}$ there
is some $c \in \mathcal{C}_{0}$ such that $d$ is isomorphic to $Fc$.
\end{definition}

\begin{definition}{(Embedding)}\label{def:embedding}\endnote{(Rmk 1.5.8. in \cite{[context]}, p. 31 (49/258))}\\
A faithful functor that is injective on objects is called an \ul{embedding} and identifies the source category
as a subcategory of the target. In this case, faithfulness implies that the functor is (globally) injective on arrows.
\end{definition}

\begin{definition}{(Full embedding / full subcategory)}\label{def:full_fully}\endnote{(ebd.)}\\
A full and faithful functor, called \ul{fully faithful} for short, that is injective on objects defines a \ul{full embedding} of the
source category into the target category. The source then defines a \ul{full subcategory} of the target category.
\end{definition}

% cut-pasted from k-Algebroid.tex
\noindent As we have seen, every category is a quiver, but in general, to become a category, a quiver is lacking identity morphisms
and the composition of morphisms. To be more precise, there is a \ul{functor} $U$ from the \ul{category of categories} $\textup{CAT}$ to the
\ul{category of quivers} $\textup{Quiv}$, called the \ul{underlying quiver} or \ul{forgetful functor}.
\[
\begin{tikzcd}
\textup{Cat} \arrow[rr,"U"] &  & \textup{Quiv}
\end{tikzcd}
\]
mapping every object $M \in \mathcal{C}_{0}$ to the same objects in $q_{0}$, mapping every arrow $\varphi \in \mathcal{C}_{1}$ to 
an arrow $a \in q_{1}$, respecting source and target, but forgetting the special role of the identity morphisms and of the composition morphisms.

\begin{example}{(Free / Forgetful functor)}\label{ex:forgetful_functor}\\
TODO

$Free : \mathbf{Quiv} \rightarrow \mathbf{Cat}$

$U : \mathbf{Cat} \rightarrow \mathbf{Quiv}$
\end{example}

% Was bisher bei Category Closure geschah...
\begin{example}{(Category closure)}\\

\noindent\begin{minipage}{.08\textwidth}
\phantom{}
\end{minipage}
\begin{minipage}{.37\textwidth}
\begin{tikzcd}[boxedcd={inner xsep=1.5em, inner ysep=3em}]
B \arrow[rrrr, "\psi"] &  &  &  & C \arrow[ddd, "\rho"] \\
 &  &  &  & \\
 &  &  &  & \\
A \arrow[uuu, "\varphi"] &  &  &  & D
\end{tikzcd}
\end{minipage}
%
\begin{minipage}{.10\textwidth}
$\xrightarrow{\text{  Free }}$
\end{minipage}
%
\begin{minipage}{.37\textwidth}
\begin{tikzcd}[boxedcd={inner xsep=1.5em, inner ysep=3em}]
B \arrow[rrrr, "\psi"] \arrow[rrrrddd, "\psi\rho", pos=0.3] \arrow["1_{B}"', loop, distance=2em, in=125, out=55] &  &  &  &
C \arrow[ddd, "\rho"] \arrow["1_{C}"', loop, distance=2em, in=125, out=55]\\
 &  &  &  & \\
 &  &  &  & \\
A \arrow[uuu, "\varphi"] \arrow[rrrruuu, "\varphi\psi", pos=0.3] \arrow[rrrr, bend left, "(\varphi\psi)\rho" ', shift right=2]
\arrow[rrrr, "\varphi(\psi\rho)", bend right] \arrow["1_{A}"', loop, distance=2em, in=305, out=235] &  &  &  &
D \arrow["1_{D}"', loop, distance=2em, in=305, out=235]
\end{tikzcd}
\end{minipage}
\begin{minipage}{.08\textwidth}
\phantom{}
\end{minipage}\\

We can think of a quiver as a prototype for a category. That means we can construct the missing data for a category
from a quiver by adding the identity morphisms and the composed arrows.
\end{example}

% to be seen how useful this example is...
\begin{example}{(Underlying quiver)}\\

\noindent\begin{minipage}{.08\textwidth}
\phantom{}
\end{minipage}
\begin{minipage}{.37\textwidth}
\begin{tikzcd}[boxedcd={inner xsep=1.5em, inner ysep=3em}]
2 \arrow[rrrr, "b"] \arrow[rrrrddd, "e", pos=0.3] \arrow["h"', loop, distance=2em, in=125, out=55] &  &  &  &
3 \arrow[ddd, "c"] \arrow["i"', loop, distance=2em, in=125, out=55]\\
 &  &  &  & \\
 &  &  &  & \\
1 \arrow[uuu, "a"] \arrow[rrrruuu, "d", pos=0.3] \arrow[rrrr, bend left, "f" ', shift right=2]
\arrow[rrrr, "f", bend right] \arrow["g"', loop, distance=2em, in=305, out=235] &  &  &  &
4 \arrow["j"', loop, distance=2em, in=305, out=235]
\end{tikzcd}
\end{minipage}
%
\begin{minipage}{.10\textwidth}
$\xleftarrow{\text{   U   }}$
\end{minipage}
%
\begin{minipage}{.37\textwidth}
\begin{tikzcd}[boxedcd={inner xsep=1.5em, inner ysep=3em}]
B \arrow[rrrr, "\psi"] \arrow[rrrrddd, "\psi\rho", pos=0.3] \arrow["1_{B}"', loop, distance=2em, in=125, out=55] &  &  &  &
C \arrow[ddd, "\rho"] \arrow["1_{C}"', loop, distance=2em, in=125, out=55]\\
 &  &  &  & \\
 &  &  &  & \\
A \arrow[uuu, "\varphi"] \arrow[rrrruuu, "\varphi\psi", pos=0.3] \arrow[rrrr, bend left, "(\varphi\psi)\rho" ', shift right=2]
\arrow[rrrr, "\varphi(\psi\rho)", bend right] \arrow["1_{A}"', loop, distance=2em, in=305, out=235] &  &  &  &
D \arrow["1_{D}"', loop, distance=2em, in=305, out=235]
\end{tikzcd}
\end{minipage}
\begin{minipage}{.08\textwidth}
\phantom{}
\end{minipage}\\

\noindent In the category on the left, associativity of composition guaranteed that $(\varphi\psi)\rho = \varphi(\psi\rho)$, so those two arrows
were already the same, so they are mapped to the same arrow $f = U((\varphi\psi)\rho) = U(\varphi(\psi\rho))$ in the quiver on the right.
We didn't have to draw both arrows for $f$, but since they are equal, there is still only one arrow in the hom-set $\textup{Hom}_{q}(1,4)=\{f,f\} = \{f\}$.\\
All the other identities are not preserved under the forgetful functor, e.g. $d$ doesn't know what it has to do with $a$ and $b$ apart from
$s(d) = s(a)$ and $t(d) = t(b)$. Especially the former identity arrows are now just endomorphisms with no defining property.\\
The paths $g^{2}f, gf$ and $fj^{3}$ are all different, while in the category, they all simplify to
$1_{A}1_{A}(\varphi\psi)\rho = 1_{A}(\varphi\psi)\rho = (\varphi\psi)\rho1_{D}1_{D}1_{D} =  (\varphi\psi)\rho$ due to the unit property and associativity.
\end{example}


\subsection{Natural transformations}

With fixed categories $\mathcal{C}$ and $\mathcal{D}$ we can consider functors $F, G \in \textup{Hom}(\mathcal{C},\mathcal{D})$ themselves
as objects in the category $\textup{Hom}(\mathcal{C},\mathcal{D})$ of functors between $\mathcal{C}$ and $\mathcal{D}$. In this \ul{functor category},
the morphisms between two functors are called \ul{natural transformations}.

\begin{definition}{(Natural transformations)}\label{def:natural_transformation}\\
\noindent Given categories $\mathcal{C}$ and $\mathcal{D}$ and functors $F : \mathcal{C} \rightarrow \mathcal{D}$ and
$G : \mathcal{C} \rightarrow \mathcal{D}$, a \ul{natural transformation} $\alpha : F \Rightarrow G$ consists of:
\begin{itemize}
\item a morphism $\alpha_{c} : Fc \rightarrow Gc \in \mathcal{D}_{1}$ for each object $c \in \mathcal{C}_{0}$, the collection of which
define the \ul{components} of the natural transformation, so that, for any morphism $f : c \rightarrow c' \in \mathcal{C}_{1}$, the following
square of morphisms in $\mathcal{D}$
\[\begin{tikzcd}
Fc \arrow[rr, "\alpha_{c}"] \arrow[dd, "Ff"] &  & Gc \arrow[dd, "Gf"] \\
                                             &  &                     \\
Fc' \arrow[rr, "\alpha_{c'}"]                &  & Gc'                
\end{tikzcd}\]

\ul{commutes}, i.e., has a a common composite $Fc \rightarrow Gc' \in \mathcal{D}_{1}$.
\end{itemize}
When each component $\alpha_{c}$ is an isomorphism, we call $\alpha$ a \ul{natural isomorphism}.
\end{definition}

\subsection{The functor category}

\begin{definition}{(The functor category)}\label{def:functor_category}\endnote{(cited from ncatlab \cite{[ncatlab_functor_category]})}\\
Given categories $\mathcal{C}$ and $\mathcal{D}$, the \ul{functor category} - written $\mathcal{D}^{\mathcal{C}}$, $\textup{Hom}(\mathcal{C},
\mathcal{D})$ or $[\mathcal{C}, \mathcal{D}]$ -
is the category whose
\begin{itemize}
\item objects are functors $F : \mathcal{C} \rightarrow \mathcal{D}$
\item morphisms are natural transformations between these functors.
\end{itemize}
Main usage of functor categories is as $\textup{Hom}$ categories in place of hom-sets (comp. \ref{def:hom_set} and \ref{ex:hom_functor}) where
we have much more than a set, namely a whole category of morphisms between two objects (together with the morphisms between morphisms).
\end{definition}


\section{Datatype convention of catreps}
Since the goal of this thesis is a translation of the package \texttt{catreps} by Peter Webb et al. into CAP, this section is
a short overview of the package catreps.

\blockquote[\cite{[Webb2020]}]{In this package a category is stored as a concrete category (i.e. a category where the objects are sets and
morphisms are maps of sets).
A category is stored as a record (cat, say) with fields cat.objects, cat.generators, cat.domain, cat.codomain.
Each object in the list cat.object is a set, and each morphism in the list of generator morphisms cat.generators
is stored as a mapping of sets, which we notate as the list of its values.}

\begin{Verbatim}[commandchars=!@|,fontsize=\small,frame=single,label=Example]
  !gapprompt@gap>| !gapinput@c3c3 := ConcreteCategory( [ [2,3,1], [4,5,6], [,,,5,6,4] ] );|
  rec( codomain := [ 1, 2, 2 ], domain := [ 1, 1, 2 ],
       generators := [ [ 2, 3, 1 ], [ 4, 5, 6 ], [ ,,, 5, 6, 4 ] ],
       objects := [ [ 1, 2, 3 ], [ 4, 5, 6 ] ], operations := rec(  ) )
\end{Verbatim}

\noindent The list of values as seen in the example above may be easy to type in, but does have its disadvantages: If for example you want to store the
morphism that maps the set $\{9\}$ to itself, i.e. the identity morphism $1_{\{9\}}$, you first have to write the eight commas that are not part of that
morphism definition \texttt{ [ ,,,,,,,,9 ] } and you might make a mistake by forgetting one comma.
Another issue is that the source object of a morphism \texttt{gen} is only implicitly given by those list entries \texttt{i} for which
\texttt{ IsBound( gen[i] ) = true }.

Using instead \texttt{MapOfFinSets} in \textsc{Cap} solves both of these issues, and it lets us use a different model for concrete categories in \textsc{Cap},
i.e. that of a subcategory of \texttt{FinSets}, for which we already have an implementation in \textsc{Cap}. 
Another advantages of this method is that a \texttt{MapOfFinSets} can cache known properties about itself:

\begin{Verbatim}[commandchars=!@|,fontsize=\small,frame=single,label=Example]
  !gapprompt@gap>| !gapinput@S := FinSet( [1,2,3] );|
  <An object in FinSets>
  !gapprompt@gap>| !gapinput@T := FinSet( [4,5,6] );|
  <An object in FinSets>
  !gapprompt@gap>| !gapinput@map1 := MapOfFinSets( S, [ [1,1], [2,2], [3,3] ], S );|
  <A morphism in FinSets>
  !gapprompt@gap>| !gapinput@IsAutomorphism( map1 );|
  true
  !gapprompt@gap>| !gapinput@map1;|
  <An automorphism in FinSets>
\end{Verbatim}

Going further in the cited example,\\
\blockquote[\cite{[Webb2020]}]{The following constructs a representation:}
\begin{Verbatim}[commandchars=!@|,fontsize=\small,frame=single,label=Example]
  !gapprompt@gap>| !gapinput@one:=One(GF(3));;|
  !gapprompt@gap>| !gapinput@d:=[[1,1,0,0,0],[0,1,1,0,0],[0,0,1,0,0],[0,0,0,1,1],[0,0,0,0,1]]*one;;|
  !gapprompt@gap>| !gapinput@e:=[[0,1,0,0],[0,0,1,0],[0,0,0,0],[0,1,0,1],[0,0,1,0]]*one;;|
  !gapprompt@gap>| !gapinput@f:=[[1,1,0,0],[0,1,1,0],[0,0,1,0],[0,0,0,1]]*one;;|
  !gapprompt@gap>| !gapinput@nine:=CatRep(c3c3,[d,e,f],GF(3));|
  rec(
category := rec( generators := [ [ 2, 3, 1 ], [ 4, 5, 6 ], [ ,,, 5, 6, 4 ] ]
, operations := rec( ), objects := [ [ 1, 2, 3 ], [ 4, 5, 6 ] ],
domain := [ 1, 1, 2 ], codomain := [ 1, 2, 2 ] ),
genimages := [ [ [ Z(3)^0, Z(3)^0, 0*Z(3), 0*Z(3), 0*Z(3) ],
[ 0*Z(3), Z(3)^0, Z(3)^0, 0*Z(3), 0*Z(3) ],
[ 0*Z(3), 0*Z(3), Z(3)^0, 0*Z(3), 0*Z(3) ],
[ 0*Z(3), 0*Z(3), 0*Z(3), Z(3)^0, Z(3)^0 ],
[ 0*Z(3), 0*Z(3), 0*Z(3), 0*Z(3), Z(3)^0 ] ],
[ [ 0*Z(3), Z(3)^0, 0*Z(3), 0*Z(3) ], [ 0*Z(3), 0*Z(3), Z(3)^0, 0*Z(3) ]
, [ 0*Z(3), 0*Z(3), 0*Z(3), 0*Z(3) ],
[ 0*Z(3), Z(3)^0, 0*Z(3), Z(3)^0 ],
[ 0*Z(3), 0*Z(3), Z(3)^0, 0*Z(3) ] ],
[ [ Z(3)^0, Z(3)^0, 0*Z(3), 0*Z(3) ], [ 0*Z(3), Z(3)^0, Z(3)^0, 0*Z(3) ]
, [ 0*Z(3), 0*Z(3), Z(3)^0, 0*Z(3) ],
[ 0*Z(3), 0*Z(3), 0*Z(3), Z(3)^0 ] ] ], field := GF(3),
dimension := [ 5, 4 ] )
\end{Verbatim}

we see that \texttt{catreps} works with \textsc{Gap} matrices directly whereas with \textsc{Cap} we use \texttt{HomalgMatrix} and
\texttt{RingsForHomalg} which lets us delegate computation to faster computer algebra systems like \texttt{Singular} or \texttt{Magma}.
What is also noticable is the big chunk of output we get as a result of \texttt{CatRep(c3c3,[d,e,f],GF(3))}. In \textsc{Cap} we hide the output
and give a short description of the resulting object or morphism, and use the \texttt{Display} function to display the whole result.

All in all, there are plenty of reasons to change to \textsc{Cap}. In the meantime, in order to still support inputs in the convention of
\texttt{catreps}, I wrote a converter function \funcref{ConvertToMapOfFinSets}.

\section{The category FinSets}

There are algorithms whose sole purpose is to convert data structures, so they are not of much interest to the mathematical theory,
and then there are algorithms that implement our category theoretical calculations, so they are important to our theory.

The algorithms \hyperref[func:ConvertToMapOfFinSets]{\texttt{ConvertToMapOfFinSets}},\\
\hyperref[func:ConcreteCategoryForCAP]{\texttt{ConcreteCategoryForCAP}} and
\hyperref[func:RightQuiverFromConcreteCategory]{\texttt{RightQuiverFromConcreteCategory}}\\
are more of the data structure conversion type, while 
\hyperref[func:RelationsOfEndomorphisms]{\texttt{RelationsOfEndomorphisms}},\\
\hyperref[func:Algebroid]{\texttt{Algebroid}},\\
\hyperref[func:EmbeddingOfSubRepresentation]{\texttt{EmbeddingOfSubRepresentation}}\\
and \hyperref[func:WeakDirectSumDecomposition]{\texttt{WeakDirectSumDecomposition}} are also important to our theory.

\subsection{MapOfFinSets}

\begin{algorithm}\capstart
   \caption{\texttt{ConvertToMapOfFinSets}}\label{algo:ConvertToMapOfFinSets}
      \SetKwInput{Input}{Input~}
      \SetKwInput{Output}{Output~}
      \Input{~a list $objects$ of objects in FinSets and a morphism $gen$ given as a list of images in the convention of catreps}
      \Output{~the corresponding map of finite sets from source $S$ to target $T$}
      \BlankLine
      let $T$ be the first object $O \in objects$ such that $gen \cap O \not= \emptyset$\;
      \If{$gen \cap O = \emptyset \, \forall O \in objects$}{
         Error "unable to find target set"
      }
      let $fl$ be the flattening of $objects$ as a list\;
      let $S$ be the sublist of $fl$ according to positions $i$ such that $gen[i]$ is bound\;
      set $S$ to be the first object $O \in objects$ such that $O = S$\;
      \If{$S \not= O \, \forall O \in objects$}{
          Error "unable to find source set"
      }
      \BlankLine
      let $G$ be the list of pairs $[ i, gen[i] ], i \in S$;
      \BlankLine
      \Return MapOfFinSets( S, G, T );
\end{algorithm}

\section{The categories Functor-Categories and Cat-Reps}
\section{Conclusion}

%% best endnotes are in Westend-Verlag style: 
%% All at the end, sorted by chapter, starting from 1 at each new chapter.

%%% insert endnotes in some way
\begingroup
     \parindent 0pt
     \parskip 2ex
     \def\enotesize{\normalsize}
     \theendnotes
\endgroup 

\addcontentsline{toc}{section}{References}
\input{bib/sources.bib}

\appendix
\renewcommand{\thesection}{\Alph{section}}
\section{Implementation in \textsc{Cap}}

\lstlistoflistings\label{lol}

\renewcommand{\lstlistingname}{Procedure}
\lstset{
		basicstyle=\ttfamily\small,
		keywordstyle=\color{red},
		identifierstyle=,
		commentstyle=\color{green!70!black},
		stringstyle=\color{blue},
		showstringspaces=false,
		gobble=2,
		columns=fullflexible,
		tabsize=4,
		numbers=none, 
		frame=single}

%%% Check firstline and lastline for each function after change in the codebase ! ! !
%%% 1st line = InstallMethod
%%% last line = end );

%%% ConvertToMapOfFinSets
\lstinputlisting[
		firstline=8,
		lastline=33, 
		label=func:ConvertToMapOfFinSets,
		caption={$\mathtt{ConvertToMapOfFinSets}$},
		language=GAP
		]{\pkgpath/catreps/gap/CatRepsWithCAP.gi}
From the package \texttt{CatReps} (d$\And$i by Tibor Grün and improved by Mohamed Barakat on 12 May 2020) \\
Back to \hyperref[lol]{Index}

%%% ConcreteCategoryForCAP
\lstinputlisting[
		firstline=36,
		lastline=81, 
		label=func:ConcreteCategoryForCAP,
		caption={$\mathtt{ConcreteCategoryForCAP}$},
		language=GAP
		]{\pkgpath/catreps/gap/CatRepsWithCAP.gi}
From the package \texttt{CatReps} (d$\And$i by Mohamed Barakat on 20 Feb 2020) \\
Back to \hyperref[lol]{Index}
		
%%% RightQuiverFromConcreteCategory
\lstinputlisting[
		firstline=228,
		lastline=255, 
		label=func:RightQuiverFromConcreteCategory,
		caption={$\mathtt{RightQuiverFromConcreteCategory}$},
		language=GAP
		]{\pkgpath/catreps/gap/CatRepsWithCAP.gi}
From the package \texttt{CatReps} (d$\And$i by Tibor Grün on 7 May 2020) \\
Back to \hyperref[lol]{Index}
		
%%% RelationsOfEndomorphisms
\lstinputlisting[
		firstline=156,
		lastline=225, 
		label=func:RelationsOfEndomorphisms,
		caption={$\mathtt{RelationsOfEndomorphisms}$},
		language=GAP
		]{\pkgpath/catreps/gap/CatRepsWithCAP.gi}
From the package \texttt{CatReps} (d$\And$i by Tibor Grün on 7 May 2020) \\
Back to \hyperref[lol]{Index}
		
%%% Algebroid
\lstinputlisting[
		firstline=84,
		lastline=153, 
		label=func:Algebroid,
		caption={$\mathtt{Algebroid}$},
		language=GAP
		]{\pkgpath/catreps/gap/CatRepsWithCAP.gi}
From the package \texttt{CatReps} (template added by Mohamed Barakat on 16 April 2020, finalized by Tibor Grün on 7 May 2020) \\
Back to \hyperref[lol]{Index}
		
%%% YonedaEmbedding 
\lstinputlisting[
		firstline=199,
		lastline=251, 
		label=func:YonedaEmbedding,
		caption={$\mathtt{YonedaEmbedding}$},
		language=GAP
		]{\pkgpath/FunctorCategories/gap/Functors.gi}
From the package \texttt{FunctorCategories} (d$\And$i by Kamal Saleh on 18 May 2020) \\
Back to \hyperref[lol]{Index}

%%% DecomposeOnceByRandomEndomorphism
\lstinputlisting[
		firstline=8,
		lastline=66, 
		label=func:DecomposeOnceByRandomEndomorphism,
		caption={$\mathtt{DecomposeOnceByRandomEndomorphism}$},
		language=GAP
		]{\pkgpath/FunctorCategories/gap/DirectSumDecomposition.gi}
From the package \texttt{FunctorCategories}\\
Back to \hyperref[lol]{Index}
		
%%% WeakDirectSumDecomposition
\lstinputlisting[
		firstline=69,
		lastline=96, 
		label=func:WeakDirectSumDecomposition,
		caption={$\mathtt{WeakDirectSumDecomposition}$},
		language=GAP
		]{\pkgpath/FunctorCategories/gap/DirectSumDecomposition.gi}
From the package \texttt{FunctorCategories}\\
Back to \hyperref[lol]{Index}
		
%%% MorphismOntoSumOfImagesOfAllMorphisms
\lstinputlisting[
		firstline=245,
		lastline=263,
		breaklines=true,
		label=func:MorphismOntoSumOfImagesOfAllMorphisms,
		caption={$\mathtt{MorphismOntoSumOfImagesOfAllMorphisms}$},
		language=GAP
		]{\pkgpath/CategoryConstructor/gap/Tools.gi}
From the package \texttt{CategoryConstructor} (d$\And$i by Mohamed Barakat on 10 April 2020) \\
Back to \hyperref[lol]{Index}
		
%%% EmbeddingOfSumOfImagesOfAllMorphisms
\lstinputlisting[
		firstline=266,
		lastline=276,
		breaklines=true,
		label=func:EmbeddingOfSumOfImagesOfAllMorphisms,
		caption={$\mathtt{EmbeddingOfSumOfImagesOfAllMorphisms}$},
		language=GAP
		]{\pkgpath/CategoryConstructor/gap/Tools.gi}
From the package \texttt{CategoryConstructor} (d$\And$i by Mohamed Barakat on 10 April 2020)\\
Back to \hyperref[lol]{Index}
		
%%% SumOfImagesOfAllMorphisms
\lstinputlisting[
		firstline=279,
		lastline=288,
		breaklines=true,
		label=func:SumOfImagesOfAllMorphisms,
		caption={$\mathtt{SumOfImagesOfAllMorphisms}$},
		language=GAP
		]{\pkgpath/CategoryConstructor/gap/Tools.gi}
From the package \texttt{CategoryConstructor} (d$\And$i by Mohamed Barakat on 10 April 2020)\\
Back to \hyperref[lol]{Index}

		


\end{document}