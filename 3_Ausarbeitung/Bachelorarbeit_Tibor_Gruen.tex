\documentclass{article}

%\usepackage[top=37mm,bottom=37mm,left=27mm,right=27mm]{geometry}

%% From Gliederung preambel
% left=2.5cm,right=2.5cm,top=2.5cm,bottom=2.5cm

%\usepackage[ % % top 37, bottom 37
  %          top=27mm,bottom=27mm,left=27mm,right=27mm,%
    %        footskip=.6cm]{geometry}
            
\usepackage[ 
  a4paper,
  footskip=0.7cm,
  margin=2.7cm,
  top=1.1cm,
  bottom=1.4cm
]{geometry}            

\def\changemargin#1#2{\list{}{\rightmargin#2\leftmargin#1}\item[]}
\let\endchangemargin=\endlist

\usepackage{microtype}

\usepackage[utf8]{inputenc}
\usepackage[T1]{fontenc}%
\usepackage{fancyvrb}
\usepackage{enumitem}

\usepackage{fixltx2e}%
\usepackage{graphicx}%

% size of table of contents
\usepackage{tocloft}
\renewcommand{\cftsecfont}{\small\bfseries}
\renewcommand{\cftsubsecfont}{\small\mdseries}
\renewcommand{\cftsubsubsecfont}{\small\mdseries}


\usepackage[pdfauthor   = {Tibor\ Gr{\"u}n},
            pdftitle    = {},
            pdfsubject  = {},
            pdfkeywords = {},
            %plainpages
            %breaklinks=true, % still buggy under Linux with \stackrel
            %bookmarks=true,
            %bookmarksopen=true,
            %bookmarksnumbered=true, % produces errors on arXiv.org
            %pagebackref=true,
            hyperindex=true,
            hyperfootnotes=true,
            colorlinks=true,
            %linkcolor=red,
            linkcolor=blue,
            %citecolor=green,
            citecolor=blue,
            %filecolor=magenta,
            filecolor=blue,
            %urlcolor=cyan,
            urlcolor=blue,
            %pdftex=true
            %ps2pdf=true
            %hypertex=true
            ]{hyperref}
\usepackage[all]{hypcap}



%%% For captions and references
%\usepackage{endnotes}
\usepackage{enotez} %[backref]
\newcommand{\Algoref}[1]{%
	\hyperref[algo:#1]{Algorithm~\ref*{algo:#1}}%
}
\newcommand{\algoref}[1]{%
	\hyperref[algo:#1]{Algorithm~\ref*{algo:#1}}%
}
\newcommand{\Funcref}[1]{%
	\hyperref[func:#1]{Function~\ref*{func:#1}}%
}
\newcommand{\funcref}[1]{%
	\hyperref[func:#1]{\texttt{#1}}%
}

%%% For footnotes at end of text
\usepackage{endnotes}

%%% hyperendnotes.sty
\makeatletter
\newif\ifenotelinks
\newcounter{Hendnote}
% Redefining portions of endnotes-package:
\let\savedhref\href
\let\savedurl\url
\def\endnotemark{%
\@ifnextchar[\@xendnotemark{%
\stepcounter{endnote}%
\protected@xdef\@theenmark{\theendnote}%
\protected@xdef\@theenvalue{\number\c@endnote}%
\@endnotemark
}%
}%
\def\@xendnotemark[#1]{%
\begingroup\c@endnote#1\relax
\unrestored@protected@xdef\@theenmark{\theendnote}%
\unrestored@protected@xdef\@theenvalue{\number\c@endnote}%
\endgroup
\@endnotemark
}%
\def\endnotetext{%
\@ifnextchar[\@xendnotenext{%
\protected@xdef\@theenmark{\theendnote}%
\protected@xdef\@theenvalue{\number\c@endnote}%
\@endnotetext
}%
}%
\def\@xendnotenext[#1]{%
\begingroup
\c@endnote=#1\relax
\unrestored@protected@xdef\@theenmark{\theendnote}%
\unrestored@protected@xdef\@theenvalue{\number\c@endnote}%
\endgroup
\@endnotetext
}%
\def\endnote{%
\@ifnextchar[\@xendnote{%
\stepcounter{endnote}%
\protected@xdef\@theenmark{\theendnote}%
\protected@xdef\@theenvalue{\number\c@endnote}%
\@endnotemark\@endnotetext
}%
}%
\def\@xendnote[#1]{%
\begingroup
\c@endnote=#1\relax
\unrestored@protected@xdef\@theenmark{\theendnote}%
\unrestored@protected@xdef\@theenvalue{\number\c@endnote}%
\show\@theenvalue
\endgroup
\@endnotemark\@endnotetext
}%
\def\@endnotemark{%
\leavevmode
\ifhmode
\edef\@x@sf{\the\spacefactor}\nobreak
\fi
\ifenotelinks
\expandafter\@firstofone
\else
\expandafter\@gobble
\fi
{%
\Hy@raisedlink{%
\hyper@@anchor{Hendnotepage.\@theenvalue}{\empty}%
}%
}%
\hyper@linkstart{link}{Hendnote.\@theenvalue}%
\makeenmark
\hyper@linkend
\ifhmode
\spacefactor\@x@sf
\fi
\relax
}%
\long\def\@endnotetext#1{%
\if@enotesopen
\else
\@openenotes
\fi
\immediate\write\@enotes{%
\@doanenote{\@theenmark}{\@theenvalue}%
}%
\begingroup
\def\next{#1}%
\newlinechar='40
\immediate\write\@enotes{\meaning\next}%
\endgroup
\immediate\write\@enotes{%
\@endanenote
}%
}%
\def\theendnotes{%
\immediate\closeout\@enotes
\global\@enotesopenfalse
\begingroup
\makeatletter
\edef\@tempa{`\string>}%
\ifnum\catcode\@tempa=12
\let\@ResetGT\relax
\else
\edef\@ResetGT{\noexpand\catcode\@tempa=\the\catcode\@tempa}%
\@makeother\>%
\fi
\def\@doanenote##1##2##3>{%
\def\@theenmark{##1}%
\def\@theenvalue{##2}%
\par
\smallskip %<-small vertical gap between endnotes
\begingroup
\def\href{\expandafter\savedhref}%
\def\url{\expandafter\savedurl}%
\@ResetGT
\edef\@currentlabel{\csname p@endnote\endcsname\@theenmark}%
\enoteformat
}%
\def\@endanenote{%
\par\endgroup
}%
% Redefine, how numbers are formatted in the endnotes-section:
\renewcommand*\@makeenmark{%
\hbox{\normalfont\@theenmark~}%
}%
% header of endnotes-section
\enoteheading
% font-size of endnotes
\enotesize
\input{\jobname.ent}%
\endgroup
}%
\def\enoteformat{%
\rightskip\z@
\leftskip1.8em
\parindent\z@
\leavevmode\llap{%
\setcounter{Hendnote}{\@theenvalue}%
\addtocounter{Hendnote}{-1}%
\refstepcounter{Hendnote}%
\ifenotelinks
\expandafter\@secondoftwo
\else
\expandafter\@firstoftwo
\fi
{\@firstofone}%
{\hyperlink{Hendnotepage.\@theenvalue}}%
{\makeenmark}%
}%
}%
% stop redefining portions of endnotes-package:
\makeatother
% Toggle switch in order to turn on/off back-links in the
% endnote-section:
\enotelinkstrue
%\enotelinksfalse
%\let\footnote{\endnote}
%% Heading of endnotes section
\renewcommand*{\notesname}{Annotations}

\makeatletter
\renewcommand*{\enoteheading}{%
   \section*{\notesname%
   \@mkboth{\MakeUppercase{\notesname}}{\MakeUppercase{\notesname}}}%
\mbox{}\par\vskip-\baselineskip}
\makeatother



%%% For switching languages in quotes
\usepackage[english]{babel}
%\usepackage[english, german]{babel} %% makes troubles

%%% For quotation
%% english guillemets have to be custom defined in /tex/latex/csquotes/csquotes.cfg
%\usepackage[english = guillemets, autostyle = true,autopunct,csdisplay = true]{csquotes}
\usepackage[autostyle = true,autopunct,csdisplay = true]{csquotes}

%%% For proper underline
\usepackage{soul}
%\setuldepth{gjpqy}
%\setuldepth\strut
\setuldepth{-1}

%%% Color
\usepackage{xcolor}
\usepackage{color}
\definecolor{FireBrick}{rgb}{0.5812,0.0074,0.0083}
\definecolor{RoyalBlue}{rgb}{0.0236,0.0894,0.6179}
\definecolor{RoyalGreen}{rgb}{0.0236,0.6179,0.0894}
\definecolor{RoyalRed}{rgb}{0.6179,0.0236,0.0894}
\definecolor{LightBlue}{rgb}{0.8544,0.9511,1.0000}
\definecolor{Black}{rgb}{0.0,0.0,0.0}

\definecolor{linkColor}{rgb}{0.0,0.0,0.554}
\definecolor{citeColor}{rgb}{0.0,0.0,0.554}
\definecolor{fileColor}{rgb}{0.0,0.0,0.554}
\definecolor{urlColor}{rgb}{0.0,0.0,0.554}
\definecolor{promptColor}{rgb}{0.0,0.0,0.589}
\definecolor{brkpromptColor}{rgb}{0.589,0.0,0.0}
\definecolor{gapinputColor}{rgb}{0.589,0.0,0.0}
\definecolor{gapoutputColor}{rgb}{0.0,0.0,0.0}

%%  for a long time these were red and blue by default,
%%  now black, but keep variables to overwrite
\definecolor{FuncColor}{rgb}{0.0,0.0,0.0}
%% strange name because of pdflatex bug:
\definecolor{Chapter }{rgb}{0.0,0.0,0.0}
\definecolor{DarkOlive}{rgb}{0.1047,0.2412,0.0064}

%% command for ColorPrompt style examples
\newcommand{\gapprompt}[1]{\color{promptColor}{\bfseries #1}}
\newcommand{\gapbrkprompt}[1]{\color{brkpromptColor}{\bfseries #1}}
\newcommand{\gapinput}[1]{\color{gapinputColor}{#1}}

%%% For source code listings
\usepackage{listings}[2013/08/05]
\input{pfad.tex}
%\lstloadlanguages{GAP}

%%% For algorithm styles
\usepackage{xspace}
\usepackage[linesnumbered,ruled]{algorithm2e}
\usepackage{algpseudocode}
\SetKw{Continue}{continue}
\SetKw{Break}{break}
\SetKw{Not}{not\xspace}
\SetKw{AndAlg}{and\xspace}

%%% Math theorem styles
\usepackage{amsthm}

\newtheorem{theorem}{Theorem}[subsection]
\theoremstyle{definition}
\newtheorem{lemma}[theorem]{Lemma}
\newtheorem{corollary}[theorem]{Corollary}
\newtheorem{definition}[theorem]{Definition}
\newtheorem{remark}[theorem]{Remark}
\newtheorem{proposition}[theorem]{Proposition}
\newtheorem{example}[theorem]{Example}
\newtheorem{doctrine}[theorem]{Doctrine}
\newtheorem{computation}[theorem]{Computation}

%%% make equations count from subsection
\usepackage{chngcntr}
\counterwithin{equation}{subsection}

%%% for nested proofs
\newenvironment{subproof}[1][\proofname]{%
  \renewcommand{\qedsymbol}{$\mathbin{/\mkern-6mu/}$}%
  \begin{proof}[#1]%
}{%
  \end{proof}%
}

%%% for nicer Product sign
\newcommand{\invamalg}{\mathbin{\rotatebox[origin=c]{180}{$\amalg$}}}

%%% For Math
\usepackage{amsmath}
\usepackage{amsfonts}
\usepackage{amsbsy}
\usepackage{amssymb}
\usepackage{mathtools}
\usepackage{esvect}
\usepackage{commath}
\usepackage[sc,osf]{mathpazo}

%%% Macros for our recurring categories
\newcommand{\kmat}{\Bbbk\textnormal{-}\mathbf{mat}}
\newcommand{\kAlgebroid}{\Bbbk\textnormal{-}\mathrm{algebroid}}
\newcommand{\Rmat}{R\textnormal{-}\mathbf{mat}}
\newcommand{\HomAkmat}{\mathrm{Hom_{\Bbbk}}(\mathcal{A},\kmat)}
\newcommand{\HomARmat}{\mathrm{Hom_{R}}(\mathcal{A},\Rmat)}
\newcommand{\HomA}{\mathrm{Hom}_{\mathcal{A}}}
\newcommand{\FinSets}{\mathrm{FinSets}}
\newcommand{\Cat}{\mathrm{\textbf{Cat}}}
\newcommand{\Set}{\mathrm{\textbf{Set}}}
\newcommand{\Quiv}{\mathrm{\textbf{Quiv}}}
\newcommand{\kChat}{\widehat{\Bbbk\mathcal{C}}}

%%% Macros for the software packages
\newcommand{\Gap}{\textsc{Gap}\xspace}
\newcommand{\QPA}{\textsc{QPA$2$}\xspace}
\newcommand{\CatReps}{\texttt{CatReps}\xspace} %CAP package CatReps
\newcommand{\catreps}{\texttt{catreps}\xspace} %Peter Webb's catreps
\newcommand{\CAP}{\textsc{CAP}\xspace}
\newcommand{\homalgProject}{\texttt{homalg\_project}\xspace}
\newcommand{\FunctorCategories}{\texttt{FunctorCategories}\xspace}

%%% vel means or in latin, easier to remember
\newcommand{\vel}{\vee}

%%% For arrows and categories
%\usepackage[all]{xy} %%not used anymore
\usepackage{tikz-cd}

%%% For calculations and loops inside tikz and latex
\usepackage{calc}
\usepackage{pgffor}

\newcounter{modresult}
\newcommand*{\themodulo}[2]{%
\setcounter{modresult}{%
#1-(#1/#2)*#2%
}%
#1 mod #2 = \themodresult\par
}

%%% For matrices
\let\ampersand =&

%%% Math operators bold
%\newcommand{\Category}{Category}
% just use /textup{#1} inside math environment instead of redefining every math operator.

%%% tikz
\usetikzlibrary{positioning}
\usetikzlibrary{arrows}

%%% for captions of tikzpictures and other figures
\usepackage{capt-of}

%%% For function restrictions
\newcommand\restrict[1]{\raisebox{-.5ex}{$|$}_{#1}}

%%% For dotted box around diagrams
\tikzcdset{
    boxedcd/.style={
        every matrix/.append style={
            draw=black,
            dotted,
            rounded corners,
            #1
        },
    },
}

%%% For dotted arrows in math and in text
%% dottedrightarrow
\makeatletter
\newbox\dottedrightarrow@box
\setbox\dottedrightarrow@box\hbox
  {%
    \begin{tikzpicture}
      \draw[dotted,->] (0,0) -- (1.5em,0);
    \end{tikzpicture}%
  }
\newcommand*\dottedrightarrow
  {\relax\ifmmode\expandafter\dottedrightarrow@m\else\expandafter\dottedrightarrow@t\fi}
\newcommand*\dottedrightarrow@t[1][1.5em]
  {\resizebox{#1}{!}{\raisebox{.5ex}{\usebox\dottedrightarrow@box}}}
\newcommand*\dottedrightarrow@m[1][]
  {%
    \if\relax\detokenize{#1}\relax
      \mathchoice% values are trial and error based\ldots
        {\dottedrightarrow@t}
        {\dottedrightarrow@t}
        {\dottedrightarrow@t[1.1em]}
        {\dottedrightarrow@t[0.9em]}%
    \else
      \dottedrightarrow@t[#1]%
    \fi
  }
\makeatother
\let\olddottedrightarrow\dottedrightarrow
\renewcommand{\dottedrightarrow}{\raisebox{-.2em}{\,\,\olddottedrightarrow\,\,}}
%% dottedleftarrow
\makeatletter
\newbox\dottedleftarrow@box
\setbox\dottedleftarrow@box\hbox
  {%
    \begin{tikzpicture}
      \draw[dotted,<-] (0,0) -- (1.5em,0);
    \end{tikzpicture}%
  }
\newcommand*\dottedleftarrow
  {\relax\ifmmode\expandafter\dottedleftarrow@m\else\expandafter\dottedleftarrow@t\fi}
\newcommand*\dottedleftarrow@t[1][1.5em]
  {\resizebox{#1}{!}{\raisebox{.5ex}{\usebox\dottedleftarrow@box}}}
\newcommand*\dottedleftarrow@m[1][]
  {%
    \if\relax\detokenize{#1}\relax
      \mathchoice% values are trial and error based\ldots
        {\dottedleftarrow@t}
        {\dottedleftarrow@t}
        {\dottedleftarrow@t[1.1em]}
        {\dottedleftarrow@t[0.9em]}%
    \else
      \dottedleftarrow@t[#1]%
    \fi
  }
\makeatother
\let\olddottedleftarrow\dottedleftarrow
\renewcommand{\dottedleftarrow}{\raisebox{-.2em}{\,\,\olddottedleftarrow\,\,}}

%%% For some big dots (they still don't look very big)
\makeatletter
\newcommand*{\bigcdot}{}% Check if undefined
\DeclareRobustCommand*{\bigcdot}{%
  \mathbin{\mathpalette\bigcdot@{}}%
}
\newcommand*{\bigcdot@scalefactor}{.5}
\newcommand*{\bigcdot@widthfactor}{1.15}
\newcommand*{\bigcdot@}[2]{%
  % #1: math style
  % #2: unused
  \sbox0{$#1\vcenter{}$}% math axis
  \sbox2{$#1\cdot\m@th$}%
  \hbox to \bigcdot@widthfactor\wd2{%
    \hfil
    \raise\ht0\hbox{%
      \scalebox{\bigcdot@scalefactor}{%
        \lower\ht0\hbox{$#1\bullet\m@th$}%
      }%
    }%
    \hfil
  }%
}
\makeatother

\begin{document}

\tableofcontents\label{toc}
\section{Preface}

\section{Introduction to quivers and category theory}
% mainfile: ../main.tex

This section serves two purposes: On the one hand, it is an introduction to quivers and category theory. On the other hand it introduces
concrete categories which we want to represent, and all the additional constructions that are needed to that goal.

\subsection{Quivers}
In this section, we first want to define the category \textbf{Quiv} and how it is the prototype for the category \textbf{Cats}.
In order to describe the category \textbf{Quiv} of quivers, we first have to define what a category is and for this we need
the definition of a quiver. Lateron we will revisit this definition as we can define quivers as the objects in the quiver category \textbf{Quiv}.

\begin{definition}{(Quiver)}\label{def:quiver}\\
A \ul{directed graph} or \ul{quiver} $q$ consists of a class of \ul{objects} (or \ul{vertices}) $q_{0} = \textup{Obj}\,q$ and
a class of \ul{morphisms} (or \ul{arrows}) $q_{1} = \textup{Mor}\,q$ together with two defining maps
\[
\begin{tikzcd}[column sep=small]
{s,t\colon q_{1}} \arrow[rr, shift left = 0.7ex] \arrow[rr, shift right = 0.7ex] & & q_{0}
\end{tikzcd}
\]
$s$ called \ul{source} and $t$ called \ul{target}.
\end{definition}

In the next definition we are giving a new characterization for $q_{1}$ by looking at all arrows between two fixed objects.

\begin{definition}{(Hom-set of a (locally) small quiver)}\label{def:hom_set}
\renewcommand{\labelenumi}{(\theenumi)}
\begin{enumerate}
\item Given two objects $M, N \in q_{0}$ we write $\textup{Hom}_{q}(M,N)$ or $q(M,N)$ for the fiber
$(s,t)^{-1} (\{(M,N)\})$ of the product map 
\begin{tikzcd}[column sep=small]
(s, t) : q_{1} \arrow[rr] &  & q_{0} \times q_{0} 
\end{tikzcd} over the pair $(M,N) \in q_{0} \times q_{0}$.
This is the class of all morphisms with source $= M$ and target $= N$.
We indicate this by writing
\begin{tikzcd}[column sep=small]
\varphi : M \arrow[rr] &  & N
\end{tikzcd} or 
\begin{tikzcd}[column sep=small]
M \arrow[rr,"\varphi"] &  & N.
\end{tikzcd} Hence $q_{1}$ is the disjoint union $\bigcup\limits^{\bigcdot}_{M,N \in q_{0}} \textup{Hom}_{q}(M,N) = q_{1}$.
As usual we define $\textup{End}_{q}(M):= \textup{Hom}_{q}(M,M)$.
\item If the class $\textup{Hom}_{q}(M,N)$ is a \ul{set} for all pairs $(M,N)$ then we call the quiver \ul{locally small}.
We therefore talk about \ul{Hom-sets}.
If additionally, $q_{0}$ is a set, then the quiver is called \ul{small}.
\item A quiver with a finite set of objects and a finite set of morphisms is called a \ul{finite} quiver.
\end{enumerate}
\end{definition}

When we don't assume the category to be locally small, but still talk about its hom-sets, we mean the class of morphisms,
if we don't explicitly use the fact that it's a set of morphisms.

\begin{example}\label{q(2)}{(Quiver with 2 objects and 3 morphisms)}\\
\[
\begin{tikzcd}
1 \arrow["a"', loop, distance=2em, in=305, out=235] \arrow[rr, "b"] &  & 2 \arrow["c"', loop, distance=2em, in=305, out=235]
\end{tikzcd}
\]
The objects of this quiver $q$ are $q_{0} = \{1, 2\}$, and the morphisms are $q_{1} = \{a, b, c\}$ with\\
$s (a) = 1 = t (a)$, $s (c) = 2 = t (c)$ and $s (b) = 1, t (b) = 2$.\\
\noindent Thus $\textup{End}_{q}(1) = \{a\}, \textup{End}_{q}(2) = \{c\}$ and $\textup{Hom}_{q}(1,2) = \{b\}$ whereas
$\textup{Hom}_{q}(2,1)=\emptyset$.\\

\noindent In \texttt{QPA} this quiver is encoded as \texttt{q(2)[a:1->1,b:1->2,c:2->2]} where the first \texttt{(2)} in parentheses stands for the total
number of objects.
\end{example}

\begin{definition}{(Composable arrows; path in a quiver)}\label{def:path}\endnote{(ref. \ref{[leit4]} 4.1)}
Let $q$ be a quiver.
\begin{enumerate}
\renewcommand{\labelenumi}{(\theenumi)}
\item We say two arrows $a, b \in q_{1}$ are \ul{composable} if $t(a) = s(b)$ or $t(b) = s(a)$. In this case we can write a
sequence of composable arrows $p = a_{1}a_{2}\cdots a_{n}$ where $t(a_{i}) = s(a_{i+1})$ for $i=1,\dots,n-1$.
We call this sequence a \ul{path} from $s(a_{1})$ to $t(a_{n})$ and the integer $n \in \mathbb{Z}_{\geq0}$ the \ul{length} $l(p)$ of the path $p$.
Although it may not be an arrow, we can define the source and target of a path $p = a_{1}\cdots a_{n}$ as $s(p) := s(a_{1})$ and $t(p) := t(a_{n})$.
Then again we define two paths $p$ and $q$ as composable, if $t(p) = s(q)$ (or $t(q) = s(p)$) and we call $pq$ (or $qp$) the \ul{concatenation} or
\ul{composition} of the two paths. We can identify each arrow again as a path of length 1.
A path $p = a_{1}\cdots a_{n}$ with $s(a_{1}) = t(a_{n})$, i.e. $s(p) = t(p)$, is called \ul{cyclic}.
\item For an endomorphism $a \in \textup{End}_{q}(M)$ we write $a^{n}$ for $aa \cdots a$ ($n$ times).
\item In the case of $n=0$ an \ul{empty path} whose source and target are the vertex $i \in q_{0}$ is called the \ul{trivial path at $i$} and
is denoted $e_{i}$. Note that the composition of paths $e_{i}e_{i}$ has length zero starting at $i$ therefore $e_{i}^{2}=e_{i}$,
in other words, each $e_{i}$ is an \ul{idempotent}.
\end{enumerate}
\end{definition}

\begin{lemma}\label{la:cyclic_paths}
Let Q be a quiver. If there is a path of length at least $\abs{Q_{0}}$, then there are cyclic paths,
and thus infinitely many paths.\cite{[leit4]}
\end{lemma}
\begin{proof}
Assume that there exists a path of length greater or equal to $\abs{Q_{0}}$. Then there exists a path of length $n = \abs{Q_{0}}$, say
$\alpha_{1}\cdots \alpha_{n}$. Consider the vertices $x_{i}=s(\alpha_{i})$ for $1 \leq i \leq n$ and $x_{n+1}=t(\alpha_{n})$. Then these
are $n+1$ vertices, thus there has to exist $i<j$ with $x_{i}=x_{j}$. Let $\omega=\alpha_{i}\cdots \alpha_{j-1}$, this is a path with source and target
$x_{i}=x_{j}$, thus a cyclic path. But then $\omega^{m}$ is a path for any natural number $m$. The path $\omega$ has length $j-i\geq1$, thus
$\omega^{m}$ has length $m(j-i)$. This shows that these paths are pairwise different.
\end{proof}

\begin{example}{(A quiver with no cycles)}\\
\[
\begin{tikzcd}
2 \arrow[rrrr, "\psi"] \arrow[rrrrddd, "\psi\rho", pos=0.3] &  &  &  &
3 \arrow[ddd, "\rho"] \\
 &  &  &  & \\
 &  &  &  & \\
1 \arrow[uuu, "\varphi"] \arrow[rrrruuu, "\varphi\psi", pos=0.3] \arrow[rrrr, "\varphi\psi\rho" '] &  &  &  & 4
\end{tikzcd}
\]
The longest path $1\rightarrow2\rightarrow3\rightarrow4$ has length 3. If after the object $4$ another arrow would go to either $1,2,3$ or $4$ itself,
we would have a cyclic path and thus infinitely many paths.
\end{example}

\begin{definition}{(Path algebra of a quiver)}\label{def:path_algebra}\endnote{(from \ref{[leit4]} 4.1 )}
Let $\Bbbk$ be a field. For a quiver $Q$ let $\Bbbk Q$ be the vector space with basis the set of all paths in $Q$, together with the
following multiplication: if $w, w'$ are paths, let $ww'$ be the concatenation of $w$ and $w'$ if they are composable, and the zero vector
otherwise, and extend this multiplication bilinearly to $\Bbbk Q$. We call $\Bbbk Q$ the \ul{path algebra} of the quiver $Q$.
\end{definition}

Note that the addition of two paths $w + w'$ doesn't necessarily yield a path as result, but instead an abstract element of the
path algebra, that you can't easily see in the quiver.

\begin{lemma}\label{la:path_algebra_is_ass_algebra}\endnote{(from \ref{[leit4]} 4.1 )}
For a quiver $Q$ and a field $\Bbbk$, the path algebra $\Bbbk Q$ is an associative $\Bbbk$-algebra.
\end{lemma}
\begin{proof}
Let $w, w', w''$ be paths. Then both $(ww')w''$ and $w(w'w'')$ are the concatenation of $w$ on the left,
$w'$ in the middle and $w''$ on the right, in case both conditions $t(w) = s(w')$ and $t(w') = s(w'')$ are satisfied, and
otherwise the zero element (since $(ww')0 = 0, 0(w'w'') = 0$, according to bilinearity).\\
Since the multiplication was defined on a basis and extended bilinearly, the axioms of an algebra are clearly satisfied.
\end{proof}

\begin{lemma}\label{la:unit_in_path_algebra}
If the set of vertices of a quiver $Q_{0}$ is finite, then $\Bbbk Q$ has a unit element $\sum_{x\in Q_{0}} e_{x}$. In this case, $\Bbbk Q$ is a unital ring.
\end{lemma}
\begin{proof}
Let $e := \sum_{x\in Q_{0}} e_{x}$. Let $w$ be a path with $s(w) = x$ and $t(w) = y$, then $e_{x}w = w$ and $e_{z}w = 0$ for all $z \neq x$,
thus $ew = e_{x}w + \sum_{z\neq x} e_{z}w = w + 0 = w$. Similarly, $we_{y} = w$ and $we_{z} = 0$ for $z \neq x$.
\end{proof}

\subsection{Categories}

\begin{definition}{(Category)}\label{def:category}\\
\noindent A \ul{category} $\mathcal{C}$ is a quiver with two further maps:
\begin{enumerate}
\renewcommand{\labelenumi}{(id)}
\item The \ul{identity map} $1_{( )}$ mapping every object $X \in\mathcal{C}_{0}$ to its \ul{identity morphism} $1_{X}$:
\[
\begin{tikzcd}[column sep=small]
\mathcal{C}_{0} \arrow[rr,"1"] &  & \mathcal{C}_{1}
\end{tikzcd}
\]
\renewcommand{\labelenumi}{($\mu$)}
\item And for any two \ul{composable} morphisms $\varphi$ and $\psi \in \mathcal{C}_{1}$, i.e. with $t(\varphi) = s(\psi)$, the
\ul{composition map} $\mu$, which maps $\varphi, \psi \in \mathcal{C}_{1}\times\mathcal{C}_{1}$ to $\mu(\varphi,\psi) \in \mathcal{C}_{1}$ which
we also write as $\varphi\psi$. 
\[
\begin{tikzcd}[column sep=small]
\mathcal{C}_{1} \times \mathcal{C}_{1} \arrow[rr,"\mu"] &  & \mathcal{C}_{1}
\end{tikzcd}
\]
\end{enumerate}
\noindent The defining properties for $1$ and $\mu$ are:
\renewcommand{\labelenumi}{(\theenumi)}
\begin{enumerate}
\item $s(1_{M}) = M = t(1_{M})$, i.e.\\
$1_{M} \in \textup{End}_{\mathcal{C}} \forall M \in \mathcal{C}$.

\item $s(\varphi\psi) = s(\varphi)$ and\\
$t(\varphi\psi) = t(\psi)$\\
for all composable morphisms $\varphi, \psi \in \mathcal{C}$.
\[
\begin{tikzcd}[column sep=small]
\mu : \textup{Hom}_{\mathcal{C}}(M,L) \times \textup{Hom}_{\mathcal{C}}(L,N) \arrow[rr] &  & \textup{Hom}_{\mathcal{C}}(M,N)
\end{tikzcd}
\]
\item \label{associativity_of_composition} \begin{minipage}{.55\textwidth} $(\varphi\psi)\rho = \varphi(\psi\rho)$ \hfill{} [associativity of composition]\end{minipage}
\begin{minipage}{.45\textwidth}\phantom{}\end{minipage}
\item \label{unit_property} \begin{minipage}{.55\textwidth} $1_{s(\varphi)}\varphi = \varphi = \varphi1_{t(\varphi)}$ \hfill{} [unit property]\end{minipage}
\begin{minipage}{.45\textwidth}\phantom{}\end{minipage}\\
The identity is a left and right unit of the composition.
\end{enumerate}
\end{definition}

\noindent So with categories you always answer the four questions
\begin{itemize}\label{category_questions}
\item What are the objects? (which includes the question What are the identity morphisms?)
\item What are the morphisms?
\item How do you compose morphisms?
\item Why is the composition associative?
\end{itemize}

\subsection{Functors}

Categories are themselves objects in the category of categories, which leads to a question: What is a morphism between categories?

\begin{definition}{(Functor)}\label{def:functor}\\
\noindent A \ul{functor} $F : \mathcal{C} \rightarrow \mathcal{D}$, between categories $\mathcal{C}$ and $\mathcal{D}$, consists of the
following data:

\begin{itemize}
\item An object $Fc\in\mathcal{D}_{0}$, for each object $c \in \mathcal{C}_{0}$.
\item A function $Ff : Fc \rightarrow Fc' \in \mathcal{D}_{1}$, for each morphism $f : c \rightarrow c' \in \mathcal{C}_{1}$, so that the
source and target of $Ff$ are, respectively, equal to $F$ applied to the source or target of $f$, in other words,
$s(Ff) = Fs(f)$ and $t(Ff) = Ft(f)$.
\end{itemize}

\noindent The assignments are required to satisfy the following two \ul{functoriality axioms}:
\begin{itemize}\label{functoriality}
\item For any composable pair $f, g \in \mathcal{C}_{1}, Fg \cdot Ff = F(g \cdot f)$.
\item For each object $c \in \mathcal{C}_{0}, F(1_{c}) = 1_{Fc}$.
\end{itemize}

Put concisely, a functor consists of a mapping on objects and a mapping on morphisms that preserves all of the structure of a category,
namely domains and codomains, composition, and identities.
\end{definition}

\noindent So with functors you always answer the four questions
\begin{itemize}\label{four_functor_questions}
\item How does it work on objects?
\item How does it work on morphisms?
\item Why does it respect composition?
\item Why does it respect identity morphisms?
\end{itemize}

We have already seen an example for a functor in definition \ref{def:hom_set} where we defined the hom-set $\textup{Hom}(M,N)$ between two
objects $M$ and $N$. There are two ways to leave blank one of the objects and thus define the 

\begin{example}{(partial Hom-functor)}\label{ex:hom_functor}
Let $\mathcal{C}$ be a category and $P \in \mathcal{C}_{0}$ any object. The \ul{Hom-functor}, also called \ul{partial Hom-functor},
\begin{enumerate}
\item $\textup{Hom}(P,-)$ is a functor from $\mathcal{C}$ to $\mathcal{C}_{1}$ where objects in $\mathcal{C}_{1}$ are the hom-sets 
$\textup{Hom}(P,N)$, and morphisms are maps from one hom-set to another.
$\textup{Hom}(P,-)$ works on objects by mapping the object $N \in \mathcal{C}_{0}$ to
the hom-set $\textup{Hom}(P,N) \in \mathcal{C}_{1}$.
$\textup{Hom}(P,-)$ works on morphisms by mapping the morphism $(f : M \rightarrow N ) \in \mathcal{C}_{1}$ to the transformation
$\textup{Hom}(P,f) : \textup{Hom}(P,M) \rightarrow \textup{Hom}(P,N); \varphi \mapsto \varphi f$, so for every morphism
$\varphi \in \textup{Hom}(P,M)$, you post-compose $f \in \textup{Hom}(M,N)$ to get a new morphism $\varphi f \in \textup{Hom}(P,N)$.

\item $\textup{Hom}(-,P)$ is a functor from $\mathcal{C}$ to $\mathcal{C}_{1}$ where objects in $\mathcal{C}_{1}$ are the hom-sets 
$\textup{Hom}(N,P)$, and morphisms are maps from one hom-set to another.
$\textup{Hom}(-,P)$ works on objects by mapping the object $N \in \mathcal{C}_{0}$ to
the hom-set $\textup{Hom}(N,P) \in \mathcal{C}_{1}$.
$\textup{Hom}(-,P)$ works on morphisms by mapping the morphism $(f : M \rightarrow N ) \in \mathcal{C}_{1}$ to the transformation
$\textup{Hom}(f,P) : \textup{Hom}(N,P) \rightarrow \textup{Hom}(M,P); \varphi \mapsto f\varphi$, so for every morphism
$\varphi \in \textup{Hom}(N,P)$, you pre-compose $f \in \textup{Hom}(M,N)$ to get a new morphism $f\varphi \in \textup{Hom}(M,P)$.
\end{enumerate}

The important difference between these two functors was how they worked on morphisms. If in both cases we take a morphism
$f : M \rightarrow N$ as given, then we have to arrange the source and target for $\textup{Hom}(P,f)$ and $\textup{Hom}(f,P)$
according to the post-composition and pre-composition. Thus if we wanted $\textup{Hom}(f,P)$ to be defined by pre-composition
$\varphi \mapsto f\varphi$, then we were forced to invert $M$ and $N$ as source and target to get 
$\textup{Hom}(f,P): \textup{Hom}(N,P) \rightarrow \textup{Hom}(M,P)$. 
This process of inverting source and target is caught in the following definition.
\end{example}

\begin{definition}{(covariant / contravariant functor)}\endnote{(Def 1.3.5. in \cite{[context]}, p. 17 (35/258))}\\
The way we defined a functor in definition \ref{def:functor} was in the \ul{covariant} way.\\
A \ul{contravariant} functor $F : \mathcal{C} \rightarrow \mathcal{D}$ works on objects the same way as a covariant one, i.e.
an object $Fc \in \mathcal{D}_{0}$ for each object $c \in \mathcal{C}_{0}$. For morphisms on the other hand, we have
a morphism $F f : Fc' \rightarrow Fc \in \mathcal{D}_{1}$ for each morphism $f : c \rightarrow c' \in \mathcal{C}_{1}$, so that
$s(F f) = F t(f)$ and $t(F f) = F s(f)$.
The \ul{functoriality axioms} are also inverted for a contravariant functor:
For any composable pair, $f, g \in \mathcal{C}_{1}$, $F f \cdot F g = F(g \cdot f)$.
For the identity morphisms, it is again the same as in the covariant case:
For each object $c \in \mathcal{C}_{0}$, $F(1_{c}) = 1_{Fc}$.
\end{definition}

In the following definitions, we define different subclasses of functors. These adjectives often come in opposite pairs, so that you may be
tempted to think, duality lets you just swap all the adjectives for the opposite ones, but be careful there. E.g. when 
$\textup{Hom}(P,-)$ is a \ul{covariant}, \ul{left-exact} functor, the opposite $\textup{Hom}(-,P)$ is a \ul{contravariant}, but still \ul{left-exact} functor.
But their respective \ul{right-exactedness} is equivalent to dual concepts concerning \ul{projective} and \ul{injective} objects.

Limiten 

(mit Beispielen / dual)
Kernel

Pullback

Terminal object

Equalizer

\begin{definition}{(Exact functor)}\label{def:exact_functor}\endnote{(Def 4.5.9. in \cite{[context]}, p. 139 (157/258))}\\
A functor is \ul{right exact} or \ul{finitely cocontinuous} if it preserves finite colimits, and \ul{left exact} or \ul{finitely continuous} if it preserves finite limits.
\end{definition}

\begin{remark}
Without going into the details of defining what a limit and a colimit is, and with \ul{pullbacks} and \ul{pushouts} as specific kinds of
finite limits or colimits, and with the following proposition characterizing monomorphisms and epimorphisms,
we can give a definition for exact functor that is useful enough for our purposes.\endnote{(For a more on exact functors see above footnote,
on limits and colimits the same \cite{[context]}, chapter 3, pages 73 (91/258) onward, on pullback and pushout Def 3.1.15 p. 78 / Ex. 3.1.22, p. 80 f)}
\end{remark}

\begin{lemma}\label{prop:mono_pullback}
A morphism $f : a \rightarrow b$ is a monomorphism if and only if
the pullback of $f$ and $f$ exists and is $a$, together with the identity maps $1_{a} : a \rightarrow a$.
In other words, $f : a \rightarrow b$ is a monomorphism if and only if the commutative square
\[
\begin{tikzcd}
a \arrow[r, "1_{a}"] \arrow[d, "1_{a}"'] & a \arrow[d, "f"] \\
a \arrow[r, "f"]                         & b               
\end{tikzcd}
\]
is a pullback square.\endnote{(Cited from \cite{[Annoying Precision]})}

A dual statement exists for epimorphisms and pushouts, which are finite colimits.
\end{lemma}

\begin{corollary}{(from \ref{prop:mono_pullback})}\label{cor:preserve_mono_epi}

\begin{enumerate}
\item Being a monomorphism is a “limit property”: more precisely, any functor which preserves pullbacks
(in particular any functor which preserves finite limits, in particular any functor which preserves all limits)
preserves monomorphisms.
\item Being an epimorphism is a “colimit property”: more precisely, any functor which preserves pushouts
(in particular any functor which preserves finite colimits, in particular any functor which preserves all colimits)
preserves epimorphisms.\endnote{(Cited from \cite{[Annoying Precision]},
after pullback square and pushout square respectively)}
\end{enumerate}
\end{corollary}

\begin{lemma}
For functors between Abelian categories, left/right exactness is equivalent to preserving monos/epis.
\end{lemma}

\begin{lemma}\label{la:hom_functor_left_exact}
The hom functors $\textup{Hom}(P,-)$ and $\textup{Hom}(-,P)$ from \ref{ex:hom_functor} are left exact, i.e. respect monos.
\begin{proof}
Let $f : M \rightarrow N \in \mathcal{C}_{1}$ be a monomorphism, and let $O \in \mathcal{C}_{0}$ be any object.
Let $\mathfrak{g} : \textup{Hom}(P,N) \rightarrow \textup{Hom}(P,O); \varphi \mapsto \mathfrak{g}(\varphi)$
and $\mathfrak{h} : \textup{Hom}(P,N) \rightarrow \textup{Hom}(P,O); \varphi \mapsto \mathfrak{h}(\varphi)$
such that $\textup{Hom}(P,f) \cdot \mathfrak{g} : \textup{Hom}(P,M) \rightarrow \textup{Hom}(P,O); \psi \mapsto \mathfrak{g}(\psi f)$
and  $\textup{Hom}(P,f) \cdot \mathfrak{h} : \textup{Hom}(P,M) \rightarrow \textup{Hom}(P,O); \psi \mapsto \mathfrak{h}(\psi f)$
yield the same morphism, i.e. $\forall \psi \in \textup{Hom}(P,M), \mathfrak{g}(\psi f) = \mathfrak{h}(\psi f)$.
We want to show that - under the assumption that $f : M \rightarrow N$ was a monomorphism, already $\mathfrak{g} = \mathfrak{h}$.
TODO
\end{proof}
\end{lemma}

\begin{definition}{(Full functor)}\label{def:full_functor}\endnote{(Def 1.5.7. in \cite{[context]}, p. 30 (48/258))}\\
A functor $F : \mathcal{C} \rightarrow \mathcal{D}$ is \ul{full} if
$\forall x, y \in \mathcal{C}_{0}$, the map $\mathcal{C}(x, y) \rightarrow \mathcal{D}(Fx, Fy)$ is surjective.
\end{definition}

\begin{definition}{(Faithful functor)}\label{def:faithful_functor}\endnote{(ebd.)}\\
A functor $F$ as in \ref{def:full_functor} is \ul{faithful} if
$\forall x, y \in \mathcal{C}_{0}$, the map $\mathcal{C}(x, y) \rightarrow \mathcal{D}(Fx, Fy)$ is injective.
\end{definition}

\begin{definition}{(Essentially surjective on objects)}\label{def:ess_surj_o_o}\endnote{(ebd.)}\\
A functor $F$ as in \ref{def:full_functor} is \ul{essentially surjective on objects} if for every object $d \in \mathcal{D}_{0}$ there
is some $c \in \mathcal{C}_{0}$ such that $d$ is isomorphic to $Fc$.
\end{definition}

\begin{definition}{(Embedding)}\label{def:embedding}\endnote{(Rmk 1.5.8. in \cite{[context]}, p. 31 (49/258))}\\
A faithful functor that is injective on objects is called an \ul{embedding} and identifies the source category
as a subcategory of the target. In this case, faithfulness implies that the functor is (globally) injective on arrows.
\end{definition}

\begin{definition}{(Full embedding / full subcategory)}\label{def:full_fully}\endnote{(ebd.)}\\
A full and faithful functor, called \ul{fully faithful} for short, that is injective on objects defines a \ul{full embedding} of the
source category into the target category. The source then defines a \ul{full subcategory} of the target category.
\end{definition}

% cut-pasted from k-Algebroid.tex
\noindent As we have seen, every category is a quiver, but in general, to become a category, a quiver is lacking identity morphisms
and the composition of morphisms. To be more precise, there is a \ul{functor} $U$ from the \ul{category of categories} $\textup{CAT}$ to the
\ul{category of quivers} $\textup{Quiv}$, called the \ul{underlying quiver} or \ul{forgetful functor}.
\[
\begin{tikzcd}
\textup{Cat} \arrow[rr,"U"] &  & \textup{Quiv}
\end{tikzcd}
\]
mapping every object $M \in \mathcal{C}_{0}$ to the same objects in $q_{0}$, mapping every arrow $\varphi \in \mathcal{C}_{1}$ to 
an arrow $a \in q_{1}$, respecting source and target, but forgetting the special role of the identity morphisms and of the composition morphisms.

\begin{example}{(Free / Forgetful functor)}\label{ex:forgetful_functor}\\
TODO

$Free : \mathbf{Quiv} \rightarrow \mathbf{Cat}$

$U : \mathbf{Cat} \rightarrow \mathbf{Quiv}$
\end{example}

% Was bisher bei Category Closure geschah...
\begin{example}{(Category closure)}\\

\noindent\begin{minipage}{.08\textwidth}
\phantom{}
\end{minipage}
\begin{minipage}{.37\textwidth}
\begin{tikzcd}[boxedcd={inner xsep=1.5em, inner ysep=3em}]
B \arrow[rrrr, "\psi"] &  &  &  & C \arrow[ddd, "\rho"] \\
 &  &  &  & \\
 &  &  &  & \\
A \arrow[uuu, "\varphi"] &  &  &  & D
\end{tikzcd}
\end{minipage}
%
\begin{minipage}{.10\textwidth}
$\xrightarrow{\text{  Free }}$
\end{minipage}
%
\begin{minipage}{.37\textwidth}
\begin{tikzcd}[boxedcd={inner xsep=1.5em, inner ysep=3em}]
B \arrow[rrrr, "\psi"] \arrow[rrrrddd, "\psi\rho", pos=0.3] \arrow["1_{B}"', loop, distance=2em, in=125, out=55] &  &  &  &
C \arrow[ddd, "\rho"] \arrow["1_{C}"', loop, distance=2em, in=125, out=55]\\
 &  &  &  & \\
 &  &  &  & \\
A \arrow[uuu, "\varphi"] \arrow[rrrruuu, "\varphi\psi", pos=0.3] \arrow[rrrr, bend left, "(\varphi\psi)\rho" ', shift right=2]
\arrow[rrrr, "\varphi(\psi\rho)", bend right] \arrow["1_{A}"', loop, distance=2em, in=305, out=235] &  &  &  &
D \arrow["1_{D}"', loop, distance=2em, in=305, out=235]
\end{tikzcd}
\end{minipage}
\begin{minipage}{.08\textwidth}
\phantom{}
\end{minipage}\\

We can think of a quiver as a prototype for a category. That means we can construct the missing data for a category
from a quiver by adding the identity morphisms and the composed arrows.
\end{example}

% to be seen how useful this example is...
\begin{example}{(Underlying quiver)}\\

\noindent\begin{minipage}{.08\textwidth}
\phantom{}
\end{minipage}
\begin{minipage}{.37\textwidth}
\begin{tikzcd}[boxedcd={inner xsep=1.5em, inner ysep=3em}]
2 \arrow[rrrr, "b"] \arrow[rrrrddd, "e", pos=0.3] \arrow["h"', loop, distance=2em, in=125, out=55] &  &  &  &
3 \arrow[ddd, "c"] \arrow["i"', loop, distance=2em, in=125, out=55]\\
 &  &  &  & \\
 &  &  &  & \\
1 \arrow[uuu, "a"] \arrow[rrrruuu, "d", pos=0.3] \arrow[rrrr, bend left, "f" ', shift right=2]
\arrow[rrrr, "f", bend right] \arrow["g"', loop, distance=2em, in=305, out=235] &  &  &  &
4 \arrow["j"', loop, distance=2em, in=305, out=235]
\end{tikzcd}
\end{minipage}
%
\begin{minipage}{.10\textwidth}
$\xleftarrow{\text{   U   }}$
\end{minipage}
%
\begin{minipage}{.37\textwidth}
\begin{tikzcd}[boxedcd={inner xsep=1.5em, inner ysep=3em}]
B \arrow[rrrr, "\psi"] \arrow[rrrrddd, "\psi\rho", pos=0.3] \arrow["1_{B}"', loop, distance=2em, in=125, out=55] &  &  &  &
C \arrow[ddd, "\rho"] \arrow["1_{C}"', loop, distance=2em, in=125, out=55]\\
 &  &  &  & \\
 &  &  &  & \\
A \arrow[uuu, "\varphi"] \arrow[rrrruuu, "\varphi\psi", pos=0.3] \arrow[rrrr, bend left, "(\varphi\psi)\rho" ', shift right=2]
\arrow[rrrr, "\varphi(\psi\rho)", bend right] \arrow["1_{A}"', loop, distance=2em, in=305, out=235] &  &  &  &
D \arrow["1_{D}"', loop, distance=2em, in=305, out=235]
\end{tikzcd}
\end{minipage}
\begin{minipage}{.08\textwidth}
\phantom{}
\end{minipage}\\

\noindent In the category on the left, associativity of composition guaranteed that $(\varphi\psi)\rho = \varphi(\psi\rho)$, so those two arrows
were already the same, so they are mapped to the same arrow $f = U((\varphi\psi)\rho) = U(\varphi(\psi\rho))$ in the quiver on the right.
We didn't have to draw both arrows for $f$, but since they are equal, there is still only one arrow in the hom-set $\textup{Hom}_{q}(1,4)=\{f,f\} = \{f\}$.\\
All the other identities are not preserved under the forgetful functor, e.g. $d$ doesn't know what it has to do with $a$ and $b$ apart from
$s(d) = s(a)$ and $t(d) = t(b)$. Especially the former identity arrows are now just endomorphisms with no defining property.\\
The paths $g^{2}f, gf$ and $fj^{3}$ are all different, while in the category, they all simplify to
$1_{A}1_{A}(\varphi\psi)\rho = 1_{A}(\varphi\psi)\rho = (\varphi\psi)\rho1_{D}1_{D}1_{D} =  (\varphi\psi)\rho$ due to the unit property and associativity.
\end{example}


\subsection{Natural transformations}

With fixed categories $\mathcal{C}$ and $\mathcal{D}$ we can consider functors $F, G \in \textup{Hom}(\mathcal{C},\mathcal{D})$ themselves
as objects in the category $\textup{Hom}(\mathcal{C},\mathcal{D})$ of functors between $\mathcal{C}$ and $\mathcal{D}$. In this \ul{functor category},
the morphisms between two functors are called \ul{natural transformations}.

\begin{definition}{(Natural transformations)}\label{def:natural_transformation}\\
\noindent Given categories $\mathcal{C}$ and $\mathcal{D}$ and functors $F : \mathcal{C} \rightarrow \mathcal{D}$ and
$G : \mathcal{C} \rightarrow \mathcal{D}$, a \ul{natural transformation} $\alpha : F \Rightarrow G$ consists of:
\begin{itemize}
\item a morphism $\alpha_{c} : Fc \rightarrow Gc \in \mathcal{D}_{1}$ for each object $c \in \mathcal{C}_{0}$, the collection of which
define the \ul{components} of the natural transformation, so that, for any morphism $f : c \rightarrow c' \in \mathcal{C}_{1}$, the following
square of morphisms in $\mathcal{D}$
\[\begin{tikzcd}
Fc \arrow[rr, "\alpha_{c}"] \arrow[dd, "Ff"] &  & Gc \arrow[dd, "Gf"] \\
                                             &  &                     \\
Fc' \arrow[rr, "\alpha_{c'}"]                &  & Gc'                
\end{tikzcd}\]

\ul{commutes}, i.e., has a a common composite $Fc \rightarrow Gc' \in \mathcal{D}_{1}$.
\end{itemize}
When each component $\alpha_{c}$ is an isomorphism, we call $\alpha$ a \ul{natural isomorphism}.
\end{definition}

\subsection{The functor category}

\begin{definition}{(The functor category)}\label{def:functor_category}\endnote{(cited from ncatlab \cite{[ncatlab_functor_category]})}\\
Given categories $\mathcal{C}$ and $\mathcal{D}$, the \ul{functor category} - written $\mathcal{D}^{\mathcal{C}}$, $\textup{Hom}(\mathcal{C},
\mathcal{D})$ or $[\mathcal{C}, \mathcal{D}]$ -
is the category whose
\begin{itemize}
\item objects are functors $F : \mathcal{C} \rightarrow \mathcal{D}$
\item morphisms are natural transformations between these functors.
\end{itemize}
Main usage of functor categories is as $\textup{Hom}$ categories in place of hom-sets (comp. \ref{def:hom_set} and \ref{ex:hom_functor}) where
we have much more than a set, namely a whole category of morphisms between two objects (together with the morphisms between morphisms).
\end{definition}


\section{Additional structure on the Hom-set of a category}
% mainfile: ../main.tex

\subsection{Additional structure on the Hom-set of a category}

....

\begin{example}{(Group as a category)}\\
\noindent A group $\mathbf{G}$ defines a category $\mathcal{B}\mathbf{G}$ with a single object $\ast$. The group elements are its morphisms, which are
all automorphisms (i.e. bijective endomorphisms) of the single object. Composition of morphisms is defined by the binary group operation.
The identity element $e \in G$ acts as the identity morphism for the unique object in this category. The hom-set of that category is itself
a group.
\end{example}

This example can be generalized to categories where the hom-set is a ring or an R-algebra. But for this we need a commutative ring R.

Our goal is to represent finite concrete categories, for this we need the source and target categories of our functors, which the
representations are.
As subcategories of $\textup{FinSets}$, our finite concrete categories only have definitions for their objects and their
morphisms, methods to check when two morphisms are congruent or equivalent, but not much else.
A competing theory to category theory is that of quivers and path algebras. We already used their terminology in
\ref{def:path}, \ref{la:cyclic_paths} and \ref{def:path_algebra}, for instance when talking about the trivial path,
which in the language of category theory is nothing but the identity morphism, composition of arrows to a path is nothing but
composition of morphisms (if you make the path explicit by writing a new arrow for every path).

So what we called a path algebra in \ref{def:path_algebra} is a different data structure for a category. 
For one, the path algebra is an algebra, i.e. a vector space with additional structure, and thus a single set, comparable to the
class of morphisms $\mathcal{C}_{1}$ of a category $\mathcal{C}$.
But as it is an algebra, it not only contains the generating morphisms of the category, but also $\Bbbk$-linear combinations of
morphisms and paths. This is what our concrete categories lack, and what additional structure we have to give them in order
to represent them by matrices.

In practise, there is already developed software for \ul{q}uivers and \ul{p}ath \ul{a}lgebras, namely the \textsc{Gap} package
\textsc{QPA$2$}\endnote{(see \cite{[QPA2]})}.
What we are actually doing to represent finite concrete categories, is going from $\mathcal{C} \in \mathbf{Cats}$ to $q \in \mathbf{Quiv}$,
in theory by \ul{forgetting} (see \ref{ex:forgetful_functor}) the category concepts of identity morphism and composition, in practise by calculating the
underlying quiver $q$, and then for a commutative ring $\Bbbk$, constructing the path algebra $\Bbbk q$. In this step the path algebra
is infinite-dimensional, since there are infinitely many paths according to lemma \ref{la:cyclic_paths}, and \textsc{QPA$2$}'s function
\texttt{BasisPathsBetweenVertices} only works for finite-dimensional path algebras. Thus in a next step we have to provide
additional data in the form of generators of ideals of the path algebra, by which we can divide and build the quotient path algebra,
which is then finite-dimensional. This is the purpose of \texttt{RelationsOfEndomorphisms}.

Once we have a finite-dimensional path algebra $\Bbbk q$, we let \textsc{QPA$2$} calculate generators of the non-endomorphism relations,
and when we have a complete set of relations, that will be our definitive quotient quiver algebra $\Bbbk q$, which we then take it back into the category
theoretical context by constructing the $\Bbbk$-\textbf{Algebroid} $\mathcal{A}$ from the path algebra $\Bbbk q$.

The source category for our representation is then the $\Bbbk$-\textbf{Algebroid} $\mathcal{A}$ and not anymore our finite concrete
category $\mathcal{C}$, but it behaves in the same way regarding composition of morphisms and which morphisms are congruent.

The target category of our category representations will be $\Bbbk$-\textbf{Mat} which we will describe in the next section,
especially all the nice properties $\Bbbk$-\textbf{Mat} has, and how they get carried over to our functor category with $\Bbbk$-\textbf{Mat} as
target.\endnote{
In \cite{[Ab-Cat]}, Posur used the equivalence between categories $\textup{mat}_{\Bbbk} \cong \textup{vec}^{\text{fd}}_{\Bbbk}$,
as described in \cite{[context]}, \textsc{Example} 1.5.6 on page 30 (48/258), to justify that $\Bbbk$-\textbf{Mat} is a good
\textbf{computational model} to
%\setquotestyle[guillemets]{english} don't do that!
\blockquote{transform otherwise inaccessible mathematical objects into computationally easily graspable entities}
\setquotestyle{default}, which is what we are doing with \textbf{CatReps}.
}

With source and target categories defined, the category where our category representations lie in is \textbf{CatReps} for which we
show that it's a subcategory of the \textbf{Functor Category}. And even more in the next section.
\[
\mathbf{CatReps_{\mathcal{C}}} = \textup{Hom}(\Bbbk\mathbf{-Algebroid_{\mathcal{C}}}, \kmat)
\]

\subsection{Generating morphisms of a category and the underlying quiver}

$\textup{gmorphisms} := \{g_{1},\dots,g_{r}\} \rightarrow$ concrete category with set of generating morphisms $\textup{gmorphisms}$.

This is the $\textup{Free}$ functor from $\mathbf{Quiv}$ to $\mathbf{Cat}$, taking a quiver and adding the missing structure of
identity morphisms and composition of arrows to that category. The result is a category.

The $\textup{forgetful}$ functor from $\mathbf{Cat}$ to $\mathbf{Quiv}$ is going the other way around and leaves all
morphisms that we now have in the category, but forgets their relations, what was identity, what was composition.

Given a field $\Bbbk$, we have the path algebra $\Bbbk q$ with all the arrows as a basis.

Given relations on endomorphisms and on the other morphisms, we make the quotient path algebra.

This is already a category, and now it has more structure.

\subsection{Ab-categories}

\begin{definition}{(Ab-category)}
An \ul{Ab-category} is a category in which all homomorphism sets are abelian groups, and composition distributes over addition.\\
In other words, a category $\mathcal{C}$ is an \ul{Ab-category} if for every pair of objects $M,N \in \mathcal{C}_{0}$,
$( \textup{Hom}_{\mathcal{C}}(M,N), + )$ is an abelian group (with the neutral element called \ul{zero morphism}),
and for all morphisms $\gamma, \delta \in \textup{Hom}_{\mathcal{C}}(M,N),
\alpha, \beta \in \textup{Hom}_{\mathcal{C}}(N,L)$
\begin{align}\label{eq:dist}
(\gamma + \delta)\alpha &= \gamma\alpha + \delta\alpha \textup{ and }\\
\gamma(\alpha+\beta) &= \gamma\alpha + \gamma\beta.
\end{align}
Note that every hom-set has its own unique zero morphism. E.g. in $\textup{Mat}_{\mathbb{Q}}$ the $2 \times 3$ zero-matrix
$\mathbf{0} \in \textup{Hom}(2,3)$ is different from the $4 \times 4$ zero-matrix $\mathbf{0} \in \textup{Hom}(4,4)$.
\end{definition}

\begin{definition}{(semisimple)}
A ring R is semisimple if ...
\end{definition}

\begin{example}{(The matrix category $\Rmat$ over a commutative ring $R$)}\label{ex:matrix_category}
\begin{itemize}
\item Objects are natural numbers $\textup{Obj}(\textup{Mat}_{R}) = \mathbb{N} = \mathbb{N}_{0} = \{0,1,2,\dots\}$
\item Morphisms $\textup{Mor}(\textup{Mat}_{R}) \ni (m \rightarrow n)$ are $m \times n$ matrices over $R$.
We write the set of morphisms between $m$ and $n$, as $R^{m\times n} := \textup{Hom}(m,n)$. Identity morphisms are the
identity matrices.
\item Composition is matrix multiplication (associative).
\item It is a skeletal category, i.e. $m$ is isomorphic to $n \Rightarrow m = n$. Only quadratic matrices ($m = n$) can be
isomorphisms.
\end{itemize}
In this category, the number $0$ is \ul{the} zero object.\\
A zero matrix (zero morphism) is a matrix factoring through the zero object $0$.\\
\begin{minipage}{.2\textwidth}\phantom{ }\end{minipage}
\begin{minipage}{.25\textwidth}
Matrix $R^{m\times n} \ni A = 0$
\end{minipage}
\begin{minipage}{.08\textwidth}
$\Longleftrightarrow$
\end{minipage}
\begin{minipage}{.32\textwidth}
\begin{tikzcd}
m \arrow[rr, "A"] \arrow[rd, "(m \times 0)"'] &                               & n \\
                                              & 0 \arrow[ru, "(0 \times n)"'] &  
\end{tikzcd}\\
$\Rightarrow A = (m \times 0) \cdot (0 \times n)$.
\end{minipage}
\begin{minipage}{.15\textwidth}\phantom{ }\end{minipage}\\
\noindent The ``matrices'' $(m \times 0)$ and $(0 \times n)$ have zero columns or zero rows respectively, but it is
important to note that for each $m \in \textup{Obj}(\textup{Mat}_{R})$ there is exactly one such matrix $(m \times 0)$ and $(0 \times m)$
(that's what initial and terminal object means), and for different $m$, these morphisms are different.
\end{example}


\begin{example}{($\kmat$ is an Ab-category)}
For two natural numbers $m,n \in {\kmat}_{0} = \mathbb{N} = \mathbb{N}_{0}$, the set of morphisms with source $m$ and target $n$ is
$\Bbbk^{m\times n}$, the set of $m \times n$-matrices. This is an abelian group:
\begin{itemize}
\item The neutral element of the addition is the $m \times n$ zero matrix $\mathbf{0}$.
\item Addition of matrices is associative and commutative, so it's an abelian group.
\end{itemize}
The distributive laws \eqref{eq:dist} for composable morphisms hold.
\end{example}









\begin{definition}{(Abelian category)}\endnote{(From \cite{[context]}, appendix E.5, Def. E.5.1)}
A category $\mathcal{C}$ is \ul{abelian} if
\begin{itemize}
\item it has a \ul{zero object} $0$, that is both initial and terminal,
\item it has all \ul{binary products} and \ul{binary coproducts},
\item it has all \ul{kernels} and \ul{cokernels}, defined repsectively to be the \ul{equalizer} and
\ul{coequalizer} of a map $f : A \rightarrow B$ with the zero map $A \rightarrow 0 \rightarrow B$, and
\item all monomorphisms and epimorphisms arise as kernels or cokernels, respectively.
\end{itemize}
\end{definition}

\begin{definition}{($R$-linear category)}
Let $R$ be a commutative ring.
\end{definition}

For $R = \mathbb{Z}$ an $R$-linear category is nothing but an Ab-category.


\begin{definition}
Once source and target categories $\mathcal{C}, \mathcal{D}$ are both $R$-linear categories we define the functor category
$\mathrm{Hom_{R}}(\mathcal{C},\mathcal{D})$ the subcategory of $R$-linear functors.
\end{definition}





\section{Datatype convention of catreps}
Since the goal of this thesis is a translation of the package \texttt{catreps} by Peter Webb et al. into CAP, this section is
a short overview of the package catreps.

\blockquote[\cite{[Webb2020]}]{In this package a category is stored as a concrete category (i.e. a category where the objects are sets and
morphisms are maps of sets).
A category is stored as a record (cat, say) with fields cat.objects, cat.generators, cat.domain, cat.codomain.
Each object in the list cat.object is a set, and each morphism in the list of generator morphisms cat.generators
is stored as a mapping of sets, which we notate as the list of its values.}

\begin{Verbatim}[commandchars=!@|,fontsize=\small,frame=single,label=Example]
  !gapprompt@gap>| !gapinput@c3c3 := ConcreteCategory( [ [2,3,1], [4,5,6], [,,,5,6,4] ] );|
  rec( codomain := [ 1, 2, 2 ], domain := [ 1, 1, 2 ],
       generators := [ [ 2, 3, 1 ], [ 4, 5, 6 ], [ ,,, 5, 6, 4 ] ],
       objects := [ [ 1, 2, 3 ], [ 4, 5, 6 ] ], operations := rec(  ) )
\end{Verbatim}

\noindent The list of values as seen in the example above may be easy to type in, but does have its disadvantages: If for example you want to store the
morphism that maps the set $\{9\}$ to itself, i.e. the identity morphism $1_{\{9\}}$, you first have to write the eight commas that are not part of that
morphism definition \texttt{ [ ,,,,,,,,9 ] } and you might make a mistake by forgetting one comma.
Another issue is that the source object of a morphism \texttt{gen} is only implicitly given by those list entries \texttt{i} for which
\texttt{ IsBound( gen[i] ) = true }.

Using instead \texttt{MapOfFinSets} in \textsc{Cap} solves both of these issues, and it lets us use a different model for concrete categories in \textsc{Cap},
i.e. that of a subcategory of \texttt{FinSets}, for which we already have an implementation in \textsc{Cap}. 
Another advantages of this method is that a \texttt{MapOfFinSets} can cache known properties about itself:

\begin{Verbatim}[commandchars=!@|,fontsize=\small,frame=single,label=Example]
  !gapprompt@gap>| !gapinput@S := FinSet( [1,2,3] );|
  <An object in FinSets>
  !gapprompt@gap>| !gapinput@T := FinSet( [4,5,6] );|
  <An object in FinSets>
  !gapprompt@gap>| !gapinput@map1 := MapOfFinSets( S, [ [1,1], [2,2], [3,3] ], S );|
  <A morphism in FinSets>
  !gapprompt@gap>| !gapinput@IsAutomorphism( map1 );|
  true
  !gapprompt@gap>| !gapinput@map1;|
  <An automorphism in FinSets>
\end{Verbatim}

Going further in the cited example,\\
\blockquote[\cite{[Webb2020]}]{The following constructs a representation:}
\begin{Verbatim}[commandchars=!@|,fontsize=\small,frame=single,label=Example]
  !gapprompt@gap>| !gapinput@one:=One(GF(3));;|
  !gapprompt@gap>| !gapinput@d:=[[1,1,0,0,0],[0,1,1,0,0],[0,0,1,0,0],[0,0,0,1,1],[0,0,0,0,1]]*one;;|
  !gapprompt@gap>| !gapinput@e:=[[0,1,0,0],[0,0,1,0],[0,0,0,0],[0,1,0,1],[0,0,1,0]]*one;;|
  !gapprompt@gap>| !gapinput@f:=[[1,1,0,0],[0,1,1,0],[0,0,1,0],[0,0,0,1]]*one;;|
  !gapprompt@gap>| !gapinput@nine:=CatRep(c3c3,[d,e,f],GF(3));|
  rec(
category := rec( generators := [ [ 2, 3, 1 ], [ 4, 5, 6 ], [ ,,, 5, 6, 4 ] ]
, operations := rec( ), objects := [ [ 1, 2, 3 ], [ 4, 5, 6 ] ],
domain := [ 1, 1, 2 ], codomain := [ 1, 2, 2 ] ),
genimages := [ [ [ Z(3)^0, Z(3)^0, 0*Z(3), 0*Z(3), 0*Z(3) ],
[ 0*Z(3), Z(3)^0, Z(3)^0, 0*Z(3), 0*Z(3) ],
[ 0*Z(3), 0*Z(3), Z(3)^0, 0*Z(3), 0*Z(3) ],
[ 0*Z(3), 0*Z(3), 0*Z(3), Z(3)^0, Z(3)^0 ],
[ 0*Z(3), 0*Z(3), 0*Z(3), 0*Z(3), Z(3)^0 ] ],
[ [ 0*Z(3), Z(3)^0, 0*Z(3), 0*Z(3) ], [ 0*Z(3), 0*Z(3), Z(3)^0, 0*Z(3) ]
, [ 0*Z(3), 0*Z(3), 0*Z(3), 0*Z(3) ],
[ 0*Z(3), Z(3)^0, 0*Z(3), Z(3)^0 ],
[ 0*Z(3), 0*Z(3), Z(3)^0, 0*Z(3) ] ],
[ [ Z(3)^0, Z(3)^0, 0*Z(3), 0*Z(3) ], [ 0*Z(3), Z(3)^0, Z(3)^0, 0*Z(3) ]
, [ 0*Z(3), 0*Z(3), Z(3)^0, 0*Z(3) ],
[ 0*Z(3), 0*Z(3), 0*Z(3), Z(3)^0 ] ] ], field := GF(3),
dimension := [ 5, 4 ] )
\end{Verbatim}

we see that \texttt{catreps} works with \textsc{Gap} matrices directly whereas with \textsc{Cap} we use \texttt{HomalgMatrix} and
\texttt{RingsForHomalg} which lets us delegate computation to faster computer algebra systems like \texttt{Singular} or \texttt{Magma}.
What is also noticable is the big chunk of output we get as a result of \texttt{CatRep(c3c3,[d,e,f],GF(3))}. In \textsc{Cap} we hide the output
and give a short description of the resulting object or morphism, and use the \texttt{Display} function to display the whole result.

All in all, there are plenty of reasons to change to \textsc{Cap}. In the meantime, in order to still support inputs in the convention of
\texttt{catreps}, I wrote a converter function \funcref{ConvertToMapOfFinSets}.

\section{Yoneda's Lemma: Completion and cocompletion of a category}

\subsection{The category of presheaves}

\begin{definition}{(The category of presheaves)}\endnote{(cited from ncatlab \cite{[ncatlab_presheaves]})}\\
For $\mathcal{C}$ a small category, its \ul{category of presheaves} is the functor category
\[ \mathrm{PSh}(\mathcal{C}) := \mathrm{Hom}(\mathcal{C}^{\text{op}}, \Set) \]
from the opposite category of $\mathcal{C}$ to $\Set$.
An object in this category is a \ul{presheaf}.\\
\noindent Taking $\mathcal{C}^{\text{op}}$ instead of $\mathcal{C}$ (and with $(\mathcal{C}^{\text{op}})^{\text{op}} = \mathcal{C}$)
we get the functor category as in \ref{def:functor_category}
\[
\mathrm{Hom}(\mathcal{C}, \Set) = \mathrm{PSh}(\mathcal{C}^{\text{op}})
\]
\end{definition}

\begin{remark}{(General properties of presheaves)}\\
The category of presheaves $\mathrm{PSh}(\mathcal{C})$ is called the \ul{free cocompletion} of $\mathcal{C}$.
\end{remark}

\begin{definition}{(Representable functor)}\label{def:repres_functor}\endnote{(from ncatlab \ref{[ncatlab_repres_functor]})}
\begin{enumerate}
\renewcommand{\labelenumi}{(\theenumi)}
\item A functor from a locally small category $\mathcal{C}$ to $\Set$ is \ul{representable} if there is an object $c \in \mathcal{C}$ and a
natural isomorphism between $F$ and $\mathrm{Hom}(c,-)$ (for a covariant $F$, otherwise $\mathrm{Hom}(-,c)$ for a contravariant $F$),
in which case one says thet the functor $F$ is \ul{represented by} the object $c$.
\item A \ul{representation} for a functor $F$ is a choice of object $c \in \mathcal{C}$ together with a specified natural isomorphism
$\mathrm{Hom}(c,-) \cong F$ (for a covariant $F$, or $\mathrm{Hom}(-,c) \cong F$ for a contravariant $F$).
\end{enumerate}
\end{definition}

\begin{lemma}{(Yoneda's Lemma)}\endnote{(Statement of Yoneda's lemma from \cite{[context]}, Lemma 2.2.4)}
For any functor $F : \mathcal{C} \rightarrow \mathrm{Set}$, whose source $\mathcal{C}$ is locally small and any
object $c \in \mathcal{C}_{0}$, there is a bijection
\[
\mathrm{Hom}(\mathrm{Hom}_{\mathcal{C}}(c,-), F) \cong Fc
\]
that associates a natural transformation $\alpha : \mathrm{Hom}_{\mathcal{C}}(c,-) \Rightarrow F$ to the element $\alpha_{c}(1_{c}) \in Fc$.
Moreover, this correspondence is natural in both $c$ and $F$.
\end{lemma}
As $\mathcal{C}$ is locally small but not necessarily small, a priori the collection of natural transformations
$\mathrm{Hom}(\mathrm{Hom}_{\mathcal{C}}(c,-),F)$ might be large. However, the bijection in the Yoneda lemma proves that this particular
collection of natural transformations indeed forms a set.
\begin{proof}
A proof of Yoneda's lemma can be found in many books on category theory, e.g. \cite{[context]}, Lemma 2.2.4, pages 57-59.
\end{proof}

\begin{definition}{(Yoneda embedding)}\label{def:yoneda_embedding}\endnote{(cited from ncatlab \cite{[ncatlab_yoneda_emb]})}\\
The \ul{Yoneda embedding} for a locally small category $\mathcal{C}$ is the functor
\[
Y : \mathcal{C} \hookrightarrow \mathrm{Hom}(\mathcal{C}^{\text{op}}, \mathrm{Set})
\]
from $\mathcal{C}$ to the category of presheaves over $\mathcal{C}$ which is the image of the hom-functor
\[
\mathrm{Hom} : \mathcal{C}^{\text{op}}\times\mathcal{C} \rightarrow \mathrm{Set}
\]
under the $\mathrm{Hom}$ adjunction
\[
\mathrm{Hom}(\mathcal{C}^{\text{op}}\times\mathcal{C}, \mathrm{Set}) \simeq
\mathrm{Hom}(\mathcal{C},\mathrm{Hom}(\mathcal{C}^{\text{op}}, \mathrm{Set}))
\]
in the closed symmetric monoidal category $\mathrm{Cat}$.
If instead we have the opposite category $\mathcal{C}^{\text{op}}$, then we get the embedding into the functor category:
\[
Y^{\text{op}} : \mathcal{C}^{\text{op}} \hookrightarrow \mathrm{Hom}(\mathcal{C},\mathrm{Set})
\]
\end{definition}

\begin{remark}[Our $\Bbbk$-linear version of Yoneda's embedding]
Let $\mathcal{A}$ be a $\Bbbk$-linear category with finite-dimensional $\Bbbk$-vector spaces as hom-sets. Then we get
$\Bbbk$-linear versions of Yoneda's lemma and Yondeda's embedding:

For any $\Bbbk$-linear functor $F : \mathcal{A} \rightarrow \kmat$ and any object $i \in \mathcal{A}_{0}$, there is a bijection
\[
\mathrm{Hom}_{\HomAkmat}(\mathrm{Hom}_{\mathcal{A}}(-,i), F) \cong F(i)
\]

\[
Y : \mathcal{A} \hookrightarrow \mathrm{Hom_{\Bbbk}}(\mathcal{A^{\text{op}}},\kmat)
\]

\[
Y^{\text{op}} : \mathcal{A}^{\text{op}} \hookrightarrow \HomAkmat
\]

\end{remark}

$\HomAkmat$ is Abelian and therefore finitely complete and finitely cocomplete.

\subsection{Projective objects and the Yoneda projective}\label{sec:projective_objects}

\begin{lemma}\label{la:Hom_exact_proj_Lift_along_epis}
Let $\mathcal{C}$ be a locally small category. For an object $P \in \mathcal{C}_{0}$ the following are equivalent:
\begin{itemize}
\item The covariant functor $\mathrm{Hom}(P,-)$ is exact.
\item For all epimorphisms $\varphi : M \twoheadrightarrow N$ and morphisms $\theta : P \rightarrow N$, there exists a
projective lift $\psi : P\dottedrightarrow M$ such that $\theta = \psi\varphi$.\\
\begin{tikzcd}
M \arrow[r, "\varphi", two heads] & N \\
	& P \arrow[u, "\theta", "=\,\psi\varphi"'] \arrow[lu, "\psi", dotted]
\end{tikzcd}
\end{itemize}
\begin{proof}
For an object $L\in \mathcal{C}_{0}$, $\mathrm{Hom}(L,-)$ is always a covariant left-exact functor, i.e. respects monos.\\
\setlist[description]{font=\normalfont}
\begin{description}
\item[``$\Leftarrow$:''] Prove that $\mathrm{Hom}(P,-)$  is right exact, i.e. respects epis.\\
For this, let $M, N \in \mathcal{C}_{0}$ and $\varphi : M \twoheadrightarrow N$ be an epi. The Hom-functor works on morphisms
by mapping the Hom-sets of the source and target objects of the morphism, i.e.
$\mathrm{Hom}(P,\varphi) : \mathrm{Hom}(P,M) \rightarrow \mathrm{Hom}(P,N)$, given by $\rho \mapsto \rho\varphi\, \forall \rho \in \mathrm{Hom}(P,M)$.
Now given that $\varphi$ is an epi, we want to show that $\mathrm{Hom}(P,\varphi)$ is also an epi.\\
Let $O \in \mathcal{C}_{0}$,  $\gamma : N \rightarrow O$ and $\varepsilon : N \rightarrow O$ such that
$\mathrm{Hom}(P,\gamma) : \mathrm{Hom}(P,N) \rightarrow \mathrm{Hom}(P,O);\, \theta \mapsto \theta\gamma$ and
$\mathrm{Hom}(P,\varepsilon) : \mathrm{Hom}(P,N) \rightarrow \mathrm{Hom}(P,O);\, \theta \mapsto \theta\varepsilon$ and
$\mathrm{Hom}(P,\varphi)\mathrm{Hom}(P,\gamma) = \mathrm{Hom}(P,\varphi)\mathrm{Hom}(P,\varepsilon)$. 
From the functoriality axioms (ref. definition \ref{def:functor} of a functor) it follows that $\mathrm{Hom}(P,\varphi\gamma) = \mathrm{Hom}(P,\varphi\varepsilon)$. This implies
\begin{equation}\label{eqn:Hom_functoriality}\rho(\varphi\gamma) = \rho(\varphi\varepsilon)\, \forall \rho \in \mathrm{Hom}(P,M)\end{equation}. 

\begin{tikzcd}
M \arrow[r, "\varphi", shift right, two heads] & N \arrow[r, "\gamma"'] \arrow[r, shift left=2, "\varepsilon"] & O \\
	& P \arrow[lu, "\rho"] \arrow[u, "\theta"] \arrow[ru, outer sep=2, pos=.55, "\rho(\varphi\gamma) = \rho(\varphi\varepsilon)"'] \arrow[ru, shift right=2]
\end{tikzcd}

We want to show that the parallel morphisms $\mathrm{Hom}(P,\gamma)$ and $\mathrm{Hom}(P,\varepsilon)$ are the same, i.e. for all
$\theta \in \mathrm{Hom}(P,N), \theta\gamma = \theta\varepsilon$. Our assumtion that there exists a projective lift helps us in this situation:
$\forall \theta \in \mathrm{Hom}(P,N)\, \exists\, \rho \in \mathrm{Hom}(P,M)$ such that $\theta = \rho\varphi$ and therefore with the above 
equation \eqref{eqn:Hom_functoriality},
$\theta\gamma = (\rho\varphi)\gamma = \rho(\varphi\gamma) = \rho(\varphi\varepsilon) = (\rho\varphi)\varepsilon = \theta\varepsilon$
and therefore $\mathrm{Hom}(P,\gamma) = \mathrm{Hom}(P,\varepsilon)$, i.e. $\mathrm{Hom}(P,\varphi)$ is epi.\\

\item[``$\Rightarrow$:''] Let $\mathrm{Hom}(P,-)$ be right exact. Let $M, N \in \mathcal{C}_{0}$, the morphism
$\varphi : M \twoheadrightarrow N$ be an epi and $\theta : P \rightarrow N$ any morphism.
We want to show the existence of a morphism $\psi : P \dottedrightarrow\, M$ such that $\theta = \psi\varphi$.
With $\mathrm{Hom}(P,-)$ being exact, we have that $\mathrm{Hom}(P,\varphi) : \mathrm{Hom}(P,M) \twoheadrightarrow \mathrm{Hom}(P,N)$ is
an epi, and is given by $\mathrm{Hom}(P,M) \ni \rho \mapsto \rho\varphi \in \mathrm{Hom}(P,N)$.\\
$\mathcal{C}$ is locally small, i.e. for the two objects $P, N \in \mathcal{C}_{0},$ there is a \ul{set} $\mathrm{Hom}(P,N)$
of morphisms between them. The $\mathrm{Hom}$-functor moves the morphisms from a general categorical context 
in $\mathcal{C}$ into the category of sets, i.e. $\mathrm{Hom}(P,\varphi)$ is a function in the category of sets.
And for those it's true that every epimorphism is surjective. Thus $\forall \theta \in \mathrm{Hom}(P,N)\, \exists \rho \in \mathrm{Hom}(P,M)$ such
that $\theta = (\mathrm{Hom}(P,\varphi))(\rho) = \rho\varphi$. This $\rho$ is the projective lift $\psi := \rho$ we were looking for.
\end{description}
\end{proof}
\end{lemma}

\begin{definition}{(Projective object)}\label{def:proj_object}\\
An object $P$ in a category $\mathcal{C}$ that satisfies one (and thus both) of the equivalent properties in Lemma
 \ref{la:Hom_exact_proj_Lift_along_epis} is called a \ul{projective object}.
\end{definition}

The dual statement to Lemma \ref{la:Hom_exact_proj_Lift_along_epis} is
\begin{lemma}\label{la:dual_Hom_exact_proj_colift}
Let $\mathcal{C}$ be a category. For an object $P \in \mathcal{C}_{0}$ the following are equivalent:
\begin{itemize}
\item The contravariant functor $\mathrm{Hom}(-,P)$ is exact.
\item For all monomorphisms $\varphi : M \hookleftarrow N$ and morphisms $\theta : P \leftarrow N$, there exists a
projective colift $\psi : P \dottedleftarrow M$ such that $\theta = \varphi\psi$.\\
\begin{tikzcd}
M \arrow[rd, "\psi"', dotted] & N \arrow[l, "\varphi", hook] \arrow[d, "=\,\varphi\psi", "\theta"'] \\
	& P 
\end{tikzcd}
\end{itemize}
\end{lemma}

\begin{definition}{(Injective object)}\label{def:inj_object}\\
An object $P$ in a category $\mathcal{C}$ that satisfies one (and thus both) of the equivalent properties in Lemma
 \ref{la:dual_Hom_exact_proj_colift} is called an \ul{injective object}.
\end{definition}

\begin{definition}{(Yoneda projective)}
Let $\mathcal{A}$ be a $\Bbbk$-algebroid with finitely many objects and finite-dimensional hom-sets over $\Bbbk$.
With the Yoneda embedding 
\[
Y^{\text{op}} : \begin{cases}\mathcal{A}^{\text{op}} \hookrightarrow \HomAkmat \\
i \mapsto \mathrm{Hom}_{\mathcal{A}}(-,i)
\end{cases}
\]
we see that the image
$Y^{\text{op}}(i)$ of the object $i \in \mathcal{A}^{\text{op}}$ is the representable functor $\mathrm{Hom}_{\mathcal{A}}(-,i)$ in $\HomAkmat$. 
It is called the \ul{$i$-th Yoneda projective}.
\end{definition}

Analogous to Prop. \ref{prop:Alg-Alg-Correspondence} one can easily prove that a $\Bbbk$-representation $F$ of $\mathcal{A}$ corresponds to a 
module $\bigoplus_{i \in \mathcal{A}_{0}} F(i)$ over the algebra $\mathbf{A} = \mathrm{Algebra}(\mathcal{A})$.
In particular the Yoneda projective $Y^{\text{op}}(i) = \mathrm{Hom}_{\mathcal{A}}(-,i)$ corresponds to the $\mathbf{A}$-module
$M_{i} = \bigoplus_{j \in \mathcal{A}_{0}} \mathrm{Hom}_{\mathcal{A}}(j,i)$.

\begin{lemma}
Yoneda projectives are projective objects in $\HomAkmat$.
\end{lemma}
\begin{proof}\phantom{}\\
We want to show that $\mathrm{Hom}_{\HomAkmat}(Y^{\text{op}}(i),-)$ is right exact, i.e. finitely cocontinuous.

\noindent For this, let
\[
\begin{tikzcd}
0 \arrow[r] & F_{1} \arrow[r, "\eta"] & F_{2} \arrow[r,"\rho"] & F_{3} \arrow[r] & 0
\end{tikzcd}
\]
be a short exact sequence in $\HomAkmat$, i.e.
\begin{enumerate}
\item $\eta$ is a monomorphism,
\item $\rho$ is an epimorphism and
\item the image of $\eta$ equals the kernel of $\rho$.
\end{enumerate}

\noindent Then applying the functor $\mathrm{Hom}_{\HomAkmat}(Y^{\text{op}}(i),-)$ on the sequence and simplifying with Yoneda's lemma

\begin{align*}
\mathrm{Hom}_{\HomAkmat}(\mathrm{Hom}_{\mathcal{A}}(-,i),F) &\simeq F(i) \\
\mathrm{Hom}_{\HomAkmat}(\mathrm{Hom}_{\mathcal{A}}(-,i),\rho) &\simeq \rho_{i}
\end{align*}

We only have to measure exactness on the components
\[
\begin{tikzcd}
0 \arrow[r] & F_{1}(i) \arrow[r, "\eta_{i}"] &
F_{2}(i) \arrow[r,"\rho_{i}"] & F_{3}(i) \arrow[r] & 0\mbox{,}
\end{tikzcd}
\]
but this coincides with the definition of the exactness of functors. So the proof is a tautology by Yoneda's lemma.
\end{proof}

\subsubsection{Abelian categories with enough projective objects (constructively)}

\begin{definition}[Enough projective objects]\label{def:enough_projectives}
A category $\mathcal{C}$ is said to have \ul{enough projective objects} if every object admits an epimorphism from a projective object,
i.e. for any object $A \in \mathcal{C}_{0}$ there exists an epimorphism $P \twoheadrightarrow A$, where $P$ is projective.
\end{definition}

We state without a proof the following fact about our functor category:

\begin{theorem}[$\HomAkmat$ has enough projectives]\phantom{}\\
Let $\mathcal{A}$ be a finite-dimensional algebroid over some field $\Bbbk$. The functor category $\HomAkmat$ has sufficiently many
projectives.
\end{theorem}
\begin{proof}
(no proof)
\end{proof}

\begin{doctrine}[Abelian category with enough projective objects]\phantom{}\\
The doctrine $\mathtt{IsAbelianCategoryWithEnoughProjectives}$ therefore involves algorithms of\\
$\mathtt{IsAbelianCategory}$ together with algorithms for
\begin{itemize}
\item $\mathtt{EpimorphismFromSomeProjectiveObject}$,
\item $\mathtt{ProjectiveLift}$,
\end{itemize}
\end{doctrine}

\subsubsection{Abelian categories with enough injective objects}
Dually to \ref{def:enough_projectives} we define

\begin{definition}[Enough injective objects]\label{def:enough_injectives}
A category $\mathcal{C}$ is said to have \ul{enough injective objects} if every object admits a monomorphism into an injective object,
i.e. for any object $A \in \mathcal{C}_{0}$ there exists a monomorphism $A \hookrightarrow J$, where $J$ is injective.
\end{definition}

\begin{doctrine}[Abelian category with enough injective objects]\phantom{}\\
The doctrine $\mathtt{IsAbelianCategoryWithEnoughInjectives}$ therefore involves algorithms of\\
$\mathtt{IsAbelianCategory}$ together with algorithms for
\begin{itemize}
\item $\mathtt{MonomorphismIntoSomeInjectiveObject}$,
\item $\mathtt{InjectiveColift}$,
\end{itemize}
\end{doctrine}


\section{The category FinSets}

There are algorithms whose sole purpose is to convert data structures, so they are not of much interest to the mathematical theory,
and then there are algorithms that implement our category theoretical calculations, so they are important to our theory.

The algorithms \hyperref[func:ConvertToMapOfFinSets]{\texttt{ConvertToMapOfFinSets}},\\
\hyperref[func:ConcreteCategoryForCAP]{\texttt{ConcreteCategoryForCAP}} and
\hyperref[func:RightQuiverFromConcreteCategory]{\texttt{RightQuiverFromConcreteCategory}}\\
are more of the data structure conversion type, while 
\hyperref[func:RelationsOfEndomorphisms]{\texttt{RelationsOfEndomorphisms}},\\
\hyperref[func:Algebroid]{\texttt{Algebroid}},\\
\hyperref[func:EmbeddingOfSubRepresentation]{\texttt{EmbeddingOfSubRepresentation}}\\
and \hyperref[func:WeakDirectSumDecomposition]{\texttt{WeakDirectSumDecomposition}} are also important to our theory.

\subsection{MapOfFinSets}

\begin{algorithm}\capstart
   \caption{\texttt{ConvertToMapOfFinSets}}\label{algo:ConvertToMapOfFinSets}
      \SetKwInput{Input}{Input~}
      \SetKwInput{Output}{Output~}
      \Input{~a list $objects$ of objects in FinSets and a morphism $gen$ given as a list of images in the convention of catreps}
      \Output{~the corresponding map of finite sets from source $S$ to target $T$}
      \BlankLine
      let $T$ be the first object $O \in objects$ such that $gen \cap O \not= \emptyset$\;
      \If{$gen \cap O = \emptyset \, \forall O \in objects$}{
         Error "unable to find target set"
      }
      let $fl$ be the flattening of $objects$ as a list\;
      let $S$ be the sublist of $fl$ according to positions $i$ such that $gen[i]$ is bound\;
      set $S$ to be the first object $O \in objects$ such that $O = S$\;
      \If{$S \not= O \, \forall O \in objects$}{
          Error "unable to find source set"
      }
      \BlankLine
      let $G$ be the list of pairs $[ i, gen[i] ], i \in S$;
      \BlankLine
      \Return MapOfFinSets( S, G, T );
\end{algorithm}

\section{The categories Functor-Categories and Cat-Reps}
\section{Conclusion}

%% best endnotes are in Westend-Verlag style: 
%% All at the end, sorted by chapter, starting from 1 at each new chapter.

%%% insert endnotes in some way
\begingroup
     \parindent 0pt
     \parskip 2ex
     \def\enotesize{\normalsize}
     \theendnotes
\endgroup 

\addcontentsline{toc}{section}{References}
\input{bib/sources.bib}

\appendix
\renewcommand{\thesection}{\Alph{section}}
\section{Implementation in \textsc{Cap}}

\lstlistoflistings\label{lol}

\renewcommand{\lstlistingname}{Procedure}
\lstset{
		basicstyle=\ttfamily\small,
		keywordstyle=\color{red},
		identifierstyle=,
		commentstyle=\color{green!70!black},
		stringstyle=\color{blue},
		showstringspaces=false,
		gobble=2,
		columns=fullflexible,
		tabsize=4,
		numbers=none, 
		frame=single}

%%% Check firstline and lastline for each function after change in the codebase ! ! !
%%% 1st line = InstallMethod
%%% last line = end );

%%% ConvertToMapOfFinSets
\lstinputlisting[
		firstline=8,
		lastline=33, 
		label=func:ConvertToMapOfFinSets,
		caption={$\mathtt{ConvertToMapOfFinSets}$},
		language=GAP
		]{\pkgpath/catreps/gap/CatRepsWithCAP.gi}
From the package \texttt{CatReps} (d$\And$i by Tibor Grün and improved by Mohamed Barakat on 12 May 2020) \\
Back to \hyperref[lol]{Index}

%%% ConcreteCategoryForCAP
\lstinputlisting[
		firstline=36,
		lastline=81, 
		label=func:ConcreteCategoryForCAP,
		caption={$\mathtt{ConcreteCategoryForCAP}$},
		language=GAP
		]{\pkgpath/catreps/gap/CatRepsWithCAP.gi}
From the package \texttt{CatReps} (d$\And$i by Mohamed Barakat on 20 Feb 2020) \\
Back to \hyperref[lol]{Index}
		
%%% RightQuiverFromConcreteCategory
\lstinputlisting[
		firstline=228,
		lastline=255, 
		label=func:RightQuiverFromConcreteCategory,
		caption={$\mathtt{RightQuiverFromConcreteCategory}$},
		language=GAP
		]{\pkgpath/catreps/gap/CatRepsWithCAP.gi}
From the package \texttt{CatReps} (d$\And$i by Tibor Grün on 7 May 2020) \\
Back to \hyperref[lol]{Index}
		
%%% RelationsOfEndomorphisms
\lstinputlisting[
		firstline=156,
		lastline=225, 
		label=func:RelationsOfEndomorphisms,
		caption={$\mathtt{RelationsOfEndomorphisms}$},
		language=GAP
		]{\pkgpath/catreps/gap/CatRepsWithCAP.gi}
From the package \texttt{CatReps} (d$\And$i by Tibor Grün on 7 May 2020) \\
Back to \hyperref[lol]{Index}
		
%%% Algebroid
\lstinputlisting[
		firstline=84,
		lastline=153, 
		label=func:Algebroid,
		caption={$\mathtt{Algebroid}$},
		language=GAP
		]{\pkgpath/catreps/gap/CatRepsWithCAP.gi}
From the package \texttt{CatReps} (template added by Mohamed Barakat on 16 April 2020, finalized by Tibor Grün on 7 May 2020) \\
Back to \hyperref[lol]{Index}
		
%%% YonedaEmbedding 
\lstinputlisting[
		firstline=199,
		lastline=251, 
		label=func:YonedaEmbedding,
		caption={$\mathtt{YonedaEmbedding}$},
		language=GAP
		]{\pkgpath/FunctorCategories/gap/Functors.gi}
From the package \texttt{FunctorCategories} (d$\And$i by Kamal Saleh on 18 May 2020) \\
Back to \hyperref[lol]{Index}

%%% DecomposeOnceByRandomEndomorphism
\lstinputlisting[
		firstline=8,
		lastline=66, 
		label=func:DecomposeOnceByRandomEndomorphism,
		caption={$\mathtt{DecomposeOnceByRandomEndomorphism}$},
		language=GAP
		]{\pkgpath/FunctorCategories/gap/DirectSumDecomposition.gi}
From the package \texttt{FunctorCategories}\\
Back to \hyperref[lol]{Index}
		
%%% WeakDirectSumDecomposition
\lstinputlisting[
		firstline=69,
		lastline=96, 
		label=func:WeakDirectSumDecomposition,
		caption={$\mathtt{WeakDirectSumDecomposition}$},
		language=GAP
		]{\pkgpath/FunctorCategories/gap/DirectSumDecomposition.gi}
From the package \texttt{FunctorCategories}\\
Back to \hyperref[lol]{Index}
		
%%% MorphismOntoSumOfImagesOfAllMorphisms
\lstinputlisting[
		firstline=245,
		lastline=263,
		breaklines=true,
		label=func:MorphismOntoSumOfImagesOfAllMorphisms,
		caption={$\mathtt{MorphismOntoSumOfImagesOfAllMorphisms}$},
		language=GAP
		]{\pkgpath/CategoryConstructor/gap/Tools.gi}
From the package \texttt{CategoryConstructor} (d$\And$i by Mohamed Barakat on 10 April 2020) \\
Back to \hyperref[lol]{Index}
		
%%% EmbeddingOfSumOfImagesOfAllMorphisms
\lstinputlisting[
		firstline=266,
		lastline=276,
		breaklines=true,
		label=func:EmbeddingOfSumOfImagesOfAllMorphisms,
		caption={$\mathtt{EmbeddingOfSumOfImagesOfAllMorphisms}$},
		language=GAP
		]{\pkgpath/CategoryConstructor/gap/Tools.gi}
From the package \texttt{CategoryConstructor} (d$\And$i by Mohamed Barakat on 10 April 2020)\\
Back to \hyperref[lol]{Index}
		
%%% SumOfImagesOfAllMorphisms
\lstinputlisting[
		firstline=279,
		lastline=288,
		breaklines=true,
		label=func:SumOfImagesOfAllMorphisms,
		caption={$\mathtt{SumOfImagesOfAllMorphisms}$},
		language=GAP
		]{\pkgpath/CategoryConstructor/gap/Tools.gi}
From the package \texttt{CategoryConstructor} (d$\And$i by Mohamed Barakat on 10 April 2020)\\
Back to \hyperref[lol]{Index}

		


\end{document}