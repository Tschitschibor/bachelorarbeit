\documentclass{article}

%\usepackage[top=37mm,bottom=37mm,left=27mm,right=27mm]{geometry}

%% From Gliederung preambel
% left=2.5cm,right=2.5cm,top=2.5cm,bottom=2.5cm

%\usepackage[ % % top 37, bottom 37
  %          top=27mm,bottom=27mm,left=27mm,right=27mm,%
    %        footskip=.6cm]{geometry}
            
\usepackage[ 
  a4paper,
  footskip=0.7cm,
  margin=2.7cm,
  top=1.1cm,
  bottom=1.4cm
]{geometry}            

\def\changemargin#1#2{\list{}{\rightmargin#2\leftmargin#1}\item[]}
\let\endchangemargin=\endlist

\usepackage{microtype}

\usepackage[utf8]{inputenc}
\usepackage[T1]{fontenc}%
\usepackage{fancyvrb}
\usepackage{enumitem}

\usepackage{fixltx2e}%
\usepackage{graphicx}%

% size of table of contents
\usepackage{tocloft}
\renewcommand{\cftsecfont}{\small\bfseries}
\renewcommand{\cftsubsecfont}{\small\mdseries}
\renewcommand{\cftsubsubsecfont}{\small\mdseries}


\usepackage[pdfauthor   = {Tibor\ Gr{\"u}n},
            pdftitle    = {},
            pdfsubject  = {},
            pdfkeywords = {},
            %plainpages
            %breaklinks=true, % still buggy under Linux with \stackrel
            %bookmarks=true,
            %bookmarksopen=true,
            %bookmarksnumbered=true, % produces errors on arXiv.org
            %pagebackref=true,
            hyperindex=true,
            hyperfootnotes=true,
            colorlinks=true,
            %linkcolor=red,
            linkcolor=blue,
            %citecolor=green,
            citecolor=blue,
            %filecolor=magenta,
            filecolor=blue,
            %urlcolor=cyan,
            urlcolor=blue,
            %pdftex=true
            %ps2pdf=true
            %hypertex=true
            ]{hyperref}
\usepackage[all]{hypcap}



%%% For captions and references
%\usepackage{endnotes}
\usepackage{enotez} %[backref]
\newcommand{\Algoref}[1]{%
	\hyperref[algo:#1]{Algorithm~\ref*{algo:#1}}%
}
\newcommand{\algoref}[1]{%
	\hyperref[algo:#1]{Algorithm~\ref*{algo:#1}}%
}
\newcommand{\Funcref}[1]{%
	\hyperref[func:#1]{Function~\ref*{func:#1}}%
}
\newcommand{\funcref}[1]{%
	\hyperref[func:#1]{\texttt{#1}}%
}

%%% For footnotes at end of text
\usepackage{endnotes}

%%% hyperendnotes.sty
\makeatletter
\newif\ifenotelinks
\newcounter{Hendnote}
% Redefining portions of endnotes-package:
\let\savedhref\href
\let\savedurl\url
\def\endnotemark{%
\@ifnextchar[\@xendnotemark{%
\stepcounter{endnote}%
\protected@xdef\@theenmark{\theendnote}%
\protected@xdef\@theenvalue{\number\c@endnote}%
\@endnotemark
}%
}%
\def\@xendnotemark[#1]{%
\begingroup\c@endnote#1\relax
\unrestored@protected@xdef\@theenmark{\theendnote}%
\unrestored@protected@xdef\@theenvalue{\number\c@endnote}%
\endgroup
\@endnotemark
}%
\def\endnotetext{%
\@ifnextchar[\@xendnotenext{%
\protected@xdef\@theenmark{\theendnote}%
\protected@xdef\@theenvalue{\number\c@endnote}%
\@endnotetext
}%
}%
\def\@xendnotenext[#1]{%
\begingroup
\c@endnote=#1\relax
\unrestored@protected@xdef\@theenmark{\theendnote}%
\unrestored@protected@xdef\@theenvalue{\number\c@endnote}%
\endgroup
\@endnotetext
}%
\def\endnote{%
\@ifnextchar[\@xendnote{%
\stepcounter{endnote}%
\protected@xdef\@theenmark{\theendnote}%
\protected@xdef\@theenvalue{\number\c@endnote}%
\@endnotemark\@endnotetext
}%
}%
\def\@xendnote[#1]{%
\begingroup
\c@endnote=#1\relax
\unrestored@protected@xdef\@theenmark{\theendnote}%
\unrestored@protected@xdef\@theenvalue{\number\c@endnote}%
\show\@theenvalue
\endgroup
\@endnotemark\@endnotetext
}%
\def\@endnotemark{%
\leavevmode
\ifhmode
\edef\@x@sf{\the\spacefactor}\nobreak
\fi
\ifenotelinks
\expandafter\@firstofone
\else
\expandafter\@gobble
\fi
{%
\Hy@raisedlink{%
\hyper@@anchor{Hendnotepage.\@theenvalue}{\empty}%
}%
}%
\hyper@linkstart{link}{Hendnote.\@theenvalue}%
\makeenmark
\hyper@linkend
\ifhmode
\spacefactor\@x@sf
\fi
\relax
}%
\long\def\@endnotetext#1{%
\if@enotesopen
\else
\@openenotes
\fi
\immediate\write\@enotes{%
\@doanenote{\@theenmark}{\@theenvalue}%
}%
\begingroup
\def\next{#1}%
\newlinechar='40
\immediate\write\@enotes{\meaning\next}%
\endgroup
\immediate\write\@enotes{%
\@endanenote
}%
}%
\def\theendnotes{%
\immediate\closeout\@enotes
\global\@enotesopenfalse
\begingroup
\makeatletter
\edef\@tempa{`\string>}%
\ifnum\catcode\@tempa=12
\let\@ResetGT\relax
\else
\edef\@ResetGT{\noexpand\catcode\@tempa=\the\catcode\@tempa}%
\@makeother\>%
\fi
\def\@doanenote##1##2##3>{%
\def\@theenmark{##1}%
\def\@theenvalue{##2}%
\par
\smallskip %<-small vertical gap between endnotes
\begingroup
\def\href{\expandafter\savedhref}%
\def\url{\expandafter\savedurl}%
\@ResetGT
\edef\@currentlabel{\csname p@endnote\endcsname\@theenmark}%
\enoteformat
}%
\def\@endanenote{%
\par\endgroup
}%
% Redefine, how numbers are formatted in the endnotes-section:
\renewcommand*\@makeenmark{%
\hbox{\normalfont\@theenmark~}%
}%
% header of endnotes-section
\enoteheading
% font-size of endnotes
\enotesize
\input{\jobname.ent}%
\endgroup
}%
\def\enoteformat{%
\rightskip\z@
\leftskip1.8em
\parindent\z@
\leavevmode\llap{%
\setcounter{Hendnote}{\@theenvalue}%
\addtocounter{Hendnote}{-1}%
\refstepcounter{Hendnote}%
\ifenotelinks
\expandafter\@secondoftwo
\else
\expandafter\@firstoftwo
\fi
{\@firstofone}%
{\hyperlink{Hendnotepage.\@theenvalue}}%
{\makeenmark}%
}%
}%
% stop redefining portions of endnotes-package:
\makeatother
% Toggle switch in order to turn on/off back-links in the
% endnote-section:
\enotelinkstrue
%\enotelinksfalse
%\let\footnote{\endnote}
%% Heading of endnotes section
\renewcommand*{\notesname}{Annotations}

\makeatletter
\renewcommand*{\enoteheading}{%
   \section*{\notesname%
   \@mkboth{\MakeUppercase{\notesname}}{\MakeUppercase{\notesname}}}%
\mbox{}\par\vskip-\baselineskip}
\makeatother



%%% For switching languages in quotes
\usepackage[english]{babel}
%\usepackage[english, german]{babel} %% makes troubles

%%% For quotation
%% english guillemets have to be custom defined in /tex/latex/csquotes/csquotes.cfg
%\usepackage[english = guillemets, autostyle = true,autopunct,csdisplay = true]{csquotes}
\usepackage[autostyle = true,autopunct,csdisplay = true]{csquotes}

%%% For proper underline
\usepackage{soul}
%\setuldepth{gjpqy}
%\setuldepth\strut
\setuldepth{-1}

%%% Color
\usepackage{xcolor}
\usepackage{color}
\definecolor{FireBrick}{rgb}{0.5812,0.0074,0.0083}
\definecolor{RoyalBlue}{rgb}{0.0236,0.0894,0.6179}
\definecolor{RoyalGreen}{rgb}{0.0236,0.6179,0.0894}
\definecolor{RoyalRed}{rgb}{0.6179,0.0236,0.0894}
\definecolor{LightBlue}{rgb}{0.8544,0.9511,1.0000}
\definecolor{Black}{rgb}{0.0,0.0,0.0}

\definecolor{linkColor}{rgb}{0.0,0.0,0.554}
\definecolor{citeColor}{rgb}{0.0,0.0,0.554}
\definecolor{fileColor}{rgb}{0.0,0.0,0.554}
\definecolor{urlColor}{rgb}{0.0,0.0,0.554}
\definecolor{promptColor}{rgb}{0.0,0.0,0.589}
\definecolor{brkpromptColor}{rgb}{0.589,0.0,0.0}
\definecolor{gapinputColor}{rgb}{0.589,0.0,0.0}
\definecolor{gapoutputColor}{rgb}{0.0,0.0,0.0}

%%  for a long time these were red and blue by default,
%%  now black, but keep variables to overwrite
\definecolor{FuncColor}{rgb}{0.0,0.0,0.0}
%% strange name because of pdflatex bug:
\definecolor{Chapter }{rgb}{0.0,0.0,0.0}
\definecolor{DarkOlive}{rgb}{0.1047,0.2412,0.0064}

%% command for ColorPrompt style examples
\newcommand{\gapprompt}[1]{\color{promptColor}{\bfseries #1}}
\newcommand{\gapbrkprompt}[1]{\color{brkpromptColor}{\bfseries #1}}
\newcommand{\gapinput}[1]{\color{gapinputColor}{#1}}

%%% For source code listings
\usepackage{listings}[2013/08/05]
\input{pfad.tex}
%\lstloadlanguages{GAP}

%%% For algorithm styles
\usepackage{xspace}
\usepackage[linesnumbered,ruled]{algorithm2e}
\usepackage{algpseudocode}
\SetKw{Continue}{continue}
\SetKw{Break}{break}
\SetKw{Not}{not\xspace}
\SetKw{AndAlg}{and\xspace}

%%% Math theorem styles
\usepackage{amsthm}

\newtheorem{theorem}{Theorem}[subsection]
\theoremstyle{definition}
\newtheorem{lemma}[theorem]{Lemma}
\newtheorem{corollary}[theorem]{Corollary}
\newtheorem{definition}[theorem]{Definition}
\newtheorem{remark}[theorem]{Remark}
\newtheorem{proposition}[theorem]{Proposition}
\newtheorem{example}[theorem]{Example}
\newtheorem{doctrine}[theorem]{Doctrine}
\newtheorem{computation}[theorem]{Computation}

%%% make equations count from subsection
\usepackage{chngcntr}
\counterwithin{equation}{subsection}

%%% for nested proofs
\newenvironment{subproof}[1][\proofname]{%
  \renewcommand{\qedsymbol}{$\mathbin{/\mkern-6mu/}$}%
  \begin{proof}[#1]%
}{%
  \end{proof}%
}

%%% for nicer Product sign
\newcommand{\invamalg}{\mathbin{\rotatebox[origin=c]{180}{$\amalg$}}}

%%% For Math
\usepackage{amsmath}
\usepackage{amsfonts}
\usepackage{amsbsy}
\usepackage{amssymb}
\usepackage{mathtools}
\usepackage{esvect}
\usepackage{commath}
\usepackage[sc,osf]{mathpazo}

%%% Macros for our recurring categories
\newcommand{\kmat}{\Bbbk\textnormal{-}\mathbf{mat}}
\newcommand{\kAlgebroid}{\Bbbk\textnormal{-}\mathrm{algebroid}}
\newcommand{\Rmat}{R\textnormal{-}\mathbf{mat}}
\newcommand{\HomAkmat}{\mathrm{Hom_{\Bbbk}}(\mathcal{A},\kmat)}
\newcommand{\HomARmat}{\mathrm{Hom_{R}}(\mathcal{A},\Rmat)}
\newcommand{\HomA}{\mathrm{Hom}_{\mathcal{A}}}
\newcommand{\FinSets}{\mathrm{FinSets}}
\newcommand{\Cat}{\mathrm{\textbf{Cat}}}
\newcommand{\Set}{\mathrm{\textbf{Set}}}
\newcommand{\Quiv}{\mathrm{\textbf{Quiv}}}
\newcommand{\kChat}{\widehat{\Bbbk\mathcal{C}}}

%%% Macros for the software packages
\newcommand{\Gap}{\textsc{Gap}\xspace}
\newcommand{\QPA}{\textsc{QPA$2$}\xspace}
\newcommand{\CatReps}{\texttt{CatReps}\xspace} %CAP package CatReps
\newcommand{\catreps}{\texttt{catreps}\xspace} %Peter Webb's catreps
\newcommand{\CAP}{\textsc{CAP}\xspace}
\newcommand{\homalgProject}{\texttt{homalg\_project}\xspace}
\newcommand{\FunctorCategories}{\texttt{FunctorCategories}\xspace}

%%% vel means or in latin, easier to remember
\newcommand{\vel}{\vee}

%%% For arrows and categories
%\usepackage[all]{xy} %%not used anymore
\usepackage{tikz-cd}

%%% For calculations and loops inside tikz and latex
\usepackage{calc}
\usepackage{pgffor}

\newcounter{modresult}
\newcommand*{\themodulo}[2]{%
\setcounter{modresult}{%
#1-(#1/#2)*#2%
}%
#1 mod #2 = \themodresult\par
}

%%% For matrices
\let\ampersand =&

%%% Math operators bold
%\newcommand{\Category}{Category}
% just use /textup{#1} inside math environment instead of redefining every math operator.

%%% tikz
\usetikzlibrary{positioning}
\usetikzlibrary{arrows}

%%% for captions of tikzpictures and other figures
\usepackage{capt-of}

%%% For function restrictions
\newcommand\restrict[1]{\raisebox{-.5ex}{$|$}_{#1}}

%%% For dotted box around diagrams
\tikzcdset{
    boxedcd/.style={
        every matrix/.append style={
            draw=black,
            dotted,
            rounded corners,
            #1
        },
    },
}

%%% For dotted arrows in math and in text
%% dottedrightarrow
\makeatletter
\newbox\dottedrightarrow@box
\setbox\dottedrightarrow@box\hbox
  {%
    \begin{tikzpicture}
      \draw[dotted,->] (0,0) -- (1.5em,0);
    \end{tikzpicture}%
  }
\newcommand*\dottedrightarrow
  {\relax\ifmmode\expandafter\dottedrightarrow@m\else\expandafter\dottedrightarrow@t\fi}
\newcommand*\dottedrightarrow@t[1][1.5em]
  {\resizebox{#1}{!}{\raisebox{.5ex}{\usebox\dottedrightarrow@box}}}
\newcommand*\dottedrightarrow@m[1][]
  {%
    \if\relax\detokenize{#1}\relax
      \mathchoice% values are trial and error based\ldots
        {\dottedrightarrow@t}
        {\dottedrightarrow@t}
        {\dottedrightarrow@t[1.1em]}
        {\dottedrightarrow@t[0.9em]}%
    \else
      \dottedrightarrow@t[#1]%
    \fi
  }
\makeatother
\let\olddottedrightarrow\dottedrightarrow
\renewcommand{\dottedrightarrow}{\raisebox{-.2em}{\,\,\olddottedrightarrow\,\,}}
%% dottedleftarrow
\makeatletter
\newbox\dottedleftarrow@box
\setbox\dottedleftarrow@box\hbox
  {%
    \begin{tikzpicture}
      \draw[dotted,<-] (0,0) -- (1.5em,0);
    \end{tikzpicture}%
  }
\newcommand*\dottedleftarrow
  {\relax\ifmmode\expandafter\dottedleftarrow@m\else\expandafter\dottedleftarrow@t\fi}
\newcommand*\dottedleftarrow@t[1][1.5em]
  {\resizebox{#1}{!}{\raisebox{.5ex}{\usebox\dottedleftarrow@box}}}
\newcommand*\dottedleftarrow@m[1][]
  {%
    \if\relax\detokenize{#1}\relax
      \mathchoice% values are trial and error based\ldots
        {\dottedleftarrow@t}
        {\dottedleftarrow@t}
        {\dottedleftarrow@t[1.1em]}
        {\dottedleftarrow@t[0.9em]}%
    \else
      \dottedleftarrow@t[#1]%
    \fi
  }
\makeatother
\let\olddottedleftarrow\dottedleftarrow
\renewcommand{\dottedleftarrow}{\raisebox{-.2em}{\,\,\olddottedleftarrow\,\,}}

%%% For some big dots (they still don't look very big)
\makeatletter
\newcommand*{\bigcdot}{}% Check if undefined
\DeclareRobustCommand*{\bigcdot}{%
  \mathbin{\mathpalette\bigcdot@{}}%
}
\newcommand*{\bigcdot@scalefactor}{.5}
\newcommand*{\bigcdot@widthfactor}{1.15}
\newcommand*{\bigcdot@}[2]{%
  % #1: math style
  % #2: unused
  \sbox0{$#1\vcenter{}$}% math axis
  \sbox2{$#1\cdot\m@th$}%
  \hbox to \bigcdot@widthfactor\wd2{%
    \hfil
    \raise\ht0\hbox{%
      \scalebox{\bigcdot@scalefactor}{%
        \lower\ht0\hbox{$#1\bullet\m@th$}%
      }%
    }%
    \hfil
  }%
}
\makeatother



\title{The category of representations of a concrete category as a functor category}

\author{Tibor Gr{\"u}n}

\begin{document}
	\pagenumbering{gobble}

	\maketitle

	\newpage

	\tableofcontents\label{toc}
	
	\newpage

	\pagenumbering{arabic}
	
\section*{Preface}
\addcontentsline{toc}{section}{Preface}

\section{Introduction to quivers and category theory}
% mainfile: ../main.tex

This section serves two purposes: On the one hand, it is an introduction to quivers and category theory. On the other hand it introduces
concrete categories which we want to represent, and all the additional constructions that are needed to that goal.

\subsection{Quivers}
In this section, we first want to define the category \textbf{Quiv} and how it is the prototype for the category \textbf{Cats}.
In order to describe the category \textbf{Quiv} of quivers, we first have to define what a category is and for this we need
the definition of a quiver. Lateron we will revisit this definition as we can define quivers as the objects in the quiver category \textbf{Quiv}.

\begin{definition}{(Quiver)}\label{def:quiver}\\
A \ul{directed graph} or \ul{quiver} $q$ consists of a class of \ul{objects} (or \ul{vertices}) $q_{0} = \textup{Obj}\,q$ and
a class of \ul{morphisms} (or \ul{arrows}) $q_{1} = \textup{Mor}\,q$ together with two defining maps
\[
\begin{tikzcd}[column sep=small]
{s,t\colon q_{1}} \arrow[rr, shift left = 0.7ex] \arrow[rr, shift right = 0.7ex] & & q_{0}
\end{tikzcd}
\]
$s$ called \ul{source} and $t$ called \ul{target}.
\end{definition}

In the next definition we are giving a new characterization for $q_{1}$ by looking at all arrows between two fixed objects.

\begin{definition}{(Hom-set of a (locally) small quiver)}\label{def:hom_set}
\renewcommand{\labelenumi}{(\theenumi)}
\begin{enumerate}
\item Given two objects $M, N \in q_{0}$ we write $\textup{Hom}_{q}(M,N)$ or $q(M,N)$ for the fiber
$(s,t)^{-1} (\{(M,N)\})$ of the product map 
\begin{tikzcd}[column sep=small]
(s, t) : q_{1} \arrow[rr] &  & q_{0} \times q_{0} 
\end{tikzcd} over the pair $(M,N) \in q_{0} \times q_{0}$.
This is the class of all morphisms with source $= M$ and target $= N$.
We indicate this by writing
\begin{tikzcd}[column sep=small]
\varphi : M \arrow[rr] &  & N
\end{tikzcd} or 
\begin{tikzcd}[column sep=small]
M \arrow[rr,"\varphi"] &  & N.
\end{tikzcd} Hence $q_{1}$ is the disjoint union $\bigcup\limits^{\bigcdot}_{M,N \in q_{0}} \textup{Hom}_{q}(M,N) = q_{1}$.
As usual we define $\textup{End}_{q}(M):= \textup{Hom}_{q}(M,M)$.
\item If the class $\textup{Hom}_{q}(M,N)$ is a \ul{set} for all pairs $(M,N)$ then we call the quiver \ul{locally small}.
We therefore talk about \ul{Hom-sets}.
If additionally, $q_{0}$ is a set, then the quiver is called \ul{small}.
\item A quiver with a finite set of objects and a finite set of morphisms is called a \ul{finite} quiver.
\end{enumerate}
\end{definition}

When we don't assume the category to be locally small, but still talk about its hom-sets, we mean the class of morphisms,
if we don't explicitly use the fact that it's a set of morphisms.

\begin{example}\label{q(2)}{(Quiver with 2 objects and 3 morphisms)}\\
\[
\begin{tikzcd}
1 \arrow["a"', loop, distance=2em, in=305, out=235] \arrow[rr, "b"] &  & 2 \arrow["c"', loop, distance=2em, in=305, out=235]
\end{tikzcd}
\]
The objects of this quiver $q$ are $q_{0} = \{1, 2\}$, and the morphisms are $q_{1} = \{a, b, c\}$ with\\
$s (a) = 1 = t (a)$, $s (c) = 2 = t (c)$ and $s (b) = 1, t (b) = 2$.\\
\noindent Thus $\textup{End}_{q}(1) = \{a\}, \textup{End}_{q}(2) = \{c\}$ and $\textup{Hom}_{q}(1,2) = \{b\}$ whereas
$\textup{Hom}_{q}(2,1)=\emptyset$.\\

\noindent In \texttt{QPA} this quiver is encoded as \texttt{q(2)[a:1->1,b:1->2,c:2->2]} where the first \texttt{(2)} in parentheses stands for the total
number of objects.
\end{example}

\begin{definition}{(Composable arrows; path in a quiver)}\label{def:path}\endnote{(ref. \ref{[leit4]} 4.1)}
Let $q$ be a quiver.
\begin{enumerate}
\renewcommand{\labelenumi}{(\theenumi)}
\item We say two arrows $a, b \in q_{1}$ are \ul{composable} if $t(a) = s(b)$ or $t(b) = s(a)$. In this case we can write a
sequence of composable arrows $p = a_{1}a_{2}\cdots a_{n}$ where $t(a_{i}) = s(a_{i+1})$ for $i=1,\dots,n-1$.
We call this sequence a \ul{path} from $s(a_{1})$ to $t(a_{n})$ and the integer $n \in \mathbb{Z}_{\geq0}$ the \ul{length} $l(p)$ of the path $p$.
Although it may not be an arrow, we can define the source and target of a path $p = a_{1}\cdots a_{n}$ as $s(p) := s(a_{1})$ and $t(p) := t(a_{n})$.
Then again we define two paths $p$ and $q$ as composable, if $t(p) = s(q)$ (or $t(q) = s(p)$) and we call $pq$ (or $qp$) the \ul{concatenation} or
\ul{composition} of the two paths. We can identify each arrow again as a path of length 1.
A path $p = a_{1}\cdots a_{n}$ with $s(a_{1}) = t(a_{n})$, i.e. $s(p) = t(p)$, is called \ul{cyclic}.
\item For an endomorphism $a \in \textup{End}_{q}(M)$ we write $a^{n}$ for $aa \cdots a$ ($n$ times).
\item In the case of $n=0$ an \ul{empty path} whose source and target are the vertex $i \in q_{0}$ is called the \ul{trivial path at $i$} and
is denoted $e_{i}$. Note that the composition of paths $e_{i}e_{i}$ has length zero starting at $i$ therefore $e_{i}^{2}=e_{i}$,
in other words, each $e_{i}$ is an \ul{idempotent}.
\end{enumerate}
\end{definition}

\begin{lemma}\label{la:cyclic_paths}
Let Q be a quiver. If there is a path of length at least $\abs{Q_{0}}$, then there are cyclic paths,
and thus infinitely many paths.\cite{[leit4]}
\end{lemma}
\begin{proof}
Assume that there exists a path of length greater or equal to $\abs{Q_{0}}$. Then there exists a path of length $n = \abs{Q_{0}}$, say
$\alpha_{1}\cdots \alpha_{n}$. Consider the vertices $x_{i}=s(\alpha_{i})$ for $1 \leq i \leq n$ and $x_{n+1}=t(\alpha_{n})$. Then these
are $n+1$ vertices, thus there has to exist $i<j$ with $x_{i}=x_{j}$. Let $\omega=\alpha_{i}\cdots \alpha_{j-1}$, this is a path with source and target
$x_{i}=x_{j}$, thus a cyclic path. But then $\omega^{m}$ is a path for any natural number $m$. The path $\omega$ has length $j-i\geq1$, thus
$\omega^{m}$ has length $m(j-i)$. This shows that these paths are pairwise different.
\end{proof}

\begin{example}{(A quiver with no cycles)}\\
\[
\begin{tikzcd}
2 \arrow[rrrr, "\psi"] \arrow[rrrrddd, "\psi\rho", pos=0.3] &  &  &  &
3 \arrow[ddd, "\rho"] \\
 &  &  &  & \\
 &  &  &  & \\
1 \arrow[uuu, "\varphi"] \arrow[rrrruuu, "\varphi\psi", pos=0.3] \arrow[rrrr, "\varphi\psi\rho" '] &  &  &  & 4
\end{tikzcd}
\]
The longest path $1\rightarrow2\rightarrow3\rightarrow4$ has length 3. If after the object $4$ another arrow would go to either $1,2,3$ or $4$ itself,
we would have a cyclic path and thus infinitely many paths.
\end{example}

\begin{definition}{(Path algebra of a quiver)}\label{def:path_algebra}\endnote{(from \ref{[leit4]} 4.1 )}
Let $\Bbbk$ be a field. For a quiver $Q$ let $\Bbbk Q$ be the vector space with basis the set of all paths in $Q$, together with the
following multiplication: if $w, w'$ are paths, let $ww'$ be the concatenation of $w$ and $w'$ if they are composable, and the zero vector
otherwise, and extend this multiplication bilinearly to $\Bbbk Q$. We call $\Bbbk Q$ the \ul{path algebra} of the quiver $Q$.
\end{definition}

Note that the addition of two paths $w + w'$ doesn't necessarily yield a path as result, but instead an abstract element of the
path algebra, that you can't easily see in the quiver.

\begin{lemma}\label{la:path_algebra_is_ass_algebra}\endnote{(from \ref{[leit4]} 4.1 )}
For a quiver $Q$ and a field $\Bbbk$, the path algebra $\Bbbk Q$ is an associative $\Bbbk$-algebra.
\end{lemma}
\begin{proof}
Let $w, w', w''$ be paths. Then both $(ww')w''$ and $w(w'w'')$ are the concatenation of $w$ on the left,
$w'$ in the middle and $w''$ on the right, in case both conditions $t(w) = s(w')$ and $t(w') = s(w'')$ are satisfied, and
otherwise the zero element (since $(ww')0 = 0, 0(w'w'') = 0$, according to bilinearity).\\
Since the multiplication was defined on a basis and extended bilinearly, the axioms of an algebra are clearly satisfied.
\end{proof}

\begin{lemma}\label{la:unit_in_path_algebra}
If the set of vertices of a quiver $Q_{0}$ is finite, then $\Bbbk Q$ has a unit element $\sum_{x\in Q_{0}} e_{x}$. In this case, $\Bbbk Q$ is a unital ring.
\end{lemma}
\begin{proof}
Let $e := \sum_{x\in Q_{0}} e_{x}$. Let $w$ be a path with $s(w) = x$ and $t(w) = y$, then $e_{x}w = w$ and $e_{z}w = 0$ for all $z \neq x$,
thus $ew = e_{x}w + \sum_{z\neq x} e_{z}w = w + 0 = w$. Similarly, $we_{y} = w$ and $we_{z} = 0$ for $z \neq x$.
\end{proof}

\subsection{Categories}

\begin{definition}{(Category)}\label{def:category}\\
\noindent A \ul{category} $\mathcal{C}$ is a quiver with two further maps:
\begin{enumerate}
\renewcommand{\labelenumi}{(id)}
\item The \ul{identity map} $1_{( )}$ mapping every object $X \in\mathcal{C}_{0}$ to its \ul{identity morphism} $1_{X}$:
\[
\begin{tikzcd}[column sep=small]
\mathcal{C}_{0} \arrow[rr,"1"] &  & \mathcal{C}_{1}
\end{tikzcd}
\]
\renewcommand{\labelenumi}{($\mu$)}
\item And for any two \ul{composable} morphisms $\varphi$ and $\psi \in \mathcal{C}_{1}$, i.e. with $t(\varphi) = s(\psi)$, the
\ul{composition map} $\mu$, which maps $\varphi, \psi \in \mathcal{C}_{1}\times\mathcal{C}_{1}$ to $\mu(\varphi,\psi) \in \mathcal{C}_{1}$ which
we also write as $\varphi\psi$. 
\[
\begin{tikzcd}[column sep=small]
\mathcal{C}_{1} \times \mathcal{C}_{1} \arrow[rr,"\mu"] &  & \mathcal{C}_{1}
\end{tikzcd}
\]
\end{enumerate}
\noindent The defining properties for $1$ and $\mu$ are:
\renewcommand{\labelenumi}{(\theenumi)}
\begin{enumerate}
\item $s(1_{M}) = M = t(1_{M})$, i.e.\\
$1_{M} \in \textup{End}_{\mathcal{C}} \forall M \in \mathcal{C}$.

\item $s(\varphi\psi) = s(\varphi)$ and\\
$t(\varphi\psi) = t(\psi)$\\
for all composable morphisms $\varphi, \psi \in \mathcal{C}$.
\[
\begin{tikzcd}[column sep=small]
\mu : \textup{Hom}_{\mathcal{C}}(M,L) \times \textup{Hom}_{\mathcal{C}}(L,N) \arrow[rr] &  & \textup{Hom}_{\mathcal{C}}(M,N)
\end{tikzcd}
\]
\item \label{associativity_of_composition} \begin{minipage}{.55\textwidth} $(\varphi\psi)\rho = \varphi(\psi\rho)$ \hfill{} [associativity of composition]\end{minipage}
\begin{minipage}{.45\textwidth}\phantom{}\end{minipage}
\item \label{unit_property} \begin{minipage}{.55\textwidth} $1_{s(\varphi)}\varphi = \varphi = \varphi1_{t(\varphi)}$ \hfill{} [unit property]\end{minipage}
\begin{minipage}{.45\textwidth}\phantom{}\end{minipage}\\
The identity is a left and right unit of the composition.
\end{enumerate}
\end{definition}

\noindent So with categories you always answer the four questions
\begin{itemize}\label{category_questions}
\item What are the objects? (which includes the question What are the identity morphisms?)
\item What are the morphisms?
\item How do you compose morphisms?
\item Why is the composition associative?
\end{itemize}

\subsection{Functors}

Categories are themselves objects in the category of categories, which leads to a question: What is a morphism between categories?

\begin{definition}{(Functor)}\label{def:functor}\\
\noindent A \ul{functor} $F : \mathcal{C} \rightarrow \mathcal{D}$, between categories $\mathcal{C}$ and $\mathcal{D}$, consists of the
following data:

\begin{itemize}
\item An object $Fc\in\mathcal{D}_{0}$, for each object $c \in \mathcal{C}_{0}$.
\item A function $Ff : Fc \rightarrow Fc' \in \mathcal{D}_{1}$, for each morphism $f : c \rightarrow c' \in \mathcal{C}_{1}$, so that the
source and target of $Ff$ are, respectively, equal to $F$ applied to the source or target of $f$, in other words,
$s(Ff) = Fs(f)$ and $t(Ff) = Ft(f)$.
\end{itemize}

\noindent The assignments are required to satisfy the following two \ul{functoriality axioms}:
\begin{itemize}\label{functoriality}
\item For any composable pair $f, g \in \mathcal{C}_{1}, Fg \cdot Ff = F(g \cdot f)$.
\item For each object $c \in \mathcal{C}_{0}, F(1_{c}) = 1_{Fc}$.
\end{itemize}

Put concisely, a functor consists of a mapping on objects and a mapping on morphisms that preserves all of the structure of a category,
namely domains and codomains, composition, and identities.
\end{definition}

\noindent So with functors you always answer the four questions
\begin{itemize}\label{four_functor_questions}
\item How does it work on objects?
\item How does it work on morphisms?
\item Why does it respect composition?
\item Why does it respect identity morphisms?
\end{itemize}

We have already seen an example for a functor in definition \ref{def:hom_set} where we defined the hom-set $\textup{Hom}(M,N)$ between two
objects $M$ and $N$. There are two ways to leave blank one of the objects and thus define the 

\begin{example}{(partial Hom-functor)}\label{ex:hom_functor}
Let $\mathcal{C}$ be a category and $P \in \mathcal{C}_{0}$ any object. The \ul{Hom-functor}, also called \ul{partial Hom-functor},
\begin{enumerate}
\item $\textup{Hom}(P,-)$ is a functor from $\mathcal{C}$ to $\mathcal{C}_{1}$ where objects in $\mathcal{C}_{1}$ are the hom-sets 
$\textup{Hom}(P,N)$, and morphisms are maps from one hom-set to another.
$\textup{Hom}(P,-)$ works on objects by mapping the object $N \in \mathcal{C}_{0}$ to
the hom-set $\textup{Hom}(P,N) \in \mathcal{C}_{1}$.
$\textup{Hom}(P,-)$ works on morphisms by mapping the morphism $(f : M \rightarrow N ) \in \mathcal{C}_{1}$ to the transformation
$\textup{Hom}(P,f) : \textup{Hom}(P,M) \rightarrow \textup{Hom}(P,N); \varphi \mapsto \varphi f$, so for every morphism
$\varphi \in \textup{Hom}(P,M)$, you post-compose $f \in \textup{Hom}(M,N)$ to get a new morphism $\varphi f \in \textup{Hom}(P,N)$.

\item $\textup{Hom}(-,P)$ is a functor from $\mathcal{C}$ to $\mathcal{C}_{1}$ where objects in $\mathcal{C}_{1}$ are the hom-sets 
$\textup{Hom}(N,P)$, and morphisms are maps from one hom-set to another.
$\textup{Hom}(-,P)$ works on objects by mapping the object $N \in \mathcal{C}_{0}$ to
the hom-set $\textup{Hom}(N,P) \in \mathcal{C}_{1}$.
$\textup{Hom}(-,P)$ works on morphisms by mapping the morphism $(f : M \rightarrow N ) \in \mathcal{C}_{1}$ to the transformation
$\textup{Hom}(f,P) : \textup{Hom}(N,P) \rightarrow \textup{Hom}(M,P); \varphi \mapsto f\varphi$, so for every morphism
$\varphi \in \textup{Hom}(N,P)$, you pre-compose $f \in \textup{Hom}(M,N)$ to get a new morphism $f\varphi \in \textup{Hom}(M,P)$.
\end{enumerate}

The important difference between these two functors was how they worked on morphisms. If in both cases we take a morphism
$f : M \rightarrow N$ as given, then we have to arrange the source and target for $\textup{Hom}(P,f)$ and $\textup{Hom}(f,P)$
according to the post-composition and pre-composition. Thus if we wanted $\textup{Hom}(f,P)$ to be defined by pre-composition
$\varphi \mapsto f\varphi$, then we were forced to invert $M$ and $N$ as source and target to get 
$\textup{Hom}(f,P): \textup{Hom}(N,P) \rightarrow \textup{Hom}(M,P)$. 
This process of inverting source and target is caught in the following definition.
\end{example}

\begin{definition}{(covariant / contravariant functor)}\endnote{(Def 1.3.5. in \cite{[context]}, p. 17 (35/258))}\\
The way we defined a functor in definition \ref{def:functor} was in the \ul{covariant} way.\\
A \ul{contravariant} functor $F : \mathcal{C} \rightarrow \mathcal{D}$ works on objects the same way as a covariant one, i.e.
an object $Fc \in \mathcal{D}_{0}$ for each object $c \in \mathcal{C}_{0}$. For morphisms on the other hand, we have
a morphism $F f : Fc' \rightarrow Fc \in \mathcal{D}_{1}$ for each morphism $f : c \rightarrow c' \in \mathcal{C}_{1}$, so that
$s(F f) = F t(f)$ and $t(F f) = F s(f)$.
The \ul{functoriality axioms} are also inverted for a contravariant functor:
For any composable pair, $f, g \in \mathcal{C}_{1}$, $F f \cdot F g = F(g \cdot f)$.
For the identity morphisms, it is again the same as in the covariant case:
For each object $c \in \mathcal{C}_{0}$, $F(1_{c}) = 1_{Fc}$.
\end{definition}

In the following definitions, we define different subclasses of functors. These adjectives often come in opposite pairs, so that you may be
tempted to think, duality lets you just swap all the adjectives for the opposite ones, but be careful there. E.g. when 
$\textup{Hom}(P,-)$ is a \ul{covariant}, \ul{left-exact} functor, the opposite $\textup{Hom}(-,P)$ is a \ul{contravariant}, but still \ul{left-exact} functor.
But their respective \ul{right-exactedness} is equivalent to dual concepts concerning \ul{projective} and \ul{injective} objects.

Limiten 

(mit Beispielen / dual)
Kernel

Pullback

Terminal object

Equalizer

\begin{definition}{(Exact functor)}\label{def:exact_functor}\endnote{(Def 4.5.9. in \cite{[context]}, p. 139 (157/258))}\\
A functor is \ul{right exact} or \ul{finitely cocontinuous} if it preserves finite colimits, and \ul{left exact} or \ul{finitely continuous} if it preserves finite limits.
\end{definition}

\begin{remark}
Without going into the details of defining what a limit and a colimit is, and with \ul{pullbacks} and \ul{pushouts} as specific kinds of
finite limits or colimits, and with the following proposition characterizing monomorphisms and epimorphisms,
we can give a definition for exact functor that is useful enough for our purposes.\endnote{(For a more on exact functors see above footnote,
on limits and colimits the same \cite{[context]}, chapter 3, pages 73 (91/258) onward, on pullback and pushout Def 3.1.15 p. 78 / Ex. 3.1.22, p. 80 f)}
\end{remark}

\begin{lemma}\label{prop:mono_pullback}
A morphism $f : a \rightarrow b$ is a monomorphism if and only if
the pullback of $f$ and $f$ exists and is $a$, together with the identity maps $1_{a} : a \rightarrow a$.
In other words, $f : a \rightarrow b$ is a monomorphism if and only if the commutative square
\[
\begin{tikzcd}
a \arrow[r, "1_{a}"] \arrow[d, "1_{a}"'] & a \arrow[d, "f"] \\
a \arrow[r, "f"]                         & b               
\end{tikzcd}
\]
is a pullback square.\endnote{(Cited from \cite{[Annoying Precision]})}

A dual statement exists for epimorphisms and pushouts, which are finite colimits.
\end{lemma}

\begin{corollary}{(from \ref{prop:mono_pullback})}\label{cor:preserve_mono_epi}

\begin{enumerate}
\item Being a monomorphism is a “limit property”: more precisely, any functor which preserves pullbacks
(in particular any functor which preserves finite limits, in particular any functor which preserves all limits)
preserves monomorphisms.
\item Being an epimorphism is a “colimit property”: more precisely, any functor which preserves pushouts
(in particular any functor which preserves finite colimits, in particular any functor which preserves all colimits)
preserves epimorphisms.\endnote{(Cited from \cite{[Annoying Precision]},
after pullback square and pushout square respectively)}
\end{enumerate}
\end{corollary}

\begin{lemma}
For functors between Abelian categories, left/right exactness is equivalent to preserving monos/epis.
\end{lemma}

\begin{lemma}\label{la:hom_functor_left_exact}
The hom functors $\textup{Hom}(P,-)$ and $\textup{Hom}(-,P)$ from \ref{ex:hom_functor} are left exact, i.e. respect monos.
\begin{proof}
Let $f : M \rightarrow N \in \mathcal{C}_{1}$ be a monomorphism, and let $O \in \mathcal{C}_{0}$ be any object.
Let $\mathfrak{g} : \textup{Hom}(P,N) \rightarrow \textup{Hom}(P,O); \varphi \mapsto \mathfrak{g}(\varphi)$
and $\mathfrak{h} : \textup{Hom}(P,N) \rightarrow \textup{Hom}(P,O); \varphi \mapsto \mathfrak{h}(\varphi)$
such that $\textup{Hom}(P,f) \cdot \mathfrak{g} : \textup{Hom}(P,M) \rightarrow \textup{Hom}(P,O); \psi \mapsto \mathfrak{g}(\psi f)$
and  $\textup{Hom}(P,f) \cdot \mathfrak{h} : \textup{Hom}(P,M) \rightarrow \textup{Hom}(P,O); \psi \mapsto \mathfrak{h}(\psi f)$
yield the same morphism, i.e. $\forall \psi \in \textup{Hom}(P,M), \mathfrak{g}(\psi f) = \mathfrak{h}(\psi f)$.
We want to show that - under the assumption that $f : M \rightarrow N$ was a monomorphism, already $\mathfrak{g} = \mathfrak{h}$.
TODO
\end{proof}
\end{lemma}

\begin{definition}{(Full functor)}\label{def:full_functor}\endnote{(Def 1.5.7. in \cite{[context]}, p. 30 (48/258))}\\
A functor $F : \mathcal{C} \rightarrow \mathcal{D}$ is \ul{full} if
$\forall x, y \in \mathcal{C}_{0}$, the map $\mathcal{C}(x, y) \rightarrow \mathcal{D}(Fx, Fy)$ is surjective.
\end{definition}

\begin{definition}{(Faithful functor)}\label{def:faithful_functor}\endnote{(ebd.)}\\
A functor $F$ as in \ref{def:full_functor} is \ul{faithful} if
$\forall x, y \in \mathcal{C}_{0}$, the map $\mathcal{C}(x, y) \rightarrow \mathcal{D}(Fx, Fy)$ is injective.
\end{definition}

\begin{definition}{(Essentially surjective on objects)}\label{def:ess_surj_o_o}\endnote{(ebd.)}\\
A functor $F$ as in \ref{def:full_functor} is \ul{essentially surjective on objects} if for every object $d \in \mathcal{D}_{0}$ there
is some $c \in \mathcal{C}_{0}$ such that $d$ is isomorphic to $Fc$.
\end{definition}

\begin{definition}{(Embedding)}\label{def:embedding}\endnote{(Rmk 1.5.8. in \cite{[context]}, p. 31 (49/258))}\\
A faithful functor that is injective on objects is called an \ul{embedding} and identifies the source category
as a subcategory of the target. In this case, faithfulness implies that the functor is (globally) injective on arrows.
\end{definition}

\begin{definition}{(Full embedding / full subcategory)}\label{def:full_fully}\endnote{(ebd.)}\\
A full and faithful functor, called \ul{fully faithful} for short, that is injective on objects defines a \ul{full embedding} of the
source category into the target category. The source then defines a \ul{full subcategory} of the target category.
\end{definition}

% cut-pasted from k-Algebroid.tex
\noindent As we have seen, every category is a quiver, but in general, to become a category, a quiver is lacking identity morphisms
and the composition of morphisms. To be more precise, there is a \ul{functor} $U$ from the \ul{category of categories} $\textup{CAT}$ to the
\ul{category of quivers} $\textup{Quiv}$, called the \ul{underlying quiver} or \ul{forgetful functor}.
\[
\begin{tikzcd}
\textup{Cat} \arrow[rr,"U"] &  & \textup{Quiv}
\end{tikzcd}
\]
mapping every object $M \in \mathcal{C}_{0}$ to the same objects in $q_{0}$, mapping every arrow $\varphi \in \mathcal{C}_{1}$ to 
an arrow $a \in q_{1}$, respecting source and target, but forgetting the special role of the identity morphisms and of the composition morphisms.

\begin{example}{(Free / Forgetful functor)}\label{ex:forgetful_functor}\\
TODO

$Free : \mathbf{Quiv} \rightarrow \mathbf{Cat}$

$U : \mathbf{Cat} \rightarrow \mathbf{Quiv}$
\end{example}

% Was bisher bei Category Closure geschah...
\begin{example}{(Category closure)}\\

\noindent\begin{minipage}{.08\textwidth}
\phantom{}
\end{minipage}
\begin{minipage}{.37\textwidth}
\begin{tikzcd}[boxedcd={inner xsep=1.5em, inner ysep=3em}]
B \arrow[rrrr, "\psi"] &  &  &  & C \arrow[ddd, "\rho"] \\
 &  &  &  & \\
 &  &  &  & \\
A \arrow[uuu, "\varphi"] &  &  &  & D
\end{tikzcd}
\end{minipage}
%
\begin{minipage}{.10\textwidth}
$\xrightarrow{\text{  Free }}$
\end{minipage}
%
\begin{minipage}{.37\textwidth}
\begin{tikzcd}[boxedcd={inner xsep=1.5em, inner ysep=3em}]
B \arrow[rrrr, "\psi"] \arrow[rrrrddd, "\psi\rho", pos=0.3] \arrow["1_{B}"', loop, distance=2em, in=125, out=55] &  &  &  &
C \arrow[ddd, "\rho"] \arrow["1_{C}"', loop, distance=2em, in=125, out=55]\\
 &  &  &  & \\
 &  &  &  & \\
A \arrow[uuu, "\varphi"] \arrow[rrrruuu, "\varphi\psi", pos=0.3] \arrow[rrrr, bend left, "(\varphi\psi)\rho" ', shift right=2]
\arrow[rrrr, "\varphi(\psi\rho)", bend right] \arrow["1_{A}"', loop, distance=2em, in=305, out=235] &  &  &  &
D \arrow["1_{D}"', loop, distance=2em, in=305, out=235]
\end{tikzcd}
\end{minipage}
\begin{minipage}{.08\textwidth}
\phantom{}
\end{minipage}\\

We can think of a quiver as a prototype for a category. That means we can construct the missing data for a category
from a quiver by adding the identity morphisms and the composed arrows.
\end{example}

% to be seen how useful this example is...
\begin{example}{(Underlying quiver)}\\

\noindent\begin{minipage}{.08\textwidth}
\phantom{}
\end{minipage}
\begin{minipage}{.37\textwidth}
\begin{tikzcd}[boxedcd={inner xsep=1.5em, inner ysep=3em}]
2 \arrow[rrrr, "b"] \arrow[rrrrddd, "e", pos=0.3] \arrow["h"', loop, distance=2em, in=125, out=55] &  &  &  &
3 \arrow[ddd, "c"] \arrow["i"', loop, distance=2em, in=125, out=55]\\
 &  &  &  & \\
 &  &  &  & \\
1 \arrow[uuu, "a"] \arrow[rrrruuu, "d", pos=0.3] \arrow[rrrr, bend left, "f" ', shift right=2]
\arrow[rrrr, "f", bend right] \arrow["g"', loop, distance=2em, in=305, out=235] &  &  &  &
4 \arrow["j"', loop, distance=2em, in=305, out=235]
\end{tikzcd}
\end{minipage}
%
\begin{minipage}{.10\textwidth}
$\xleftarrow{\text{   U   }}$
\end{minipage}
%
\begin{minipage}{.37\textwidth}
\begin{tikzcd}[boxedcd={inner xsep=1.5em, inner ysep=3em}]
B \arrow[rrrr, "\psi"] \arrow[rrrrddd, "\psi\rho", pos=0.3] \arrow["1_{B}"', loop, distance=2em, in=125, out=55] &  &  &  &
C \arrow[ddd, "\rho"] \arrow["1_{C}"', loop, distance=2em, in=125, out=55]\\
 &  &  &  & \\
 &  &  &  & \\
A \arrow[uuu, "\varphi"] \arrow[rrrruuu, "\varphi\psi", pos=0.3] \arrow[rrrr, bend left, "(\varphi\psi)\rho" ', shift right=2]
\arrow[rrrr, "\varphi(\psi\rho)", bend right] \arrow["1_{A}"', loop, distance=2em, in=305, out=235] &  &  &  &
D \arrow["1_{D}"', loop, distance=2em, in=305, out=235]
\end{tikzcd}
\end{minipage}
\begin{minipage}{.08\textwidth}
\phantom{}
\end{minipage}\\

\noindent In the category on the left, associativity of composition guaranteed that $(\varphi\psi)\rho = \varphi(\psi\rho)$, so those two arrows
were already the same, so they are mapped to the same arrow $f = U((\varphi\psi)\rho) = U(\varphi(\psi\rho))$ in the quiver on the right.
We didn't have to draw both arrows for $f$, but since they are equal, there is still only one arrow in the hom-set $\textup{Hom}_{q}(1,4)=\{f,f\} = \{f\}$.\\
All the other identities are not preserved under the forgetful functor, e.g. $d$ doesn't know what it has to do with $a$ and $b$ apart from
$s(d) = s(a)$ and $t(d) = t(b)$. Especially the former identity arrows are now just endomorphisms with no defining property.\\
The paths $g^{2}f, gf$ and $fj^{3}$ are all different, while in the category, they all simplify to
$1_{A}1_{A}(\varphi\psi)\rho = 1_{A}(\varphi\psi)\rho = (\varphi\psi)\rho1_{D}1_{D}1_{D} =  (\varphi\psi)\rho$ due to the unit property and associativity.
\end{example}


\subsection{Natural transformations}

With fixed categories $\mathcal{C}$ and $\mathcal{D}$ we can consider functors $F, G \in \textup{Hom}(\mathcal{C},\mathcal{D})$ themselves
as objects in the category $\textup{Hom}(\mathcal{C},\mathcal{D})$ of functors between $\mathcal{C}$ and $\mathcal{D}$. In this \ul{functor category},
the morphisms between two functors are called \ul{natural transformations}.

\begin{definition}{(Natural transformations)}\label{def:natural_transformation}\\
\noindent Given categories $\mathcal{C}$ and $\mathcal{D}$ and functors $F : \mathcal{C} \rightarrow \mathcal{D}$ and
$G : \mathcal{C} \rightarrow \mathcal{D}$, a \ul{natural transformation} $\alpha : F \Rightarrow G$ consists of:
\begin{itemize}
\item a morphism $\alpha_{c} : Fc \rightarrow Gc \in \mathcal{D}_{1}$ for each object $c \in \mathcal{C}_{0}$, the collection of which
define the \ul{components} of the natural transformation, so that, for any morphism $f : c \rightarrow c' \in \mathcal{C}_{1}$, the following
square of morphisms in $\mathcal{D}$
\[\begin{tikzcd}
Fc \arrow[rr, "\alpha_{c}"] \arrow[dd, "Ff"] &  & Gc \arrow[dd, "Gf"] \\
                                             &  &                     \\
Fc' \arrow[rr, "\alpha_{c'}"]                &  & Gc'                
\end{tikzcd}\]

\ul{commutes}, i.e., has a a common composite $Fc \rightarrow Gc' \in \mathcal{D}_{1}$.
\end{itemize}
When each component $\alpha_{c}$ is an isomorphism, we call $\alpha$ a \ul{natural isomorphism}.
\end{definition}

\subsection{The functor category}

\begin{definition}{(The functor category)}\label{def:functor_category}\endnote{(cited from ncatlab \cite{[ncatlab_functor_category]})}\\
Given categories $\mathcal{C}$ and $\mathcal{D}$, the \ul{functor category} - written $\mathcal{D}^{\mathcal{C}}$, $\textup{Hom}(\mathcal{C},
\mathcal{D})$ or $[\mathcal{C}, \mathcal{D}]$ -
is the category whose
\begin{itemize}
\item objects are functors $F : \mathcal{C} \rightarrow \mathcal{D}$
\item morphisms are natural transformations between these functors.
\end{itemize}
Main usage of functor categories is as $\textup{Hom}$ categories in place of hom-sets (comp. \ref{def:hom_set} and \ref{ex:hom_functor}) where
we have much more than a set, namely a whole category of morphisms between two objects (together with the morphisms between morphisms).
\end{definition}


%%% Limits + Abelian categories
\section{Limits and colimits}
% mainfile: ../main.tex

\subsection{Limit and colimit of a functor}

\begin{definition}{(Source of a functor)}
Let $D : \mathbf{I} \rightarrow \mathcal{C}$ be a functor. A \ul{source} of $D$ consists of the following data:
\begin{enumerate}
\renewcommand{\labelenumi}{(\theenumi)}
\item An object $S \in \mathcal{C}$.
\item A dependent function $s$ mapping an object $i \in \mathbf{I}_{0}$ to a morphism
$s(i) : S \rightarrow D(i)$ such that for all $i, j \in \mathbf{I}, \iota : i \rightarrow j$, we have $D(\iota) \cdot s(i) = s(j)$.
\end{enumerate}
\end{definition}

\begin{definition}{(Limit and colimit of a functor)}\label{def:limit}
Let $D : \mathbf{I} \rightarrow \mathcal{C}$ be a functor. A \ul{limit} of $D$ consists of the
following data:
\begin{enumerate}
\renewcommand{\labelenumi}{(\theenumi)}
\item A source of $D$ given by the data $(\mathrm{lim}\, D, (\lambda(i) : \mathrm{lim}\, D \rightarrow D(i))_{i\in\mathbf{I}_{0}})$.
\item A dependent function $u$, called the \ul{lift}, mapping every source $\tau = (T, (\tau(i) : T \rightarrow D(i))_{i \in \mathbf{I}})$ to a
morphism $u(\tau) : T \rightarrow \mathrm{lim}\, D$ such that $\lambda(i) \cdot u(\tau) = \tau(i)$ for all $i \in \mathbf{I}$.
This dependent function $u$ is unique with this property.
\end{enumerate}
A \ul{colimit} in $\mathcal{C}$ is a limit in $\mathcal{C}^{\mathrm{op}}$.
\end{definition}

\begin{definition}[Limits of type \textbf{I}]
Let $\mathbf{I}$ be a category. We say a category $\mathcal{C}$ \ul{has limits of type} $\mathbf{I}$ if it is
equipped with a dependent function $\lambda$ mapping a functor $D : \mathbf{I} \rightarrow \mathcal{C}$ to a limit
$(\mathrm{lim}\, D, \lambda_{D}, u_{D})$ of $D$.
We say $\mathcal{C}$ \ul{has colimits of type} $\mathbf{I}$ if $\mathcal{C}^{\mathrm{op}}$ has limits of that type.
\end{definition}

\begin{example}\label{ex:limits}
Depending on \textbf{I} some limits and colimits have special names:
\begin{center}
\begin{tabular}{c|c|c}
generating quiver of $\mathbf{I}$ & limit & colimit \\
\hline
$\emptyset$ & terminal object & initial object \\
$\cdot \text{\phantom{$\rightarrow$}} \cdot$ & binary product & binary coproduct \\
$\cdot \rightarrow \cdot \leftarrow \cdot$ & binary pullback & - \\
$\cdot \leftarrow \cdot \rightarrow \cdot$  & - & binary pushout \\
$ \cdot \rightrightarrows \cdot$ & binary equalizer & binary coequalizer
\end{tabular}
\end{center}
\end{example}

A category having or lacking limits and colimits of a certain type formalizes the notion of what one can or cannot \textit{do} in a category.
Since we are \textit{doing} constructive category theory, this all boils down to which limits and colimits we can \ul{compute}, i.e. for which we
have algorithms, and what needs to be true for the category in order for those algorithms to terminate with a correct output.
Just as in algebra words like ``field'' or ``ring'' or ``abelian group'' are established names and adjectives for sets with additional structure,
in category theory we give special names for categories which have certain limits. This is summarized under the vaguely defined
notion of ``doctrine''.

We say a category is of a certain \ul{categorical doctrine}, if it has all categorical operations appearing in the
defining axioms of the doctrine, in particular if it has all limits and colimits of a required set of types.
In this thesis, we will define a categorical doctrine by specifying a set of algorithms for all existential quantifyers and
disjunctions appearing in the defining axioms of the doctrine, in particular by specifying all algorithms needed to
compute the required (co-)limits.

Being a skeletal category ($\mathtt{IsSkeletalCategory}$) is not a categorical doctrine. Still, it is of high computational benefit
if we can represent a category by a skeletal model.

\subsection{A hierarchy of categorical doctrines}

In this subsection, we give explicit definitions for the (co-)limits in \ref{ex:limits} and define the
doctrines for our categories with such (co-)limits. The doctrine we are interested in is that of an Abelian category with
enough projectives and enough injectives. For now we will describe the doctrines up to and including Abelian categories, and
leave projective and injective objects for section 5. In section 3 we will work on the doctrine of $\Bbbk$-algebroids or $\Bbbk$-linear
categories.

\begin{doctrine}[Category]\label{doc:category}
The doctrine $\mathtt{IsCategory}$ involves the two algorithms
\begin{itemize}
\item $\mathtt{Source}$\endnote{The two algorithms $\mathtt{Source}$ and $\mathtt{Range}$ are implicit since
each morphism in \CAP has to be defined with source and target.},
\item $\mathtt{Range}$,
\item $\mathtt{PreCompose}$,
\item $\mathtt{IdentityMorphism}$.
\end{itemize}
\end{doctrine}


\subsubsection{Ab-categories}
We are starting simple with Ab-categories. All we need is an abelian group structure on the hom-set between two objects.

\begin{definition}[Zero morphism]\label{def:zero_morphism}\phantom{}\\
A \ul{zero morphism} from $M$ to $N$ is the neutral element of the abelian group $\mathrm{Hom}_{\mathcal{C}}(M,N)$.
\end{definition}
Note that every hom-set has its own unique zero morphism. E.g. in $\kmat$ the $2 \times 3$ zero-matrix
$0_{2,3} \in \textup{Hom}_{\kmat}(2,3)$ is different from the $4 \times 4$ zero-matrix $0_{4,4} \in \textup{Hom}_{\kmat}(4,4)$.

\begin{definition}[Ab-category]
An \ul{Ab-category} (also called \ul{pre-additive category}) is a category in which all homomorphism sets are abelian groups,
and composition distributes over addition.\\
In other words, a category $\mathcal{C}$ is an Ab-category if for every pair of objects $M,N \in \mathcal{C}_{0}$,
$( \mathrm{Hom}_{\mathcal{C}}(M,N), + )$ is an abelian group (with the zero morphism $0_{M,N}$ as the neutral element),
and for all morphisms $\gamma, \delta \in \mathrm{Hom}_{\mathcal{C}}(M,N),
\alpha, \beta \in \mathrm{Hom}_{\mathcal{C}}(N,L)$
\begin{align}
(\gamma + \delta)\alpha &=\label{eq:dist1} \gamma\alpha + \delta\alpha\quad \mathrm{ and }\\
\gamma(\alpha+\beta) &=\label{eq:dist2} \gamma\alpha + \gamma\beta.
\end{align}
\end{definition}

\begin{doctrine}[Ab-category]\label{doc:ab-category}
The doctrine $\mathtt{IsAbCategory}$ therefore involves algorithms for $\mathtt{IsCategory}$ and
\begin{itemize}
\item $\mathtt{AdditionForMorphisms}$,
\item $\mathtt{AdditiveInverseForMorphisms}$,
\item ($\mathtt{SubtractionForMorphisms}$),
\item $\mathtt{ZeroMorphism}$,
\item $\mathtt{IsZeroForMorphisms}$.
\end{itemize}
\end{doctrine}

\begin{definition}[Ab-functor]
A functor between Ab-categories is called an \ul{Ab-functor} if in the functor definition \ref{def:functor} the function $Ff$ for each
morphism $f$ is a homomorphism of abelian groups, i.e. $F(f+g) = Ff + Fg$.
\end{definition}

\subsubsection{Categories with a zero object}

\begin{remark}[Terminal object, initial object, zero object]\label{def:init_term_zero_object}
The limit / colimit of $\emptyset$.
\renewcommand{\labelenumi}{(\theenumi)}
\begin{enumerate}
\item A \ul{terminal object} $T$ in a category $\mathcal{C}$ is an object such that $\textup{Hom}_{\mathcal{C}}(-,T)$ is a singleton.
\item An \ul{initial object} $I$ in a category $\mathcal{C}$ is an object such that $\textup{Hom}_{\mathcal{C}}(I,-)$ is a singleton.
\item An object $Z$ or $0$ is a \ul{zero object} if it is both initial and terminal. (Bilimit of $\emptyset$).
\end{enumerate}
\end{remark}

\begin{definition}[Zero morphism]\label{def:zero_morphism}\phantom{}\\
A \ul{zero morphism} in a category with a zero object $0$ is a morphism factoring over $0$, i.e. $\varphi : M \rightarrow N$ is called a zero
morphism, if\\
\begin{minipage}{.35\textwidth}
\begin{tikzcd}
M \arrow[rr, "\varphi"] \arrow[rd, "\varphi_{1}"] &                              & N \\
                                                  & 0 \arrow[ru, "\varphi_{2}"'] &  
\end{tikzcd}
\end{minipage}
\begin{minipage}{.65\textwidth}
$\exists \varphi_{1} : M \rightarrow 0, \varphi_{2} : 0 \rightarrow N$\\
such that $\varphi = \varphi_{1}\varphi_{2}$.
\end{minipage}
Since the zero object $0$ is both initial and terminal, a zero morphism is uniquely defined by its source and target, thus we can
talk about \textit{the} zero morphism from $M$ to $N$, which we denote by $0_{M,N}$.\endnote{We could define the zero morphism without
the zero object, just as the neutral element of the abelian group $\mathrm{Hom}_{\mathcal{C}}(M,N)$. Therefore $\mathtt{ZeroMorphism}(M,N)$
is needed in the doctrine of $\mathtt{IsAbCategory}$, but the $\mathtt{ZeroObject}$ is not. Eventually at additive categories the
zero object arrises naturally as the direct sum of $\emptyset$, which is why $\mathtt{ZeroObject}$ is listed only there.}
\end{definition}

\begin{doctrine}[Category with zero object]
The doctrine $\mathtt{IsCategoryWithZeroObject}$ therefore involves algorithms for
\begin{itemize}
 \item $\mathtt{ZeroObject}$,
 \item $\mathtt{UniversalMorphismFromZeroObject}$,
 \item $\mathtt{UniversalMorphismIntoZeroObject}$.
\end{itemize}
\end{doctrine}

If an Ab-category has a zero object, then both notions of a zero morphism coincide.

\subsubsection{Additive categories}
The definition of the binary operation $+$ in a pre-additive category came as arbitrary outside data and could be defined
in multiple ways. A category with the following limits is additive in at most one way, warrenting the name additive category.

\begin{definition}[Product, coproduct]\label{def:prod_coprod}
The limit / colimit of a set of objects $\,\cdot \text{\phantom{$\rightarrow$}} \cdot$\\
Let $I$ be an index set and $\{A_{i}\}_{i\in I}$ a family of objects in a category $\mathcal{C}$.
\setlist[description]{font=\normalfont}
\begin{description}
\item[(prod)] The \ul{product} of the family $\{A_{i}\}_{i\in I}$ is an object $\invamalg A_{i}$ together with a family of morphisms
\[
\{ \pi_{i} : \invamalg A_{i} \rightarrow A_{i} \}
\]
called \ul{projections}, such that the following universal property is satisfied:\\
For any object $M \in \mathcal{C}_{0}$ and any family $\{ \varphi_{i} : M \rightarrow A_{i} \}_{i\in I}$ of morphisms, there exists
a unique morphism $\varphi : M \rightarrow \invamalg A_{i}$ called the \ul{product morphism} such that
\[
\varphi \pi_{i} = \varphi_{i} \, \forall i \in I.
\]
\begin{tikzcd}
                                                                                                                            &  &                                                          & A_{1} \\
M \arrow[rrru, "\varphi_{1}", bend left] \arrow[rrrd, "\varphi_{2}"', bend right] \arrow[rr, "\exists^{1} \varphi", dashed] &  & A_{1}\invamalg A_{2} \arrow[ru, "\pi_{1}"] \arrow[rd, "\pi_2"] &       \\
                                                                                                                            &  &                                                          & A_{2}
\end{tikzcd}
\item[(coprod)] The dual notion to product is the \ul{coproduct} of the family $\{A_{i}\}_{i\in I}$, that is an object $\amalg A_{i}$ together with
a family of morphisms
\[
\{ \iota_{i} : A_{i} \rightarrow \amalg A_{i} \}
\]
called \ul{coprojections} or sometimes \ul{injections}, \ul{inclusions} or \ul{embeddings}, such that the following universal property is satisfied:\\
For any object $M \in \mathcal{C}$ and any family $\{ \psi_{i} : A_{i} \rightarrow M \}$ of morphisms, there exists a unique
morphism $\psi : \amalg A_{i} \rightarrow M$ called the \ul{coproduct morphism} such that
\[
\iota_{i} \psi = \psi_{i} \, \forall i \in I.
\]
\begin{tikzcd}
  &  &                                                           & A_{1} \arrow[llld, "\psi_{1}"', bend right] \arrow[ld, "\iota_{1}"'] \\
M &  & A_{1}\amalg A_{2} \arrow[ll, "\exists^{1} \psi"', dashed] &                                                                      \\
  &  &                                                           & A_{2} \arrow[lllu, "\psi_{2}", bend left] \arrow[lu, "\iota_2"]     
\end{tikzcd}
\end{description}
\end{definition}

\begin{definition}[Biproduct]\label{def:biproduct}
Let $I$ be an index set and $\{S_{i}\}_{i\in I}$ a family of objects in a category $\mathcal{C}$.
A \ul{biproduct} is a product and a coproduct simultaneously, i.e. consists of the following data:
\begin{itemize}
\item an object $S \in \mathcal{C}_{0}$,
\item a family of morphisms $\pi = \{ \pi_{i} : S \rightarrow S_{i} \}_{i\in I}$,
\item a family of morphisms $\iota = \{ \iota_{i} : S_{i} \rightarrow S \}_{i\in I}$,
\item a dependent function $u_{\text{in}}$ mapping every family $\tau = \{ \tau_{i} : T \rightarrow S_{i} \}_{i\in I}$ of morphisms
with the same source $T$ to a morphism
$u_{\text{in}}(\tau) : T \rightarrow S$ such that $u_{\text{in}}(\tau) \pi_{i} \sim \tau_{i}$ for all $i \in I$,
\item a dependent function $u_{\text{out}}$ mapping every family $\rho = \{ \rho_{i} : S_{i} \rightarrow R \}_{i\in I}$ of morphisms
with the same target $R$ to a morphism
$u_{\text{out}}(\rho) : S \rightarrow R$ such that $\iota_{i} u_{\text{out}}(\rho) \sim \rho_{i}$ for all $i \in I$,
\end{itemize}
\end{definition}

\begin{definition}{(Direct sum)}\label{def:direct_sum}
A \ul{direct sum} is a biproduct of objects in an Ab-category such that
\begin{itemize}
\item $\sum_{i\in I}  \pi_{i} \iota_{i} \sim 1_{S}$,
\item $ \iota_{i} \pi_{j} \sim \delta_{i, j} =  \begin{cases}
            1_{S_{i}} & \text{ if } i = j  \\
            0_{ij} & \text{ if } i \neq j
        \end{cases}$,
\end{itemize}
where $\delta_{i, j} \in \mathrm{Hom}(S_{i}, S_{j})$ is the identity if $i = j$, and the zero morphism $0_{ij} := 0_{S_{i}, S_{j}}$ otherwise.
\end{definition}

\begin{example}[The direct sum of $\emptyset$]\label{ex:sum_of_empty}
For the index set $I = \emptyset$ we get an empty family of objects in a category $\mathcal{C}$. Its direct sum is
\begin{itemize}
\item an object $Z \in \mathcal{C}_{0}$,
\item empty morphism sets $\pi$ and $\iota$,
\item a dependent function $u_{\text{in}}$ maps an empty collection $\tau$ of morphisms with same source $T$ to a unique
morphism $u_{\text{in}}(\tau) : T \rightarrow Z$.
\item a dependent function $u_{\text{out}}$ maps an empty collection $\rho$ of morphisms with same target $R$ to a unique
morphism $u_{\text{out}}(\rho) : Z \rightarrow R$.
\end{itemize}
The unique morphisms $T \rightarrow Z$ and $Z \rightarrow T$ for any object $T$ (the empty family $\tau$ imposing no further condition)
suggests that our object $Z = \mathrm{DirectSum}(\emptyset)$ is in fact the zero object from \ref{def:init_term_zero_object}, and the
unique morphisms are the unique zero morphisms in $\mathrm{Hom}(Z,T)$ and $\mathrm{Hom}(T,Z)$ from \ref{def:zero_morphism}.
\end{example}

\begin{definition}\label{def:additive_category}
An \ul{additive category} is a pre-additive category $\mathcal{C}$ together with a dependent function $\oplus^{\mathcal{C}}$ mapping
a finite set $I$ and a family $(A_{i})_{i\in I}$ of objects in $\mathcal{C}$ to a corresponding direct sum $(\oplus_{i\in I}^{\mathcal{C}} A_{i},
(\pi_{i})_{i\in I}, (\iota_{i})_{i\in I})$.
\end{definition}

\begin{remark}[Addition of morphisms]\label{rmk:addition_derived_from_direct_sum}
In an additive category the abelian group structure on $\mathrm{Hom}_{\mathcal{C}}(M,N)$ can be derived from the direct sum:
For $\rho_{1}, \rho_{2} \in \mathrm{Hom}_{\mathcal{C}}(M,N)$
\[
\rho_{1} + \rho_{2} = u_{\mathrm{in}}(1_{M},1_{M}) u_{\mathrm{out}}(\rho_{1},\rho_{2}) 
= u_{\mathrm{in}}(\rho_{1},\rho_{2}) u_{\mathrm{out}}(1_{N},1_{N})
\]
The above equation illustrates that an additive category is
pre-additive in at most one way, i.e. we don't have a choice how we define the abelian group structure on the hom-sets.
\endnote{This result is already implemented in \textsc{Cap} as a derivation of \texttt{AdditionForMorphisms} from the four morphisms
\texttt{UniversalMorphismIntoDirectSum}, \texttt{IdentityMorphism}, \texttt{UniversalMorphismFromDirectSum} and \texttt{PreCompose}.
See 
\url{https://github.com/homalg-project/CAP_project/blob/v2019.06.06/CAP/gap/DerivedMethods.gi\#L1024}}
\end{remark}

\begin{doctrine}[Additive category]
The doctrine $\mathtt{IsAdditiveCategory}$ therefore involves algorithms of $\mathtt{IsAbCategory}$ and
$\mathtt{IsCategoryWithZeroObject}$ together with algorithms for
\begin{itemize}
 \item $\mathtt{DirectSum}$,
 \item $\mathtt{ProjectionInFactorOfDirectSum}$,
 \item $\mathtt{InjectionOfCofactorOfDirectSum}$,
 \item $\mathtt{UniversalMorphismIntoDirectSum}$,
 \item $\mathtt{UniversalMorphismFromDirectSum}$.
\end{itemize}
\end{doctrine}

\subsubsection{Pre-abelian categories}

A pre-abelian category is an additive category with kernels and cokernels, and hence images and coimages. To define
kernels and cokernels we need the following definition.

\begin{definition}[binary equalizer]
The limit of two parallel morphisms $\,\cdot \rightrightarrows \cdot$\\
If it exists in a category $\mathcal{C}$, the \ul{equalizer} of two morphisms $f, g : A \rightrightarrows B \in \mathcal{C}_{1}$
consists of the data
\begin{itemize}
\item an object $E := \mathrm{Eq}(f,g) \in \mathcal{C}_{0}$
\item a morphism $\iota := E \hookrightarrow A$ such that pulled back to $E$, both morphisms are equal $\iota\,f = \iota\,g$:
\begin{align*}
&E \xrightarrow{\iota} A \xrightarrow{f} B \\
=\, &E \xrightarrow{\iota} A \xrightarrow{g} B
\end{align*}
\item a dependent function $u$ such that for any other morphism $\tau : T \rightarrow A$ with
\begin{align*}
&T \xrightarrow{\tau} A \xrightarrow{f} B \\
=\, &T \xrightarrow{\tau} A \xrightarrow{g} B
\end{align*}
we have a unique morphism $u( \tau ) : T \rightarrow E$ such that $u( \tau ) \iota = \tau$.
\[
\begin{tikzcd}
E \arrow[r, "\iota", hook]                              & A \arrow[r, "f", shift left] \arrow[r, "g"', shift right] & B \\
T \arrow[ru, "\tau"] \arrow[u, "u(\tau)", dashed] &                                                           &  
\end{tikzcd}
\]
\end{itemize}
The dual concept is that of a \ul{coequalizer}, which is the colimit of $\,\cdot \rightrightarrows \cdot$.
\end{definition}

The following definition is a special case of an equalizer where $g = 0_{A,B}$. We will write it all out explicitly with their
own names for objects, morphisms and (co-)lifts.

\begin{definition}[Kernel]\label{def:kernel}\phantom{}\\
In an additive category $\mathcal{C}$, the \ul{kernel} of a morphism $f : A \rightarrow B \in \mathcal{C}_{1}$ is the equalizer
of $f$ and $0_{A,B}$, i.e. consists of the data
\begin{enumerate}
\renewcommand{\labelenumi}{(\theenumi)}
\item An object $K = \mathrm{Ker}(f)$
\item A morphism $\mathrm{KernelEmbedding}(f) := \kappa : K \rightarrow A$ such that $\kappa\,f = \kappa\,0_{A,B} = 0_{K,B}$:
\begin{align*}
\begin{tikzcd}[
  ampersand replacement=\&,
  row sep=1em,
]
{\phantom{=\, }K} \arrow[r, "\kappa"]                 \& A \arrow[r, "f"] \& B \\
{=\, K} \arrow[r, "\kappa"]   \& A \arrow[r, "0_{A,B}"] \& B \\
{= \, K} \arrow[rr, "{0_{K,B}}"] \&                            \& B
\end{tikzcd}
\end{align*}
\item A unique dependent function $\mathrm{KernelLift}(f,-) := ( - /\kappa)$ mapping a morphism $\tau : T \rightarrow A$ with $\tau f = 0_{T,B}$ to a
morphism $\mathrm{KernelLift}(f,\tau) = (\tau / \kappa) : T \rightarrow K$ such that
\[
\tau = (\tau / \kappa)\, \kappa.
\]
\end{enumerate}
\[
\begin{tikzcd}
K \arrow[r, "\kappa"]                            & A \arrow[r, "f"] & B \\
T \arrow[ru, "\tau"'] \arrow[u, "(\tau/\kappa)", dashed] &                  &  
\end{tikzcd}
\]
The last property means that the kernel embedding $\kappa$ dominates every such $\tau$, and that the kernel lift
$(\tau/\kappa)$ is the unique lift of $\tau$ along $\kappa$ in the sense of \ref{def:lift_colift_codominate}(1).
\end{definition}

\begin{definition}[Cokernel]
In an additive category $\mathcal{C}$, the \ul{cokernel} of a morphism $f : A \rightarrow B \in \mathcal{C}_{1}$ is the coequalizer of
$f$ and $0_{A,B}$, i.e. consists of the data
\begin{enumerate}
\renewcommand{\labelenumi}{(\theenumi)}
\item An object $C = \mathrm{Coker}(f)$
\item A morphism $\mathrm{CokernelProjection}(f) := \varepsilon : B \rightarrow C$ such that $f\,\varepsilon = 0_{A,B}\,\varepsilon = 0_{A,C}$:
\begin{align*}
\begin{tikzcd}[
  ampersand replacement=\&,
  row sep=1em,
]
{\phantom{=\, }A} \arrow[r, "f"]                 \& B \arrow[r, "\varepsilon"] \& C \\
{=\, A} \arrow[r, "{0_{A,B}}"]   \& B \arrow[r, "\varepsilon"] \& C \\
{= \, A} \arrow[rr, "{0_{A,C}}"] \&                            \& C
\end{tikzcd}
\end{align*}
\item A unique dependent function $\mathrm{CokernelColift}(f,-) := ( \varepsilon \backslash -)$ mapping a morphism $\tau : B \rightarrow T$ with
$f \tau  = 0_{A,T}$ to a morphism $\mathrm{CokernelColift}(f,\tau) = ( \varepsilon \backslash \tau) : C \rightarrow T$ such that
\[
\tau =\label{eq:cokernel_colift} \varepsilon \, (\varepsilon \backslash \tau).
\]
\end{enumerate}
\[
\begin{tikzcd}
A \arrow[r, "f"] & B \arrow[r, "\varepsilon", two heads] \arrow[rd, "\tau"] & C \arrow[d, "(\varepsilon\backslash\tau)", dashed] \\
                 &                                                          & T                                                 
\end{tikzcd}
\]
The last property means that the cokernel projection $\varepsilon$ codominates every such $\tau$, and that the cokernel colift
$(\varepsilon \backslash \tau)$ is the unique colift of $\tau$ along $\varepsilon$ in the sense of \ref{def:lift_colift_codominate}(2).
\end{definition}

\begin{definition}[Image]
In an additive category $\mathcal{C}$ we define the \ul{image} of a morphism\\
$f : A \rightarrow B \in \mathcal{C}_{1}$ as the kernel of its cokernel:\\
\begin{minipage}{.06\textwidth} \phantom{} \end{minipage}
\begin{minipage}{.39\textwidth}
\[
\begin{tikzcd}[
  ampersand replacement=\&,
]
A \arrow[r, "f"] \arrow[d, "(f/\kappa_{\varepsilon})"', dashed]         \& B \arrow[r, "\varepsilon", shift left, two heads]
\arrow[r, "{0_{B,C}}"', shift right] \& C \\
K \arrow[ru, "\kappa_{\varepsilon}", hook]                                    \&                            \&   \\
T \arrow[u, "(\tau/\kappa_{\varepsilon})", dashed] \arrow[ruu, "\tau"'] \&                            \&  
\end{tikzcd}
\]
\end{minipage}
\begin{minipage}{.49\textwidth}
Since $\varepsilon$ is the cokernel projection of $f$, the morphism $f : A \rightarrow B$ with $f\,\varepsilon = 0_{A,C}$ plays the
same role as any $\tau : T \rightarrow B$ with $\tau\,\varepsilon = 0_{T,C}$ in that it factors over the kernel object $K$ in a
unique way: 
\[
f = (f/\kappa_{\varepsilon})\,\kappa_{\varepsilon}
\]
\end{minipage}
\begin{minipage}{.06\textwidth} \phantom{} \end{minipage}

Thus, the image is
\begin{itemize}
\item An object $\mathrm{Im}(f) := \mathrm{Ker}(\varepsilon)$ where $\varepsilon = \mathrm{CokernelProjection}(f)$,
\item A morphism $\mathrm{ImageEmbedding}(f) := \mathrm{KernelEmbedding}(\mathrm{CokernelProjection}(f))$
\end{itemize}
To differentiate the image of $f$ (which is a kernel, but not the kernel of $f$) from the kernel of $f$,
we are using $I$ for $\mathrm{Im}(f)$ and $\iota$ for $\mathrm{ImageEmbedding}(f)$.
\end{definition}

\begin{definition}[Coimage]
In an additive category $\mathcal{C}$ we define the \ul{coimage} of a morphism $f : A \rightarrow B \in \mathcal{C}_{1}$ as the 
cokernel of its kernel:\\
\begin{minipage}{.06\textwidth} \phantom{} \end{minipage}
\begin{minipage}{.39\textwidth}
\[
\begin{tikzcd}[
  ampersand replacement=\&,
]
K \arrow[r, "\kappa", hook, shift left] \arrow[r, "{0_{K,A}}"', shift right] \& A \arrow[r, "f"] \arrow[rd, "{\varepsilon_{\kappa}}", two heads] \arrow[rdd, "\tau"'] \& B                                                                                                   \\
                                                                             \&                                                                            \& C \arrow[u, "({\varepsilon_{\kappa}}\backslash f)"', dashed] \arrow[d, "({\varepsilon_{\kappa}}\backslash \tau)", dashed] \\
                                                                             \&                                                                            \& T                                                                                                  
\end{tikzcd}
\]
\end{minipage}
\begin{minipage}{.49\textwidth}
Since $\kappa$ is the kernel embedding of $f$, the morphism $f : A \rightarrow B$ with $\kappa\,f = 0_{K,B}$ plays the
same role as any $\tau : A \rightarrow T$ with $\kappa\,\tau = 0_{K,T}$ in that it factors over the cokernel object $C$ in a
unique way:
\[
f = \varepsilon_{\kappa}\,(\varepsilon_{\kappa}\backslash f)
\]
\end{minipage}
\begin{minipage}{.06\textwidth} \phantom{} \end{minipage}

Thus, the coimage is
\begin{itemize}
\item An object $\mathrm{Coim}(f) := \mathrm{Coker}(\kappa)$ where $\kappa = \mathrm{KernelEmbedding}(f)$,
\item A morphism $\varepsilon_{\kappa} := \mathrm{CoimageProjection}(f) := \mathrm{CokernelProjection}(\mathrm{KernelEmbedding}(f))$
\end{itemize}
Since it was reserved for cokernel, we are not using $C$ to denote the coimage object.
\end{definition}

\begin{definition}{(Pre-Abelian category)}
A \ul{pre-abelian category} consists of the following data:
\begin{enumerate}
\renewcommand{\labelenumi}{(\theenumi)}
\item An additive category $\mathcal{C}$.
\item A dependent function mapping every morphism $f : A \rightarrow B$ for $A, B \in \mathcal{C}_{0}$ to a
kernel of $f$.
\item A dependent function mapping every morphism $f : A \rightarrow B$ for $A, B \in \mathcal{C}_{0}$ to a
cokernel of $f$.
\end{enumerate}
\end{definition}

\begin{doctrine}[Pre-abelian category]\label{doc:pre-abelian}
The doctrine $\mathtt{IsPreAbelianCategory}$ therefore involves algorithms of $\mathtt{IsAdditiveCategory}$ together with algorithms for
\begin{itemize}
\item $\mathtt{KernelObject}$,
\item $\mathtt{KernelEmbedding}$,
\item $\mathtt{KernelLift}$,
\item $\mathtt{CokernelObject}$,
\item $\mathtt{CokernelProjection}$,
\item $\mathtt{CokernelColift}$
\end{itemize}
\end{doctrine}

\begin{remark}[Kernel lift of cokernel colift $=$ cokernel colift of kernel lift]\label{rmk:kerl_col_col_kerl}\phantom{}\\
In categories with kernels and cokernels we can compute images and coimages together with an induced morphism
\begin{align}
\overline{\varphi} : \mathrm{Coim}(\varphi) \rightarrow \mathrm{Im}(\varphi).
\end{align}
This morphism being an isomorphism is one of the defining axioms of Abelian categories.
\end{remark}
\begin{proof}
Given a morphism $\varphi : M \rightarrow N$ in a pre-Abelian category $\mathcal{C}$, we can compute its kernel embedding
$\kappa : \mathrm{Ker}(\varphi) \hookrightarrow M$ and its cokernel projection
$\varepsilon : N \twoheadrightarrow \mathrm{Coker}(\varphi)$. They are the equalizer and coequalizer of $\varphi$ and $0_{M,N}$,
in particular we have $\kappa \, \varphi = 0_{\mathrm{Ker}(\varphi),N}$ and
$\varphi \, \varepsilon = 0_{M,\mathrm{Coker}(\varphi)}$ (the two $0$-arrows on the top).

For the kernel embedding $\kappa$ we can compute its cokernel projection
$\varepsilon_{\kappa} : M \twoheadrightarrow \mathrm{Coim}(\varphi)$ which we called the coimage projection of $\varphi$.
Keeping in mind, that it is the cokernel of $\kappa$, i.e. the coequalizer of $\kappa$ and $0_{\mathrm{Ker}(\varphi),M}$, we get
the zero morphism $0_{\mathrm{Ker}(\varphi),\mathrm{Coim}(\varphi)}$ (the bottom left $0$-arrow).

Dually for the cokernel projection $\varepsilon$ we can compute its kernel embedding
$\kappa_{\varepsilon} : \mathrm{Im} \hookrightarrow N$ which we called the image embedding of $\varphi$. Again we see
that we have the zero morphism $0_{\mathrm{Im}(\varphi),\mathrm{Coker}(\varphi)}$ (the bottom right $0$-arrow).

The two morphisms $\varphi$ and its coimage projection $\varepsilon_{\kappa}$ have the same source $M$ and
composed with $\kappa$, both yield zero morphisms. Since $\varepsilon_{\kappa}$ codominates $\varphi$, we have
a unique cokernel colift\\
$(\varepsilon_{\kappa}\backslash \varphi) : \mathrm{Coim}(\varphi) \rightarrow N$, which is one of the two diagonals
in the square diagram, and for which\\
$\varepsilon_{\kappa} \cdot (\varepsilon_{\kappa}\backslash \varphi) = \varphi$.

Now $\kappa_{\varepsilon}$ and the above colift $(\varepsilon_{\kappa}\backslash \varphi)$ both have the same target $N$.
For $\kappa_{\varepsilon}$ we already know that composed with $\varepsilon$ we get the zero morphism.
If we also proved that
$(\varepsilon_{\kappa}\backslash \varphi) \cdot \varepsilon = 0_{\mathrm{Coim}(\varphi),\mathrm{Coker}(\varphi)}$, i.e.
the red $0$-arrow in the picture, then since $\kappa_{\varepsilon}$ dominates $(\varepsilon_{\kappa}\backslash \varphi)$
we get the existence of the kernel lift of the cokernel colift
\[
(\varepsilon_{\kappa}\backslash \varphi) / \kappa_{\varepsilon} : \mathrm{Coim}(\varphi) \rightarrow \mathrm{Im}(\varphi)
\]

\[
\begin{tikzcd}
\mathrm{Ker}(\varphi) \arrow[rd, "\kappa", hook] \arrow[rddd, "0"', bend right] \arrow[rrrd, "0", pos=0.3, bend left] &                                                                                                                                                                                             &  &                                                                                               & \mathrm{Coker}(\varphi) \\
                                                                                                             & M \arrow[rrru, "0", pos=0.7, bend left] \arrow[rr, "\varphi"] \arrow[dd, "\varepsilon_{\kappa}"', two heads]                                                                                         &  & N \arrow[ru, "\varepsilon", two heads]                                                        &                         \\
                                                                                                             &                                                                                                                                                                                             &  &                                                                                               &                         \\
                                                                                                             & \mathrm{Coim}(\varphi) \arrow[rruu, "(\varepsilon_{\kappa}\backslash \varphi)"'] \arrow[rr, "(\varepsilon_{\kappa}\backslash\varphi)/\kappa_{\varepsilon}"'] \arrow[rrruuu, "0", pos=0.7, color=red, bend left] &  & \mathrm{Im}(\varphi) \arrow[uu, "\kappa_{\varepsilon}"', hook] \arrow[ruuu, "0"', bend right] &                        
\end{tikzcd}
\]

\begin{subproof}[Proof that $(\varepsilon_{\kappa}\backslash \varphi) \cdot \varepsilon = 0_{\mathrm{Coim}(\varphi),\mathrm{Coker}(\varphi)}$]\phantom{}\\
We have $\varepsilon_{\kappa} \cdot (\varepsilon_{\kappa}\backslash \varphi) = \varphi$, i.e. the top left triangle
of the square commutes. But then since composition with a zero morphism gives a zero morphism we have
\begin{align*}
&\varepsilon_{\kappa} \cdot 0_{\mathrm{Coim}(\varphi),\mathrm{Coker}(\varphi)} : M \rightarrow \mathrm{Coker}(\varphi) \\
=\, &0_{M,\mathrm{Coker}(\varphi)} \\
=\, &\varphi \cdot \varepsilon \\
=\, &\varepsilon_{\kappa} \cdot (\varepsilon_{\kappa}\backslash \varphi) \cdot \varepsilon
\end{align*}
And pre-cancellation of the epimorphism $\varepsilon_{\kappa}$ gives
\begin{align*}
0_{\mathrm{Coim}(\varphi),\mathrm{Coker}(\varphi)} = (\varepsilon_{\kappa}\backslash \varphi) \cdot \varepsilon.
\end{align*}
\end{subproof}

Dual to the above picture, the two morphisms $\varphi$ and its image embedding $\kappa_{\varepsilon}$ both have the same
target $N$ and composed with $\varepsilon$, both yield zero morphisms. Since $\kappa_{\varepsilon}$ dominates $\varphi$,
we have a unique kernel lift
$(\varphi / \kappa_{\varepsilon}) : M \rightarrow \mathrm{Im}(\varphi)$, which is the other diagonal in the square diagram,
and for which $(\varphi / \kappa_{\varepsilon})\cdot \kappa_{\varepsilon} = \varphi$.

Now $\varepsilon_{\kappa}$ and $(\varphi / \kappa_{\varepsilon})$ both have the same source $M$. For $\varepsilon_{\kappa}$
we already know that composed with $\kappa$ we get the zero morphism. If we also proved that
$\kappa \cdot (\varphi / \kappa_{\varepsilon}) = 0_{\mathrm{Ker}(\varphi),\mathrm{Im}(\varphi)}$, i.e. the red $0$-arrow in the
picture, then, since $\varepsilon_{\kappa}$ codominates $(\varphi / \kappa_{\varepsilon})$ we get the existence of the
cokernel colift of the kernel lift
\[
\varepsilon_{\kappa}\backslash(\varphi/\kappa_{\varepsilon}) : \mathrm{Coim}(\varphi) \rightarrow \mathrm{Im}(\varphi)
\]

\[
\begin{tikzcd}
\mathrm{Ker}(\varphi) \arrow[rd, "\kappa", hook] \arrow[rddd, "0"', bend right] \arrow[rrrd, "0", pos=0.3, bend left] \arrow[rrrddd, "0"', pos=0.25, color=red, shift left=2, bend right] &                                                                                                                                                    &  &                                                                                               & \mathrm{Coker}(\varphi) \\
                                                                                                                                             & M \arrow[rrru, "0", pos=0.7, bend left] \arrow[rr, "\varphi"] \arrow[dd, "\varepsilon_{\kappa}"', two heads] \arrow[rrdd, "(\varphi/\kappa_{\varepsilon})"] &  & N \arrow[ru, "\varepsilon", two heads]                                                        &                         \\
                                                                                                                                             &                                                                                                                                                    &  &                                                                                               &                         \\
                                                                                                                                             & \mathrm{Coim}(\varphi) \arrow[rr, "\varepsilon_{\kappa}\backslash(\varphi/\kappa_{\varepsilon})"']                                                 &  & \mathrm{Im}(\varphi) \arrow[uu, "\kappa_{\varepsilon}"', hook] \arrow[ruuu, "0"', bend right] &                        
\end{tikzcd}
\]

\begin{subproof}[Proof that $\kappa \cdot (\varphi / \kappa_{\varepsilon}) = 0_{\mathrm{Ker}(\varphi),\mathrm{Im}(\varphi)}$]\phantom{}\\
We have $(\varphi / \kappa_{\varepsilon}) \cdot \kappa_{\varepsilon} = \varphi$, i.e. the top right triangle
of the square commutes. But then since composition with a zero morphism gives a zero morphism we have
\begin{align*}
&0_{\mathrm{Ker}(\varphi),\mathrm{Im}(\varphi)} \cdot \kappa_{\varepsilon} : \mathrm{Ker}(\varphi) \rightarrow N \\
=\, &0_{\mathrm{Ker}(\varphi),N} \\
=\, &\kappa \cdot \varphi \\
=\, &\kappa \cdot (\varphi / \kappa_{\varepsilon}) \cdot \kappa_{\varepsilon}
\end{align*}
And post-cancellation of the monomorphism $\kappa_{\varepsilon}$ gives
\begin{align*}
0_{\mathrm{Ker}(\varphi),\mathrm{Im}(\varphi)} = \kappa \cdot (\varphi / \kappa_{\varepsilon}).
\end{align*}
\end{subproof}

So we have two morphisms from $\mathrm{Coim}(\varphi)$ to $\mathrm{Im}(\varphi)$:
\begin{alignat*}{3}
\underbrace{(\varepsilon_{\kappa}\backslash \varphi) / \kappa_{\varepsilon}}_{\alpha} & \quad\text{and}\quad
&& \underbrace{\varepsilon_{\kappa}\backslash(\varphi/\kappa_{\varepsilon})}_{\beta}.
\end{alignat*}
In fact they are equal since the two commutative traingles give commutative squares.
\begin{align*}
\varepsilon_{\kappa}\, \alpha \, \kappa_{\varepsilon} = \varphi = \varepsilon_{\kappa}\, \beta \, \kappa_{\varepsilon} 
\end{align*}
Since the epi $\varepsilon_{\kappa}$ is pre-cancelable and the mono $\kappa_{\varepsilon}$ is post-cancelable, we get $\alpha = \beta$, i.e.
\begin{align*}
(\varepsilon_{\kappa}\backslash \varphi) / \kappa_{\varepsilon} = \varepsilon_{\kappa}\backslash(\varphi/\kappa_{\varepsilon})
\end{align*}
This justifies the notation
\begin{align}\label{eq:natural_morphism}
\overline{\varphi} := \varepsilon_{\kappa}\backslash \varphi / \kappa_{\varepsilon} : \mathrm{Coim}(\varphi) \rightarrow \mathrm{Im}(\varphi).
\end{align}
In this context associativity holds, i.e. we can leave out the parentheses of the lift and colift.
\end{proof}






\newpage
\subsubsection{Abelian categories}

In this section we give three equivalent definitions for Abelian categories, starting abstractly from the above diagrams
and ending with the existence of two algorithms needed for the doctrine $\mathtt{IsAbelianCategory}$.

\begin{definition}[Abelian category]
A pre-Abelian category $\mathcal{C}$ is Abelian if for every morphism $\varphi \in \mathcal{C}_{1}$, the natural morphism
in \eqref{eq:natural_morphism}
\[
\overline{\varphi} := \varepsilon_{\kappa}\backslash \varphi/\kappa_{\varepsilon} : \mathrm{Coim}(\varphi)
\xrightarrow{\sim} \mathrm{Im}(\varphi)
\]
is an isomorphism.
\end{definition}
Therefore in the doctrine of Abelian categories we will not need to distinguish between coimages and images\endnote{Especially
in a skeletal category such as $\kmat$.}, and
the diagrams from \ref{rmk:kerl_col_col_kerl} simplify into an epi-mono factorization diagram.

\begin{corollary}[Epi-mono-factorization]\label{cor:epi_mono_factorization}
Every morphism $\varphi : M \rightarrow N$ in an abelian category $\mathcal{C}$ can be factored as the
composition of an epimorphism $\pi : M \twoheadrightarrow I$ and a monomorphism $\iota : I \hookrightarrow N$\\
where $I \cong \mathrm{Im}(\varphi) \cong \mathrm{Coim}(\varphi)$.
\[
\begin{tikzcd}
K \arrow[rd, "\kappa", hook] &                                                       &                             &                                        & C \\
                             & M \arrow[rr, "\varphi"] \arrow[rd, "\pi"', two heads] &                             & N \arrow[ru, "\varepsilon", two heads] &   \\
                             &                                                       & I \arrow[ru, "\iota", hook] &                                        &  
\end{tikzcd}
\]
The factorization is unique up to unique isomorphism.
\end{corollary}

\begin{definition}[Abelian category]
An \ul{abelian category} is
\begin{itemize}
\item a pre-abelian category $\mathcal{C}$ where
\item every monomorphism $\kappa$ is a kernel-embedding of its cokernel-projection and
\item every epimorphism $\varepsilon$ is a cokernel-projection of  its kernel-embedding.
\end{itemize}
\end{definition}

\begin{remark}
The above definition, that we can regard
\begin{itemize}
\item every monomorphism $\kappa : K \hookrightarrow A$ as a kernel-embedding
of its cokernel-projection $\varepsilon_{\kappa} : A \twoheadrightarrow C$, with $K$ being the kernel object
$\mathrm{Ker}(\varepsilon_{\kappa}) = s(\kappa)$,
\item and every epimorphism
$\varepsilon : B \twoheadrightarrow C$ as a cokernel-projection of its kernel-embedding $\kappa_{\varepsilon} : K \hookrightarrow B$,
with $C$ being the cokernel object $\mathrm{Coker}(\kappa_{\varepsilon}) = t(\varepsilon)$
\end{itemize}
implies that we also have the third ingredient
for kernels and cokernels, namely the dependent functions
\begin{itemize}
\item kernel lift $(-/\kappa) : \underline{\phantom{T}} \rightarrow K$ mapping a morphism $\tau : T \rightarrow A$ with same
target $t(\tau) = t(\kappa) = A$ and same cokernel $\varepsilon_{\kappa} = \varepsilon_{\tau}$ (i.e. $\tau\varepsilon_{\kappa} = 0_{T,C}$)
as $\kappa$ to a unique kernel lift $(\tau / \kappa) : T \rightarrow K$
\[
\begin{tikzcd}
K \arrow[r, "\kappa", hook] \arrow[r, "{0_{K,A}}", bend left, shift left=2]                                & A \arrow[r, "\varepsilon_{\kappa}", two heads] \arrow[r, "\varepsilon_{\tau}"', two heads] & C \\
T \arrow[ru, "{0_{T,A}}"', bend right, shift right=2] \arrow[ru, "\tau"] \arrow[u, "(\tau/\kappa)", dashed] &                                                                                            &  
\end{tikzcd}
\]
\item cokernel colift $(\varepsilon \backslash - ) : C \rightarrow \underline{\phantom{T}}$ mapping a morphism $\tau : B \rightarrow T$ with
same source $s(\tau) = s(\varepsilon) = B$ and same kernel $\kappa_{\varepsilon} = \kappa_{\tau}$
(i.e. $\kappa_{\varepsilon}\tau = 0_{K,T}$)  as $\varepsilon$ to a unique cokernel
colift $(\varepsilon \backslash \tau) : C \rightarrow T$.
\[
\begin{tikzcd}
K \arrow[r, "\kappa_{\varepsilon}", hook] \arrow[r, "\kappa_{\tau}"', hook] & B \arrow[r, "\varepsilon", two heads] \arrow[r, "{0_{B,C}}", bend left, shift left=2] \arrow[rd, "\tau"] \arrow[rd, "{0_{B,T}}"', bend right, shift right=2] & C \arrow[d, "(\varepsilon\backslash\tau)", dashed] \\
                                                                            &                                                                                                                                                   & T                                                 
\end{tikzcd}
\]
\end{itemize}
\end{remark}

The existence of \ul{lifts along monos} and \ul{colifts along epis} (in the sense of the above remark)
in a pre-abelian category is therefore an equivalent definition of an abelian category.

\begin{definition}[Abelian category]\label{def:abelian_category}
An \ul{Abelian category} consists of the following data:
\begin{enumerate}
\renewcommand{\labelenumi}{(\theenumi)}
\item A pre-abelian category $\mathcal{C}$.
\item A dependent function $( - / - )$ mapping a monomorphism $\kappa : K \hookrightarrow A$ and a morphism $\tau : T \rightarrow A$ with
the same target $t(\tau) = t(\kappa)$ and the same cokernel projection $\varepsilon_{\tau} = \varepsilon_{\kappa}$ to a lift $(\tau / \kappa)$ of
$\tau$ along $\kappa$.
\item A dependent function $( - \backslash - )$ mapping an epimorphism $\varepsilon : B \twoheadrightarrow C$ and a morphism
$\tau : B \rightarrow T$ with same source $s(\tau) = s(\varepsilon)$ and the same kernel embedding $\kappa_{\tau} = \kappa_{\varepsilon}$
to a colift $(\varepsilon \backslash \tau)$ of $\tau$ along $\varepsilon$.
\end{enumerate}
\end{definition}

\begin{doctrine}[Abelian category]
The doctrine $\mathtt{IsAbelianCategory}$ therefore involves algorithms of $\mathtt{IsPreAbelianCategory}$ together
with two additional algorithms
\begin{itemize}
\item $\mathtt{LiftAlongMonomorphism}$,
\item $\mathtt{ColiftAlongEpimorphism}$,
\end{itemize}
fulfilling the specification of definition \ref{def:abelian_category}
\end{doctrine}

\newpage
\subsection{The matrix category $\kmat$ is an abelian category}

The following is an example of a category which has all the limits and colimits mentioned so far, and has them implemented constructively.
We will check the four doctrines $\mathtt{IsAbCategory}$, $\mathtt{IsAdditiveCategory}$, $\mathtt{IsPreAbelianCategory}$ and
$\mathtt{IsAbelianCategory}$ by providing the needed algorithms.

\begin{example}{(The matrix category $\kmat$ over a commutative ring $\Bbbk$)}\label{ex:kmat_skeletal}
\begin{itemize}
\item Objects are natural numbers $\kmat_{0} = \mathbb{N} = \mathbb{N}_{0} = \{0,1,2,\dots\}$ for wich we use small latin letters
($m, n, k,\dots$).
\item Morphisms $(m \rightarrow n) \in \kmat_{1}$ are $m \times n$ matrices over $\Bbbk$.
We write the set of morphisms between $m$ and $n$, as $\Bbbk^{m\times n} := \textup{Hom}_{\kmat}(m,n)$. 
For variables that are Matrices we use small greek letters ($\varphi, \psi,\dots$) or capital latin letters ($A, B, C,\dots$). When only source and target are relevant,
we write $(m \times n)$.
\item $s(\varphi) = \mathtt{Source}(\varphi) := \mathtt{NrRows}(\varphi)$
\item $t(\varphi) = \mathtt{Range}(\varphi) := \mathtt{NrColumns}(\varphi)$
\item Identity morphisms are the identity matrices.
\[
1_{m} = \mathtt{IdentityMorphism}(m) := \mathtt{IdentityMat}(m,\Bbbk).
\]
\item Composition is matrix multiplication which is associative.
\[
\varphi\psi = \mathtt{PreCompose}(\varphi,\psi) := \mathtt{MatMul}(\varphi,\psi).
\]
\item It is a skeletal category, i.e. $m \cong n \Rightarrow m = n$. Only quadratic matrices ($m = n$) can be
isomorphisms.
\end{itemize}
\end{example}

\begin{example}[$\kmat$ is an Ab-category]\label{ex:kmat_pre-additive}
In $\kmat$, the number $0$ is the zero object. $\mathtt{ZeroObject := 0}$\\
A zero matrix (zero morphism) is a matrix factoring through the zero object $0$.\\
\begin{minipage}{.2\textwidth}\phantom{ }\end{minipage}
\begin{minipage}{.25\textwidth}
$\Bbbk^{m\times n} \ni A = 0_{m,n}$
\end{minipage}
\begin{minipage}{.08\textwidth}
$\Longleftrightarrow$
\end{minipage}
\begin{minipage}{.32\textwidth}
\begin{tikzcd}
m \arrow[rr, "A"] \arrow[rd, "(m \times 0)"'] &                               & n \\
                                              & 0 \arrow[ru, "(0 \times n)"'] &  
\end{tikzcd}\\
$\Rightarrow A = (m \times 0) \cdot (0 \times n)$.
\end{minipage}
\begin{minipage}{.15\textwidth}\phantom{ }\end{minipage}\\

\noindent The matrices $(m \times 0)$ and $(0 \times n)$ have zero columns or zero rows respectively, but it is
important to note that for each $m \in \kmat_{0}$ there is exactly one such matrix $(m \times 0)$ and $(0 \times m)$
(that's what initial and terminal object means), and for different $m$, these morphisms are different.\endnote{It is
challenging to decide between different types of zero matrices with zero rows or zero columns, if they
are represented by lists of lists with their entries. How many rows does the $(3 \times 0)$ matrix have, if you represent it
by the empty list \texttt{[ ]}? That's why the implenentation in \homalgProject uses a special function
\texttt{HomalgZeroMatrix}, and why the morphisms in \CAP are always implemented with their source and target defined.}

\begin{itemize}
\item For the matrix $(m \times 0)$ we have\\
$\mathtt{UniversalMorphismIntoZeroObject (m) := ZeroMorphism(m, 0) := ZeroMatrix( m, 0 )}$,
\item For the matrix $(0 \times n)$ we have\\
$\mathtt{UniversalMorphismFromZeroObject (n) := ZeroMorphism(0, n) := ZeroMatrix( 0, n )}$,
\item For zero matrices $(m \times n)$ we write $\mathtt{ZeroMorphism(m, n) := ZeroMatrix( m, n )}$.
\end{itemize}

For two natural numbers $m,n \in {\kmat}_{0} = \mathbb{N} = \mathbb{N}_{0}$, the set of morphisms with source $m$ and target $n$ is
$\Bbbk^{m\times n}$, the set of $m \times n$-matrices. This is an abelian group:
\begin{itemize}
\item The neutral element of the addition is the $m \times n$ zero matrix $0_{m,n}$.
\item Addition of matrices $\mathtt{AdditionForMorphisms( phi, psi ) := phi + psi}$ is associative and commutative.
\item For every matrix $A \in \Bbbk^{m\times n}$ there is a negative matrix $-A \in \Bbbk^{m \times n}$ such that $A + (-A) = 0_{m,n}$.
\end{itemize}
Composition of matrices is defined as matrix multiplication, which is bilinear, i.e. satisfies the distributive laws \eqref{eq:dist1} and
\eqref{eq:dist2}.\\
It is an easy exercise to provide the algorithms for an Ab-category mentioned in doctrine \ref{doc:ab-category}.
\end{example}

\begin{example}[$\kmat$ is an additive category]\label{ex:kmat_additive}
Let for $I = \{1,\dots,N\},$ the set $\{n_{1},\dots,n_{N}\}$ be a family of objects in $\kmat_{0}$. Their direct sum is
\begin{itemize}
\item the object $n := \bigoplus_{i=1}^{N} n_{i} := \sum_{i=1}^{N} n_{i} = \mathtt{Sum}$
\item For $i \in I$ we have as identity morphism $1_{n_{i}}$ of the object $n_{i}$ the $n_{i} \times n_{i}$ identity matrix.
Define
\[
n_{<i} := \sum_{j=1}^{i-1} n_{j}\quad \text{ and }\quad n_{>i} := \sum_{j=i+1}^{N} n_{j}.
\]
Then we have
\item The projection $\pi_{i} : n \rightarrow n_{i}$ is an $n \times n_{i}$ matrix that is a stacked matrix of the $n_{j}\times n_{i}$
zero matrices not including $0_{n_{i},n_{i}}$ and the identity matrix $1_{n_{i}}$.
\begin{align}
\pi_{i} = \label{eq:projection_direct_sum_matrix}
\begin{pmatrix}
0_{n_{1},\,n_{i}} \\
0_{n_{2},\,n_{i}} \\
\vdots \\
0_{n_{i-1},\,n_{i}} \\
1_{n_{i}} \\
0_{n_{i+1},\,n_{i}} \\
\vdots \\
0_{n_{N},\,n_{i}}
\end{pmatrix}
=
\begin{pmatrix}
0_{n_{<i},\,n_{i}} \\
1_{n_{i}} \\
0_{n_{>i},\,n_{i}}
\end{pmatrix}
\end{align}
\item The coprojection $\iota_{i} : n_{i} \rightarrow n$ is the transposed matrix $\iota_{i} = \pi_{i}^{T}$, i.e. an $n_{i} \times n$ matrix of
the $n_{i} \times n_{j}$ zero matrices not including $0_{n_{i},n_{i}}$ and the identity matrix $1_{n_{i}}$ lined up next to each other.
\begin{align}
\iota_{i} = \label{eq:coprojection_direct_sum_matrix}
\begingroup
\setlength\arraycolsep{2pt}
\begin{pmatrix}
0_{n_{i},n_{1}} & 0_{n_{i},n_{2}} & \dots & 0_{n_{i},n_{i-1}} & 1_{n_{i}} & 0_{n_{i},n_{i+1}} & \dots & 0_{n_{i},n_{N}}
\end{pmatrix}
= \begin{pmatrix}
0_{n_{i},\,n_{<i}} & 1_{n_{i}} & 0_{n_{i},\,n_{>i}}
\end{pmatrix} \endgroup
\end{align}

\item For a family $\tau = (\tau_{i} : t \rightarrow n_{i})_{i\in I}$ we have the morphism $u_{\text{in}}(\tau)$ which is a $t \times n$ block matrix of
the $\tau_{i}$:
\begin{align}
u_{\text{in}}(\tau) = \label{eq:u_in_direct_sum_matrix}
\begin{pmatrix}
\tau_{1} & \cdots & \tau_{N}
\end{pmatrix}
\end{align}
with $u_{\text{in}}(\tau) \pi_{i} = \tau_{i}$.
\item For a family $\tau = (\tau_{i} : n_{i} \rightarrow t)_{i \in I}$ we have the morphism $u_{\text{out}}(\tau)$ which is an $n \times t$ block matrix
of the $\tau_{i}$:
\begin{align}
u_{\text{out}}(\tau) = \label{eq:u_out_direct_sum_matrix}
\begin{pmatrix}
\tau_{1} \\
\vdots \\
\tau_{N}
\end{pmatrix}
\end{align}
with $\iota_{i} u_{\text{out}}(\tau) = \tau_{i}$.
\end{itemize}
One can easily verify the conditions for $\pi$, $\iota$, $u_{\text{in}}$ and $u_{\text{out}}$ in Definitions \ref{def:biproduct} and \ref{def:direct_sum}.
\end{example}

\begin{computation} 
If we assume algorithms for adding natural numbers
\begin{alignat*}{3}
&\mathtt{Sum}( [ m, n &&] ) = m + n, \\
&\mathtt{Sum}( [  &&] ) = 0,
\end{alignat*}
for stacking two matrices with the same number of columns (same target) on top of each other
\begin{align*}
\varphi &: k \rightarrow n \\
\psi &: m \rightarrow n\\
\mathtt{StackMatrix}( \varphi, \psi ) =
\begin{pmatrix}
\varphi \\
\psi
\end{pmatrix} &: k + m \rightarrow n,
\end{align*}
and for aligning two matrices with the same number of rows (same source) next to each other
\begin{align*}
\varphi &: m \rightarrow k \\
\psi &: m \rightarrow n\\
\mathtt{AugmentMatrix}( \varphi, \psi ) =
\begin{pmatrix}
\varphi & \psi
\end{pmatrix} &: m \rightarrow k + n.
\end{align*}
Together with the algorithms $\mathtt{IdentityMorphism( n )}$ and $\mathtt{ZeroMorphism( m, n )}$ from $\mathtt{IsAbCategory}$
we can then calculate all the algorithms in the doctrine $\mathtt{IsAdditiveCategory}$:

The direct sum of the objects $\mathtt{D := [ V1, V2, \dots, VN ]}$ in $\kmat$ is defined as
\begin{itemize}
\item \texttt{DirectSum ( D ) := VectorSpaceObject( Sum( List( D, V $\mapsto$ Dimension( V ) ) ), k )}
\item \texttt{ProjectionInFactorOfDirectSum( D, i ) := StackMatrix( [ \\
\phantom{x}\hspace{3ex}List( D[1, \dots, (i-1)], V $\mapsto$ ZeroMorphism( V, D[i] ) ), \\
\phantom{x}\hspace{3ex}IdentityMorphism( D[i] ), \\
\phantom{x}\hspace{3ex}List( D[(i+1), \dots, N], V $\mapsto$ ZeroMorphism( V, D[i] ) ) ] )}
\item \texttt{InjectionOfCofactorOfDirectSum( D, i ) := AugmentMatrix( [ \\
\phantom{x}\hspace{3ex}List( D[1, \dots, (i-1)], V $\mapsto$ ZeroMorphism( D[i], V ) ), \\
\phantom{x}\hspace{3ex}IdentityMorphism( D[i] ), \\
\phantom{x}\hspace{3ex}List( D[(i+1), \dots, N], V $\mapsto$ ZeroMorphism( D[i], V ) ) ] )}
\item \texttt{UniversalMorphismIntoDirectSum( [ phi, psi ] ) := StackMatrix( [ phi, psi ] )}
\item \texttt{UniversalMorphismFromDirectSum( [ phi, psi ] ) := AugmentMatrix( [ phi, psi ] )}
\end{itemize}

In the following \Gap{} session, we demonstrate the algorithms of $\mathtt{IsAdditiveCategory}$ and verify in
an example the two axioms of the direct sum in \ref{def:direct_sum}. To verify two these properties for a general case
of a direct sum of objects in the matrix category is left as an exercise.

The implementation of objects in the matrix category $\kmat$ in \textsc{Cap} is slightly different than simply natural numbers, since
we always have to mention the commutative ring $\Bbbk$ for the objects. We are using the finite field
$\Bbbk := \mathbb{F}_{3}$ for our calculations below.
\begin{Verbatim}[commandchars=!@|,fontsize=\small,frame=single,label=Example]
  !gapprompt@gap>| !gapinput@LoadPackage("LinearAlgebraForCAP");|
  true
  !gapprompt@gap>| !gapinput@GF3 := HomalgRingOfIntegers( 3 );|
  GF(3)
  !gapprompt@gap>| !gapinput@V3 := VectorSpaceObject( 3, GF3 );|
  <A vector space object over GF(3) of dimension 3>
  !gapprompt@gap>| !gapinput@V5 := VectorSpaceObject( 5, GF3 );|
  <A vector space object over GF(3) of dimension 5>
  !gapprompt@gap>| !gapinput@V2 := VectorSpaceObject( 2, GF3 );|
  <A vector space object over GF(3) of dimension 2>
  !gapprompt@gap>| !gapinput@D := [ V3, V5, V2 ];|
  [ <A vector space object over GF(3) of dimension 3>,
    <A vector space object over GF(3) of dimension 5>,
    <A vector space object over GF(3) of dimension 2> ]
  !gapprompt@gap>| !gapinput@S := DirectSum( D );|
  <A vector space object over GF(3) of dimension 10>
  !gapprompt@gap>| !gapinput@zero35 := ZeroMorphism( V3, V5 );|
  <A zero morphism in Category of matrices over GF(3)>
  !gapprompt@gap>| !gapinput@Display( zero35 );|
   . . . . .
   . . . . .
   . . . . .
   
   A zero morphism in Category of matrices over GF(3)
  !gapprompt@gap>| !gapinput@one5 := IdentityMorphism( V5 );|
  <An identity morphism in Category of matrices over GF(3)>
  !gapprompt@gap>| !gapinput@Display( one5 );|
   1 . . . .
   . 1 . . .
   . . 1 . .
   . . . 1 .
   . . . . 1
   
   An identity morphism in Category of matrices over GF(3)
  !gapprompt@gap>| !gapinput@zero25 := ZeroMorphism( V2, V5 );|
  <A zero morphism in Category of matrices over GF(3)>
  !gapprompt@gap>| !gapinput@Display( zero25 );|
   . . . . .
   . . . . .
   
   A zero morphism in Category of matrices over GF(3)
  !gapprompt@gap>| !gapinput@pi2 := ProjectionInFactorOfDirectSum( D, 2 );|
  <A morphism in Category of matrices over GF(3)>
  !gapprompt@gap>| !gapinput@Display( pi2 );|
   . . . . .
   . . . . .
   . . . . .
   1 . . . .
   . 1 . . .
   . . 1 . .
   . . . 1 .
   . . . . 1
   . . . . .
   . . . . .
  
  A morphism in Category of matrices over GF(3)
  !gapprompt@gap>| !gapinput@iota1 := InjectionOfCofactorOfDirectSum( D, 1 );|
  <A morphism in Category of matrices over GF(3)>
  !gapprompt@gap>| !gapinput@Display( iota1 );|
   1 . . . . . . . . .
   . 1 . . . . . . . .
   . . 1 . . . . . . .
   
   A morphism in Category of matrices over GF(3)
  !gapprompt@gap>| !gapinput@Display( PreCompose( iota1, pi2 ) );|
   . . . . .
   . . . . .
   . . . . .
   
   A morphism in Category of matrices over GF(3)
  !gapprompt@gap>| !gapinput@IsEqualForMorphisms( PreCompose( iota1, pi2 ), ZeroMorphism( V3, V5 ) );|
  true
  !gapprompt@gap>| !gapinput@iota2 := InjectionOfCofactorOfDirectSum( D, 2 );;|
  !gapprompt@gap>| !gapinput@Display( iota2 );|
   . . . 1 . . . . . .
   . . . . 1 . . . . .
   . . . . . 1 . . . .
   . . . . . . 1 . . .
   . . . . . . . 1 . .
   
   A morphism in Category of matrices over GF(3)
  !gapprompt@gap>| !gapinput@Display( PreCompose( iota2, pi2 ) );|
   1 . . . .
   . 1 . . .
   . . 1 . .
   . . . 1 .
   . . . . 1
   
   A morphism in Category of matrices over GF(3)
  !gapprompt@gap>| !gapinput@IsEqualForMorphisms( PreCompose( iota2, pi2 ), IdentityMorphism( V5 ) );|
  true
  !gapprompt@gap>| !gapinput@Display( PreCompose( pi2, iota2 ) );|
   . . . . . . . . . .
   . . . . . . . . . .
   . . . . . . . . . .
   . . . 1 . . . . . .
   . . . . 1 . . . . .
   . . . . . 1 . . . .
   . . . . . . 1 . . .
   . . . . . . . 1 . .
   . . . . . . . . . .
   . . . . . . . . . .
   
   A morphism in Category of matrices over GF(3)
  !gapprompt@gap>| !gapinput@iota3 := InjectionOfCofactorOfDirectSum( D, 3 );;|
  !gapprompt@gap>| !gapinput@pi1 := ProjectionInFactorOfDirectSum( D, 1 );;|
  !gapprompt@gap>| !gapinput@pi3 := ProjectionInFactorOfDirectSum( D, 3 );;|
  !gapprompt@gap>| !gapinput@Display( PreCompose( pi1, iota1 ) + PreCompose( pi2, iota2 )|
  !gapprompt@>| !gapinput@   + PreCompose( pi3, iota3 ) );|
   1 . . . . . . . . .
   . 1 . . . . . . . .
   . . 1 . . . . . . .
   . . . 1 . . . . . .
   . . . . 1 . . . . .
   . . . . . 1 . . . .
   . . . . . . 1 . . .
   . . . . . . . 1 . .
   . . . . . . . . 1 .
   . . . . . . . . . 1
   
   A morphism in Category of matrices over GF(3)
  !gapprompt@gap>| !gapinput@IsEqualForMorphisms(|
  !gapprompt@>| !gapinput@     PreCompose( pi1, iota1 )|
  !gapprompt@>| !gapinput@   + PreCompose( pi2, iota2 )|
  !gapprompt@>| !gapinput@   + PreCompose( pi3, iota3 ),|
  !gapprompt@>| !gapinput@   IdentityMorphism( S ) );|
  true
\end{Verbatim}

We can also verify the result in \ref{rmk:addition_derived_from_direct_sum} that the abelian group operation
$\mathtt{AdditionForMorphisms}$ can be derived from $\mathtt{UniversalMorphismIntoDirectSum}$,
$\mathtt{UniversalMorphismFromDirectSum}$,\\
$\mathtt{IdentityMorphism}$ and
$\mathtt{PreCompose}$. As an example we add two $\mathrm{GF}_{3}$-matrices from $\mathtt{V2}$ to $\mathtt{V3}$ in three
different ways.

\begin{Verbatim}[commandchars=!@|,fontsize=\small,frame=single,label=Example]
  !gapprompt@gap>| !gapinput@mat1 := HomalgMatrix( [ 0, 1, 2, 1, 1, 2 ], 2, 3, GF3 );|
  <A 2 x 3 matrix over an internal ring>
  !gapprompt@gap>| !gapinput@mat2 := HomalgMatrix( [ 1, 1, 1, 1, 1, 1 ], 2, 3, GF3 );|
  <A 2 x 3 matrix over an internal ring>
  !gapprompt@>| !gapinput@mor1 := VectorSpaceMorphism( V2, mat1, V3 );|
  <A morphism in Category of matrices over GF(3)>
  !gapprompt@gap>| !gapinput@Display( mor1 );|
   . 1 2
   1 1 2
   
   A morphism in Category of matrices over GF(3)
  !gapprompt@>| !gapinput@mor2 := VectorSpaceMorphism( V2, mat2, V3 );|
  <A morphism in Category of matrices over GF(3)>
  !gapprompt@gap>| !gapinput@Display( mor2 );|
   1 1 1
   1 1 1
   
   A morphism in Category of matrices over GF(3)
  !gapprompt@>| !gapinput@result1 := mor1 + mor2;|
  <A morphism in Category of matrices over GF(3)>
  !gapprompt@gap>| !gapinput@Display( result1 );|
   1 2 .
   2 2 .
   
   A morphism in Category of matrices over GF(3)
  !gapprompt@gap>| !gapinput@one2 := IdentityMorphism( V2 );|
  <An identity morphism in Category of matrices over GF(3)>
  !gapprompt@gap>| !gapinput@result2 := PreCompose( UniversalMorphismIntoDirectSum( [ one2, one2 ] ),|
  !gapprompt@>| !gapinput@   UniversalMorphismFromDirectSum( [ mor1, mor2 ] ) );|
  <A morphism in Category of matrices over GF(3)>
  !gapprompt@gap>| !gapinput@Display( result2 );|
   1 2 .
   2 2 .
   
   A morphism in Category of matrices over GF(3)
  !gapprompt@gap>| !gapinput@result1 = result2;|
  true
  !gapprompt@gap>| !gapinput@one3 := IdentityMorphism( V3 );|
  <An identity morphism in Category of matrices over GF(3)>
  !gapprompt@gap>| !gapinput@result3 := PreCompose( UniversalMorphismIntoDirectSum( [ mor1, mor2 ] ),|
  !gapprompt@>| !gapinput@  UniversalMorphismFromDirectSum( [ one3, one3 ] ) );|
  <A morphism in Category of matrices over GF(3)>
  !gapprompt@gap>| !gapinput@Display( result3 );|
   1 2 .
   2 2 .
   
   A morphism in Category of matrices over GF(3)
  !gapprompt@gap>| !gapinput@result3 = result2;|
  true
\end{Verbatim}
\end{computation}

Next we provide the algorithms from \ref{doc:pre-abelian} that make $\kmat$ into a pre-abelian category.
They are all based on the well-known $\mathtt{Gauss}$ algorithm that gives us the row echolon form (REF) and
the column echolon form (CEF) of a matrix.

\begin{computation}\label{comp:gauss-algorithms}
Let $\varphi : m \rightarrow n \in \kmat_{1}$ be a matrix. We assume algorithms for 
\begin{itemize}
\item The rank of a matrix, $r := \mathtt{Rank( phi )}$, defined as $\Bbbk$-dimension of the column space
$\mathrm{dim_{Col}}\,( \varphi ) = r$ which is the
same\endnote{The result ``row-rank = column-rank'' whose importance a first-year student might not understand
right away, is a very nice property of matrices.}
as the $\Bbbk$-dimension of the row space $\mathrm{dim_{Row}}( \varphi ) = r$,
\item The left nullspace of a matrix $\varphi$ is a matrix $x = \mathtt{LeftNullSpace( phi )}$ satisfying $x\, \varphi = 0$ and
for each matrix $y$ with $y\,\varphi = 0$ there exists a matrix $z$ with $zx = y$.
\item The right nullspace of a matrix $\varphi$ is a matrix $x = \mathtt{RightNullSpace( phi )}$ satisfying $\varphi \,x= 0$ and
for each matrix $y$ with $\varphi \,y= 0$ there exists a matrix $z$ with $xz = y$.
\item As the standardized form to represent these subspaces, the $\mathtt{Gauss}$-algorithm can calculate the
row-echolon-form $\mathtt{REF( LeftNullSpace( phi ) )}$ and the\\
column-echolon-form $\mathtt{CEF( RightNullSpace( phi ) )}$.
\end{itemize}
\end{computation}

\begin{example}[$\kmat$ is a pre-abelian category]\label{ex:kmat_pre-abelian}
With the algorithms in \ref{comp:gauss-algorithms} taken as given, we now give all the algorithms needed for the doctrine
$\mathtt{IsPreAbelianCategory}$ in \ref{doc:pre-abelian}:
\begin{itemize}
\item $\mathtt{KernelObject( phi ) := NrRows( phi ) - Rank( phi )}$
\item $\mathtt{KernelEmbedding( phi ) := REF( LeftNullSpace( phi ) )}$
\item $\mathtt{KernelLift( phi, tau ) := Solve( x \cdot REF( LeftNullSpace( phi ) ) = tau )}$
\item $\mathtt{CokernelObject( phi ) := NrColumns( phi ) - Rank( phi )}$
\item $\mathtt{CokernelProjection( phi ) := CEF( RightNullSpace( phi ) )}$
\item $\mathtt{CokernelColift( phi, tau ) := Solve( CEF( RightNullSpace( phi ) ) \cdot x = tau )}$
\end{itemize}
With these constructions, $\kmat$ becomes a pre-abelian category.
\end{example}

Note that the right-hand side $B$ in the equation 
\[
A \cdot x = B
\]
can be more than a single column vector, but a matrix with multiple columns, as long as they have the same number of rows as $A$.
This corresponds to solving the system of equations simultaneously for different right-hand sides. In case that it is solvable,
we get a particular solution as a matrix $x = \mathtt{LeftDivide( A, B )}$.

Dually for $B$ and $A$ matrices having the same number of columns, for the equation
\[
x \cdot A = B
\]
we get a particular solution as a matrix $x = \mathtt{RightDivide( B, A )}$, if it exists.

\begin{example}[$\kmat$ is an Abelian category]\phantom{}\\
\begin{enumerate}
\renewcommand{\labelenumi}{(\theenumi)}
\item Let $\kappa : K \hookrightarrow A \in \kmat_{1}$ and $\tau : T \rightarrow A \in \kmat_{1}$ be as in
\ref{def:abelian_category}(2).
Then with the algorithms from \ref{comp:gauss-algorithms} we define
\begin{itemize}
\item $\mathtt{LiftAlongMonomorphism( kappa, tau ) := Solve( x \cdot kappa = tau )}$\\
$\mathtt{ := RightDivide( tau, kappa )}$.
\end{itemize}
\item Let $\varepsilon : B \twoheadrightarrow C \in \kmat_{1}$ and $\tau : B \rightarrow T \in \kmat_{1}$ be as in
\ref{def:abelian_category}(3).
Then with the algorithms from \ref{comp:gauss-algorithms} we define
\begin{itemize}
\item $\mathtt{ColiftAlongEpimorphism( epsilon, tau ) := Solve( epsilon \cdot x = tau )}$\\
$\mathtt{ := LeftDivide( epsilon, tau )}$
\end{itemize}
\end{enumerate}

Since a matrix $\kappa : m \rightarrow n$ is a monomorphism iff its kernel is $0$ iff it has full row rank $\mathtt{Rank(kappa) = m}$,
the lift along monomorphism $\mathtt{RightDivide( tau, kappa )}$ always exists.

Since a matrix $\varepsilon : m \rightarrow n$ is an epimorphism iff its image is $n$ iff it has full column rank
$\mathtt{Rank(kappa) = n}$, the colift along epimorphism $\mathtt{LeftDivide( epsilon, tau )}$ always exists.

With these algorithms, $\kmat$ becomes an abelian category. In particular we have
\begin{align*}
\mathrm{Coim}(\varphi) &\cong \mathrm{Im}(\varphi)\quad\text{and since $\kmat$ is skeletal}\\
\Rightarrow\, \mathrm{Coim}(\varphi) &=\mathrm{Im}(\varphi).
\end{align*}
\end{example}


The following situation where we have a family of matrices $\{a_{i} : m_{i} \rightarrow n_{i}\}_{i\in I}$,
i.e. a family $\{m_{i}\}_{i\in I}$ of sources and $\{n_{i}\}_{i\in I}$ of targets, is useful to understand. There are two
different direct sums involved, one of the $m_{i}$'s and one of the $n_{i}$'s. We will need this construction in section 4
for the direct sum of functors, and in section 6 for the Sylvester equations.

\begin{example}[Block-Diagonal matrices]\phantom{}\label{ex:block_diagonal_matrix}\\
In a situation with an index set $I = \{1,\dots,N\}$, two families of objects $\{m_{i}\}_{i\in I}, \{n_{i}\}_{i\in I}$ and a family of
morphisms $\{a_{i} : m_{i} \rightarrow n_{i}\}_{i\in I}$ in $\kmat$, we have the two direct sums
\begin{alignat}{4}
m &:= \bigoplus_{i\in I} m_{i},\quad &&(\pi_{i})_{m} : m \rightarrow m_{i},\quad &&(\iota_{i})_{m} : m_{i} \rightarrow m, \\
n &:= \bigoplus_{i\in I} n_{i},\quad &&(\pi_{i})_{n} : n \rightarrow n_{i},\quad &&(\iota_{i})_{n} : n_{i} \rightarrow n.
\end{alignat}
This situation is displayed in the following diagram
\[
\begin{tikzcd}
m \arrow[dd, "(\pi_{i})_{m}", shift left=2] \arrow[rr, "a"]           &  & n \arrow[dd, "(\pi_{i})_{n}", shift left=2]       \\
                                                                      &  &                                                   \\
m_{i} \arrow[uu, "(\iota_{i})_{m}", shift left=2] \arrow[rr, "a_{i}"] &  & n_{i} \arrow[uu, "(\iota_{i})_{n}", shift left=2]
\end{tikzcd}
\]
The morphism $a : m \rightarrow n$ defined as
\begin{align}
a = \sum_{i \in I} (\pi_{i})_{m} a_{i} (\iota_{i})_{n}
\end{align}
satisfies
\begin{align}
(\iota_{i})_{m}\, a &= a_{i}\, (\iota_{i})_{n}, \\
a\, (\pi_{i})_{n} &= (\pi_{i})_{m}\, a_{i}\,\text{ and }\\
(\iota_{i})_{m}\, a\, (\pi_{i})_{n} &= a_{i}.
\end{align}
This an be interpreted in two ways:
\begin{enumerate}
\renewcommand{\labelenumi}{(\theenumi)}
\item For the family $\iota_{m} a := \{ (\iota_{i})_{m} a : m_{i} \rightarrow n \} := \{ a_{i}\,(\iota_{i})_{n} : m_{i} \rightarrow n \}$
of morphisms with same target $n$, we have the morphism
$u_{\text{out}}(\iota_{m} a) : m \rightarrow n$ such that\\
$(\iota_{i})_{m} u_{\text{out}}(\iota_{m} a) = (\iota_{i})_{m} a = a_{i}\,(\iota_{i})_{n}$, and
\item For the family $a \pi_{n} := \{ a (\pi_{i})_{n} : m \rightarrow n_{i} \} := \{ (\pi_{i})_{m}\,a_{i} : m \rightarrow n_{i} \}$ of
morphisms with same source $m$, we have the morphism
$u_{\text{in}}(a \pi_{n}) : m \rightarrow n$ such that\\
$u_{\text{in}}(a \pi_{n}) (\pi_{i})_{n} = a (\pi_{i})_{n} = (\pi_{i})_{m}\,a_{i}$.
\end{enumerate}
So we have
\begin{alignat}{3}
(\iota_{i})_{m} u_{\text{out}}(\iota_{m} a) (\pi_{i})_{n} &= a_{i} &&= (\iota_{i})_{m} u_{\text{in}}(a \pi_{n}) (\pi_{i})_{n}\,
\text{ and }\\
u_{\text{out}}(\iota_{m} a) &= a &&= u_{\text{in}}(a \pi_{n})
\end{alignat}
\end{example}

%%% functors between abelian categories.

\begin{theorem}
The functor category has all limits, colimits and bilimits which exist in the target category.
\end{theorem}

Instead of proving this in general, we prove this as part of \ref{thm:functor_category_abelian} for the direct sum, from which
the procedure of the general proof becomes apparent.



\section{$\Bbbk$-linear closure of a finite concrete category}


\subsection{Finite concrete categories and the free/forgetful adjunction}\label{sec:fin_concrete_cat_free_forgetful}

Our model for a finite concrete category $\mathcal{C}$ is that of a finite subcategory of $\FinSets$. In particular we restrict ourselves
to finite concrete categories that are generated by a finite set of morphisms $\mathtt{SetOfGeneratingMorphisms}$ and whose endomorphism
monoids are explicitly cyclic.

The $\mathtt{SetOfGeneratingMorphisms} = \{ a_{1},a_{2},\dots,a_{n} \}$ already defines a finite quiver:

\begin{definition}{(Finite quiver generated by a finite set of morphisms)}\label{def:quiver_generated}
Let $M = \{ a_{1}, a_{2}, \dots, a_{n} \}$ be a finite set of morphisms. We say a quiver $q$ is \ul{generated by $M$}, if
\begin{align}
q_{1} &= M, \text{ and } \\
q_{0} &= \{ o : \exists a \in M, s(a) = o \vel t(a) = o \}
\end{align}
In this case the quiver $q$ is finite.
\end{definition}

The fact that every category is also a quiver can be expressed in the following as the existence of a certain forgetful functor:

\begin{example}{(Forgetful functor $U : \mathcal{C} \rightarrow \mathcal{D}$)}
\begin{enumerate}
\renewcommand{\labelenumi}{(\theenumi)}
\item We denote by the letter $U$ (for \textit{underlying}) a \ul{forgetful functor} between two categories $U : \mathcal{C} \rightarrow \mathcal{D}$ if
we can identify every object $c \in \mathcal{C}$ as an object $Uc \in \mathcal{D}$ by \textit{forgetting} some additional structure that $c$ had
in $\mathcal{C}$ but that is not defined for objects in $\mathcal{D}$. The object $Uc$ is called the \ul{underlying object} of $c$ (e.g. the
\ul{underlying set} of a group).
\item For a morphism $a : c \rightarrow c' \in \mathcal{C}$ that was some structure-preserving map between $c$ and $c'$, if that structure doesn't
exist in the category $\mathcal{D}$ then $Ua : Uc \rightarrow Uc'$ \textit{forgets} the structure-preserving property of $a$.
\item There are other conceivable functors that even map morphisms between two objects
(e.g. functors between two categories are morphisms in $\mathrm{\textbf{Cat}}$) to objects (e.g. functors in the functor category). If you now
want to get back the morphism from the object, you again are using a forgetful functor (e.g. to get the \ul{underlying functor} of the functor object).
\end{enumerate}
Some authors define a forgetful functor in the strict sense that its target category $\mathcal{D} = \mathrm{\textbf{Set}}$, i.e. it forgets all structure;
and functors that only forget some but not all of the algebraic structure are called \ul{intermediate forgetful functors}.
\end{example}

\noindent We are interested in the forgetful functor with $\mathcal{C} = \Cat$ and $\mathcal{D} = \Quiv$:

\begin{example}{(Forgetful functor $U  : \Cat \hookrightarrow \Quiv$)}
Let $\mathcal{C} \in \Cat$ be a category. The quiver $q = U\mathcal{C}$ is defined by
\begin{align}
q_{0} &= \mathcal{C}_{0} \\
q_{1} &= \mathcal{C}_{1}
\end{align}
In particular every identity morphism $1_{c} \in \mathcal{C}_{1}$ for an object $c \in \mathcal{C}$ now is just any other endomorphism
on that object (but it is still true that $s(1_{c}) = t(1_{c}) = c$).
And every morphism $\varphi\psi \in \mathcal{C}_{1}$ that was the composition of $\varphi$ with $\psi$ is now just
any morphism without much deeper connection to $\varphi$ and $\psi$ apart from
\begin{align}
s(\varphi\psi) &= s(\varphi) \text{ and } \\
t(\varphi\psi) &= t(\psi),
\end{align}
which is still true in $\Quiv$. Of course, associativity and unital property of the composition $\mu$ doesn't exist in $\Quiv$ since there is no composition
of arrows.
\end{example}

The following algorithm maps the objects and generating morphisms of a finite concrete category $\mathcal{C}$ to a
right quiver with the same number of objects and with an arrow for each generating morphism. 

\begin{algorithm}[H]\capstart
    \caption{\texttt{RightQuiverFromConcreteCategory}}\label{algo:RightQuiverFromConcreteCategory}
	\SetKwInput{Input}{Input~}
	\SetKwInput{Output}{Output~}
	\Input{~a finite concrete category $C$ with $n$ objects}
	\Output{~the right quiver $q$ with the same number of objects and an arrow for each generating morphism}
	\BlankLine
	$Obj := \mathtt{SetOfObjects}(C)$\;
	$n := \mathtt{Length}(Obj)$\;
	$gMor := \mathtt{SetOfGeneratingMorphisms}(C)$\;
	$A := \emptyset$\tcp*{this will be the set of arrows as pairs of natural numbers}
	$i := 1$\;
	\ForEach{\textnormal{morphism} $mor \in gMor$}{
	    $A_{i,1} := $ the position of $\mathtt{Source}( mor )$ in $Obj$\;
	    $A_{i,2} := $ the position of $\mathtt{Range}( mor )$ in $Obj$\;
	    $i := i+1$\;
	}
	\BlankLine
	$q := \mathtt{RightQuiver}$ with vertices $\{1,\dots,n\}$ and arrows $A$.
	\BlankLine
	\Return $q$\;
\end{algorithm}

\begin{example}{(Underlying quiver)}\label{ex:underlying_quiver}\\

\noindent\begin{minipage}{.08\textwidth}
\phantom{}
\end{minipage}
\begin{minipage}{.37\textwidth}
\begin{tikzcd}[boxedcd={inner xsep=1.5em, inner ysep=3em}]
2 \arrow[rrrr, "b"] \arrow[rrrrddd, "e", pos=0.3] \arrow["h"', loop, distance=2em, in=125, out=55] &  &  &  &
3 \arrow[ddd, "c"] \arrow["i"', loop, distance=2em, in=125, out=55]\\
 &  &  &  & \\
 &  &  &  & \\
1 \arrow[uuu, "a"] \arrow[rrrruuu, "d", pos=0.3] \arrow[rrrr, bend left, "f" ', shift right=2]
\arrow[rrrr, "f", bend right] \arrow["g"', loop, distance=2em, in=305, out=235] &  &  &  &
4 \arrow["j"', loop, distance=2em, in=305, out=235]
\end{tikzcd}
\end{minipage}
%
\begin{minipage}{.10\textwidth}
\center$\xhookleftarrow{\text{   U   }}$
\end{minipage}
%
\begin{minipage}{.37\textwidth}
\begin{tikzcd}[boxedcd={inner xsep=1.5em, inner ysep=3em}]
B \arrow[rrrr, "\psi"] \arrow[rrrrddd, "\psi\rho", pos=0.3] \arrow["1_{B}"', loop, distance=2em, in=125, out=55] &  &  &  &
C \arrow[ddd, "\rho"] \arrow["1_{C}"', loop, distance=2em, in=125, out=55]\\
 &  &  &  & \\
 &  &  &  & \\
A \arrow[uuu, "\varphi"] \arrow[rrrruuu, "\varphi\psi", pos=0.3] \arrow[rrrr, bend left, "(\varphi\psi)\rho" ', shift right=2]
\arrow[rrrr, "\varphi(\psi\rho)", bend right] \arrow["1_{A}"', loop, distance=2em, in=305, out=235] &  &  &  &
D \arrow["1_{D}"', loop, distance=2em, in=305, out=235]
\end{tikzcd}
\end{minipage}
\begin{minipage}{.08\textwidth}
\phantom{}
\end{minipage}\\

\noindent In the category on the right, associativity of composition guaranteed that $(\varphi\psi)\rho = \varphi(\psi\rho)$, so those two arrows
were already the same, so they are mapped to the same arrow $f = U((\varphi\psi)\rho) = U(\varphi(\psi\rho))$ in the quiver on the left.
We didn't have to draw both arrows for $f$, but since they are equal, there is still only one arrow in the hom-set $\textup{Hom}_{q}(1,4)=\{f,f\} = \{f\}$.\\
All the other identities are not preserved under the forgetful functor, e.g. $d$ doesn't know what it has to do with $a$ and $b$ apart from
$s(d) = s(a)$ and $t(d) = t(b)$. Especially the former identity arrows are now just endomorphisms with no defining property.\\
The paths $g^{2}f, gf$ and $fj^{3}$ are all different, while in the category, they all simplify to
$1_{A}1_{A}(\varphi\psi)\rho = 1_{A}(\varphi\psi)\rho = (\varphi\psi)\rho1_{D}1_{D}1_{D} =  (\varphi\psi)\rho$ due to the unit property and associativity.
\end{example}

The category $\mathcal{C}$ in the last example has the set of morphisms $\mathcal{C}_{1} =
\{ 1_{A}, 1_{B}, 1_{C}, 1_{D}, \varphi, \psi, \rho, \varphi\psi, \psi\rho, \varphi\psi\rho \}$, i.e. 10 morphisms. But once the three morphisms
$\varphi, \psi$ and $\rho$ were defined, the other seven morphisms were forced from the unit and composition axioms of a category.

\begin{example}{(Category generated by one endomorphism)}\label{ex:category_generated_by_one_endomorphism}
As another example, take a category $\mathcal{M}$ with one object $\ast$ and apart from $1_{\ast}$ one other endomorphism
$\alpha : \ast \rightarrow \ast$. It already has a priori countably infinitely many morphisms
$\mathcal{M}_{1} = \{ 1_{\ast}, \alpha, \alpha^{2}, \alpha^{3}, \dots \}$. But the information to generate that category is all encoded in the
one morphism $\alpha$.
\end{example}

What we are looking for is a construction of a finite concrete category from a finite set of generating morphisms. For this we can take
the generated quiver from \ref{def:quiver_generated} and from it the free category.

\begin{definition}{(The free category $F : \Quiv \rightarrow \Cat$)}\label{def:free_category}\endnote{Ref. \cite{[context]} example 4.1.13.}
The \ul{free category} $Fq$ of a quiver $q$ has $q_{0}$ as its set of objects. The set $(Fq)_{1}$ of morphisms consists of all finite paths of arrows in
$q_{1}$. The identity morphism $1_{c}$ of an object $c \in q_{0}$ is defined as the empty path from $c$ to itself. Composition is defined by
concatenation of paths.
\end{definition}

\begin{definition}{(Free $\dashv$ forgetful adjunction)}
The functor pair $F : \Quiv \leftrightarrows \Cat : U$ is an example for an adjunction, i.e. for the functors $F : \Quiv \rightarrow \Cat$ and
$U : \Cat \rightarrow \Quiv$ there is an isomorphism
\begin{align}
\mathrm{Hom}_{\Cat}(Fq, \mathcal{C}) &\cong \mathrm{Hom}_{\Quiv}(q,U\mathcal{C})
\end{align}
for each $q \in \Quiv$ and $\mathcal{C} \in \Cat$, that is natural in both $q$ and $\mathcal{C}$. Here $U$ is \ul{right adjoint} to $F$ and
the forgetful functor $U$ admits a \ul{left adjoint}, free construction $F$.
\end{definition}

We can view the free category functor in action in two different ways: Where does the category $\mathcal{C}$ in example \ref{ex:underlying_quiver}
come from, i.e. what is the quiver $q$ such that $\mathcal{C} = Fq$? And where does it go after we forget the category structure, i.e. what is
the category $F(U(\mathcal{C}))$? We will illustrate the answers to both questions in the next example:

\begin{example}{(Generating quiver $\xrightarrow{F}$ category $\xrightarrow{U}$ underlying quiver $\xrightarrow{F}$ category)}
\begin{enumerate}
\renewcommand{\labelenumi}{(\theenumi)}
\item The free category generated by the quiver:
\[
\noindent\begin{minipage}{.08\textwidth}
\phantom{}
\end{minipage}
%
\begin{minipage}{.37\textwidth}
\begin{tikzcd}[boxedcd={inner xsep=1.5em, inner ysep=3em}]
B \arrow[rrrr, "\psi"] &  &  &  & C \arrow[ddd, "\rho"] \\
 &  &  &  & \\
 &  &  &  & \\
A \arrow[uuu, "\varphi"] &  &  &  & D
\end{tikzcd}
\end{minipage}
%
\begin{minipage}{.10\textwidth}
\center$\xrightarrow{\text{     }F\text{     }}$
\end{minipage}
%
\begin{minipage}{.37\textwidth}
\begin{tikzcd}[boxedcd={inner xsep=1.5em, inner ysep=3em}]
B \arrow[rrrr, "\psi"] \arrow[rrrrddd, "\psi\rho", pos=0.3] \arrow["1_{B}"', loop, distance=2em, in=125, out=55] &  &  &  &
C \arrow[ddd, "\rho"] \arrow["1_{C}"', loop, distance=2em, in=125, out=55]\\
 &  &  &  & \\
 &  &  &  & \\
A \arrow[uuu, "\varphi"] \arrow[rrrruuu, "\varphi\psi", pos=0.3] \arrow[rrrr, bend left, "(\varphi\psi)\rho" ', shift right=2]
\arrow[rrrr, "\varphi(\psi\rho)", bend right] \arrow["1_{A}"', loop, distance=2em, in=305, out=235] &  &  &  &
D \arrow["1_{D}"', loop, distance=2em, in=305, out=235]
\end{tikzcd}
\end{minipage}
\begin{minipage}{.08\textwidth}
\phantom{}
\end{minipage}
\]
\item The free category generated by the underlying quiver:
\[
\noindent\begin{minipage}{.08\textwidth}
\phantom{}
\end{minipage}
\begin{minipage}{.37\textwidth}
\begin{tikzcd}[boxedcd={inner xsep=1.5em, inner ysep=3em}]
2 \arrow[rrrr, "b"] \arrow[rrrrddd, "e", pos=0.3] \arrow["h"', loop, distance=2em, in=125, out=55] &  &  &  &
3 \arrow[ddd, "c"] \arrow["i"', loop, distance=2em, in=125, out=55]\\
 &  &  &  & \\
 &  &  &  & \\
1 \arrow[uuu, "a"] \arrow[rrrruuu, "d", pos=0.3] \arrow[rrrr, bend left, "f" ', shift right=2]
\arrow["g"', loop, distance=2em, in=305, out=235] &  &  &  &
4 \arrow["j"', loop, distance=2em, in=305, out=235]
\end{tikzcd}
\end{minipage}
%
\begin{minipage}{.10\textwidth}
\center$\xrightarrow{\text{     }F\text{     }}$
\end{minipage}
%
\begin{minipage}{.37\textwidth}
\begin{tikzcd}[boxedcd={inner xsep=3em, inner ysep=3em}]
2 \arrow[rrrr, "b"] \arrow[rrrrddd, "e", pos=0.75] \arrow["h"', loop, distance=2em, in=125, out=55] \arrow["1_{2}"', loop, distance=2em, in=215, out=145] \arrow[rrrrddd, "bc", pos=0.75, shift left=5]                                                                                                                                                                                                                                                                                       &  &  &  & 3 \arrow[ddd, "c"] \arrow["i"', loop, distance=2em, in=125, out=55] \arrow["1_{3}"', loop, distance=2em, in=35, out=325] \\
                                                                                                                                                                                                                                                                                                                                                                                                                                                                          &  &  &  &                                                                                                                          \\
                                                                                                                                                                                                                                                                                                                                                                                                                                                                          &  &  &  &                                                                                                                          \\
1 \arrow[uuu, "gah"', pos=0.65, shift right=3] \arrow[rrrruuu, "d", pos=0.75] \arrow[rrrr, "f", bend left, shift right=2] \arrow[rrrr, "abc", bend right] \arrow["g"', loop, distance=2em, in=305, out=235] \arrow["1_{1}"', loop, distance=2em, in=215, out=145] \arrow[uuu, "ah" description, pos=0.6] \arrow[uuu, "ga" description, pos=0.45, shift left=3] \arrow[uuu, "a" description, pos=0.3, shift left=6] \arrow[rrrruuu, "ab", pos=0.67, shift left=6] \arrow[rrrr, "dc", shift right=3] \arrow[rrrr, "ae", shift left=2] &  &  &  & 4 \arrow["j"', loop, distance=2em, in=305, out=235] \arrow["1_{4}"', loop, distance=2em, in=35, out=325]                
\end{tikzcd}
\end{minipage}
%
\begin{minipage}{.08\textwidth}
\phantom{}
\end{minipage}
\]
As you can see, this picture gets cluttered very fast (not all morphisms were drawn in the picture). The reason for this is the existence of non-identity endomorphisms.
As we have shown in lemma \ref{la:cyclic_paths}, one non-identity endomorphism is enough for a quiver to have infinitely many paths.
Here it is even worse than in the example \ref{ex:category_generated_by_one_endomorphism} where we had countably many endomorphisms
$\alpha^{n}, n \in \mathbb{N}$. Through the arrow $a : 1 \rightarrow 2$ we can concatenate countably many morphisms
$g^{m}ah^{n}, (m,n) \in \mathbb{N}\times\mathbb{N}$ and even $g^{n_{1}}ah^{n_{2}}bi^{n_{3}}cj^{n_{4}}, (n_{1}, n_{2}, n_{3}, n_{4}) \in \mathbb{N}^{4}$.
If we were to construct the path algebra (see \ref{def:path_algebra}) on the quiver, it already had an infinite basis.
\end{enumerate}
\end{example}

\begin{example}{(Continued example \ref{ex:category_generated_by_one_endomorphism})}\label{ex:U-F-U-F_from_singleton}\\

As a last example to see how bad it can get from seemingly innocent quivers, take the category with 1 object and its identity morphism:
\[
\noindent\begin{minipage}{.005\textwidth}
\phantom{}
\end{minipage}
\begin{minipage}{.08\textwidth}
\begin{tikzcd}
\ast \arrow["1_{\ast}"', loop, distance=2em, in=305, out=235]
\end{tikzcd}
\end{minipage}
%
\begin{minipage}{.05\textwidth}
$\xrightarrow{\text{     }U\text{     }}$
\end{minipage}
%
\begin{minipage}{.08\textwidth}
\begin{tikzcd}
\ast \arrow["a"', loop, distance=2em, in=305, out=235]
\end{tikzcd}
\end{minipage}
%
\begin{minipage}{.05\textwidth}
$\xrightarrow{\text{     }F\text{     }}$
\end{minipage}
%
\begin{minipage}{.15\textwidth}
\begin{tikzcd}
\ast \arrow["a"', loop, distance=2em, in=305, out=235] \arrow["1_{\ast}"', loop, distance=2em, in=125, out=55] \arrow["{a^{2}, a^{3},\dots}"', loop, distance=2em, in=35, out=325]
\end{tikzcd}
\end{minipage}
%
\begin{minipage}{.05\textwidth}
$\xrightarrow{\text{     }U\text{     }}$
\end{minipage}
%
\begin{minipage}{.08\textwidth}
\begin{tikzcd}
\ast \arrow["{a, b, c,\dots}"', loop, distance=2em, in=305, out=235]
\end{tikzcd}
\end{minipage}
%
\begin{minipage}{.05\textwidth}
$\xrightarrow{\text{     }F\text{     }}$
\end{minipage}
%
\begin{minipage}{.25\textwidth}
\begin{tikzcd}
\ast \arrow["{a, b, c,\dots}"', loop, distance=2em, in=305, out=235] \arrow["1_{\ast}"', loop, distance=2em, in=125, out=55] \arrow["{a^{2},\dots,ab,\dots,ababbaba,\dots,b^{2},\dots,c^{2},\dots}"', loop, distance=2em, in=35, out=325]
\end{tikzcd}
\end{minipage}
\begin{minipage}{.10\textwidth}
\phantom{}
\end{minipage}
\]
After the first forgetful functor, we are in the situation of \ref{ex:category_generated_by_one_endomorphism} where then the first free functor
gives us countably many morphisms.
The following forgetful functor only renames them to countably many distinct morphisms.
In the last step, we are constructing the free monoid on countably many generators. Among other things it contains the free monoid on two
generators.\endnote{As my father rightly remarked when I showed this to him, \enquote{You can hide the whole world inside there!}}
\end{example}

We can learn two lessons from these examples:
\begin{enumerate}
\item An adjunction is more general than an equivalence of categories. The free functor $F$ doesn't just \textit{undo} the forgetful functor $U$.
You will end up with much more than you started with.
\item If we want to still work with finite categories, we really need to control the size of our hom-sets, especially regarding the endomorphisms.
This is the topic of the next section.
\end{enumerate}

\vspace{4em}

%\begin{minipage}{.01\textwidth}\phantom{}
%\end{minipage}
%
\noindent\begin{minipage}[t]{.55\textwidth}\vspace{0pt}%
\subsection{Relations of endomorphisms}
\begin{lemma}{($\sigma$-Lemma)}\label{la:sigma-lemma}
\begin{enumerate}
\renewcommand{\labelenumi}{(\theenumi)}
\item Let $\mathcal{C}$ be a finite concrete category. In remark \ref{rmk:endo_monoid} we showed that for each object
$M \in \mathcal{C}_{0}$ the set $\mathrm{End}_{\mathcal{C}}(M)$ is a monoid. For each endomorphism $f \in \mathrm{End}_{\mathcal{C}}(M)$
there exist $m,n \in \mathbb{N}, n\geq 1$, such that $f^{m+n}=f^{m}$.
\item When we restrict both the source and target of $f$ to $\mathrm{Im}(f^{m})$, then $f\restrict{\mathrm{Im}(f^{m})}$ is an isomorphism, i.e.
$f\restrict{\mathrm{Im}(f^{m})} \in \mathrm{Aut}_{\mathcal{C}}(\mathrm{Im}(f^{m}))$ with
$f\restrict{\mathrm{Im}(f^{m})}^{n} = 1_{\mathrm{Im}(f^{m})}$.
\item If $m = 0$ and $n \geq 1$ then $f \in \mathrm{Aut}_{\mathcal{C}}(M)$ is an automorphism with $f^{-1} = f^{n-1}$ and has order $n$.
\item An upper bound for $m$ is $\abs{M}$ and for $n$ is $g(N)$ with $N = \abs{M}$ and Landau's function\endnote{My thanks to
Felix Potthast and Michael Figelius who both sent me in the direction of Landau's function} $g(N)$
defined as the largest order of an element of the symmetric group $S_{N}$.
\end{enumerate}
\end{lemma}
\noindent
%m=1, n=1
\begin{minipage}[b]{.45\textwidth}
\begin{tikzpicture}[->,>=stealth',auto,node distance=1cm,
  thick,main node/.style={circle,draw,font=\sffamily\Large\bfseries}]

  % nodes on the leg of sigma
  \node[main node] (1) [circle, fill, inner sep=2pt] {};
  \node[main node] (2) [left of=1, circle, fill, inner sep=2pt] {};
  
  \path[every node/.style={font=\sffamily\small}]
  % arrows on the leg of sigma
    (1) edge[-] node [left] {} (2)
    (1) edge["$f^{0} = 1_{M}$"', pos=0.60, out=300, in=40, min distance=7mm, looseness=8] (1)
    (1) edge["$f^{1}$"', bend right] node [right] {} (2)
    (2) edge["$f^{1+1}$"', pos=0.55, out=220, in=300, min distance=7mm, looseness=8] (2);
\end{tikzpicture}
\captionof{figure}{$f$ is idempotent. $m=1$ and $n=1$.}
\end{minipage}\hfill
%
\begin{minipage}[b]{.02\textwidth}
\phantom{}
\end{minipage}
\begin{minipage}[b]{.45\textwidth}
%m=0, n=2
\begin{tikzpicture}[->,>=stealth',auto,node distance=0.5cm,
  thick,main node/.style={circle,draw,font=\sffamily\Large\bfseries}]

  % nodes on the leg of sigma
  \node[main node] (1) [circle, fill, inner sep=2pt] {};
  % draw the circle
  \node[main node] (C) [below of=1, circle, draw, inner sep=0.35cm] {};
  
  \node[main node] (2) [below of=C, circle, fill, inner sep=2pt] {};
  
  \path[every node/.style={font=\sffamily\small}]
  % arrows on the leg of sigma
    (1) edge["$f^{0} = 1_{M}$"', pos=0.32, out=40, in=100, min distance=7mm, looseness=8] (1)
    (1) edge["$f^{0+1}$"', bend right=120, min distance=11mm] node [right] {} (2)
%    (2) edge["$f^{0+2}$", pos=0.5, bend right=55] node [right] {} (1);
    (2) edge["$f^{0+2}$"', bend right=120, min distance=11mm] node [right] {} (1);
\end{tikzpicture}
\captionof{figure}{$f$ is bijective with order $2$. $m=0$ and $n=2$.}
\end{minipage}
\end{minipage}
%
\begin{minipage}{.01\textwidth}
\phantom{}
\end{minipage}
%
\begin{minipage}[t]{.45\textwidth}\vspace{0pt}
%[$\sigma$ lemma illustrated for different sizes of $m$ and $n$]
%m=3, n=5
\begin{tikzpicture}[->,>=stealth',auto,node distance=1cm,
  thick,main node/.style={circle,draw,font=\sffamily\Large\bfseries}]

  % nodes on the leg of sigma
  \node[main node] (1) [circle, fill, inner sep=2pt] {};
  \node[main node] (2) [left of=1, circle, fill, inner sep=2pt] {};
  \node[main node] (3) [left of=2, circle, fill, inner sep=2pt] {};
  \node[main node] (4) [left of=3, circle, fill, inner sep=2pt] {};
  
  % nodes on the circle of sigma
  \node[main node] (5) [above right=-0.84098cm and -1.10106cm of 4, circle, fill, inner sep=2pt] {};
  \node[main node] (6) [above right=-1.95902cm and -0.73779cm of 4, circle, fill, inner sep=2pt] {};
  \node[main node] (7) [above right=-1.95902cm and 0.43779cm of 4, circle, fill, inner sep=2pt] {};
  \node[main node] (8) [above right=-0.84098cm and 0.80106cm of 4, circle, fill, inner sep=2pt] {};
  \node[main node] (9) [above right=-0.150cm and -0.150cm of 4, circle, fill, inner sep=2pt] {};
  
  % draw the circle
  \node[main node] (C) [below of=4, circle, draw, inner sep=0.70cm] {};
  
  \path[every node/.style={font=\sffamily\small}]
  % arrows on the leg of sigma
    (1) edge[-] node [left] {} (2)
    (2) edge[-] node [left] {} (3)
    (3) edge[-] node [left] {} (4)
    (1) edge["$f^{0} = 1_{M}$"', pos=0.60, out=300, in=40, min distance=7mm, looseness=8] (1)
    (1) edge["$f^{1}$"', bend right] node [right] {} (2)
    (2) edge["$f^{2}$"', bend right] node [right] {} (3)
    (3) edge["$f^{3}$"', bend right] node [right] {} (4)
    % arrows on the circle of sigma
    (9) edge["$f^{3+1}$"', bend right=55] node [right] {} (5)
    (5) edge["$f^{3+2}$"', bend right=55] node [right] {} (6)
    (6) edge["$f^{3+3}$"', bend right=55] node [right] {} (7)
    (7) edge["$f^{3+4}$"', bend right=55] node [right] {} (8)
    (8) edge["$f^{3+5}$", pos=0.4, bend left=55] node [right] {} (9);
\end{tikzpicture}
\captionof{figure}{$\sigma$ lemma with $m=3$ and $n=5$.}
%m=1, n=10
\begin{tikzpicture}[->,>=stealth',auto,node distance=1cm,
  thick,main node/.style={circle,draw,font=\sffamily\Large\bfseries}]

  % nodes on the leg of sigma
  \node[main node] (1) [circle, fill, inner sep=2pt] {};
  \node[main node] (2) [left of=1, circle, fill, inner sep=2pt] {};
  
  % draw the circle (radius = 0.51540)
%  \node[main node] (C) [above right=-2cm and -1.5cm of 1.center, circle, draw, inner sep=1.4577cm] {};
  \node[main node] (C) [above right=-2.4cm and -2.52cm of 1.center, circle, draw, inner sep=1.0308cm] {};
  
  % nodes on the circle of sigma
  \node[main node] (3) [left of=2, circle, fill, inner sep=2pt] {};
  \node[main node] (10) [below right=1.25cm and 0.88cm of 2.center, circle, fill, inner sep=2pt] {};
  \node[main node] (5) [left=2.70cm of 10, circle, fill, inner sep=2pt] {};
  \node[main node] (8) [below=2.5cm of 2, circle, fill, inner sep=2pt] {};
  \node[main node] (7) [left of=8, circle, fill, inner sep=2pt] {};
  \node[main node] (11) [below right=0.40cm and 0.60cm of 2.center, circle, fill, inner sep=2pt] {};
  \node[main node] (4) [left=2.10cm of 11, circle, fill, inner sep=2pt] {};
  \node[main node] (9) [below=1.55cm of 11, circle, fill, inner sep=2pt] {};
  \node[main node] (6) [below=1.55cm of 4, circle, fill, inner sep=2pt] {};
  
  \path[every node/.style={font=\sffamily\small}]
    % arrows on the leg of sigma
    (1) edge[-] node [left] {} (2)
    (1) edge["$f^{0} = 1_{M}$"', pos=0.60, out=300, in=40, min distance=7mm, looseness=8] (1)
    (1) edge["$f^{1}$"', bend right] node [right] {} (2)
    % arrows on the circle of sigma
    (2) edge["$f^{1+1}$"', bend right=55] node [right] {} (3)
    (3) edge["$f^{1+2}$"', bend right=55] node [right] {} (4)
    (4) edge["$f^{1+3}$"', bend right=55] node [right] {} (5)
    (5) edge["$f^{1+4}$"', bend right=55] node [right] {} (6)
    (6) edge["$f^{1+5}$"', bend right=55] node [right] {} (7)
    (7) edge["$f^{1+6}$"', bend right=55] node [right] {} (8)
    (8) edge["$f^{1+7}$"', bend right=55] node [right] {} (9)
    (9) edge["$f^{1+8}$"', pos=0.65, bend right=55] node [right] {} (10)
    (10) edge["$f^{1+9}$"', pos=0.3, bend right=55] node [right] {} (11)
    (11) edge["$f^{1+10}$", bend left=55] node [right] {} (2);
\end{tikzpicture}
\captionof{figure}{$\sigma$ lemma with $m=1$ and $n=10$.}
\end{minipage}

\begin{proof}[Proof\nopunct]
\begin{subproof}[of (1)]
The set of all endomorphisms $\mathrm{End}_{\mathcal{C}}(M)$ of the finite set $M$ with $\abs{M} < \infty$ is also finite:
\begin{align}
\abs{\textup{End}_{\mathcal{C}}(M)} = \abs{M^{M}} = \abs{M}^{\abs{M}} < \infty.
\end{align}
Let $f \in \mathrm{End}_{\mathcal{C}}(M)$. Then $\{f^{k} : k \in \mathbb{N}_{0} \} \subseteq \textup{End}_{\mathcal{C}}(M)$ is a finite set
$\{f^{k} : k \in \mathbb{N}_{0} \} = \{f^{0},f^{1},\dots,f^{N}\}$. There is a minimal $l\in \mathbb{N}_{\geq 1}$ such that
$\{f^{0},f^{1},\dots,f^{l-1}\} = \{f^{0},f^{1},\dots,f^{l}\}$, i.e. the first repetition $f^{l} = f^{j}$ for $j \in \{0,\dots,l-1\}$. Since $l$ is the first
such number, the $j$ is also uniquely determined. With $m := j$ and $n := l-j \geq l \geq 1$ we have found $m,n$ satisfying
\begin{align}
f^{m+n} = f^{j + l-j} = f^{l} = f^{j} = f^{m}.
\end{align}
\end{subproof}
\begin{subproof}[Proof of (2).]
\begin{align*}
&f^{m+n} &&= f^{m} \\
\Rightarrow &f^{n}f^{m} &&= f^{m} \\
\Rightarrow &f^{n}(y) &&= y, \text{ if } y = f^{m}(x), x \in M \\
\Rightarrow &f^{n}(y) &&= y, \text{ if } y \in \mathrm{Im}(f^{m}) \\
\Rightarrow &f^{n} &&= 1_{\mathrm{Im}(f^{m})}
\end{align*}
The identity on $\mathrm{Im}(f^{m})$ is $1_{\mathrm{Im}(f^{m})} : \mathrm{Im}(f^{m}) \rightarrow \mathrm{Im}(f^{m})$
and maps $\mathrm{Im}(f^{m})\ni y \mapsto y \in \mathrm{Im}(f^{m})$.
Let $x \in M$ and $y := f^{m}(x)$. Then we can identify as the identity on $\mathrm{Im}(f^{m})$ the function
that maps $y \mapsto y$ iff it maps $f^{m}(x) \mapsto f^{m}(x)$.\\
For $f : \mathrm{Im}(f^{m}) \rightarrow \mathrm{Im}(f^{m})$ we have
$f^{n-1} : \mathrm{Im}(f^{m}) \rightarrow \mathrm{Im}(f^{m})$ such that for all
$x \in M$ we have $(f\,f^{n-1})(f^{m}(x)) = f^{m+n}(x) = f^{m}(x)$ and
$(f^{n-1}\,f)(f^{m}(x)) = f^{m+n}(x) = f^{m}(x)$. Therefore $f : \mathrm{Im}(f^{m}) \rightarrow \mathrm{Im}(f^{m})$
is an iso with inverse $f^{n-1}$.
\end{subproof}
\begin{subproof}[Proof of (3).]
Since $f^{m+n} = 1_{\mathrm{Im}(f^{m})}$, for $m = 0$ we have
$\mathrm{Im}(f^{0}) = \mathrm{Im}(1_{M}) = M$ and therefore
$f^{n} = 1_{M}$, i.e. $f$ is bijective with $f^{-1} = f^{n-1}$.
\end{subproof}
\begin{subproof}[Proof of (4).]
In the case of $m = 0$, the morphism $f$ is a bijective function, i.e. an element of $S_{M}$. With the obvious
bijection $M \cong \{1,\dots,N\}$ we also have a bijection of groups $\varphi : S_{M} \cong S_{N}$. By the definition
of Landau's function $g$, there is a permutation $a \in S_{N}$ of order at most $g(N)$ with $a = \varphi(f)$, i.e.
$\mathrm{ord}(a) \leq g(N)$, in other words, $\exists n\leq g(N) : a^{n} = 1_{\{1,\dots,N\}}$. For the same $n$ we have
$f^{n} = (\varphi^{-1}a)^{n} = \varphi^{-1}(a^{n}) = \varphi^{-1}(1_{\{1,\dots,N\}}) = 1_{M}$.\\
In \textit{(2)} we proved that $f\restrict{\mathrm{Im}(f^{m})}$ is bijective on $\mathrm{Im}(f^{m})$, therefore an element of
$S_{\mathrm{Im}(f^{m})}$ and we can find with the same argument as above an upper bound $g(N') \leq g(N), N' := \abs{\mathrm{Im}(f^{m})}$
since $\mathrm{Im}(f^{m}) \subseteq M$. This gives us an upper bound for $n$ in the general case.\\
For an upper bound for $m$ let's imagine an example with $m=11$ and $n=1$:\\

%m=11, n=1
\begin{tikzpicture}[->,>=stealth',auto,node distance=1cm,
  thick,main node/.style={circle,draw,font=\sffamily\Large\bfseries}]

  % nodes on the leg of sigma
  \node[main node] (1) [circle, fill, inner sep=2pt] {};
  \node[main node] (2) [left of=1, circle, fill, inner sep=2pt] {};
  \node[main node] (3) [left of=2, circle, fill, inner sep=2pt] {};
  \node[main node] (4) [left of=3, circle, fill, inner sep=2pt] {};
  \node[main node] (5) [left of=4, circle, fill, inner sep=2pt] {};
  \node[main node] (6) [left of=5, circle, fill, inner sep=2pt] {};
  \node[main node] (7) [left of=6, circle, fill, inner sep=2pt] {};
  \node[main node] (8) [left of=7, circle, fill, inner sep=2pt] {};
  \node[main node] (9) [left of=8, circle, fill, inner sep=2pt] {};
  \node[main node] (10) [left of=9, circle, fill, inner sep=2pt] {};
  \node[main node] (11) [left of=10, circle, fill, inner sep=2pt] {};
  \node[main node] (12) [left of=11, circle, fill, inner sep=2pt] {};
  
  % draw the circle
  
  \path[every node/.style={font=\sffamily\small}]
    % arrows on the leg of sigma
    (1) edge[-] node [left] {} (12)
    (1) edge["$f^{0} = 1_{M}$"', pos=0.60, out=300, in=40, min distance=7mm, looseness=8] (1)
    (12) edge["$f^{11+1}$"', pos=0.60, out=140, in=220, min distance=7mm, looseness=8] (12)
    (1) edge["$f^{1}$"', bend right] node [right] {} (2)
    (2) edge["$f^{2}$"', bend right] node [right] {} (3)
    (3) edge["$f^{3}$"', bend right] node [right] {} (4)
    (4) edge["$f^{4}$"', bend right] node [right] {} (5)
    (5) edge["$f^{5}$"', bend right] node [right] {} (6)
    (6) edge["$f^{6}$"', bend right] node [right] {} (7)
    (7) edge["$f^{7}$"', bend right] node [right] {} (8)
    (8) edge["$f^{8}$"', bend right] node [right] {} (9)
    (9) edge["$f^{9}$"', bend right] node [right] {} (10)
    (10) edge["$f^{10}$"', bend right] node [right] {} (11)
    (11) edge["$f^{11}$"', bend right] node [right] {} (12);
\end{tikzpicture}\\

If by $f^{10}$ we had already nine distinct functions $f^{1},\dots,f^{9}$ with no repetitions, and also the inclusions
$M \supsetneq \mathrm{Im}(f) \supsetneq \mathrm{Im}(f^{2}) \supsetneq \dots \supsetneq \mathrm{Im}(f^{9})$, we can only have
so many proper subsets of M. If in the chain of images, $|\mathrm{Im}(f^{k+1})| = |\mathrm{Im}(f^{k})| - 1$ then after
$m = \abs{M}$ steps the set $\mathrm{Im}(f^{m})$ is a singleton and the next function $f : \mathrm{Im}(f^{m}) \rightarrow \mathrm{Im}(f^{m})$
is forced to be bijective.
\end{subproof}
\end{proof}

\begin{example}{(Endomorphism $f$ satisfying the relation $f^{m+n} = f^{m}$)}
\begin{enumerate}
\renewcommand{\labelenumi}{(\theenumi)}
\item For any $m, n \in \mathbb{N}, n\geq 1$ we can find a finite set $M$ and an endomorphism $f \in \mathrm{End}_{\mathrm{FinSet}}(M)$
with $f^{m+n} = f^{m}$:\\
Let $\Bbbk$ be a field, e.g. $\Bbbk :=\{\cdot,1\}$, then on the vector space $V := \Bbbk^{m+n}$ with
standard basis $B = \{e_{1},\dots,e_{m+n}\}$ we can define the endomorphism $f$ via the companion matrix $\mathcal{M}_{p}$ to the polynomial
$p(x) = x^{m+n}-x^{m} \in \Bbbk[x]$. This matrix defines an endomorphism\\
$f : \{1,\dots,m+n\} \rightarrow \{1,\dots,m+n\}; 1 \mapsto 2, 2\mapsto 3, \dots, m+n-1 \mapsto m+n, m+n \mapsto m+1$
on the finite set $M := \{1,\dots,m+n\}$ satisfying $f^{m+n} = f^{m}$.
\item For $m = 3$ and $n = 5$ that is the matrix
\[
\mathcal{M}_{x^{8}-x^{3}} = \begin{pmatrix}
\cdot & \cdot & \cdot & \cdot & \cdot & \cdot & \cdot & \cdot \\
1 & \cdot & \cdot & \cdot & \cdot & \cdot & \cdot & \cdot \\
\cdot & 1 & \cdot & \cdot & \cdot & \cdot & \cdot & \cdot \\
\cdot & \cdot & 1 & \cdot & \cdot & \cdot & \cdot & \cdot \\
\cdot & \cdot & \cdot & 1 & \cdot & \cdot & \cdot & \cdot \\
\cdot & \cdot & \cdot & \cdot & 1 & \cdot & \cdot & 1 \\
\cdot & \cdot & \cdot & \cdot & \cdot & 1 & \cdot & \cdot \\
\cdot & \cdot & \cdot & \cdot & \cdot & \cdot & 1 & \cdot
\end{pmatrix}
\]
which has $p(x) = x^{3+5}-x^{3}$ as its minimal polynomial, i.e. $\mathcal{M}_{p}^{3+5} = \mathcal{M}_{p}^{3}$. The endomorphism of
vector spaces can be restricted to an endomorphism of the set of standard basis elements $\{e_{1},\dots,e_{8}\}$,
\[
f : \{e_{1},\dots,e_{8}\} \rightarrow \{e_{1},\dots,e_{8}\}; e_{1} \mapsto e_{2}, e_{2} \mapsto e_{3},\dots, e_{7} \mapsto e_{8}, e_{8}
\mapsto e_{6}
\]
and satisfies $f^{3+5} = f^{3}$.
\end{enumerate}
\end{example}

With the $\sigma$ lemma \ref{la:sigma-lemma} as a tool in our toolbox, we can tackle the problem of the free category generated by a finite
quiver of $\mathrm{FinSets}$ with endomorphisms.\\
Recall definition \ref{def:path_algebra} of the path algebra $\Bbbk q$ of a quiver $q$, which is the
$\Bbbk$-vector space with basis set of all paths in $q$ and concatenation of composable paths as multiplication. By lemma \ref{la:cyclic_paths} one cyclic
path, i.e. an endomorphism, in $q$ is enough for there to be infinitely many paths, i.e. $\Bbbk q$ is infinite-dimensional.

Our algorithm can deal with endomorphism monoids that are explicitly cyclic.

\begin{definition}[Cyclic quiver]\label{def:cyclic_quiver}
We call a quiver $q$ with relations \ul{cyclic}, if it has at most one loop at each vertex and the endomorphism monoid of
each vertex in the category generated by $q$ is generated by the corresponding loop in the quiver. We call the
category generated by $q$ a category with \ul{explicitly cyclic endomorphism monoids}.
\end{definition}

We will give some examples and counterexamples of cyclic quivers and which arrow's existence violates the above condition.

\begin{example}{(Examples and counterexamples for cyclic quivers)}
\begin{enumerate}
\renewcommand{\labelenumi}{(\theenumi)}
\item A cyclic quiver

\[
\begin{minipage}{.10\textwidth}
\phantom{}
\end{minipage}
\begin{minipage}{.30\textwidth}
\begin{tikzcd}
{X} \arrow["a"', loop, distance=2em, in=305, out=235] \arrow[rr, "b"] &  & {Y} \arrow["c"', loop, distance=2em, in=305, out=235]
\end{tikzcd}
\end{minipage}
%
\begin{minipage}{.15\textwidth}
$\xrightarrow{\text{     }F\text{     }}$
\end{minipage}
%
\begin{minipage}{.35\textwidth}
\begin{tikzcd}
{X} \arrow["{a,a^{2},\dots}"', loop, distance=2em, in=305, out=235] \arrow[rr, "b", shift left] \arrow[rr, "{\begin{matrix} \text{$a^{m}bc^{n}$}
\\ \text{$(m,n\in\mathbb{N})$} \end{matrix}}"', shift right] &  & {Y} \arrow["{c,c^{2},\dots}"', loop, distance=2em, in=305, out=235]
\end{tikzcd}
\end{minipage}
%
\begin{minipage}{.10\textwidth}
\phantom{}
\end{minipage}
\]
\item This quiver is not cyclic.
\[
\begin{tikzcd}
X \arrow["a"', loop, distance=2em, in=305, out=235] \arrow[rr, "b", bend left] &  & Y \arrow["c"', loop, distance=2em, in=305, out=235] \arrow[ll, "d", bend left] & {} \arrow[r, "F"] & {} & X \arrow["{\begin{matrix}a,a^{2},\dots, \\ (bd),(bd)^{2},\dots, \\ a(bd),\dots \end{matrix}}"', loop, distance=2em, in=305, out=235] \arrow[rr, "{b,\dots}", bend left] &  & Y \arrow[ll, "{d,\dots}", bend left] \arrow["{\begin{matrix} c,c^{2},\dots, \\(db),(db)^{2},\dots, \\ (db)c(db),\dots \end{matrix}}"', loop, distance=2em, in=305, out=235]
\end{tikzcd}
\]
With the path $bd : X \rightarrow X$ we have two paths
$a, bd \in \mathrm{End}_{Fq}(X)$ and thus an endomorphism monoid that is generated by two generators $a$ and $bd$, and thus is
not explicitly cyclic.\endnote{The difference between the two quivers (1) and (2) is that the first one has some sense of order. Object $Y$ comes
\textit{after} $X$ in the sense that you have an arrow from $X$ to $Y$, but not the other way around. If $a$ and $c$ were the identity morphisms,
then (1) would be an example for a \ul{poset} or partially ordered set, i.e. a category such that for any pair of objects $x, y$ there is at most one
morphism from $x$ to $y$, and if there is a morphism from $x$ to $y$ and a morphism from $y$ to $x$, then $x = y$, i.e. the objects are equal.
An exercise for the reader could be to formulate the exact relations between posets and cyclic quivers.}
\item The first quiver from example \ref{ex:U-F-U-F_from_singleton} is cyclic:\\
\[
\begin{tikzcd}
\ast \arrow["a"', loop, distance=2em, in=305, out=235] & {} \arrow[r, "F"] & {} & \ast \arrow["a"', loop, distance=2em, in=305, out=235] \arrow["1_{\ast}"', loop, distance=2em, in=125, out=55] \arrow["{{a^{2}, a^{3},\dots}}"', loop, distance=2em, in=35, out=325]
\end{tikzcd}
\]
\item But after another forgetful functor $U$ the underlying quiver is not cyclic:\\
\[
\begin{tikzcd}
\ast \arrow["{a, b, c,\dots}"', loop, distance=2em, in=305, out=235] & {} \arrow[r, "F"] & {} & \ast \arrow["{a, b, c,\dots}"', loop, distance=2em, in=305, out=235] \arrow["1_{\ast}"', loop, distance=2em, in=125, out=55] \arrow["{a^{2},\dots,ab,\dots,ababbaba,\dots,b^{2},\dots,c^{2},\dots}"', loop, distance=2em, in=35, out=325]
\end{tikzcd}
\]
\end{enumerate}
\end{example}

\begin{definition}{(Ideal of an algebra)}\label{def:ideal_of_algebra}
Let $\mathcal{A}$ be a $\Bbbk$-algebra over a field $\Bbbk$.
A subset $L \subset \mathcal{A}$ is a \ul{left ideal} of $\mathcal{A}$, denoted $L \unlhd \mathcal{A}$, if for every $x, y \in L$,
$z \in \mathcal{A}$ and $c \in \Bbbk$ we have the following three statements:
\begin{align}
x + y &\in L\, \text{ ($L$ is closed under addition) },\label{eq:ideal_cl_add} \\
cx &\in L\, \text{ ($L$ is closed under scalar multiplication) }, \label{eq:ideal_cl_scm} \\
z \cdot x &\in L\, \text{ ($L$ is closed under left multiplication by arbitrary elements) }. \label{eq:ideal_cl_lm}
\end{align}
\noindent If \eqref{eq:ideal_cl_lm} were replaced with
\begin{align}
x \cdot z \in L\, \text{ ($L$ is closed under right multiplication by arbitrary elements) }, \label{eq:ideal_cl_rm}
\end{align}
then this would define a \ul{right ideal}.
A \ul{two-sided ideal} is a subset that is both a left and a right ideal.
\end{definition}

\begin{remark}{(Ideal of a unital algebra)}
If in \ref{def:ideal_of_algebra} $\mathcal{A}$ is a unital associative algebra with unit $e$, then we only need statements
(1) and (3). The statement (2) follows from (3).
\end{remark}
\begin{proof}
For $c \in \Bbbk$ and $x\in L$, $ce \in \mathcal{A}$ and with (3) we have $L \ni (ce)\cdot x = c(e\cdot x) = cx$, i.e. (2).
\end{proof}

\begin{remark}
Let $\mathcal{A}$ be a unital $\Bbbk$-algebra, and $\mathcal{I} \unlhd \mathcal{A}$ a two-sided ideal.
The \ul{quotient algebra} of$\mathcal{A}$ modulo $\mathcal{I}$ is the set
\[
\mathcal{A}\,/\,\mathcal{I} := \{ z + I : z \in \mathcal{A} \}
\]
of cosets $z + I$
together with the operations of addition, scalar multiplication and multiplication defined respectively for all
$z+I, w+I \in \mathcal{A}\,/\,\mathcal{I}$ by:
\begin{alignat}{3}
+ &: (z+I) + (w+I) &&:= (z+w)+I \\
\cdot &: \lambda \cdot (z+I) &&:= (\lambda z)+I \\
\cdot &: (z+I) \cdot (w+I) &&:= (zw)+I
\end{alignat}
These operations are independend of the representatives.
\end{remark}
\begin{proof}
Two cosets $(z+I), (z'+I)$ are equal iff $z-z' \in I$.
Let $z, z'\in z+I$ and $w, w' \in w+I$ be two representatives for $z+I$ and $w+I$ respectively, and let $\lambda \in \Bbbk$.
Therefore $z-z' \in I$ and $w-w' \in I$.
\begin{align}
(z+w) - (z'+w') = (z-z') + (w-w') \in I\, \text{ by \eqref{eq:ideal_cl_add} }\, \Rightarrow (z+w)+I = (z'+w')+I.\\
\lambda z - \lambda z' = \lambda(z-z') \in I\, \text{ by \eqref{eq:ideal_cl_scm} }\, \Rightarrow (\lambda z)+I = (\lambda z')+I.
\end{align}
For well-definedness of multiplication, let $z' = z + i_{1}$ and $w' = w + i_{2}$ with $i_{1}, i_{2} \in I$. Then
\begin{align}
zw - z'w' &= zw - (z+i_{1})(w+i_{2}) \\
&= zw - zw -zi_{2} -i_{1}w -i_{1}i_{2} \\
&= -zi_{2} -i_{1}w -i_{1}i_{2} \in I, \\
\Rightarrow zw+I &= z'w'+I
\end{align}
because of \eqref{eq:ideal_cl_add}, \eqref{eq:ideal_cl_scm} and the closure under left and right multiplication \eqref{eq:ideal_cl_lm} and
\eqref{eq:ideal_cl_rm}.
\end{proof}

\begin{remark}{(Relations of Endomorphisms)}
Let the generating quiver $q$ of our finite concrete category $\mathcal{C}$ be explicitly cyclic with one generating endomorphism
$\alpha_{i} : M_{i} \rightarrow M_{i}$ for each finite set $M_{i} \in \mathcal{C}_{0}, i = 1,\dots,N$ with $N := \abs{\mathcal{C}_{0}}$.

The algorithm $\mathtt{RelationsOfEndomorphism}$ calculates a list of equivalence classes under the equivalence relation
$\mathtt{IsCongruentForMorphisms}$, i.e. a list of pairs of paths $[\alpha^{m+n}, \alpha^{m}]$,
\begin{equation}
\mathtt{relsEndo} :=\label{eq:pairs_of_endos}
[ [\alpha_{1}^{m_{1}+n_{1}}, \alpha_{1}^{m_{1}}], \dots, [\alpha_{N}^{m_{N}+n_{N}}, \alpha_{N}^{m_{N}}] ].
\end{equation}
From the $\sigma$-lemma \ref{la:sigma-lemma} we know that each endomorphism relation is of the form
$\alpha_{i}^{m_{i}+n_{i}} = \alpha_{i}^{m_{i}}$ for some natural numbers $m_{i}, n_{i} \in \mathbb{N}_{0}$.
Especially with $n_{i} \geq 1$, the pairs in \eqref{eq:pairs_of_endos} have different exponents $m_{i}+n_{i} \neq m_{i}$.

With the list $\mathtt{relsEndo}$ of relations of endomorphisms that we got from our algorithm, we can calculate the
finitely presented category $\text{fp}\mathcal{C}$ from the quiver $q$ and the relations $\mathtt{relEndo}$,
\[
\text{fp}\mathcal{C} := \mathtt{Category}(\mathtt{q}, \mathtt{relsEndo}).
\]
Apart from the endomorphism relations, there are other, non-endomorphism paths that are equivalent under the relation
$\mathtt{IsCongruentForMorphisms}$. These can be identified in order to present the finitely presented category
$\text{fp}\mathcal{C}$ in a shorter form. However calculating the endomorphism relations was the more important task,
since these are responsible for the infinite hom-sets.
\end{remark}

\begin{algorithm}[H]\capstart
    \caption{\texttt{RelationsOfEndomorphisms}}\label{algo:RelationsOfEndomorphisms}
	\SetKwInput{Input}{~Input}
	\SetKwInput{Output}{~Output}
	\Input{~a finite concrete category $C$}
	\Output{~the endomorphism relations of the category $C$ given as list $\mathtt{relsEndo}$ of pairs of paths}
	\BlankLine
	$q := \mathtt{RightQuiverFromConcreteCategory}(C)$\;
	$kq := \mathtt{PathAlgebra}(k, q)$\;
	$gMor := \mathtt{SetOfGeneratingMorphisms}(C)$\;
	$A := \mathtt{Arrows}(q)$\;
	$relsEndo := \emptyset$\;
	\For{$i = 1, \dots, \mathtt{Length}(gMor)$}{
	    let $mor := gMor_i$\\
	    \If{\Not{$\mathtt{IsEndomorphism}(mor)$}}{
		continue\;
	    }
	    $m := 0$\;
	    $mPowers := \emptyset$\;
	    $foundEqual := \mathtt{false}$\;
	    \While{$mor^{m}\notin mPowers$}{
		let $n := 1$\;
	    	$nPowers := \emptyset$\;
		\While{\Not{} $foundEqual$ \AndAlg{} $mor^{(m+n)} \notin nPowers$}{
		    \If{$\mathtt{IsCongruentForMorphisms}(mor^{(m+n)}, mor^{m})$}{
		    	Add the relation $[kq.(A_{i})^{(m+n)}, kq.(A_{i})^{m}]$ to relsEndo\;
		    	$foundEqual := \mathtt{true}$\;
		    }
		    Add $mor^{(m+n)}$ to mpowers\;
		    n := n+1\;
		}
		Add $mor^{m}$ to mpowers\;
		m := m+1\;
	    }
	}
	\Return{$\mathtt{relsEndo}$}\;
\end{algorithm}

%% 
\newpage
\subsection{$\Bbbk$-linear categories, $\Bbbk$-algebroids and $\Bbbk$-algebras with orthogonal idempotents}

\begin{definition}[$\Bbbk$-linear category, $\Bbbk$-$\mathrm{algebroid}$]
Let $\Bbbk$ be a commutative unital ring. A \ul{$\Bbbk$-linear category}, also called \ul{$\Bbbk$-$\mathrm{algebroid}$},$\,\mathcal{A}$ is a
category where every hom-set is a $\Bbbk$-module, and where for $x,y,z \in \mathcal{A}_{0}$ composition of morphisms
\[
\mu : \mathrm{Hom}_{\mathcal{A}}(x,y) \times \mathrm{Hom}_{\mathcal{A}}(y,z) \rightarrow \mathrm{Hom}_{\mathcal{A}}(x,z)
\]
is $\Bbbk$-bilinear.

Note that this does imply that a $\Bbbk$-linear category is an Ab-category, but it need not be additive nor does it need to have a zero object.
\end{definition}

\begin{doctrine}[$\Bbbk$-algebroid]
The doctrine $\mathtt{IsAlgebroid}$ also known as\\
$\mathtt{IsLinearCategoryOverCommutativeRing}$ involves algorithms for $\mathtt{IsAbCategory}$ and
\begin{itemize}
\item $\mathtt{MultiplyWithElementOfCommutativeRingForMorphisms}$.
\end{itemize}
\end{doctrine}

\begin{definition}{($\Bbbk$-linear functor)}
Let $\Bbbk$ be a commutative ring. A \ul{functor of $\Bbbk$-linear categories} or a \ul{$\Bbbk$-linear functor} is a functor
$F : \mathcal{A} \rightarrow \mathcal{B}$ between $\Bbbk$-linear categories $\mathcal{A}$ and $\mathcal{B}$,
where for all objects $x, y \in \mathcal{A}_{0}$, the map
\[
F : \mathrm{Hom}_{\mathcal{A}}(x,y) \rightarrow \mathrm{Hom}_{\mathcal{B}}(F(x), F(y))
\]
is a homomorphism of $\Bbbk$-modules.
\end{definition}

\noindent We can associate to any category a $\Bbbk$-linear category called its $\Bbbk$-linear closure:

\begin{definition}{($\Bbbk$-linear closure of a category)}
Let $\mathcal{C}$ be a category, and $\Bbbk$ a commutative unital ring. We define $\Bbbk \mathcal{C}$ to be
the $\Bbbk$-algebroid with the same object set as $\mathcal{C}$ and with 
\[
\mathrm{Hom}_{\Bbbk\mathcal{C}}(a,b) = \bigoplus_{\varphi \in \mathrm{Hom}_{\mathcal{C}}(a,b)} \Bbbk \cdot \varphi,
\]
the free $\Bbbk$-module on the set of free generators $\mathrm{Hom}_{\mathcal{C}}(a,b)$. We call $\Bbbk\mathcal{C}$ the
\ul{$\Bbbk$-linear closure} of $\mathcal{C}$.
\end{definition}

\begin{definition}{(Idempotent)}\label{def:idempotent}
Let $\mathbf{A}$ be a unital algebra over the commutative ring $\Bbbk$.
\begin{enumerate}
\renewcommand{\labelenumi}{(\theenumi)}
\item An element $e\in \mathbf{A}$ is an \ul{idempotent} if $e^{2} = e$.
\item If $e_{1}, e_{2} \in \mathbf{A}$ are idempotents, then we will say that they are \ul{orthogonal} iff $e_{1}e_{2} = e_{2}e_{1} = 0$.
\item An idempotent is called \ul{trivial} if it is either $0$ or $1$.
\item An idempotent $e \in \mathbf{A}$ is called \ul{primitive} if $e = e_{1} + e_{2}$ with idempotents $e_{1}, e_{2}$, then
$e_{1}$ or $e_{2}$ is trivial.
\item A finite set $\{e_{1},\dots,e_{n}\}$ of orthogonal idempotents is called \ul{complete} if $\sum_{i=1}^{n} e_{i} = 1$.
\end{enumerate}
\end{definition}

\begin{proposition}\label{prop:idempotent}
Let $\mathbf{A}$ be a unital algebra over the commutative ring $\Bbbk$, and 
$M_{1},\dots,M_{n}$ $\mathbf{A}$-submodules of $\mathbf{A}$ such that
\begin{align}
\mathbf{A} = \bigoplus_{i=1}^{n} M_{i},
\end{align}
i.e.
\begin{align*}
\mathbf{A} &= \{ m_{1} + \dots + m_{n} : m_{i} \in M_{i} \}\, \text{ and }\\
M_{i} \cap M_{j} &= \{0\}\, \text{ if }\, i\neq j 
\end{align*}
Write $1 = \sum_{i=1}^{n} e_{i}$ for some $e_{i} \in M_{i}$. Then the $e_{i}$ are orthogonal idempotents in $\mathbf{A}$, and
\begin{align*}
M_{i} = \mathbf{A} e_{i},\, 1\leq i \leq n.
\end{align*}
Furthermore, $M_{i}$ is indecomposable as a module if and only if $e_{i}$ is primitive.
\end{proposition}
\begin{proof}[Proof\nopunct]
\begin{subproof}[that the $e_{i}$ are orthogonal idempotents]
\phantom{}\\
\begin{align*}
1 &= \sum_{i=1}^{n} e_{i}\\
e_{j} &= e_{j}\sum_{i=1}^{n} e_{i} = e_{j}^{2} + \sum_{i\neq j} e_{j} e_{i}
\end{align*}
Since we can write $e_{j} = m_{1} + \dots + m_{n}$ in a unique way with $m_{j} \in M_{j}$, it follows that
\begin{align*}
e_{j} = m_{j} = e_{j}^{2}\, \text{ and }\\
m_{i} = e_{j} e_{i} = 0\,\forall i \neq j,
\end{align*}
i.e. the $e_{j}$ are orthogonal idempotents in $\mathbf{A}$.
\end{subproof}
\begin{subproof}[Proof that $M_{i} = \mathbf{A}e_{i}$]\phantom{}\\
Since the $M_{i}$ are $\mathbf{A}$-submodules, they are already closed under multiplication by elements in $\mathbf{A}$.
So with $e_{i} \in M_{i}$ we have $ze_{i} \in M_{i}\, \forall z \in \mathbf{A}$, i.e. $\mathbf{A}e_{i} \subseteq M_{i}$.
We only have to show the other inclusion.
\setlist[description]{font=\normalfont}
\begin{description}
\item[``$\subseteq$:''] Let $x \in M_{i}$. We are done, if we show that $x = xe_{i}$. With $1 = \sum_{i=1}^{n} e_{i}$ we have
\begin{align*}
x &= x1 = x\sum_{j=1}^{n} e_{j} = xe_{i} + \sum_{j\neq i} xe_{j}
\end{align*}
By the above, we have $xe_{j} \in M_{j}$ for $j\neq i$, and since $x \in M_{i}$, we again have the unique sum $x = m_{1} + \dots + m_{n}$
with $m_{i} = xe_{i}$ and $m_{j} = xe_{j} = 0$ for $j \neq i$. Therefore $x = xe_{i}$, i.e. $M_{i} \subseteq \mathbf{A}e_{i}$.
\end{description}
\end{subproof}
\begin{subproof}[Proof of equivalence $M_{i}$ indecomposable $\Leftrightarrow$ $m_{i}$ primitive]\phantom{}\\
\setlist[description]{font=\normalfont}
\begin{description}
\item[``$\Rightarrow$:''] Let $M_{i}$ be indecomposable. Let $m_{i} = n + p$ for some idempotent $n, p \in \mathbf{A}$. Then
$M_{i} = m_{i}\mathbf{A} = (n+p)\mathbf{A} = \{nx + py : x,y \in \mathbf{A}\} = n\mathbf{A} \oplus p\mathbf{A}$. Since $M_{i}$ was
indecomposable, it follows that $n\mathbf{A} = \{0\} \vel p\mathbf{A} = \{0\}$. Therefore $n = 0 \vel p = 0$, i.e. $m_{i}$ is primitive.

\item[``$\Leftarrow$:''] Let $m_{i}$ be primitive. Let $m_{i}\mathbf{A} = M_{i} = N \oplus P$. Since
\begin{align*}
\mathbf{A} = \bigoplus_{j=1}^{n} M_{j} = \bigoplus_{\begin{smallmatrix}j=1,\\ j \neq i\end{smallmatrix}}^{n} M_{j} \oplus (N \oplus P)
\end{align*}
there exist orthogonal idempotent $n, p \in \mathbf{A}$ such that $N = n\mathbf{A}$ and $P = p\mathbf{A}$ and also
$1 = \sum_{j=1}^{n} m_{j} = \sum_{j\neq i} m_{j} + n + p$. Therefore $m_{i} = n + p$. With $m_{i}$ primitive, we have
$n = 0 \vel p = 0$, and therefore $M_{i} = \{0\} \oplus (p\mathbf{A}) \vel M_{i} = (n\mathbf{A}) \oplus \{0\}$, i.e. $M_{i}$ is indecomposable.
\end{description}
\end{subproof}
\end{proof}

A unital associative $\Bbbk$-algebra is a $\Bbbk$-algebroid with one object.
Let $\Bbbk$ be a commutative unital ring and $\mathbf{A}$ a unital associative algebra over $\Bbbk$. This defines a category $\mathcal{A}$
with a single object $\ast$ and the morphisms being the elements of the algebra, which are all endomorphisms since there is only one object.
Composition of morphisms is defined by the multiplication in $\mathbf{A}$, which is assumed to be associative.
The unit $e$ of the algebra acts as the identity morphism. The set $\{ e \}$ is trivially a set of orthogonal idempotents.
The hom-set $\mathrm{Hom}_{\mathcal{A}}(\ast,\ast) = \mathrm{End}_{\mathcal{A}}(\ast)$ is the whole algebra $\mathbf{A}$, which is
a $\Bbbk$ module with bilinear multiplication as composition, i.e. $\mathcal{A}$ is a $\kAlgebroid$.

\begin{proposition}\label{prop:Alg-Alg-Correspondence}
Each $\Bbbk$-algebra $\mathbf{A}$ with $\{e_{1},\dots,e_{n}\}$ a finite complete system of (not necessarily primitive) orthogonal
idempotents defines a $\Bbbk$-algebroid $\mathcal{A}$ with object set $\mathcal{A}_{0} := \{1,\dots,n\}$. More precisely:
\begin{enumerate}
\renewcommand{\labelenumi}{(\theenumi)}
\item The set of morphisms between two objects $i,j$ is defined as
$\mathrm{Hom}_{\mathcal{A}}(i,j) := e_{i}\mathbf{A}e_{j}$, i.e. a morphism is an element
$e_{i}\mathbf{A}e_{j} \ni \alpha = e_{i}ae_{j}, a\in \mathbf{A}$.
It follows that the identity morphism $1_{i} = e_{i}$ for $1\leq i \leq n$.

\item $\mathbf{A} := \mathrm{Algebra}(\mathcal{A}) := \bigoplus_{i,j} \mathrm{Hom}_{\mathcal{A}}(i,j)$ is again a unital algebra with
multiplication 
\[
\varphi_{i,j} \cdot \psi_{k,l} := \begin{cases}\varphi_{i,j} \psi_{k,l} & \text{ if } j = k\\
0 \in \mathbf{A} & \text{ if } j \neq k, \end{cases}
\]
where the first case is the composition in the algebroid. It follows that $1 = \sum_{i} 1_{i}$ is the multiplicative identity of $\mathbf{A}$.

\end{enumerate}
These two constructions are mutually inverse.
\end{proposition}
\begin{proof}
\begin{enumerate}
\renewcommand{\labelenumi}{(\theenumi)}
\item We can write $\mathbf{A}$ as
\[
\mathbf{A} = 1\mathbf{A}1 = \left(\sum_{i}e_{i}\right) \mathbf{A} \left(\sum_{j} e_{j}\right) = \bigoplus_{i,j} e_{i}\mathbf{A}e_{j},
\]
which is a direct sum decomposition of $\mathbf{A}$ in $\Bbbk$-submodules. The last sum is a direct sum:
\begin{align}
e_{i}\mathbf{A}e_{j} &\cap e_{i'}\mathbf{A}e_{j'} = \delta_{i,i'} \cdot \delta_{j,j'},\,\text{ since }\\
x &= e_{i} \cdot a \cdot e_{j} = e_{i'} \cdot b \cdot e_{j'} \\
x &= \underbrace{e_{i}e_{i}}_{e_{i}} \cdot a \cdot e_{j} = \underbrace{e_{i}e_{i'}}_{= 0} \cdot b \cdot e_{j'},\, &e_{i} \neq e_{i'} \\
x &= e_{i} \cdot a \cdot \underbrace{e_{j}e_{j}}_{e_{j}} = e_{i'} \cdot b \cdot \underbrace{e_{j'}e_{j}}_{= 0},\, &e_{j} \neq e_{j'}
\end{align}

\item Composition of morphisms
$i \xrightarrow{\alpha} j \xrightarrow{\beta} k$ is defined by the multiplication in $\mathbf{A}$.
Let $\alpha = e_{i}ae_{j}, \beta = e_{j}be_{k}$ with $a, b\in \mathbf{A}$. Then
\begin{align}
\alpha \beta &= (e_{i}ae_{j})(e_{j}be_{k}) \\
&= e_{i}(ae_{j}b)e_{k} \\
&= e_{i}ce_{k}\,\text{ with }\, c := ae_{j}b \in \mathbf{A}.
\end{align}
Associativity of composition follows from associativity of multiplication in $\mathbf{A}$.

\item It's an easy exercise to see that the above defined $1$ is indeed a multiplicative unit of the algebra $\mathbf{A}$.
We define $M_{i} := \bigoplus_{j=1}^{n} \mathrm{Hom}_{\mathcal{A}}(j,i)$ an $\mathbf{A}$-submodule of $\mathbf{A}$.
Then $\mathbf{A} = \bigoplus_{i=1}^{n} M_{i}$ as in Proposition \ref{prop:idempotent} and it follows that the idempotent $e_{i} = 1_{i}$.
\end{enumerate}
\end{proof}

\begin{definition}{(Path algebra of a quiver)}\label{def:path_algebra}
We define the \ul{path algebra} of a quiver $q$ over a commutative ring $\Bbbk$ by
\[
\Bbbk q := \mathrm{Algebra}( \Bbbk Fq ),
\]
where $Fq$ is the free category of the quiver $q$ from definition \ref{def:free_category}.
\end{definition}

\begin{lemma}\label{la:path_algebra_is_ass_algebra}
For a quiver $q$ and a field $\Bbbk$, the path algebra $\Bbbk q$ is an associative $\Bbbk$-algebra.\endnote{From \cite{[leit4]} 4.1}
\end{lemma}
\begin{proof}
Let $w, w', w''$ be paths. Then both $(ww')w''$ and $w(w'w'')$ are the concatenation of $w$ on the left,
$w'$ in the middle and $w''$ on the right, in case both conditions $t(w) = s(w')$ and $t(w') = s(w'')$ are satisfied, and
otherwise the zero element (since $(ww')0 = 0, 0(w'w'') = 0$, according to bilinearity).\\
Since the multiplication was defined on a basis and extended bilinearly, the axioms of an algebra are clearly satisfied.
\end{proof}

\begin{lemma}\label{la:unit_in_path_algebra}
If the set of vertices of a quiver $q_{0}$ is finite, then $\Bbbk q$ has a unit element $\sum_{x\in q_{0}} e_{x}$. In this case, $\Bbbk q$ is a unital
associative algebra, i.e. a unital ring that is also a vector space.\endnote{From \cite{[leit4]} 4.1}
\end{lemma}
\begin{proof}
Let $e := \sum_{x\in q_{0}} e_{x}$. Let $w$ be a path with $s(w) = x$ and $t(w) = y$, then $e_{x}w = w$ and $e_{z}w = 0$ for all $z \neq x$,
thus $ew = e_{x}w + \sum_{z\neq x} e_{z}w = w + 0 = w$. Similarly, $we_{y} = w$ and $we_{z} = 0$ for $z \neq x$.
\end{proof}

\begin{computation}[A finitely presented category isomorphic to the finite concrete category $C_{3}C_{3}C_{3}$]\phantom{}\\
\begin{Verbatim}[commandchars=!@|,fontsize=\small,frame=single,label=Example]
  !gapprompt@gap>| !gapinput@c3c3c3 := ConcreteCategoryForCAP(|
  !gapprompt@>| !gapinput@                  [ [2,3,1], [4,5,6], [,,,5,6,4],|
  !gapprompt@>| !gapinput@                    [,,,7,8,9], [,,,,,,8,9,7], [7,8,9] ] );|
  A finite concrete category
  !gapprompt@gap>| !gapinput@objects := SetOfObjects( c3c3c3 );|
  [ An object in subcategory given by: <An object in FinSets>,
    An object in subcategory given by: <An object in FinSets>,
    An object in subcategory given by: <An object in FinSets> ]
  !gapprompt@gap>| !gapinput@gmorphisms := SetOfGeneratingMorphisms( c3c3c3 );|
  [ A morphism in subcategory given by: <A morphism in FinSets>,
    A morphism in subcategory given by: <A morphism in FinSets>,
    A morphism in subcategory given by: <A morphism in FinSets>,
    A morphism in subcategory given by: <A morphism in FinSets>,
    A morphism in subcategory given by: <A morphism in FinSets>,
    A morphism in subcategory given by: <A morphism in FinSets> ]
  !gapprompt@gap>| !gapinput@q := RightQuiverFromConcreteCategory( c3c3c3 );|
  q(3)[a:1->1,b:1->2,c:2->2,d:2->3,e:3->3,f:1->3]
  !gapprompt@gap>| !gapinput@relEndo := RelationsOfEndomorphisms( c3c3c3 );|
  [ [ (a*a*a), (1) ], [ (c*c*c), (2) ], [ (e*e*e), (3) ] ]
  !gapprompt@gap>| !gapinput@C := AsFpCategory( c3c3c3 );|
  Category generated by the right quiver
  q(3)[a:1->1,b:1->2,c:2->2,d:2->3,e:3->3,f:1->3] with relations
\end{Verbatim}
The underlying quiver algebra of $\mathcal{C}$ is a quotient algebra by the following relations. It also has a finite
set of vertices, thus it has a unit.
\begin{Verbatim}[commandchars=!@|,fontsize=\small,frame=single,label=Example]
  !gapprompt@gap>| !gapinput@A := UnderlyingQuiverAlgebra( C );|
  (Q * q) / [ 1*(a*a*a) - 1*(1), 1*(c*c*c) - 1*(2), 1*(e*e*e) - 1*(3),
  1*(b*c) - 1*(a*b), 1*(b*d) - 1*(f), 1*(f*e) - 1*(a*f), 1*(d*e) - 1*(c*d) ]
  !gapprompt@gap>| !gapinput@unit := A.1 + A.2 + A.3;|
  { 1*(3) + 1*(2) + 1*(1) }
  !gapprompt@gap>| !gapinput@unit * A.a = A.a;|
  true
  !gapprompt@gap>| !gapinput@A.f * unit = A.f;|
  true
\end{Verbatim}

\begin{minipage}{0.40\textwidth}
\[
\begin{tikzcd}
{\{1,2,3\}} \arrow["\begin{pmatrix} 1\mapsto 2 \\ 2\mapsto 3\\ 3\mapsto 1\end{pmatrix}"', loop, distance=2em, in=125, out=55] \arrow[rr, "\begin{pmatrix} 1\mapsto 4 \\ 2\mapsto 5\\ 3\mapsto 6\end{pmatrix}"] \arrow[rd, "\begin{pmatrix} 1\mapsto 7 \\ 2\mapsto 8\\ 3\mapsto 9\end{pmatrix}"'] &                                                                                                                                & {\{4,5,6\}} \arrow["\begin{pmatrix} 4\mapsto 5 \\ 5\mapsto 6\\ 6\mapsto 4\end{pmatrix}"', loop, distance=2em, in=125, out=55] \arrow[ld, "\begin{pmatrix} 4\mapsto 7 \\ 5\mapsto 8\\ 6\mapsto 9\end{pmatrix}"] \\
                                                                                                                                                                                                                                                                                                 & {\{7,8,9\}} \arrow["\begin{pmatrix} 7\mapsto 8 \\ 8\mapsto 9\\ 9\mapsto 7\end{pmatrix}"', loop, distance=2em, in=305, out=235] &                                                                                                                                                                                                               
\end{tikzcd}
\]
\end{minipage}
\begin{minipage}{0.05\textwidth}
\phantom{}
\end{minipage}
\begin{minipage}{0.10\textwidth}
\[
\begin{tikzcd}
{} \arrow[rr, "\mathtt{q}"] &  & {}
\end{tikzcd}
\]
\end{minipage}
\begin{minipage}{0.02\textwidth}
\phantom{}
\end{minipage}
\begin{minipage}{0.43\textwidth}
\[
\begin{tikzcd}
1 \arrow["a"', loop, distance=2em, in=125, out=55] \arrow[rr, "b"] \arrow[rd, "f"'] &                                                     & 2 \arrow["c"', loop, distance=2em, in=125, out=55] \arrow[ld, "d"] \\
                                                                                    & 3 \arrow["e"', loop, distance=2em, in=305, out=235] &                                                                   
\end{tikzcd}
\]
\end{minipage}
\end{computation}









\section{Finite-dimensional Linear Categories, Algebroids and Idempotents}
% mainfile: ../main.tex

\subsection{Additional structure on the Hom-set of a category}

....

\begin{example}{(Group as a category)}\\
\noindent A group $\mathbf{G}$ defines a category $\mathcal{B}\mathbf{G}$ with a single object $\ast$. The group elements are its morphisms, which are
all automorphisms (i.e. bijective endomorphisms) of the single object. Composition of morphisms is defined by the binary group operation.
The identity element $e \in G$ acts as the identity morphism for the unique object in this category. The hom-set of that category is itself
a group.
\end{example}

This example can be generalized to categories where the hom-set is a ring or an R-algebra. But for this we need a commutative ring R.

Our goal is to represent finite concrete categories, for this we need the source and target categories of our functors, which the
representations are.
As subcategories of $\textup{FinSets}$, our finite concrete categories only have definitions for their objects and their
morphisms, methods to check when two morphisms are congruent or equivalent, but not much else.
A competing theory to category theory is that of quivers and path algebras. We already used their terminology in
\ref{def:path}, \ref{la:cyclic_paths} and \ref{def:path_algebra}, for instance when talking about the trivial path,
which in the language of category theory is nothing but the identity morphism, composition of arrows to a path is nothing but
composition of morphisms (if you make the path explicit by writing a new arrow for every path).

So what we called a path algebra in \ref{def:path_algebra} is a different data structure for a category. 
For one, the path algebra is an algebra, i.e. a vector space with additional structure, and thus a single set, comparable to the
class of morphisms $\mathcal{C}_{1}$ of a category $\mathcal{C}$.
But as it is an algebra, it not only contains the generating morphisms of the category, but also $\Bbbk$-linear combinations of
morphisms and paths. This is what our concrete categories lack, and what additional structure we have to give them in order
to represent them by matrices.

In practise, there is already developed software for \ul{q}uivers and \ul{p}ath \ul{a}lgebras, namely the \textsc{Gap} package
\textsc{QPA$2$}\endnote{(see \cite{[QPA2]})}.
What we are actually doing to represent finite concrete categories, is going from $\mathcal{C} \in \mathbf{Cats}$ to $q \in \mathbf{Quiv}$,
in theory by \ul{forgetting} (see \ref{ex:forgetful_functor}) the category concepts of identity morphism and composition, in practise by calculating the
underlying quiver $q$, and then for a commutative ring $\Bbbk$, constructing the path algebra $\Bbbk q$. In this step the path algebra
is infinite-dimensional, since there are infinitely many paths according to lemma \ref{la:cyclic_paths}, and \textsc{QPA$2$}'s function
\texttt{BasisPathsBetweenVertices} only works for finite-dimensional path algebras. Thus in a next step we have to provide
additional data in the form of generators of ideals of the path algebra, by which we can divide and build the quotient path algebra,
which is then finite-dimensional. This is the purpose of \texttt{RelationsOfEndomorphisms}.

Once we have a finite-dimensional path algebra $\Bbbk q$, we let \textsc{QPA$2$} calculate generators of the non-endomorphism relations,
and when we have a complete set of relations, that will be our definitive quotient quiver algebra $\Bbbk q$, which we then take it back into the category
theoretical context by constructing the $\Bbbk$-\textbf{Algebroid} $\mathcal{A}$ from the path algebra $\Bbbk q$.

The source category for our representation is then the $\Bbbk$-\textbf{Algebroid} $\mathcal{A}$ and not anymore our finite concrete
category $\mathcal{C}$, but it behaves in the same way regarding composition of morphisms and which morphisms are congruent.

The target category of our category representations will be $\Bbbk$-\textbf{Mat} which we will describe in the next section,
especially all the nice properties $\Bbbk$-\textbf{Mat} has, and how they get carried over to our functor category with $\Bbbk$-\textbf{Mat} as
target.\endnote{
In \cite{[Ab-Cat]}, Posur used the equivalence between categories $\textup{mat}_{\Bbbk} \cong \textup{vec}^{\text{fd}}_{\Bbbk}$,
as described in \cite{[context]}, \textsc{Example} 1.5.6 on page 30 (48/258), to justify that $\Bbbk$-\textbf{Mat} is a good
\textbf{computational model} to
%\setquotestyle[guillemets]{english} don't do that!
\blockquote{transform otherwise inaccessible mathematical objects into computationally easily graspable entities}
\setquotestyle{default}, which is what we are doing with \textbf{CatReps}.
}

With source and target categories defined, the category where our category representations lie in is \textbf{CatReps} for which we
show that it's a subcategory of the \textbf{Functor Category}. And even more in the next section.
\[
\mathbf{CatReps_{\mathcal{C}}} = \textup{Hom}(\Bbbk\mathbf{-Algebroid_{\mathcal{C}}}, \kmat)
\]

\subsection{Generating morphisms of a category and the underlying quiver}

$\textup{gmorphisms} := \{g_{1},\dots,g_{r}\} \rightarrow$ concrete category with set of generating morphisms $\textup{gmorphisms}$.

This is the $\textup{Free}$ functor from $\mathbf{Quiv}$ to $\mathbf{Cat}$, taking a quiver and adding the missing structure of
identity morphisms and composition of arrows to that category. The result is a category.

The $\textup{forgetful}$ functor from $\mathbf{Cat}$ to $\mathbf{Quiv}$ is going the other way around and leaves all
morphisms that we now have in the category, but forgets their relations, what was identity, what was composition.

Given a field $\Bbbk$, we have the path algebra $\Bbbk q$ with all the arrows as a basis.

Given relations on endomorphisms and on the other morphisms, we make the quotient path algebra.

This is already a category, and now it has more structure.

\subsection{Ab-categories}

\begin{definition}{(Ab-category)}
An \ul{Ab-category} is a category in which all homomorphism sets are abelian groups, and composition distributes over addition.\\
In other words, a category $\mathcal{C}$ is an \ul{Ab-category} if for every pair of objects $M,N \in \mathcal{C}_{0}$,
$( \textup{Hom}_{\mathcal{C}}(M,N), + )$ is an abelian group (with the neutral element called \ul{zero morphism}),
and for all morphisms $\gamma, \delta \in \textup{Hom}_{\mathcal{C}}(M,N),
\alpha, \beta \in \textup{Hom}_{\mathcal{C}}(N,L)$
\begin{align}\label{eq:dist}
(\gamma + \delta)\alpha &= \gamma\alpha + \delta\alpha \textup{ and }\\
\gamma(\alpha+\beta) &= \gamma\alpha + \gamma\beta.
\end{align}
Note that every hom-set has its own unique zero morphism. E.g. in $\textup{Mat}_{\mathbb{Q}}$ the $2 \times 3$ zero-matrix
$\mathbf{0} \in \textup{Hom}(2,3)$ is different from the $4 \times 4$ zero-matrix $\mathbf{0} \in \textup{Hom}(4,4)$.
\end{definition}

\begin{definition}{(semisimple)}
A ring R is semisimple if ...
\end{definition}

\begin{example}{(The matrix category $\Rmat$ over a commutative ring $R$)}\label{ex:matrix_category}
\begin{itemize}
\item Objects are natural numbers $\textup{Obj}(\textup{Mat}_{R}) = \mathbb{N} = \mathbb{N}_{0} = \{0,1,2,\dots\}$
\item Morphisms $\textup{Mor}(\textup{Mat}_{R}) \ni (m \rightarrow n)$ are $m \times n$ matrices over $R$.
We write the set of morphisms between $m$ and $n$, as $R^{m\times n} := \textup{Hom}(m,n)$. Identity morphisms are the
identity matrices.
\item Composition is matrix multiplication (associative).
\item It is a skeletal category, i.e. $m$ is isomorphic to $n \Rightarrow m = n$. Only quadratic matrices ($m = n$) can be
isomorphisms.
\end{itemize}
In this category, the number $0$ is \ul{the} zero object.\\
A zero matrix (zero morphism) is a matrix factoring through the zero object $0$.\\
\begin{minipage}{.2\textwidth}\phantom{ }\end{minipage}
\begin{minipage}{.25\textwidth}
Matrix $R^{m\times n} \ni A = 0$
\end{minipage}
\begin{minipage}{.08\textwidth}
$\Longleftrightarrow$
\end{minipage}
\begin{minipage}{.32\textwidth}
\begin{tikzcd}
m \arrow[rr, "A"] \arrow[rd, "(m \times 0)"'] &                               & n \\
                                              & 0 \arrow[ru, "(0 \times n)"'] &  
\end{tikzcd}\\
$\Rightarrow A = (m \times 0) \cdot (0 \times n)$.
\end{minipage}
\begin{minipage}{.15\textwidth}\phantom{ }\end{minipage}\\
\noindent The ``matrices'' $(m \times 0)$ and $(0 \times n)$ have zero columns or zero rows respectively, but it is
important to note that for each $m \in \textup{Obj}(\textup{Mat}_{R})$ there is exactly one such matrix $(m \times 0)$ and $(0 \times m)$
(that's what initial and terminal object means), and for different $m$, these morphisms are different.
\end{example}


\begin{example}{($\kmat$ is an Ab-category)}
For two natural numbers $m,n \in {\kmat}_{0} = \mathbb{N} = \mathbb{N}_{0}$, the set of morphisms with source $m$ and target $n$ is
$\Bbbk^{m\times n}$, the set of $m \times n$-matrices. This is an abelian group:
\begin{itemize}
\item The neutral element of the addition is the $m \times n$ zero matrix $\mathbf{0}$.
\item Addition of matrices is associative and commutative, so it's an abelian group.
\end{itemize}
The distributive laws \eqref{eq:dist} for composable morphisms hold.
\end{example}









\begin{definition}{(Abelian category)}\endnote{(From \cite{[context]}, appendix E.5, Def. E.5.1)}
A category $\mathcal{C}$ is \ul{abelian} if
\begin{itemize}
\item it has a \ul{zero object} $0$, that is both initial and terminal,
\item it has all \ul{binary products} and \ul{binary coproducts},
\item it has all \ul{kernels} and \ul{cokernels}, defined repsectively to be the \ul{equalizer} and
\ul{coequalizer} of a map $f : A \rightarrow B$ with the zero map $A \rightarrow 0 \rightarrow B$, and
\item all monomorphisms and epimorphisms arise as kernels or cokernels, respectively.
\end{itemize}
\end{definition}

\begin{definition}{($R$-linear category)}
Let $R$ be a commutative ring.
\end{definition}

For $R = \mathbb{Z}$ an $R$-linear category is nothing but an Ab-category.


\begin{definition}
Once source and target categories $\mathcal{C}, \mathcal{D}$ are both $R$-linear categories we define the functor category
$\mathrm{Hom_{R}}(\mathcal{C},\mathcal{D})$ the subcategory of $R$-linear functors.
\end{definition}





\section{The functor category with values in an abelian category is an abelian category}
\label{sect:abelian_cat}
Goal is to show that

The functor category with values in $\kmat$ is an Abelian category with enough projectives (constructively) and direct sum decomposition (constructively).


\begin{definition}[The functor category $\HomAkmat$]
Let $\Bbbk$ be a commutative ring, and let $\mathcal{A}$ be a finite-dimensional algebroid over $\Bbbk$.\\
The \ul{functor category} $\HomAkmat$ has as objects $\Bbbk$-linear functors $F : \mathcal{A} \rightarrow \kmat \in \HomAkmat_{0}$, where each
object $i \in \mathcal{A}$ gets mapped to a natural number $F(i) \in \kmat_{0} = \mathbb{N}_{0}$, and each
morphism $b : i \rightarrow j \in \mathcal{A}_{1}$ gets mapped to an $F(i) \times F(j)$ matrix $F(b) : F(i) \rightarrow F(j) \in \kmat_{1}$.
A morphism $\alpha : F \rightarrow G \in \HomAkmat_{1}$ is a natural transformation $\alpha : F \Rightarrow G$ with each component
$\alpha_{i} : F(i) \rightarrow G(i) \in \kmat_{1}$ for an object $i \in \mathcal{A}$ being a $F(i) \times G(i)$ matrix satisfying the
naturality conditions in \eqref{eq:naturality_condition}.
\end{definition}

\begin{example}[Limits and colimits in $\kmat$]
In our target category $\kmat$ which is an abelian category, we have the following limits, colimits and special bilimts:
\begin{itemize}
\item A zero object $0 \in \kmat_{0}$
\item For every finite set $I$ and family $\{S_{i}\}_{i \in I}$ we have the direct sum
\[
(S = \bigoplus_{i\in I} S_{i}, \{\pi_{i} : S \rightarrow S_{i}\}, \{\iota_{i} : S_{i} \rightarrow S\}, u_{\text{in}}(\tau), u_{\text{out}}(\tau))
\]
\item An equalizer of the diagram 
\[
\begin{tikzcd}
A \arrow[rr, "{0_{A,B}}"', shift right] \arrow[rd] \arrow[rr, "\alpha", shift left] &              & B \\
                                                                               & 0 \arrow[ru] &  
\end{tikzcd}
\]
called kernel, i.e. for the matrix $\alpha : A \rightarrow B$ we have
\[
(K = \mathrm{ker}(\alpha), \mathrm{KernelEmbedding}(\alpha), \mathrm{KernelLift}(\alpha,\tau))
\]

\item A co-equalizer of the same diagram called cokernel, i.e. for the matrix $A \in \Bbbk^{m\times n}$ we have
\[
(C = \mathrm{coker}(\alpha), \mathrm{CokernelProjection}(\alpha), \mathrm{CokernelColift}(\alpha,\tau))
\]



\end{itemize}
\end{example}

\begin{example}[Lift along mono and colift along epi]
In an abelian category there exist lifts along monos and colifts along epis, more precisely there exist two dependent functions:
\begin{itemize}
\item A dependent function $( - / - )$ mapping a pair $\iota : K \rightarrow A, \tau : T \rightarrow A$ to a lift $(\tau / \iota )$ of
$\tau$ along $\iota$, where $A, K, T \in \kmat_{0}$, 

$\iota$ is a monomorphism and 

$\tau\,\mathrm{CokernelProjection}(\iota) = 0$.

\item A dependent function $( - \backslash - )$ mapping a pair $\varepsilon : A \rightarrow C, \tau : A \rightarrow T$ to a colift
$(\varepsilon \backslash \tau )$ of
$\tau$ along $\varepsilon$, where $A, C, T \in \kmat_{0}$,

$\varepsilon$ is an epimorphism and

$\mathrm{KernelEmbedding}(\varepsilon)\tau = 0$.

\end{itemize}
Lifts along monos and colifts along epis are unique.
\end{example}

\begin{remark}
The existence of lifts along monos and colifts along epis is equivalent to \ref{def:abelian_category}.
\end{remark}
\begin{proof}

Each mono is a kernel embedding, each epi is cokernel projection.
\end{proof}


Our category $\HomAkmat$ has all of these limits.

\subsection{$\HomAkmat$ is an abelian category}

\begin{theorem}\label{thm:functor_category_abelian}
Let $\mathcal{A}$ be a finite-dimensional algebroid over some field $\Bbbk$. The functor category $\HomAkmat$
is an abelian category.
\end{theorem}
\begin{proof}
We prove that $\HomAkmat$ is an Ab-category, then that it is also an additive category and a pre-abelian category and finally that
it is an abelian category.
\begin{enumerate}
\renewcommand{\labelenumi}{(\theenumi)}
\item In order to show that $\HomAkmat$ is an Ab-category, we must show that 
\begin{enumerate}
\renewcommand{\labelenumii}{(\roman{enumii})}
\item for any two objects $F,G \in \HomAkmat_{0}$ the hom-set $\mathrm{Hom}_{\HomAkmat}(F,G)$ between them is an Abelian group, and
\item that the composition of two morphisms\\
$\mu : \mathrm{Hom}_{\HomAkmat}(F,G) \times \mathrm{Hom}_{\HomAkmat}(G,H) \rightarrow \mathrm{Hom}_{\HomAkmat}(F,H)$ is a
bilinear map, i.e. for $\eta, \varepsilon \in \mathrm{Hom}_{\HomAkmat}(F,G)$ and $\varphi, \psi \in \mathrm{Hom}_{\HomAkmat}(G,H)$ and
for $x \in \Bbbk$
\begin{align}
(\eta + \varepsilon)\varphi &= \eta\varphi + \varepsilon\varphi \\
\eta(\varphi + \psi) &= \eta\varphi + \eta\psi \\
(x\eta)\varphi &= \eta(x\varphi) = x(\eta\varphi).
\end{align}
\end{enumerate}

\begin{subproof}[Proof of (i)]
For any object $c \in \mathcal{A}$, the set of components $\mathrm{Hom}_{\kmat}(Fc,Gc)$ is a $\Bbbk$-vector space and therefore an
Abelian group.

We define the addition and scalar multiplication
\begin{align*}
+ :&& \mathrm{Hom}_{\HomAkmat}(F,G) &\times \mathrm{Hom}_{\HomAkmat}(F,G) \rightarrow \mathrm{Hom}_{\HomAkmat}(F,G)\\
\cdot :&& \Bbbk &\times \mathrm{Hom}_{\HomAkmat}(F,G) \rightarrow \mathrm{Hom}_{\HomAkmat}(F,G)
\end{align*}
component-wise: $\forall \eta, \varepsilon \in \mathrm{Hom}_{\HomAkmat}(F,G), \forall x \in \Bbbk, \forall c \in \mathcal{A}$
\begin{align}
(\eta+\varepsilon)_{c} &:= \eta_{c} + \varepsilon_{c}\\
(x \eta)_{c} &:= x\eta_{c}
\end{align}
where the right-hand side is the usual addition and scalar multiplication of matrices.

We identify as the neutral element $0_{F,G} \in \mathrm{Hom}_{\HomAkmat}(F,G)$ (or simply $0$ when the context is clear)
the natural transformation $0$ with each component $0_{c} = 0_{Fc,Gc}$ being the $Fc\times Gc$ zero matrix.
For each natural transformation $\eta$ the additive inverse $-\eta$ is defined component-wise as
\begin{align}
(-\eta)_{c} &:= -\eta_{c}.
\end{align}
We also confirm that the addition is commutative:
\begin{align}
(\eta+\varepsilon)_{c} &= \eta_{c} + \varepsilon_{c}\\
    &= \varepsilon_{c} + \eta_{c}\\
    &= (\varepsilon + \eta)_{c}
\end{align}
This concludes that for each $F, G \in \HomAkmat,\, \mathrm{Hom}_{\HomAkmat}(F,G)$ is an Abelian group.
\end{subproof}
\begin{subproof}[Proof of (ii)]
Let $F, G, H \in \HomAkmat$ and let $\eta, \varepsilon \in \mathrm{Hom}_{\HomAkmat}(F,G)$,
$\varphi, \psi \in \mathrm{Hom}_{\HomAkmat}(G,H)$ and $x \in \Bbbk$.\\
The composition $\eta\varphi \in \mathrm{Hom}_{\HomAkmat}(F,H)$ is defined by component-wise matrix-multiplication,
i.e. $\forall c \in \mathcal{A}$
\begin{align*}
(\eta\varphi)_{c} := \eta_{c}\varphi_{c}
\end{align*}
and from this follows the bilinearity of the composition, since the matrix multiplication is bilinear.\\
This concludes the first part of the proof, i.e. $\HomAkmat$ is an Ab-category.
\end{subproof}

\item Next we show that $\HomAkmat$ is an additive category, i.e. it is
\begin{enumerate}
\renewcommand{\labelenumii}{(\roman{enumii})}
\item An Ab-category with
\item A dependent function $\oplus$ mapping a finite set $I$ and a collection $(F_{i})_{i\in I}$ of objects in $\HomAkmat$
to a corresponding direct sum $( \oplus_{i\in I} F_{i}, (\pi_{i})_{i\in I}, (\iota_{i})_{i\in I}, u_{\mathrm{in}}, u_{\mathrm{out}} )$.
\end{enumerate}
\begin{subproof}[Proof of (ii)]
We will make extensive use of the direct sum in $\kmat$ as described in \ref{ex:kmat_additive} and \ref{ex:block_diagonal_matrix} to
define the direct sum in $\HomAkmat$.
Let $I = \{1,\dots,N\}$ be a finite set, and $\{F_{i} \}_{i \in I}$ a family of objects in $\HomAkmat_{0}$.
\begin{itemize}
\item The object $F := \oplus_{i \in I} F_{i}$ is a functor defined by its image on objects $c \in \mathcal{A}_{0}$ and its
image on morphisms $a : c \rightarrow c' \in \mathcal{A}_{1}$ in the following way:
\begin{itemize}
\item For an object $c \in \mathcal{A}_{0}$ we have a family $\{F_{i} c \}_{i \in I}$ of objects in $\kmat_{0}$ for which we can define
their direct sum $\oplus_{i \in I} F_{i} c = \sum_{i \in I} F_{i} c =: F c$.
\item The projections $\pi_{i} : F \rightarrow F_{i}$ and coprojections $\iota_{i} : F_{i} \rightarrow F$ are defined component-wise
exactly as in \eqref{eq:projection_direct_sum_matrix} and \eqref{eq:coprojection_direct_sum_matrix}, i.e.
\begin{align}
(\pi_{i})_{c} =
\begin{pmatrix}
0_{\sum_{j=1}^{i-1} F_{j}(c), F_{i}(c)} \\
1_{F_{i}(c)} \\
0_{\sum_{j=i+1}^{N} F_{j}(c), F_{i}(c)}
\end{pmatrix} \\
(\iota_{i})_{c} = 
\begin{pmatrix}
0_{F_{i}(c),\,\sum\limits_{j=1}^{i-1} F_{j}(c)} & 1_{F_{i}(c)} & 0_{F_{i}(c),\,\sum\limits_{j=i+1}^{N} F_{j}(c)}
\end{pmatrix}
\end{align}

We verify the property of a direct sum, that

\begin{align}
\sum_{i \in I} (\pi_{i}) (\iota_{i}) &= 1_{F}\,\text{ and } \\
(\iota_{i}) (\pi_{j}) &= (\delta_{i,j}) = \begin{cases}
1_{F_{i}} & \text{ if } i = j \\
0_{F_{i}, F_{j}} & \text{ if } i \neq j
\end{cases}
\end{align}
which again can be checked component-wise
\begin{align}
\sum_{i \in I} (\pi_{i})_{c} (\iota_{i})_{c} &= 1_{Fc}\,\text{ and } \\
(\iota_{i})_{c}(\pi_{j})_{c} &= (\delta_{i,j})_{c} = \begin{cases}
1_{F_{i}c} & \text{ if } i = j \\
0_{F_{i}c, F_{j}c} & \text{ if } i \neq j
\end{cases}
\end{align}

\item For a morphism $a : c \rightarrow c' \in \mathcal{A}_{1}$ we have a family of morphisms 
$\{F_{i} a : F_{i} c \rightarrow F_{i} c'\}_{i \in I}$. Analogous to \ref{ex:block_diagonal_matrix} we define

\begin{align}
F a := \sum_{i \in I} (\pi_{i})_{c} F_{i} a (\iota_{i})_{c'} : Fc \rightarrow Fc'
\end{align}
which satisfies
\begin{align}
(\iota_{i})_{c} Fa (\pi_{i})_{c'} &= (\iota_{i})_{c} \sum_{j \in I} (\pi_{j})_{c} F_{j} a (\iota_{j})_{c'} (\pi_{i})_{c'} \\
&= \sum_{j \in I} (\iota_{i})_{c} (\pi_{j})_{c} F_{j} a (\iota_{j})_{c'}(\pi_{i})_{c'} \\
&= \sum_{j \in I} (\delta_{i,j})_{c} F_{j} a (\delta_{j,i})_{c'} \\
&= 1_{F_{i} c} F_{i} a 1_{F_{i} c'} \\
&= F_{i} a
\end{align}
\end{itemize}
This defines how $F$ works on objects and on morphisms.

\item For a family $\tau = \{ \tau_{i} : G \rightarrow F_{i} \}_{i \in I}$ of natural transformations with the same source
$G \in \HomAkmat_{0}$ we have for each object $c \in \mathcal{A}_{0}$ a family
$\tau_{c} = \{ (\tau_{i})_{c} : Gc \rightarrow F_{i}c \}_{i \in I}$ of morphisms in $\kmat_{1}$ with same source $Gc \in \kmat_{0}$.
For these we have by \eqref{eq:u_in_direct_sum_matrix} the morphism $u_{\text{in}}(\tau_{c}) : Gc \rightarrow Fc$ such that
\begin{align}
u_{\text{in}}(\tau_{c}) (\pi_{i})_{c} =
(\tau_{i})_{c}
\end{align}
We can thus define the natural transformation $u_{\text{in}}(\tau) : G \rightarrow F \in \HomAkmat_{1}$ by the components
\begin{align}
(u_{\text{in}}(\tau))_{c} := u_{\text{in}}(\tau_{c})
\end{align}
and we can verify that $u_{\text{in}}(\tau) (\pi_{i})$ is the natural transformation with components
\begin{align}
(u_{\text{in}}(\tau) (\pi_{i}))_{c} &= (u_{\text{in}}(\tau))_{c} (\pi_{i})_{c} \\
&= u_{\text{in}}(\tau_{c}) (\pi_{i})_{c} \\
&= (\tau_{i})_{c}
\end{align}
per definition, i.e. $u_{\text{in}}(\tau)$ is the natural transformation fulfilling $u_{\text{in}}(\tau) (\pi_{i}) = \tau_{i}$.
\item Analogous for a family $\rho = \{ \rho_{i} : F_{i} \rightarrow H \}$ of natural transformations with the same target
$H \in \HomAkmat_{0}$ we have for each object $c \in \mathcal{A}_{0}$ a family
$\rho_{c} = \{ (\rho_{i})_{c} : F_{i} c \rightarrow Hc \}_{i \in I}$ and get the
natural transformation $u_{\text{out}}(\rho) : F \rightarrow H \in \HomAkmat_{1}$ with components
\begin{align}
(u_{\text{out}}(\rho))_{c} := u_{\text{out}}(\rho_{c})
\end{align}
fulfilling $\iota_{i} u_{\text{out}}(\rho) = \rho_{i}$.
\end{itemize}

All in all we have 

\begin{align*}
\forall c \in \mathcal{A}_{0},&& &&  (\oplus_{i \in I} F_{i}) c &= \oplus_{i \in I} (F_{i} c) \\
\forall c \in \mathcal{A}_{0},&& \forall i \in I,&& (\pi_{i} : F \rightarrow F_{i})_{c} &= (\pi_{i})_{c} : Fc \rightarrow F_{i} c \\
\forall c \in \mathcal{A}_{0},&& \forall i \in I,&& (\iota_{i} : F_{i} \rightarrow F)_{c} &= (\iota_{i})_{c} : F_{i} c \rightarrow Fc \\
\forall c \in \mathcal{A}_{0},&& \forall \tau = \{ \tau_{i} : G \rightarrow F_{i} \}_{i = 1,\dots,n},&&
(u_{\mathrm{in}}(\tau))_{c} &= u_{\mathrm{in}}(\tau_{c}) \\
\forall c \in \mathcal{A}_{0},&& \forall \rho = \{ \rho_{i} : F_{i} \rightarrow H \}_{i = 1,\dots,n},&&
(u_{\mathrm{out}}(\rho))_{c} &= u_{\mathrm{out}}(\rho_{c})
\end{align*}

where $\tau_{c} = \{ (\tau_{i})_{c} : Gc \rightarrow F_{i}c \}_{i \in I}$ and $\rho_{c} = \{ (\rho_{i})_{c} : F_{i} c \rightarrow Hc \}_{i \in I}$.

\end{subproof}

\item Next we show that $\HomAkmat$ is a pre-abelian category, i.e. an
\begin{enumerate}
\renewcommand{\labelenumii}{(\roman{enumii})}
\item additive category with
\item a dependent function mapping a morphism $\eta : F \rightarrow G \in \HomAkmat_{1}$ to a kernel of $\eta$ and
\item a dependent function mapping a morphism $\eta : F \rightarrow G \in \HomAkmat_{1}$ to a cokernel of $\eta$.
\end{enumerate}
\begin{subproof}
For a morphism $\eta : F \rightarrow G \in \HomAkmat_{1}$ we want to define its kernel, i.e. the triple
\begin{itemize}
\item $K := \mathrm{KernelObject}(\eta)$
\item $\kappa := \mathrm{KernelEmbedding}(\eta) : K \rightarrow F$ such that $\kappa \eta = 0_{K,G}$.
\item For any other $\tau : T \rightarrow F$ such that $\tau\eta = 0_{T,G}$ we have a unique morphism
$(\tau / \eta) := \mathrm{KernelLift}(\eta,\tau)$ such that $\tau = (\tau / \eta) \kappa$
\end{itemize}

For the components $\eta_{c} : Fc \rightarrow Gc \in \kmat_{1}$ we have in $\kmat$ the kernel object $Kc := \mathrm{KernelObject}(\eta_{c})$
and the kernel embedding
$\kappa_{c} := \mathrm{KernelEmbedding}(\eta_{c}) : Kc \rightarrow Fc$ such that
$\kappa_{c} \eta_{c} = 0_{Kc,Fc}$.
We define the kernel embedding of $\eta$ as $\kappa : K \rightarrow F$ by its components
$\kappa_{c} = \mathrm{KernelEmbedding}(\eta)_{c} := \mathrm{KernelEmbedding}(\eta_{c}) : Kc \rightarrow Fc$ with source
$K = \mathrm{KernelObject}(\eta)$ defined on objects as $Kc := \mathrm{KernelObject}(\eta_{c})$ which is the source of the
kernel embedding of $\eta_{c}$.\\
Rather than proving that the so defined $\kappa : K \rightarrow F$ is a natural transformation, we assume it is and fill in the blanks, i.e. 
how $K$ works on morphisms.

We have for each morphism $a : c \rightarrow c' \in \mathcal{A}_{1}$ the morphisms $Fa : Fc \rightarrow Fc'$ and
$Ga : Gc \rightarrow Gc'$. Then the kernel embedding $\kappa$ of a natural transformation $\eta : F \rightarrow G$ has components
$\kappa_{c} : Fc \rightarrow Gc$ with $\kappa_{c} = \mathrm{KernelEmbedding}(\eta_{c})$ and the kernel object $K$ acts on objects
as $\mathrm{KernelObject}(\eta)c = \mathrm{KernelObject}(\eta_{c})$.
And on morphisms $Ka := \mathrm{KernelLift}(\eta_{c},\kappa_{c}Fa)$.

\[
\begin{tikzcd}
Kc \arrow[dd, "Ka"] \arrow[rr, "\kappa_{c}", hook] \arrow[rrdd, "\kappa_{c}Fa", shift left] \arrow[rrdd, "Ka\kappa_{c'}"', shift right] \arrow[dd, "(\kappa_{c}Fa/\eta_{c'})"', dashed, shift right=2] &  & Fc \arrow[rr, "\eta_{c}"] \arrow[dd, "Fa"] &  & Gc \arrow[dd, "Ga"] \arrow[rr, "\varepsilon_{c}", two heads] \arrow[rrdd, "Ga\varepsilon_{c'}"', shift right] \arrow[rrdd, "\varepsilon_{c}Ca", shift left] &  & Cc \arrow[dd, "Ca"'] \arrow[dd, "(Ga\varepsilon_{c'}\backslash \eta_{c})", dashed, shift left=2] \\
                                                                                                                                                                                                       &  &                                            &  &                                                                                                                                                             &  &                                                                                                  \\
Kc' \arrow[rr, "\kappa_{c'}", hook]                                                                                                                                                                    &  & Fc' \arrow[rr, "\eta_{c'}"]                &  & Gc' \arrow[rr, "\varepsilon_{c'}", two heads]                                                                                                               &  & Cc'                                                                                             
\end{tikzcd}
\]


As an example, let $\mathcal{A}$ be our concrete category $C_{3}C_{3}$ with objects $1,2$ and morphisms
$a : 1\rightarrow 1, b : 1 \rightarrow 2, c : 2 \rightarrow 2$. Then the following rectangle describes how the functors $F,G$ and $K$ act on
$\mathcal{A}$:

\[
\begin{tikzcd}
K1 \arrow["Ka"', loop, distance=2em, in=125, out=55] \arrow[d, "Kb"] \arrow[r, "\kappa_{1}"] & F1 \arrow["Fa"', loop, distance=2em, in=125, out=55] \arrow[d, "Fb"] \arrow[r, "\eta_{1}"] & G1 \arrow["Ga"', loop, distance=2em, in=125, out=55] \arrow[d, "Gb"] \\
K2 \arrow["Kc"', loop, distance=2em, in=305, out=235] \arrow[r, "\kappa_{2}"]                & F2 \arrow["Fc"', loop, distance=2em, in=305, out=235] \arrow[r, "\eta_{2}"]                & G2 \arrow["Gc"', loop, distance=2em, in=305, out=235]               
\end{tikzcd}
\]
Naturality of $\kappa$ and $\eta$ gives the following equations:\\
\begin{minipage}{.20\textwidth}
\phantom{}
\end{minipage}
\begin{minipage}{.28\textwidth}
\begin{subequations}
\begin{align}
Ka \kappa_{1} &=\label{eq:natural_Ka_kappa_kappa_Fa}
 \kappa_{1} Fa \\
Kb \kappa_{2} &= \kappa_{1} Fb \\
Kc \kappa_{2} &= \kappa_{2} Fc
\end{align}
\end{subequations}
\end{minipage}
\begin{minipage}{.04\textwidth}
\phantom{}
\end{minipage}
\begin{minipage}{.28\textwidth}
\begin{subequations}
\begin{align}
Fa \eta_{1} &= \eta_{1} Ga \\
Fb \eta_{2} &= \eta_{1} Gb \\
Fc \eta_{2} &= \eta_{2} Gc
\end{align}
\end{subequations}
\end{minipage}
\begin{minipage}{.20\textwidth}
\phantom{}
\end{minipage}

which together with $\kappa \eta = 0_{K,G}$ gives the following equations:

\begin{subequations}
\begin{alignat}{4}
Ka\kappa_{1}\eta_{1} &= \kappa_{1}Fa\eta_{1} &= \kappa_{1}\eta_{1}Ga &= 0 : K1 \rightarrow G1 \\
Kb\kappa_{2}\eta_{2} &= \kappa_{1}Fb\eta_{2} &= \kappa_{1}\eta_{1}Gb &= 0 : K1 \rightarrow G2 \\
Kc\kappa_{2}\eta_{2} &= \kappa_{2}Fc\eta_{2} &= \kappa_{2}\eta_{2}Gc &= 0 : K2 \rightarrow G2
\end{alignat}
\end{subequations}

To determine the morphism $Ka$, we are writing $\kappa_{1} : K1 \rightarrow F1$ and $Ka \kappa_{1} : K1 \rightarrow F1$ in one diagram:
\[
\begin{tikzcd}
K1 \arrow[rr, "\kappa_{1}", hook]                                 &                                   & F1 \arrow[rr, "\eta_{1}"] &  & G1 \\
                                                                  & K1 \arrow[ru, "\kappa_{1}", hook] &                           &  &    \\
K1 \arrow[ru, "Ka"] \arrow[uu, "(Ka\kappa_{1}/\eta_{1})", dashed] &                                   &                           &  &   
\end{tikzcd}
\]
Then we have two competing morphisms, $\kappa_{1}$ and $Ka \kappa_{1}$ both from $K1 \rightarrow F1$, and both with
\begin{align*}
\kappa_{1}\eta_{1} &= 0_{K1,G1} \\
Ka \kappa_{1}\eta_{1} &= 0_{K1,G1}
\end{align*}
Then we get a unique morphism $(Ka\kappa_{1}/\eta_{1}) := \mathrm{KernelLift}(\eta_{1},Ka\kappa_{1})$ such that
\begin{align*}
Ka\kappa_{1} &= (Ka\kappa_{1}/\eta_{1})\kappa_{1}
\end{align*}
But since $\kappa_{1}$ is a monomorphism, we get that
\begin{align*}
Ka = (Ka\kappa_{1}/\eta_{1}) = \mathrm{KernelLift}(\eta_{1},Ka\kappa_{1})
\end{align*}
which a priori doesn't help calculating $Ka$, since it stands on both sides of the equation, but with equation
\eqref{eq:natural_Ka_kappa_kappa_Fa} we get
\begin{align*}
Ka\kappa_{1} &= \kappa_{1} Fa \\
\end{align*}
and therefore
\begin{align*}
Ka = \mathrm{KernelLift}(\eta_{1},\kappa_{1} Fa)
\end{align*}

We have for each morphism $a : c \rightarrow c' \in \mathcal{A}_{1}$ the morphisms $Fa : Fc \rightarrow Fc'$ and
$Ga : Gc \rightarrow Gc'$. Then the kernel embedding $\kappa$ of a natural transformation $\eta : F \rightarrow G$ has components
$\kappa_{c} : Fc \rightarrow Gc$ with $\kappa_{c} = \mathrm{KernelEmbedding}(\eta_{c})$ and the kernel object $K$ acts on objects
as $\mathrm{KernelObject}(\eta)c = \mathrm{KernelObject}(\eta_{c})$.
And on morphisms $Ka := \mathrm{KernelLift}(\eta_{c},\kappa_{c}Fa)$.

\[
\begin{tikzcd}
Kc \arrow[dd, "Ka"] \arrow[rr, "\kappa_{c}", hook] \arrow[rrdd, "\kappa_{c}Fa", shift left] \arrow[rrdd, "Ka\kappa_{c'}"', shift right] \arrow[dd, "(\kappa_{c}Fa/\eta_{c'})"', dashed, shift right=2] &  & Fc \arrow[rr, "\eta_{c}"] \arrow[dd, "Fa"] &  & Gc \arrow[dd, "Ga"] \arrow[rr, "\varepsilon_{c}", two heads] \arrow[rrdd, "Ga\varepsilon_{c'}"', shift right] \arrow[rrdd, "\varepsilon_{c}Ca", shift left] &  & Cc \arrow[dd, "Ca"'] \arrow[dd, "(Ga\varepsilon_{c'}\backslash \eta_{c})", dashed, shift left=2] \\
                                                                                                                                                                                                       &  &                                            &  &                                                                                                                                                             &  &                                                                                                  \\
Kc' \arrow[rr, "\kappa_{c'}", hook]                                                                                                                                                                    &  & Fc' \arrow[rr, "\eta_{c'}"]                &  & Gc' \arrow[rr, "\varepsilon_{c'}", two heads]                                                                                                               &  & Cc'                                                                                             
\end{tikzcd}
\]

KernelLift:
\[
\begin{tikzcd}
K \arrow[r, "\kappa"]                                   & F \arrow[r, "\eta"] & G \\
T \arrow[ru, "\tau"] \arrow[u, "(\tau / \eta)", dashed] &                     &  
\end{tikzcd}
\]

CokernelColift:
\[
\begin{tikzcd}
F \arrow[r, "\eta"] & G \arrow[r, "\varepsilon"] \arrow[rd, "\tau"'] & C                                            \\
                    &                                                & T \arrow[u, "(\tau\backslash\eta)"', dashed]
\end{tikzcd}
\]

This describes the kernel object $K$.

Analogous we have the cokernel object $C$ with $Cc = \mathrm{CokernelObject}(\eta_{c})$ and
$Ca = \mathrm{CokernelColift}(\eta_{c}, Ga\varepsilon_{c'})$.

Now for the $\mathrm{KernelLift}(\eta, \tau)$.

$\eta : F \rightarrow G$

$\tau : T \rightarrow F$ with $\tau \eta = 0_{T,G}$. For all $c \in \mathcal{A}_{0}$ we have
$\tau_{c} : Tc \rightarrow Fc$ with $\tau_{c} \eta_{c} = 0_{Tc,Gc}$. Therefore in $\kmat$ we have
$(\tau_{c} / \eta_{c} ) := \mathrm{KernelLift}( \eta_{c}, \tau_{c} )$ such that $\tau_{c} = (\tau_{c} / \eta_{c} ) \kappa_{c}$.

The natural transformation $\lambda : T \rightarrow K$ defined by its components
$\lambda_{c} := (\tau_{c} / \eta_{c}) := \mathrm{KernelLift}( \eta_{c}, \tau_{c} )$ satisfies
$\tau_{c} = \lambda_{c} \kappa_{c} = (\tau_{c} / \eta_{c} ) \kappa_{c}$ and therefore
$\tau = \lambda \kappa$, i.e. $\lambda$ is the kernel lift $\mathrm{KernelLift}(\eta, \tau)$.

\end{subproof}

\item Finally to show that $\HomAkmat$ is an abelian category, we need to show that it is
\begin{enumerate}
\renewcommand{\labelenumii}{(\roman{enumii})}
\item a pre-abelian category with
\item a dependent function $(-/-)$ mapping a pair $\iota : K \rightarrow A, \tau : T \rightarrow A$ to a lift $\tau / \iota$ of
$\tau$ along $\iota$, where $A, K, T \in \HomAkmat_{0}$, $\iota$ is a monomorphism and $\tau \mathrm{CokernelProjection}(\iota) = 0$.
\item a dependent function $(-\backslash -)$ mapping a pair $\varepsilon : A \rightarrow C, \tau : A \rightarrow T$ to a colift
$\epsilon \backslash \tau$ of $\tau$ along $\varepsilon$, where $A, K, T \in \mathcal{A}_{0}$, $\varepsilon$ is an epimorphism and
$\mathrm{KernelEmbedding}(\varepsilon)\,\tau = 0$.
\end{enumerate}
\begin{subproof}
Let $A, K, T \in \HomAkmat_{0}$ and $\iota, \tau \in \HomAkmat_{1}$ such that 
\begin{itemize}
\item $\iota : K \rightarrow A$ is a monomorphism,
\item $\tau : T \rightarrow A$ is a morphism with $\tau \, \mathrm{CokernelProjection}(\iota) = 0$.
\end{itemize}
Then for each $c \in \mathcal{A}_{0}$ we have $\iota_{c}, \tau_{c} \in \kmat_{1}$ such that
\begin{itemize}
\item $\iota_{c} : Kc \rightarrow Ac$ is a monomorphism, and
\item $\tau_{c} \, \mathrm{CokernelProjection}(\iota_{c}) = 0$.
\end{itemize}
Since $\kmat$ is an abelian category, we have a lift $\tau_{c} / \iota_{c}$ of $\tau_{c}$ along $\iota_{c}$, i.e.
$(\tau_{c} / \iota_{c}) \iota_{c} = \tau_{c}$.
This lift $(\tau_{c} / \iota_{c}) \in \kmat_{1}$ defines the components for $(\tau / \iota) \in \HomAkmat_{1}$. And since
$\iota$ is a monomorphism, the lift along $\iota$ is necessarily unique.\\
Thus we have a dependent function $( - / - )$ mapping a
a pair $\iota : K \rightarrow A, \tau : T \rightarrow A$ to a lift $(\tau / \iota)$ of
$\tau$ along $\iota$, where $A, K, T \in \HomAkmat_{0}$, $\iota$ is a monomorphism and $\tau \, \mathrm{CokernelProjection}(\iota) = 0$
with $(\tau / \iota)$ defined as $(\tau / \iota)_{c} := (\tau_{c} / \iota_{c})$.



\end{subproof}
\end{enumerate}
\end{proof}


%%% Yoneda projectives + enough projectives
\section{Yoneda's Lemma: Completion and cocompletion of a category}

\subsection{The category of presheaves}

\begin{definition}{(The category of presheaves)}\endnote{(cited from ncatlab \cite{[ncatlab_presheaves]})}\\
For $\mathcal{C}$ a small category, its \ul{category of presheaves} is the functor category
\[ \mathrm{PSh}(\mathcal{C}) := \mathrm{Hom}(\mathcal{C}^{\text{op}}, \Set) \]
from the opposite category of $\mathcal{C}$ to $\Set$.
An object in this category is a \ul{presheaf}.\\
\noindent Taking $\mathcal{C}^{\text{op}}$ instead of $\mathcal{C}$ (and with $(\mathcal{C}^{\text{op}})^{\text{op}} = \mathcal{C}$)
we get the functor category as in \ref{def:functor_category}
\[
\mathrm{Hom}(\mathcal{C}, \Set) = \mathrm{PSh}(\mathcal{C}^{\text{op}})
\]
\end{definition}

\begin{remark}{(General properties of presheaves)}\\
The category of presheaves $\mathrm{PSh}(\mathcal{C})$ is called the \ul{free cocompletion} of $\mathcal{C}$.
\end{remark}

\begin{definition}{(Representable functor)}\label{def:repres_functor}\endnote{(from ncatlab \ref{[ncatlab_repres_functor]})}
\begin{enumerate}
\renewcommand{\labelenumi}{(\theenumi)}
\item A functor from a locally small category $\mathcal{C}$ to $\Set$ is \ul{representable} if there is an object $c \in \mathcal{C}$ and a
natural isomorphism between $F$ and $\mathrm{Hom}(c,-)$ (for a covariant $F$, otherwise $\mathrm{Hom}(-,c)$ for a contravariant $F$),
in which case one says thet the functor $F$ is \ul{represented by} the object $c$.
\item A \ul{representation} for a functor $F$ is a choice of object $c \in \mathcal{C}$ together with a specified natural isomorphism
$\mathrm{Hom}(c,-) \cong F$ (for a covariant $F$, or $\mathrm{Hom}(-,c) \cong F$ for a contravariant $F$).
\end{enumerate}
\end{definition}

\begin{lemma}{(Yoneda's Lemma)}\endnote{(Statement of Yoneda's lemma from \cite{[context]}, Lemma 2.2.4)}
For any functor $F : \mathcal{C} \rightarrow \mathrm{Set}$, whose source $\mathcal{C}$ is locally small and any
object $c \in \mathcal{C}_{0}$, there is a bijection
\[
\mathrm{Hom}(\mathrm{Hom}_{\mathcal{C}}(c,-), F) \cong Fc
\]
that associates a natural transformation $\alpha : \mathrm{Hom}_{\mathcal{C}}(c,-) \Rightarrow F$ to the element $\alpha_{c}(1_{c}) \in Fc$.
Moreover, this correspondence is natural in both $c$ and $F$.
\end{lemma}
As $\mathcal{C}$ is locally small but not necessarily small, a priori the collection of natural transformations
$\mathrm{Hom}(\mathrm{Hom}_{\mathcal{C}}(c,-),F)$ might be large. However, the bijection in the Yoneda lemma proves that this particular
collection of natural transformations indeed forms a set.
\begin{proof}
A proof of Yoneda's lemma can be found in many books on category theory, e.g. \cite{[context]}, Lemma 2.2.4, pages 57-59.
\end{proof}

\begin{definition}{(Yoneda embedding)}\label{def:yoneda_embedding}\endnote{(cited from ncatlab \cite{[ncatlab_yoneda_emb]})}\\
The \ul{Yoneda embedding} for a locally small category $\mathcal{C}$ is the functor
\[
Y : \mathcal{C} \hookrightarrow \mathrm{Hom}(\mathcal{C}^{\text{op}}, \mathrm{Set})
\]
from $\mathcal{C}$ to the category of presheaves over $\mathcal{C}$ which is the image of the hom-functor
\[
\mathrm{Hom} : \mathcal{C}^{\text{op}}\times\mathcal{C} \rightarrow \mathrm{Set}
\]
under the $\mathrm{Hom}$ adjunction
\[
\mathrm{Hom}(\mathcal{C}^{\text{op}}\times\mathcal{C}, \mathrm{Set}) \simeq
\mathrm{Hom}(\mathcal{C},\mathrm{Hom}(\mathcal{C}^{\text{op}}, \mathrm{Set}))
\]
in the closed symmetric monoidal category $\mathrm{Cat}$.
If instead we have the opposite category $\mathcal{C}^{\text{op}}$, then we get the embedding into the functor category:
\[
Y^{\text{op}} : \mathcal{C}^{\text{op}} \hookrightarrow \mathrm{Hom}(\mathcal{C},\mathrm{Set})
\]
\end{definition}

\begin{remark}[Our $\Bbbk$-linear version of Yoneda's embedding]
Let $\mathcal{A}$ be a $\Bbbk$-linear category with finite-dimensional $\Bbbk$-vector spaces as hom-sets. Then we get
$\Bbbk$-linear versions of Yoneda's lemma and Yondeda's embedding:

For any $\Bbbk$-linear functor $F : \mathcal{A} \rightarrow \kmat$ and any object $i \in \mathcal{A}_{0}$, there is a bijection
\[
\mathrm{Hom}_{\HomAkmat}(\mathrm{Hom}_{\mathcal{A}}(-,i), F) \cong F(i)
\]

\[
Y : \mathcal{A} \hookrightarrow \mathrm{Hom_{\Bbbk}}(\mathcal{A^{\text{op}}},\kmat)
\]

\[
Y^{\text{op}} : \mathcal{A}^{\text{op}} \hookrightarrow \HomAkmat
\]

\end{remark}

$\HomAkmat$ is Abelian and therefore finitely complete and finitely cocomplete.

\subsection{Projective objects and the Yoneda projective}\label{sec:projective_objects}

\begin{lemma}\label{la:Hom_exact_proj_Lift_along_epis}
Let $\mathcal{C}$ be a locally small category. For an object $P \in \mathcal{C}_{0}$ the following are equivalent:
\begin{itemize}
\item The covariant functor $\mathrm{Hom}(P,-)$ is exact.
\item For all epimorphisms $\varphi : M \twoheadrightarrow N$ and morphisms $\theta : P \rightarrow N$, there exists a
projective lift $\psi : P\dottedrightarrow M$ such that $\theta = \psi\varphi$.\\
\begin{tikzcd}
M \arrow[r, "\varphi", two heads] & N \\
	& P \arrow[u, "\theta", "=\,\psi\varphi"'] \arrow[lu, "\psi", dotted]
\end{tikzcd}
\end{itemize}
\begin{proof}
For an object $L\in \mathcal{C}_{0}$, $\mathrm{Hom}(L,-)$ is always a covariant left-exact functor, i.e. respects monos.\\
\setlist[description]{font=\normalfont}
\begin{description}
\item[``$\Leftarrow$:''] Prove that $\mathrm{Hom}(P,-)$  is right exact, i.e. respects epis.\\
For this, let $M, N \in \mathcal{C}_{0}$ and $\varphi : M \twoheadrightarrow N$ be an epi. The Hom-functor works on morphisms
by mapping the Hom-sets of the source and target objects of the morphism, i.e.
$\mathrm{Hom}(P,\varphi) : \mathrm{Hom}(P,M) \rightarrow \mathrm{Hom}(P,N)$, given by $\rho \mapsto \rho\varphi\, \forall \rho \in \mathrm{Hom}(P,M)$.
Now given that $\varphi$ is an epi, we want to show that $\mathrm{Hom}(P,\varphi)$ is also an epi.\\
Let $O \in \mathcal{C}_{0}$,  $\gamma : N \rightarrow O$ and $\varepsilon : N \rightarrow O$ such that
$\mathrm{Hom}(P,\gamma) : \mathrm{Hom}(P,N) \rightarrow \mathrm{Hom}(P,O);\, \theta \mapsto \theta\gamma$ and
$\mathrm{Hom}(P,\varepsilon) : \mathrm{Hom}(P,N) \rightarrow \mathrm{Hom}(P,O);\, \theta \mapsto \theta\varepsilon$ and
$\mathrm{Hom}(P,\varphi)\mathrm{Hom}(P,\gamma) = \mathrm{Hom}(P,\varphi)\mathrm{Hom}(P,\varepsilon)$. 
From the functoriality axioms (ref. definition \ref{def:functor} of a functor) it follows that $\mathrm{Hom}(P,\varphi\gamma) = \mathrm{Hom}(P,\varphi\varepsilon)$. This implies
\begin{equation}\label{eqn:Hom_functoriality}\rho(\varphi\gamma) = \rho(\varphi\varepsilon)\, \forall \rho \in \mathrm{Hom}(P,M)\end{equation}. 

\begin{tikzcd}
M \arrow[r, "\varphi", shift right, two heads] & N \arrow[r, "\gamma"'] \arrow[r, shift left=2, "\varepsilon"] & O \\
	& P \arrow[lu, "\rho"] \arrow[u, "\theta"] \arrow[ru, outer sep=2, pos=.55, "\rho(\varphi\gamma) = \rho(\varphi\varepsilon)"'] \arrow[ru, shift right=2]
\end{tikzcd}

We want to show that the parallel morphisms $\mathrm{Hom}(P,\gamma)$ and $\mathrm{Hom}(P,\varepsilon)$ are the same, i.e. for all
$\theta \in \mathrm{Hom}(P,N), \theta\gamma = \theta\varepsilon$. Our assumtion that there exists a projective lift helps us in this situation:
$\forall \theta \in \mathrm{Hom}(P,N)\, \exists\, \rho \in \mathrm{Hom}(P,M)$ such that $\theta = \rho\varphi$ and therefore with the above 
equation \eqref{eqn:Hom_functoriality},
$\theta\gamma = (\rho\varphi)\gamma = \rho(\varphi\gamma) = \rho(\varphi\varepsilon) = (\rho\varphi)\varepsilon = \theta\varepsilon$
and therefore $\mathrm{Hom}(P,\gamma) = \mathrm{Hom}(P,\varepsilon)$, i.e. $\mathrm{Hom}(P,\varphi)$ is epi.\\

\item[``$\Rightarrow$:''] Let $\mathrm{Hom}(P,-)$ be right exact. Let $M, N \in \mathcal{C}_{0}$, the morphism
$\varphi : M \twoheadrightarrow N$ be an epi and $\theta : P \rightarrow N$ any morphism.
We want to show the existence of a morphism $\psi : P \dottedrightarrow\, M$ such that $\theta = \psi\varphi$.
With $\mathrm{Hom}(P,-)$ being exact, we have that $\mathrm{Hom}(P,\varphi) : \mathrm{Hom}(P,M) \twoheadrightarrow \mathrm{Hom}(P,N)$ is
an epi, and is given by $\mathrm{Hom}(P,M) \ni \rho \mapsto \rho\varphi \in \mathrm{Hom}(P,N)$.\\
$\mathcal{C}$ is locally small, i.e. for the two objects $P, N \in \mathcal{C}_{0},$ there is a \ul{set} $\mathrm{Hom}(P,N)$
of morphisms between them. The $\mathrm{Hom}$-functor moves the morphisms from a general categorical context 
in $\mathcal{C}$ into the category of sets, i.e. $\mathrm{Hom}(P,\varphi)$ is a function in the category of sets.
And for those it's true that every epimorphism is surjective. Thus $\forall \theta \in \mathrm{Hom}(P,N)\, \exists \rho \in \mathrm{Hom}(P,M)$ such
that $\theta = (\mathrm{Hom}(P,\varphi))(\rho) = \rho\varphi$. This $\rho$ is the projective lift $\psi := \rho$ we were looking for.
\end{description}
\end{proof}
\end{lemma}

\begin{definition}{(Projective object)}\label{def:proj_object}\\
An object $P$ in a category $\mathcal{C}$ that satisfies one (and thus both) of the equivalent properties in Lemma
 \ref{la:Hom_exact_proj_Lift_along_epis} is called a \ul{projective object}.
\end{definition}

The dual statement to Lemma \ref{la:Hom_exact_proj_Lift_along_epis} is
\begin{lemma}\label{la:dual_Hom_exact_proj_colift}
Let $\mathcal{C}$ be a category. For an object $P \in \mathcal{C}_{0}$ the following are equivalent:
\begin{itemize}
\item The contravariant functor $\mathrm{Hom}(-,P)$ is exact.
\item For all monomorphisms $\varphi : M \hookleftarrow N$ and morphisms $\theta : P \leftarrow N$, there exists a
projective colift $\psi : P \dottedleftarrow M$ such that $\theta = \varphi\psi$.\\
\begin{tikzcd}
M \arrow[rd, "\psi"', dotted] & N \arrow[l, "\varphi", hook] \arrow[d, "=\,\varphi\psi", "\theta"'] \\
	& P 
\end{tikzcd}
\end{itemize}
\end{lemma}

\begin{definition}{(Injective object)}\label{def:inj_object}\\
An object $P$ in a category $\mathcal{C}$ that satisfies one (and thus both) of the equivalent properties in Lemma
 \ref{la:dual_Hom_exact_proj_colift} is called an \ul{injective object}.
\end{definition}

\begin{definition}{(Yoneda projective)}
Let $\mathcal{A}$ be a $\Bbbk$-algebroid with finitely many objects and finite-dimensional hom-sets over $\Bbbk$.
With the Yoneda embedding 
\[
Y^{\text{op}} : \begin{cases}\mathcal{A}^{\text{op}} \hookrightarrow \HomAkmat \\
i \mapsto \mathrm{Hom}_{\mathcal{A}}(-,i)
\end{cases}
\]
we see that the image
$Y^{\text{op}}(i)$ of the object $i \in \mathcal{A}^{\text{op}}$ is the representable functor $\mathrm{Hom}_{\mathcal{A}}(-,i)$ in $\HomAkmat$. 
It is called the \ul{$i$-th Yoneda projective}.
\end{definition}

Analogous to Prop. \ref{prop:Alg-Alg-Correspondence} one can easily prove that a $\Bbbk$-representation $F$ of $\mathcal{A}$ corresponds to a 
module $\bigoplus_{i \in \mathcal{A}_{0}} F(i)$ over the algebra $\mathbf{A} = \mathrm{Algebra}(\mathcal{A})$.
In particular the Yoneda projective $Y^{\text{op}}(i) = \mathrm{Hom}_{\mathcal{A}}(-,i)$ corresponds to the $\mathbf{A}$-module
$M_{i} = \bigoplus_{j \in \mathcal{A}_{0}} \mathrm{Hom}_{\mathcal{A}}(j,i)$.

\begin{lemma}
Yoneda projectives are projective objects in $\HomAkmat$.
\end{lemma}
\begin{proof}\phantom{}\\
We want to show that $\mathrm{Hom}_{\HomAkmat}(Y^{\text{op}}(i),-)$ is right exact, i.e. finitely cocontinuous.

\noindent For this, let
\[
\begin{tikzcd}
0 \arrow[r] & F_{1} \arrow[r, "\eta"] & F_{2} \arrow[r,"\rho"] & F_{3} \arrow[r] & 0
\end{tikzcd}
\]
be a short exact sequence in $\HomAkmat$, i.e.
\begin{enumerate}
\item $\eta$ is a monomorphism,
\item $\rho$ is an epimorphism and
\item the image of $\eta$ equals the kernel of $\rho$.
\end{enumerate}

\noindent Then applying the functor $\mathrm{Hom}_{\HomAkmat}(Y^{\text{op}}(i),-)$ on the sequence and simplifying with Yoneda's lemma

\begin{align*}
\mathrm{Hom}_{\HomAkmat}(\mathrm{Hom}_{\mathcal{A}}(-,i),F) &\simeq F(i) \\
\mathrm{Hom}_{\HomAkmat}(\mathrm{Hom}_{\mathcal{A}}(-,i),\rho) &\simeq \rho_{i}
\end{align*}

We only have to measure exactness on the components
\[
\begin{tikzcd}
0 \arrow[r] & F_{1}(i) \arrow[r, "\eta_{i}"] &
F_{2}(i) \arrow[r,"\rho_{i}"] & F_{3}(i) \arrow[r] & 0\mbox{,}
\end{tikzcd}
\]
but this coincides with the definition of the exactness of functors. So the proof is a tautology by Yoneda's lemma.
\end{proof}

\subsubsection{Abelian categories with enough projective objects (constructively)}

\begin{definition}[Enough projective objects]\label{def:enough_projectives}
A category $\mathcal{C}$ is said to have \ul{enough projective objects} if every object admits an epimorphism from a projective object,
i.e. for any object $A \in \mathcal{C}_{0}$ there exists an epimorphism $P \twoheadrightarrow A$, where $P$ is projective.
\end{definition}

We state without a proof the following fact about our functor category:

\begin{theorem}[$\HomAkmat$ has enough projectives]\phantom{}\\
Let $\mathcal{A}$ be a finite-dimensional algebroid over some field $\Bbbk$. The functor category $\HomAkmat$ has sufficiently many
projectives.
\end{theorem}
\begin{proof}
(no proof)
\end{proof}

\begin{doctrine}[Abelian category with enough projective objects]\phantom{}\\
The doctrine $\mathtt{IsAbelianCategoryWithEnoughProjectives}$ therefore involves algorithms of\\
$\mathtt{IsAbelianCategory}$ together with algorithms for
\begin{itemize}
\item $\mathtt{EpimorphismFromSomeProjectiveObject}$,
\item $\mathtt{ProjectiveLift}$,
\end{itemize}
\end{doctrine}

\subsubsection{Abelian categories with enough injective objects}
Dually to \ref{def:enough_projectives} we define

\begin{definition}[Enough injective objects]\label{def:enough_injectives}
A category $\mathcal{C}$ is said to have \ul{enough injective objects} if every object admits a monomorphism into an injective object,
i.e. for any object $A \in \mathcal{C}_{0}$ there exists a monomorphism $A \hookrightarrow J$, where $J$ is injective.
\end{definition}

\begin{doctrine}[Abelian category with enough injective objects]\phantom{}\\
The doctrine $\mathtt{IsAbelianCategoryWithEnoughInjectives}$ therefore involves algorithms of\\
$\mathtt{IsAbelianCategory}$ together with algorithms for
\begin{itemize}
\item $\mathtt{MonomorphismIntoSomeInjectiveObject}$,
\item $\mathtt{InjectiveColift}$,
\end{itemize}
\end{doctrine}


%%% Direct sum decomposition
\section{Direct Sum decomposition (constructively)}

We have shown in section \ref{sect:abelian_cat} that for a family $\{F_{i}\}_{i\in I}$ of functors in $\HomAkmat$, there is the
direct sum $F := \bigoplus_{i\in I} F_{i}$ with embeddings $\iota_{i} : F_{i} \rightarrow F$ (and projections $\pi_{i} : F \rightarrow F_{i}$).
In this section we show constructively the other direction, i.e. that for each $F$ there is a direct sum decomposition
$\{F_{i}\}_{i\in I}$ and $\iota_{i} : F_{i} \rightarrow F$ such that
\[
F = \bigoplus_{i\in I} F_{i}
\]

\subsection{The algorithm $\mathtt{DecomposeOnceByRandomEndomorphism}$}

\begin{algorithm}[H]\capstart
    \caption{\texttt{DecomposeOnceByRandomEndomorphism}}\label{algo:DecomposeOnceByRandomEndomorphism}
	\SetKwInput{Input}{Input~}
	\SetKwInput{Output}{Output~}
	\Input{~a functor $F$ in a functor category}
	\Output{~a pair $[\iota : I \rightarrow F, \kappa : K \rightarrow F]$ of morphisms such that $I \oplus K = F$ with $I \neq 0$ and $K \neq 0$ or
	$\mathtt{fail}$ if it was unable to further decompose $F$; }
	\BlankLine
	$d := \max \{ \mathrm{dim}_{\Bbbk}Fc \}_{c \in \mathcal{A}_{0}}$\;
	\If{$d = 0$}{
	    \Return $\mathtt{fail}$\;
	}
	$\mathcal{B} = [\beta_{1},\dots,\beta_{h}]$ be a $\Bbbk$-basis of $\mathrm{Hom}_{\HomAkmat}(F,F)$\;
	add $0_{F,F}$ to $\mathcal{B}$\;
	$n := \log_{2}(d) + 1$\;
	\BlankLine
	\For{$b \in [h+1, h, \dots,2]$}{
	    $\alpha := \beta_{b} + \mathrm{random}(\Bbbk) \cdot \beta_{b-1}$\;
	    \For{$i \in [ 1, \dots, n ]$}{
	        $\alpha_{2} := \alpha^{2}$\;
	        \nl\tcc{We do not expect the exponentiation to produce an idempotent, still this is a very cheap test:}
	        \If{$\alpha = \alpha_{2}$}{
	            \Break\;
	        }
	        $\alpha := \alpha_{2}$\;
	    }
	    \BlankLine
	    
	    \If{$\alpha = 0$}{
	        \Continue\tcp*{try another endomorphism}
	    }
	    
	    $\kappa := \mathrm{KernelEmbedding}(\alpha)$\;
	    
	    \If{$\kappa = 0$}{
	        \Continue\tcp*{try another endomorphism}
	    }
	    \BlankLine
	    $\iota := \mathrm{ImageEmbedding}(\alpha)$\;
	    \Return $[ \iota, \kappa ]$\;
	}
	\BlankLine
	\Return $\mathtt{fail}$\tcp*{The input functor $F$ is indecomposable with a high probability.}
\end{algorithm}

To justify proposition \ref{prop:Decompose_terminates_correct} below we need the following lemma which is a linear analogue of the $\sigma$-lemma
\ref{la:sigma-lemma}.

\begin{lemma}
If $\alpha$ is an endomorphism of a $d$-dimensional vector space, then $\alpha^{d}\restrict{\mathrm{Im}(\alpha^{d})}$ is an automorphism.
\end{lemma}
\begin{proof}
This is a corollary of the Jordan normal form.
\end{proof}

\begin{proposition}\label{prop:Decompose_terminates_correct}
\algoref{DecomposeOnceByRandomEndomorphism} terminates with the correct output.
\end{proposition}
\begin{proof} We assert certain truths about the algorithm by formulating invariants, and how they stay constant in each line of the algorithm.\\

$\alpha^{2^{k}+2^{k}} = \alpha^{2^{k}}$.

\noindent For the input $F = 0$ we have $Fc = 0\,\forall c \in \mathcal{A}$ and thus $d = 0$ which returns $\mathtt{fail}$, i.e. there is no
decomposition $0 = I \oplus K$ with $I \neq 0$ and $K \neq 0$.\\

\noindent For any other input $F \neq 0$ there is a $c \in \mathcal{A}_{0}$ with $Fc > 0$, thus in line 1 we have $d > 0$.\\
Since $F \neq 0$ we also have $1_{F} \neq 0_{F,F}$, thus after line 6 we can assume $\mathrm{Length}(\mathcal{B}) \geq 2$ and
the list $[\mathrm{Length}(\mathcal{B}),\mathrm{Length}(\mathcal{B}) - 1,\dots,2]$ in line 8 to be nonempty, thus we will enter the for loop
at least once.\\

\noindent In line 9, the endomorphism $\alpha$ is a linear combination of $\beta_{b}$ and $\beta_{b-1}$. Note that for the initial
$b = \mathrm{Length}(\mathcal{B})$, we have $\beta_{b} = 0_{F,F}$ and, as we added the zero morphism last.\\
Without loss of generality, we can assume that $\alpha_{c}$ is in Jordan normal form for each $c \in \mathcal{A}$.
\end{proof}

\endnote{Die Menge der Endomorphismen, die nicht in der Lage wären, einen zerlegbaren Modul zu zerlegen, liegt auf einer Hyperfläche.
Somit ist das Komplement Zariski-dicht.}

\subsection{The algorithm $\mathtt{WeakDirectSumDecomposition}$}

\begin{algorithm}[H]\capstart
    \caption{\texttt{WeakDirectSumDecomposition}}\label{algo:WeakDirectSumDecomposition}
	\SetKwInput{Input}{Input~}
	\SetKwInput{Output}{Output~}
	\Input{~a functor $F$ in a functor category}
	\Output{~a list $[\eta_i : F_{i} \rightarrow F]$ of embeddings such that $\oplus_{i} F_{i} = F$ and each $F_{i}$ is indecomposable
	(with a high probability).}
	\BlankLine
	$\mathtt{queue} := [ 1_{F} ]$\;
	$\mathtt{summands} := [ \quad ]$\;
	
	\While{ $\mathtt{queue} \neq \emptyset$ }{
	    let $\eta$ be the last element in $\mathtt{queue}$ and delete $\eta$ from $\mathtt{queue}$\;
	    $\mathtt{result} := \mathtt{DecomposeOnceByRandomEndomorphism}(s(\eta))$\;
	    \eIf(\tcp*[f]{$s(\eta)$ was indecomposable (with a high probability)}){$\mathtt{result} = \mathtt{fail}$}{
	        add $\eta$ to $\mathtt{summands}$\;
	    }{
	        $[\iota,\kappa] = \mathtt{result}$\;
	        append $[\iota\eta, \kappa\eta]$ to $\mathtt{queue}$\;
	    }
	}
	\BlankLine
	\Return $\mathtt{summands}$\;
\end{algorithm}

\begin{proposition}
\algoref{WeakDirectSumDecomposition} terminates with the correct output.
\end{proposition}
\begin{proof} We assert certain truths about the algorithm by formulating invariants, and how they stay constant in each line of the algorithm.\\

\begin{subproof}[Proof that the output is correct]\phantom{}\\
\noindent In line 1, the morphism $1_{F} : F \rightarrow F$, which is initially the only morphism in $\mathtt{queue}$, satisfies $t(1_{F}) = F$.\\
In line 10, since $\iota : I \rightarrow s(\eta)$ and $\kappa : K \rightarrow s(\eta)$ are each composable with $\eta$, then we are
appending the list $[\iota\eta, \kappa\eta]$ of morphisms with target $t(\iota\eta) = t(\kappa\eta) = F$ to the $\mathtt{queue}$.\\
Thus in each step of the algorithm the $\mathtt{queue}$ only contains morphisms $\eta$ with target $t(\eta) = F$.\\

\noindent In line 2, the list $\mathtt{summands}$ is initially empty.\\
In line 7, we add a morphism $\eta$ from the $\mathtt{queue}$ to the list $\mathtt{summands}$ only if
in line 6 we checked that $s(\eta)$ is indecomposable.\\
Thus in each step of the algorithm the list $\mathtt{summands}$ only contains indecomposable morphisms with target $F$.\\

\noindent Initially with $\mathtt{queue} = [1_{F}]$ and $\mathtt{summands} = \emptyset$ we have
\begin{align}
F = \label{eq:direct_sum_decomposition}
\bigoplus_{\eta \in \mathtt{queue}} s(\eta) \oplus \bigoplus_{\eta \in \mathtt{summands}} s(\eta)
\end{align}
For the first run of the while loop we take $\eta_{1} := 1_{F}$ from the $\mathtt{queue}$. Then there are two possibilities:\\
If $F$ was indecomposable, we now add $\eta_{1}$ to $\mathtt{summands}$, and have $\mathtt{queue} = \emptyset$ and
$\mathtt{summands} = [1_{F}]$ which also satisfies \eqref{eq:direct_sum_decomposition}.\\
Otherwise we get a decomposition of $F$ with $\iota_{1} : I_{1} \rightarrow F$ and $\kappa_{1} : K_{1} \rightarrow F$. In this case
$\mathtt{summands}$ stays empty, and instead we have $\mathtt{queue} = [\iota_{1}\eta_{1}, \kappa_{1}\eta_{1}]$. For
\eqref{eq:direct_sum_decomposition} to hold, we need to prove that
\[
I_{1} \oplus K_{1} = F
\]
which is exactly what we assert for the output of $\mathtt{DecomposeOnceByRandomEndomorphism}$.\\
Thus after the first while loop, equation \eqref{eq:direct_sum_decomposition} holds.\\
In each run of the while loop, we are replacing $I_{j}$ with $I_{j,1}$ and $K_{j,1}$ with $I_{j} = I_{j,1} \oplus K_{j,1}$ and
$K_{j}$ with $I_{j,2} \oplus K_{j,2}$. That is we have 
\begin{alignat}{5}
F &=  &&I_{1} &&\oplus &&K_{1} \\
&= (I_{11} &&\oplus K_{11}) &&\oplus (I_{12} &&\oplus K_{12})
\end{alignat}

\begin{subproof}[Proof that the $\mathtt{queue}$ will be empty]\phantom{}\\
\noindent In each run of the while loop, a morphism $\eta$ from the $\mathtt{queue}$ gets either deleted and not replaced,
since $s(\eta)$ was indecomposable, or it gets deleted and replaced by two morphisms $[\iota\eta, \kappa\eta]$.\\
If at some point, $s(\eta)$ is indecomposable for every $\eta \in \mathtt{queue}$, then the $\eta$ will be deleted from the
$\mathtt{queue}$ (and added to $\mathtt{summands}$) until the $\mathtt{queue}$ is empty. Then the while loop will end.\\

\noindent The case that each $\eta$ gets replaced by $[\iota\eta, \kappa\eta]$ where again the $s(\iota\eta)$ and $s(\kappa\eta)$ are
decomposable must come to an end after a finite number of steps:\\

\noindent Whenever $G := s(\eta)$ is decomposable with $I \oplus K = G$ and $I, K \neq 0$, we have
%informeller: Invariante d_G fällt bei echter Zerlegung: 0 < d_I < d_G, 0 < d_K < d_G
\begin{align*}
Gc = (I \oplus K)c = Ic \oplus Kc = Ic + Kc \,\forall c \in \mathcal{A}_{0} \\
d_{G} :=\max_{c \in \mathcal{A}_{0}} Gc = \max_{c \in \mathcal{A}_{0}} Ic + Kc \\
d_{I} := \max_{c \in \mathcal{A}_{0}} Ic > 0 \\
d_{K} := \max_{c \in \mathcal{A}_{0}} Kc > 0 \\
0 < d_{I} \leq d_{G} \leq d_{I} + d_{K} \\
0 < d_{K} \leq d_{G} \leq d_{I} + d_{K} \\
\Rightarrow 0 < d_{I} < d_{G} \\
\Rightarrow 0 < d_{K} < d_{G}
\end{align*}
Thus with each step where $\eta : G \rightarrow F$ gets replaced with $\iota\eta : I \rightarrow F$ and $\kappa\eta : K \rightarrow F$,
both $d_{I} < d_{G}$ and $d_{K} < d_{G}$,
and both have the lower bound of $0$. Thus after a finite number of steps there are only $\eta$ with indecomposable $s(\eta)$ in the $\mathtt{queue}$
so that by the above result, $\mathtt{queue}$ will eventually become empty.
\end{subproof}

\noindent With the $\mathtt{queue} = \emptyset$ and $\mathtt{summands}$ containing only $\eta$ with indecomposable $s(\eta)$,
equation \eqref{eq:direct_sum_decomposition} becomes
\begin{align}
F = \bigoplus_{\eta \in \mathtt{summands}} s(\eta)
\end{align}
\end{subproof}

\end{proof}


\[
\begin{tikzcd}
                                                          &                                                                                   &                                                          & F                                                      &                                                           &                                                                                       &                                                          \\
                                                          & F \arrow[rru, "\eta_{1}", bend left]                                              &                                                          &                                                        &                                                           & F \arrow[llu, "\eta_{1}"', bend right]                                                &                                                          \\
                                                          & I_{1} \arrow[u, "\iota_{1}"] \arrow[rr, dash] \arrow[rruu, "\iota_{1}\eta_{1}"] &                                                          & \bigoplus \arrow[uu, "I_{1} \oplus K_{1}" description] &                                                           & K_{1} \arrow[u, "\kappa_{1}"'] \arrow[ll, dash] \arrow[lluu, "\kappa_{1}\eta_{1}"'] &                                                          \\
I_{1} \arrow[ru, "\eta_{11}"', bend left=49, shift right] &                                                                                   & I_{1} \arrow[lu, "\eta_{11}", bend right=49, shift left] &                                                        & K_{1} \arrow[ru, "\eta_{12}"', bend left=49, shift right] &                                                                                       & K_{1} \arrow[lu, "\eta_{12}", bend right=49, shift left] \\
I_{11} \arrow[u, "\iota_{11}"] \arrow[r, dash]          & \bigoplus \arrow[uu, "I_{11}\oplus K_{11}" description]                           & K_{11} \arrow[u, "\kappa_{11}"] \arrow[l, dash]        &                                                        & I_{12} \arrow[u, "\iota_{12}"] \arrow[r, dash]          & \bigoplus \arrow[uu, "I_{12} \oplus K_{12}" description]                              & K_{12} \arrow[u, "\kappa_{12}"'] \arrow[l, dash]      
\end{tikzcd}
\]



Invariant:

In each step:

\[
F = \bigoplus_{\eta \in \mathtt{queue}} s(\eta) \oplus \bigoplus_{\varphi \in \mathtt{summands}} s(\varphi).
\]

When $\mathtt{queue} = \emptyset$, then
\[
F = \bigoplus_{\varphi \in \mathtt{summands}} s(\varphi).
\]

Why is $\mathtt{queue}$ going to be $\emptyset$?

$5 = 4+1 = 2+2+1 = 2+2+1$.

In each step we remove the last $\eta$ from the $\mathtt{queue}$. If $s(\eta)$ was indecomposable, it doesn't get replaced, i.e.
$\mathtt{queue}$ becomes smaller by 1. If $s(\eta)$ was decomposable, we replace $\eta$ by $\iota\eta$ and $\kappa\eta$ with

\begin{align}
\iota : I \rightarrow s(\eta) \\
\kappa : K \rightarrow s(\eta)
\end{align}
and in some sense
\begin{align}
I \preceq s(\eta) \\
K \preceq s(\eta)
\end{align}
Which sense?
\begin{align}
I \oplus K &= s(\eta) \\
s(\eta)(i) &= (I \oplus K)(i) = I(i) + K(i)\, \forall i \in \mathcal{A}_{0}
\end{align}
Since $\iota \neq 0$ and $\kappa \neq 0$, $I \neq 0$ and $K \neq 0$, so for some objects $i,j \in \mathcal{A}_{0}$ we have
$I(i) > 0$ and $K(j) > 0$, and thus $s(\eta)(i) > 0$ for some $i$.

For all objects $i\in \mathcal{A}_{0}$ $I(i) \leq s(\eta)(i)$ and $K(i) \leq s(\eta)(i)$.

$\mathtt{queue} := \{1_{F}\} \mapsto \{\iota 1_{F}, \kappa 1_{F}\} = \{\iota, \kappa \}
\mapsto \{ \iota_{1}\iota, \kappa_{1}\iota, \iota_{2}\kappa, \kappa_{2}\kappa \}$


Input: Functor $F$.\\
Output: Tupel $(\mathrm{ImageEmbedding}( \alpha ), \mathrm{KernelEmbedding}( \alpha ) )$

such that
\[
\begin{tikzcd}
                                       & F &                                         \\
s(\eta) \arrow[ru, "\eta", bend left=49]     &   & s(\eta) \arrow[lu, "\eta"', bend right=49]    \\
I \arrow[u, "\iota"] &   & K \arrow[u, "\kappa"']
\end{tikzcd}
\]
and
\[
F = I \oplus K
\]

When we input I and K into DecomposeOnce, we get a further decomposition:
\[
\begin{tikzcd}
                                                 &                                                &                                                  & F &                                                  &                                                 &                                                   \\
                                                 & F \arrow[rru, "\eta_{1}", bend left]           &                                                  &   &                                                  & F \arrow[llu, "\eta_{1}"', bend right]          &                                                   \\
                                                 & I_{1} \arrow[u, "\mathrm{ImgEmb}(\alpha_{1})"] &                                                  &   &                                                  & K_{1} \arrow[u, "\mathrm{KerEmb}(\alpha_{1})"'] &                                                   \\
I_{1} \arrow[ru, "\eta_{11}", bend left=49]      &                                                & I_{1} \arrow[lu, "\eta_{11}", bend right=49]     &   & K_{1} \arrow[ru, "\eta_{12}", bend left=49]      &                                                 & K_{1} \arrow[lu, "\eta_{12}", bend right=49]      \\
I_{11} \arrow[u, "\mathrm{ImgEmb}(\alpha_{11})"] &                                                & K_{11} \arrow[u, "\mathrm{KerEmb}(\alpha_{11})"] &   & I_{12} \arrow[u, "\mathrm{ImgEmb}(\alpha_{12})"] &                                                 & K_{12} \arrow[u, "\mathrm{KerEmb}(\alpha_{12})"']
\end{tikzcd}
\]

The direct sum decomposition in each step looks like this:

\[
\begin{tikzcd}
                                                                   &                                                                   &                                                                    & F                                                      &                                                                    &                                                                    &                                                                     \\
                                                                   & F \arrow[rru, "\eta_{1}", bend left]                              &                                                                    &                                                        &                                                                    & F \arrow[llu, "\eta_{1}"', bend right]                             &                                                                     \\
                                                                   & I_{1} \arrow[u, "\mathrm{ImgEmb}(\alpha_{1})"] \arrow[rr, dash] &                                                                    & \bigoplus \arrow[uu, "I_{1} \oplus K_{1}" description] &                                                                    & K_{1} \arrow[u, "\mathrm{KerEmb}(\alpha_{1})"'] \arrow[ll, dash] &                                                                     \\
I_{1} \arrow[ru, "\eta_{11}"', bend left=49, shift right]           &                                                                   & I_{1} \arrow[lu, "\eta_{11}", bend right=49, shift left]           &                                                        & K_{1} \arrow[ru, "\eta_{12}"', bend left=49, shift right]                        &                                                                    & K_{1} \arrow[lu, "\eta_{12}", bend right=49, shift left]                        \\
I_{11} \arrow[u, "\mathrm{ImgEmb}(\alpha_{11})"] \arrow[r, dash] & \bigoplus \arrow[uu, "I_{11}\oplus K_{11}" description]           & K_{11} \arrow[u, "\mathrm{KerEmb}(\alpha_{11})"] \arrow[l, dash] &                                                        & I_{12} \arrow[u, "\mathrm{ImgEmb}(\alpha_{12})"] \arrow[r, dash] & \bigoplus \arrow[uu, "I_{12} \oplus K_{12}" description]           & K_{12} \arrow[u, "\mathrm{KerEmb}(\alpha_{12})"'] \arrow[l, dash]
\end{tikzcd}
\]
That is we have 
\begin{alignat}{5}
F &=  &&I_{1} &&\oplus &&K_{1} \\
F &= (I_{11} &&\oplus K_{11}) &&\oplus (I_{12} &&\oplus K_{12})
\end{alignat}




\begin{enumerate}
\renewcommand{\labelenumi}{(\theenumi)}
\item Let $F \neq 0$ be a functor in $\HomAkmat$. Then there is a maximum dimension $d := \max_{i \in \mathcal{A}_{0}} \mathrm{dim}\, F(i)$ with
$d \geq 1$.
\item The vector space $\mathrm{Hom}_{\HomAkmat}(F,F) = \mathrm{End}_{\HomAkmat}(F)$ is finite-dimensional, i.e. we can find a
finite basis $\mathtt{endbas}$ of our external hom using the algorithm $\mathtt{BasisOfExternalHom}(F,F)$. Since $F \neq 0$, the
identity morphism $1_{F}$ 
\item If $\mathrm{End}_{\HomAkmat}(F)$ has dimension 1, i.e. each morphism in $\mathrm{End}_{\HomAkmat}(F)$ is of the form
$\lambda \alpha$ for some $\lambda \in \Bbbk$ and the one $\alpha \in \mathtt{endbas}$, then $\alpha = 1_{F}$ is the identity morphism.
In this case, 

The process $\alpha \mapsto \alpha^{2}$ will end when for some $k \in \mathbb{N},\, \alpha^{2^{k+1}} = \alpha^{2^{k}},$ i.e.
$\alpha^{2^{k}+2^{k}} = \alpha^{2^{k}}$ which is the $\sigma$-lemma with $m = n = 2^{k}$.

\end{enumerate}

The algorithm $\mathtt{DecomposeOnceByRandomEndomorphism}$ will terminate without $\mathtt{fail}$ only if
\[
\mathrm{ImgEmb}(\alpha) \neq 0\,\text{ and }\, \mathrm{KerEmb}(\alpha) \neq 0
\]
for some $\alpha$.

When we have $\mathrm{ImgEmb}(\alpha) = 0$ or $\mathrm{KerEmb}(\alpha) = 0$ for all $\alpha : F \rightarrow F$, we have
\begin{alignat}{4}
&F &&= \{0\} &&\oplus K\quad\text{ or }\\
&F &&= I &&\oplus \{0\},
\end{alignat}
i.e. F was indecomposable.



\begin{tikzcd}
F1 \arrow["Fa"', loop, distance=2em, in=125, out=55] \arrow[rr, "Fb"] \arrow[dd, "\eta_{1}"]
&  & F2 \arrow["Fc"', loop, distance=2em, in=125, out=55] \arrow[dd, "\eta_{2}"] \\
&  &                                                                                 \\
G1 \arrow["Ga"', loop, distance=2em, in=305, out=235] \arrow[rr, "Gb"]
&  & G2 \arrow["Gc"', loop, distance=2em, in=305, out=235]
\end{tikzcd}

\begin{align}
Fa\,\eta_{1} &= \eta_{1} Ga \\
Fb\,\eta_{2} &= \eta_{1} Gb \\
Fc\,\eta_{2} &= \eta_{2} Gc
\end{align}

For the first equation we have

\begin{align}
Fa\,\eta_{1} - \eta_{1} Ga = 0
\end{align}

thus

\begin{align}
\left( Fa\,\eta_{1} - \eta_{1} Ga \right)_{i,j} = 0,\, 1\leq i \leq 5, 1\leq j \leq 3
\end{align}

Sylvester equations

\texttt{BasisOfExternalHom}
\texttt{WeakDirectSumDecomposition}

InstallMethod( DecomposeOnceByRandomEndomorphism,
        "for an object in a Hom-category",
        [ IsCapCategoryObjectInHomCategory ],
        
  function( F )
    local d, n, endbas, k, b, alpha, i, alpha2, keremb;
    
    d := Maximum( List( ValuesOnAllObjects( F ), Dimension ) );
    
    endbas := ShallowCopy( BasisOfExternalHom( F, F ) );
    
    Add( endbas, ZeroMorphism( F, F ) );
    
    k := CommutativeRingOfLinearCategory( CapCategory( F ) );
    
    n := Int( Log2( Float( d ) ) ) + 1;
    
    for b in Reversed( [ 2 .. Length( endbas ) ] ) do
        
        alpha := endbas[b] + Random( k ) * endbas[b-1];
        
        SetFilterObj( alpha, IsMultiplicativeElementWithInverse );
        
        for i in [ 1 .. n ] do
            alpha2 := PreCompose( alpha, alpha );
            if IsCongruentForMorphisms( alpha, alpha2 ) then
                break;
            fi;
            alpha := alpha2;
        od;
        
        if IsZero( alpha ) then
            continue;
        fi;
        
        keremb := KernelEmbedding( alpha );
        
        if IsZero( keremb ) then
            continue;
        fi;
        
        return [ ImageEmbedding( alpha ), keremb ];
        
    od;
    
    return fail;
    
end );


InstallMethod( WeakDirectSumDecomposition,
        "for an object in a Hom-category",
        [ IsCapCategoryObjectInHomCategory ],
        
  function( F )
    local queue, summands, eta, result;
    
    queue := [ IdentityMorphism( F ) ];
    
    summands := [ ];
    
    while not IsEmpty( queue ) do
        
        eta := Remove( queue );
        
        result := DecomposeOnceByRandomEndomorphism( Source( eta ) );
        
        if result = fail then
            Add( summands, eta );
        else
            Append( queue, List( result, emb -> PreCompose( emb, eta ) ) );
        fi;
        
    od;
    
    return summands;
    
end );






















\section{Conclusion}

%% best endnotes are in Westend-Verlag style: 
%% All at the end, sorted by chapter, starting from 1 at each new chapter.

%%% insert endnotes in some way
\begingroup
     \parindent 0pt
     \parskip 2ex
     \def\enotesize{\normalsize}
     \theendnotes
\endgroup 

\addcontentsline{toc}{section}{References}
\input{bib/sources.bib}

\appendix
\renewcommand{\thesection}{\Alph{section}}
\section{Implementation in \textsc{Cap}}

\lstlistoflistings\label{lol}

\renewcommand{\lstlistingname}{Procedure}
\lstset{
		basicstyle=\ttfamily\small,
		keywordstyle=\color{red},
		identifierstyle=,
		commentstyle=\color{green!70!black},
		stringstyle=\color{blue},
		showstringspaces=false,
		gobble=2,
		columns=fullflexible,
		tabsize=4,
		numbers=none, 
		frame=single}

%%% Check firstline and lastline for each function after change in the codebase ! ! !
%%% 1st line = InstallMethod
%%% last line = end );

%%% ConvertToMapOfFinSets
\lstinputlisting[
		firstline=8,
		lastline=33, 
		label=func:ConvertToMapOfFinSets,
		caption={$\mathtt{ConvertToMapOfFinSets}$},
		language=GAP
		]{\pkgpath/catreps/gap/CatRepsWithCAP.gi}
From the package \texttt{CatReps} (d$\And$i by Tibor Grün and improved by Mohamed Barakat on 12 May 2020) \\
Back to \hyperref[lol]{Index}

%%% ConcreteCategoryForCAP
\lstinputlisting[
		firstline=36,
		lastline=81, 
		label=func:ConcreteCategoryForCAP,
		caption={$\mathtt{ConcreteCategoryForCAP}$},
		language=GAP
		]{\pkgpath/catreps/gap/CatRepsWithCAP.gi}
From the package \texttt{CatReps} (d$\And$i by Mohamed Barakat on 20 Feb 2020) \\
Back to \hyperref[lol]{Index}
		
%%% RightQuiverFromConcreteCategory
\lstinputlisting[
		firstline=228,
		lastline=255, 
		label=func:RightQuiverFromConcreteCategory,
		caption={$\mathtt{RightQuiverFromConcreteCategory}$},
		language=GAP
		]{\pkgpath/catreps/gap/CatRepsWithCAP.gi}
From the package \texttt{CatReps} (d$\And$i by Tibor Grün on 7 May 2020) \\
Back to \hyperref[lol]{Index}
		
%%% RelationsOfEndomorphisms
\lstinputlisting[
		firstline=156,
		lastline=225, 
		label=func:RelationsOfEndomorphisms,
		caption={$\mathtt{RelationsOfEndomorphisms}$},
		language=GAP
		]{\pkgpath/catreps/gap/CatRepsWithCAP.gi}
From the package \texttt{CatReps} (d$\And$i by Tibor Grün on 7 May 2020) \\
Back to \hyperref[lol]{Index}
		
%%% Algebroid
\lstinputlisting[
		firstline=84,
		lastline=153, 
		label=func:Algebroid,
		caption={$\mathtt{Algebroid}$},
		language=GAP
		]{\pkgpath/catreps/gap/CatRepsWithCAP.gi}
From the package \texttt{CatReps} (template added by Mohamed Barakat on 16 April 2020, finalized by Tibor Grün on 7 May 2020) \\
Back to \hyperref[lol]{Index}
		
%%% YonedaEmbedding 
\lstinputlisting[
		firstline=199,
		lastline=251, 
		label=func:YonedaEmbedding,
		caption={$\mathtt{YonedaEmbedding}$},
		language=GAP
		]{\pkgpath/FunctorCategories/gap/Functors.gi}
From the package \texttt{FunctorCategories} (d$\And$i by Kamal Saleh on 18 May 2020) \\
Back to \hyperref[lol]{Index}

%%% DecomposeOnceByRandomEndomorphism
\lstinputlisting[
		firstline=8,
		lastline=66, 
		label=func:DecomposeOnceByRandomEndomorphism,
		caption={$\mathtt{DecomposeOnceByRandomEndomorphism}$},
		language=GAP
		]{\pkgpath/FunctorCategories/gap/DirectSumDecomposition.gi}
From the package \texttt{FunctorCategories}\\
Back to \hyperref[lol]{Index}
		
%%% WeakDirectSumDecomposition
\lstinputlisting[
		firstline=69,
		lastline=96, 
		label=func:WeakDirectSumDecomposition,
		caption={$\mathtt{WeakDirectSumDecomposition}$},
		language=GAP
		]{\pkgpath/FunctorCategories/gap/DirectSumDecomposition.gi}
From the package \texttt{FunctorCategories}\\
Back to \hyperref[lol]{Index}
		
%%% MorphismOntoSumOfImagesOfAllMorphisms
\lstinputlisting[
		firstline=245,
		lastline=263,
		breaklines=true,
		label=func:MorphismOntoSumOfImagesOfAllMorphisms,
		caption={$\mathtt{MorphismOntoSumOfImagesOfAllMorphisms}$},
		language=GAP
		]{\pkgpath/CategoryConstructor/gap/Tools.gi}
From the package \texttt{CategoryConstructor} (d$\And$i by Mohamed Barakat on 10 April 2020) \\
Back to \hyperref[lol]{Index}
		
%%% EmbeddingOfSumOfImagesOfAllMorphisms
\lstinputlisting[
		firstline=266,
		lastline=276,
		breaklines=true,
		label=func:EmbeddingOfSumOfImagesOfAllMorphisms,
		caption={$\mathtt{EmbeddingOfSumOfImagesOfAllMorphisms}$},
		language=GAP
		]{\pkgpath/CategoryConstructor/gap/Tools.gi}
From the package \texttt{CategoryConstructor} (d$\And$i by Mohamed Barakat on 10 April 2020)\\
Back to \hyperref[lol]{Index}
		
%%% SumOfImagesOfAllMorphisms
\lstinputlisting[
		firstline=279,
		lastline=288,
		breaklines=true,
		label=func:SumOfImagesOfAllMorphisms,
		caption={$\mathtt{SumOfImagesOfAllMorphisms}$},
		language=GAP
		]{\pkgpath/CategoryConstructor/gap/Tools.gi}
From the package \texttt{CategoryConstructor} (d$\And$i by Mohamed Barakat on 10 April 2020)\\
Back to \hyperref[lol]{Index}

		


\end{document}