\documentclass{article}
\usepackage[top=37mm,bottom=37mm,left=27mm,right=27mm]{geometry}

\usepackage[utf8]{inputenc}
\usepackage{fancyvrb}

%%% For captions and references
\usepackage{hyperref}
\usepackage{hypcap}
\newcommand{\Algoref}[1]{%
	\hyperref[algo:#1]{Algorithm~\ref*{algo:#1}}%
}
\newcommand{\algoref}[1]{%
	\hyperref[algo:#1]{Algorithm~\ref*{algo:#1}}%
}
\newcommand{\Funcref}[1]{%
	\hyperref[func:#1]{Function~\ref*{func:#1}}%
}
\newcommand{\funcref}[1]{%
	\hyperref[func:#1]{\texttt{#1}}%
}

%%% For quotation
\usepackage{csquotes}

\usepackage{xcolor}
\usepackage{color}
\definecolor{FireBrick}{rgb}{0.5812,0.0074,0.0083}
\definecolor{RoyalBlue}{rgb}{0.0236,0.0894,0.6179}
\definecolor{RoyalGreen}{rgb}{0.0236,0.6179,0.0894}
\definecolor{RoyalRed}{rgb}{0.6179,0.0236,0.0894}
\definecolor{LightBlue}{rgb}{0.8544,0.9511,1.0000}
\definecolor{Black}{rgb}{0.0,0.0,0.0}

\definecolor{linkColor}{rgb}{0.0,0.0,0.554}
\definecolor{citeColor}{rgb}{0.0,0.0,0.554}
\definecolor{fileColor}{rgb}{0.0,0.0,0.554}
\definecolor{urlColor}{rgb}{0.0,0.0,0.554}
\definecolor{promptColor}{rgb}{0.0,0.0,0.589}
\definecolor{brkpromptColor}{rgb}{0.589,0.0,0.0}
\definecolor{gapinputColor}{rgb}{0.589,0.0,0.0}
\definecolor{gapoutputColor}{rgb}{0.0,0.0,0.0}

%%  for a long time these were red and blue by default,
%%  now black, but keep variables to overwrite
\definecolor{FuncColor}{rgb}{0.0,0.0,0.0}
%% strange name because of pdflatex bug:
\definecolor{Chapter }{rgb}{0.0,0.0,0.0}
\definecolor{DarkOlive}{rgb}{0.1047,0.2412,0.0064}

%% command for ColorPrompt style examples
\newcommand{\gapprompt}[1]{\color{promptColor}{\bfseries #1}}
\newcommand{\gapbrkprompt}[1]{\color{brkpromptColor}{\bfseries #1}}
\newcommand{\gapinput}[1]{\color{gapinputColor}{#1}}

%%% For source code listings
\usepackage{listings}[2013/08/05]
\def \pkgpath {C:/Users/Tibor/AppData/Local/Packages/CanonicalGroupLimited.UbuntuonWindows_79rhkp1fndgsc/LocalState/rootfs/home/user/.gap/pkg}
%\lstloadlanguages{GAP}

%%% For algorithm styles
\usepackage[linesnumbered,ruled]{algorithm2e}

%%% Math theorem styles
\usepackage{amsthm}

\newtheorem{thm}{Theorem}[subsection]
\newtheorem{lemma}[thm]{Lemma}
\theoremstyle{definition}
\newtheorem{definition}[thm]{Definition}
\newtheorem{remark}[thm]{Remark}
\newtheorem{example}[thm]{Example}

\begin{document}
\tableofcontents\label{toc}
\section{Preface}
\section{Introduction to quivers and category theory}

\section{Datatype convention of catreps}
Since the goal of this thesis is a translation of the package \texttt{catreps} by Peter Webb et al. into CAP, this section is
a short overview of the package catreps.

\blockquote[\cite{[Webb2020]}]{In this package a category is stored as a concrete category (i.e. a category where the objects are sets and
morphisms are maps of sets).
A category is stored as a record (cat, say) with fields cat.objects, cat.generators, cat.domain, cat.codomain.
Each object in the list cat.object is a set, and each morphism in the list of generator morphisms cat.generators
is stored as a mapping of sets, which we notate as the list of its values.}

\begin{Verbatim}[commandchars=!@|,fontsize=\small,frame=single,label=Example]
  !gapprompt@gap>| !gapinput@c3c3 := ConcreteCategory( [ [2,3,1], [4,5,6], [,,,5,6,4] ] );|
  rec( codomain := [ 1, 2, 2 ], domain := [ 1, 1, 2 ],
       generators := [ [ 2, 3, 1 ], [ 4, 5, 6 ], [ ,,, 5, 6, 4 ] ],
       objects := [ [ 1, 2, 3 ], [ 4, 5, 6 ] ], operations := rec(  ) )
\end{Verbatim}

\noindent The list of values as seen in the example above may be easy to type in, but does have its disadvantages: If for example you want to store the
morphism that maps the set $\{9\}$ to itself, i.e. the identity morphism $1_{\{9\}}$, you first have to write the eight commas that are not part of that
morphism definition \texttt{ [ ,,,,,,,,9 ] } and you might make a mistake by forgetting one comma.
Another issue is that the source object of a morphism \texttt{gen} is only implicitly given by those list entries \texttt{i} for which
\texttt{ IsBound( gen[i] ) = true }.

Using instead \texttt{MapOfFinSets} in \textsc{Cap} solves both of these issues, and it lets us use a different model for concrete categories in \textsc{Cap},
i.e. that of a subcategory of \texttt{FinSets}, for which we already have an implementation in \textsc{Cap}. 
Another advantages of this method is that a \texttt{MapOfFinSets} can cache known properties about itself:

\begin{Verbatim}[commandchars=!@|,fontsize=\small,frame=single,label=Example]
  !gapprompt@gap>| !gapinput@S := FinSet( [1,2,3] );|
  <An object in FinSets>
  !gapprompt@gap>| !gapinput@T := FinSet( [4,5,6] );|
  <An object in FinSets>
  !gapprompt@gap>| !gapinput@map1 := MapOfFinSets( S, [ [1,1], [2,2], [3,3] ], S );|
  <A morphism in FinSets>
  !gapprompt@gap>| !gapinput@IsAutomorphism( map1 );|
  true
  !gapprompt@gap>| !gapinput@map1;|
  <An automorphism in FinSets>
\end{Verbatim}

Going further in the cited example,\\
\blockquote[\cite{[Webb2020]}]{The following constructs a representation:}
\begin{Verbatim}[commandchars=!@|,fontsize=\small,frame=single,label=Example]
  !gapprompt@gap>| !gapinput@one:=One(GF(3));;|
  !gapprompt@gap>| !gapinput@d:=[[1,1,0,0,0],[0,1,1,0,0],[0,0,1,0,0],[0,0,0,1,1],[0,0,0,0,1]]*one;;|
  !gapprompt@gap>| !gapinput@e:=[[0,1,0,0],[0,0,1,0],[0,0,0,0],[0,1,0,1],[0,0,1,0]]*one;;|
  !gapprompt@gap>| !gapinput@f:=[[1,1,0,0],[0,1,1,0],[0,0,1,0],[0,0,0,1]]*one;;|
  !gapprompt@gap>| !gapinput@nine:=CatRep(c3c3,[d,e,f],GF(3));|
  rec(
category := rec( generators := [ [ 2, 3, 1 ], [ 4, 5, 6 ], [ ,,, 5, 6, 4 ] ]
, operations := rec( ), objects := [ [ 1, 2, 3 ], [ 4, 5, 6 ] ],
domain := [ 1, 1, 2 ], codomain := [ 1, 2, 2 ] ),
genimages := [ [ [ Z(3)^0, Z(3)^0, 0*Z(3), 0*Z(3), 0*Z(3) ],
[ 0*Z(3), Z(3)^0, Z(3)^0, 0*Z(3), 0*Z(3) ],
[ 0*Z(3), 0*Z(3), Z(3)^0, 0*Z(3), 0*Z(3) ],
[ 0*Z(3), 0*Z(3), 0*Z(3), Z(3)^0, Z(3)^0 ],
[ 0*Z(3), 0*Z(3), 0*Z(3), 0*Z(3), Z(3)^0 ] ],
[ [ 0*Z(3), Z(3)^0, 0*Z(3), 0*Z(3) ], [ 0*Z(3), 0*Z(3), Z(3)^0, 0*Z(3) ]
, [ 0*Z(3), 0*Z(3), 0*Z(3), 0*Z(3) ],
[ 0*Z(3), Z(3)^0, 0*Z(3), Z(3)^0 ],
[ 0*Z(3), 0*Z(3), Z(3)^0, 0*Z(3) ] ],
[ [ Z(3)^0, Z(3)^0, 0*Z(3), 0*Z(3) ], [ 0*Z(3), Z(3)^0, Z(3)^0, 0*Z(3) ]
, [ 0*Z(3), 0*Z(3), Z(3)^0, 0*Z(3) ],
[ 0*Z(3), 0*Z(3), 0*Z(3), Z(3)^0 ] ] ], field := GF(3),
dimension := [ 5, 4 ] )
\end{Verbatim}

we see that \texttt{catreps} works with \textsc{Gap} matrices directly whereas with \textsc{Cap} we use \texttt{HomalgMatrix} and
\texttt{RingsForHomalg} which lets us delegate computation to faster computer algebra systems like \texttt{Singular} or \texttt{Magma}.
What is also noticable is the big chunk of output we get as a result of \texttt{CatRep(c3c3,[d,e,f],GF(3))}. In \textsc{Cap} we hide the output
and give a short description of the resulting object or morphism, and use the \texttt{Display} function to display the whole result.

All in all, there are plenty of reasons to change to \textsc{Cap}. In the meantime, in order to still support inputs in the convention of
\texttt{catreps}, I wrote a converter function \funcref{ConvertToMapOfFinSets}.

\section{The category FinSets}

There are algorithms whose sole purpose is to convert data structures, so they are not of much interest to the mathematical theory,
and then there are algorithms that implement our category theoretical calculations, so they are important to our theory.

The algorithms \hyperref[func:ConvertToMapOfFinSets]{\texttt{ConvertToMapOfFinSets}}, 
\hyperref[func:ConcreteCategoryForCAP]{\texttt{ConcreteCategoryForCAP}} and 
\hyperref[func:RightQuiverFromConcreteCategory]{\texttt{RightQuiverFromConcreteCategory}} are
more of the data structure conversion type, while 
\hyperref[func:RelationsOfEndomorphisms]{\texttt{RelationsOfEndomorphisms}}, 
\hyperref[func:Algebroid]{\texttt{Algebroid}},
\hyperref[func:EmbeddingOfSubRepresentation]{\texttt{EmbeddingOfSubRepresentation}} and 
\hyperref[func:WeakDirectSumDecomposition]{\texttt{WeakDirectSumDecomposition}} are also important to our theory.

\subsection{MapOfFinSets}

\begin{algorithm}\capstart
   \caption{\texttt{ConvertToMapOfFinSets}}\label{algo:ConvertToMapOfFinSets}
      \SetKwInput{Input}{Input~}
      \SetKwInput{Output}{Output~}
      \Input{~a list $objects$ of objects in FinSets and a morphism $gen$ given as a list of images in the convention of catreps}
      \Output{~the corresponding map of finite sets from source $S$ to target $T$}
      \BlankLine
      let $T$ be the first object $O \in objects$ such that $gen \cap O \not= \emptyset$\;
      \If{$gen \cap O = \emptyset \, \forall O \in objects$}{
         Error "unable to find target set"
      }
      let $fl$ be the flattening of $objects$ as a list\;
      let $S$ be the sublist of $fl$ according to positions $i$ such that $gen[i]$ is bound\;
      set $S$ to be the first object $O \in objects$ such that $O = S$\;
      \If{$S \not= O \, \forall O \in objects$}{
          Error "unable to find source set"
      }
      \BlankLine
      let $G$ be the list of pairs $[ i, gen[i] ], i \in S$;
      \BlankLine
      \Return MapOfFinSets( S, G, T );
\end{algorithm}

\section{The categories Functor-Categories and Cat-Reps}
\section{Conclusion}

\addcontentsline{toc}{section}{References}
\input{bib/sources.bib}

\appendix
\renewcommand{\thesection}{\Alph{section}}
\section{Implementation in \textsc{Cap}}

\lstlistoflistings\label{lol}

\renewcommand{\lstlistingname}{Procedure}
\lstset{
		basicstyle=\ttfamily\small,
		keywordstyle=\color{red},
		identifierstyle=,
		commentstyle=\color{green!70!black},
		stringstyle=\color{blue},
		showstringspaces=false,
		gobble=2,
		columns=fullflexible,
		tabsize=4,
		numbers=none, 
		frame=single}

%%% Check firstline and lastline for each function after change in the codebase ! ! !
%%% 1st line = InstallMethod
%%% last line = end );

%%% ConvertToMapOfFinSets
\lstinputlisting[
		firstline=8,
		lastline=33, 
		label=func:ConvertToMapOfFinSets,
		caption={$\mathtt{ConvertToMapOfFinSets}$},
		language=GAP
		]{\pkgpath/catreps/gap/CatRepsWithCAP.gi}
Back to \hyperref[lol]{Index}

%%% ConcreteCategoryForCAP
\lstinputlisting[
		firstline=36,
		lastline=81, 
		label=func:ConcreteCategoryForCAP,
		caption={$\mathtt{ConcreteCategoryForCAP}$},
		language=GAP
		]{\pkgpath/catreps/gap/CatRepsWithCAP.gi}
Back to \hyperref[lol]{Index}
		
%%% RightQuiverFromConcreteCategory
\lstinputlisting[
		firstline=228,
		lastline=255, 
		label=func:RightQuiverFromConcreteCategory,
		caption={$\mathtt{RightQuiverFromConcreteCategory}$},
		language=GAP
		]{\pkgpath/catreps/gap/CatRepsWithCAP.gi}
Back to \hyperref[lol]{Index}
		
%%% RelationsOfEndomorphisms
\lstinputlisting[
		firstline=156,
		lastline=225, 
		label=func:RelationsOfEndomorphisms,
		caption={$\mathtt{RelationsOfEndomorphisms}$},
		language=GAP
		]{\pkgpath/catreps/gap/CatRepsWithCAP.gi}
Back to \hyperref[lol]{Index}
		
%%% Algebroid
\lstinputlisting[
		firstline=84,
		lastline=153, 
		label=func:Algebroid,
		caption={$\mathtt{Algebroid}$},
		language=GAP
		]{\pkgpath/catreps/gap/CatRepsWithCAP.gi}
Back to \hyperref[lol]{Index}
		
%%% YonedaEmbedding 
\lstinputlisting[
		firstline=199,
		lastline=251, 
		label=func:YonedaEmbedding,
		caption={$\mathtt{YonedaEmbedding}$},
		language=GAP
		]{\pkgpath/FunctorCategories/gap/Functors.gi}
From the package \texttt{FunctorCategories} (defined and implemented by Kamal Saleh on 18 May 2020) \\
Back to \hyperref[lol]{Index}

%%% DecomposeOnceByRandomEndomorphism
\lstinputlisting[
		firstline=8,
		lastline=66, 
		label=func:DecomposeOnceByRandomEndomorphism,
		caption={$\mathtt{DecomposeOnceByRandomEndomorphism}$},
		language=GAP
		]{\pkgpath/FunctorCategories/gap/DirectSumDecomposition.gi}
From the package \texttt{FunctorCategories}\\
Back to \hyperref[lol]{Index}
		
%%% WeakDirectSumDecomposition
\lstinputlisting[
		firstline=65,
		lastline=92, 
		label=func:WeakDirectSumDecomposition,
		caption={$\mathtt{WeakDirectSumDecomposition}$},
		language=GAP
		]{\pkgpath/FunctorCategories/gap/DirectSumDecomposition.gi}
From the package \texttt{FunctorCategories}\\
Back to \hyperref[lol]{Index}

		


\end{document}